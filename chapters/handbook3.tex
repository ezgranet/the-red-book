\part{Scope of Practice, Authorisation and Licensing Rules}




\chap{Part 3
- A. Application}


\rulesubsection{rS1}~

Section 3.B applies to all \emph{BSB regulated persons} and
\emph{unregistered barristers} and ``You'' and ``Your'' should be
construed accordingly. It provides that you \textcolor{myred}{\textbf{must not }}carry on any
\emph{reserved legal activity} or practise as a \emph{barrister} unless
you are authorised to do so, and explains the different capacities
within which you may work if you are so authorised and any limitations
on the scope of your \emph{practice}. It also explains the further
requirements which you \textcolor{myred}{\textbf{must }}follow if you intend to work in more than
one capacity.~

\rulesubsection{rS2}

Section 3.C applies to \emph{barristers} and \emph{registered European
lawyers} and sets out the basis on which they may apply for a
\emph{practising certificate} which will entitle them to practise within
England and Wales.

\rulesubsection{rS3}

Section 3.D applies to \emph{European lawyers} and provides details
about how to apply to become a \emph{registered European lawyer} in
England and Wales, thus entitling them to apply for a \emph{practising
certificate} in accordance with the provisions of 3.C.

\rulesubsection{rS4}

Section 3.E applies to all entities wishing to be regulated by the BSB
and sets out the basis upon which entities may be:
\nl\item authorised to practise as a \emph{BSB authorised body}; or
\item licensed to practise as a \emph{BSB licensed body}.
\ln
\rulesubsection{rS5}

Section 3.F applies to all \emph{BSB entities}. It contains the
continuing compliance requirements which apply to them.

\chap{Part 3
- B. Scope of Practice}



\rulesection{Part 3
- B1. No practice without authorisation (Rule S6)}


\rulesubsection{rS6}

You \textcolor{myred}{\textbf{must not }}carry on any \emph{reserved legal activity} unless you are
entitled to do so under the \emph{LSA}.


\guidancesection{Guidance
to Rule S6}


\guidancesubsubsection{gS1}

You are not entitled to carry on any \emph{reserved legal activity},
whether on your own behalf or acting as a \emph{manager} or employee,
unless you are either authorised or exempt in respect of that
\emph{reserved legal activity}. Where you are a \emph{manager} or
employee of a \emph{person} who, as part of their \emph{practice},
supplies services to \emph{the public} or to a section of \emph{the
public} (with or without a view to profit), which consist of, or
include, the carrying on of \emph{reserved legal activities}, that
\emph{person} must also be entitled to carry on that \emph{reserved
legal activity} under the LSA. Authorisation in accordance with this
Part 3 permits you to carry on the \emph{reserved legal activities}
specified in your authorisation.

\rulesection{Part 3
- B1. No practice without authorisation (Rules S7-S15)}


\rulesubsection{rS7}

You \textcolor{myred}{\textbf{must not }}permit any third party who is not authorised to provide
\emph{reserved legal activities} to provide such \emph{reserved legal
activities} on your behalf.

\rulesubsection{rS8}

If:
\nl\item you are an individual and do not have a \emph{practising
certificate}; or
\item you are an entity and you have not been authorised or licensed to
provide \emph{reserved legal activities} in accordance with Section 3.E,
then:
\al
\item you may not practise as a \emph{barrister} or a \emph{registered
European lawyer} or as a \emph{BSB entity} (as appropriate); and
\item you are not authorised by the \emph{Bar Standards Board} to carry on
any \emph{reserved legal activity}.\la
\ln
\rulesubsection{rS9}

For the purposes of this \emph{Handbook}, you practise as a
\emph{barrister} or a \emph{registered European lawyer}, or a \emph{BSB
entity} if you are supplying \emph{legal services} and:
\nl\item you are an individual and you hold a \emph{practising certificate};
or
\item you hold yourself out as a \emph{barrister} or a \emph{registered
European lawyer} (as appropriate) or
\item you are an entity and you have been authorised or licensed to provide
\emph{reserved legal activities} in accordance with Section 3.E; or
\item you act as a \emph{manager} of, or have an ownership interest in, an
\emph{authorised (non-BSB) body} and as such you are required by the
rules of that body's \emph{Approved Regulator} to hold a
\emph{practising certificate} issued by the \emph{Bar Standards
Board}~(as the case may be).
\ln
\rulesubsubsection{rS10}

For the purposes of this Section 3.B1 any reference to the supply of
\emph{legal services} includes an offer to supply such services.

\rulesubsubsection{rS11}

Rule rS9.1 above does not apply to you if you are a \emph{pupil} without
a provisional practising certificate if and insofar as you accept a
noting brief with the permission of your \emph{pupil supervisor} or head
of \emph{chambers} or \emph{HOLP}.

\rulesubsubsection{rS12}

If you are an \emph{unregistered barrister} or \emph{registered European
lawyer} but do not hold a \emph{practising certificate} and you supply
\emph{legal services} in the manner provided for in Rules rS13, rS14 and
rS15 below, then you shall not, by reason of supplying those services:
\nl\item be treated for the purposes of this Section B of Part 3 as
\emph{practising barrister} or a \emph{registered European lawyer}; or
\item be subject to the rules in Part 2 of this \emph{Handbook} or the
rules in this Section 3.B which apply to \emph{practising barristers}.
\ln
\rulesubsubsection{rS13}

Rule rS12 applies to you if and insofar as:
\nl\item you are practising as a \emph{foreign lawyer}; and
\item you do not:
\begin{alphlist}
\item give advice on \emph{English Law}; or

\item supply \emph{legal services} in connection with any proceedings or
contemplated proceedings in England and Wales (other than as an expert
witness on foreign law).\end{alphlist}
\ln
\rulesubsubsection{rS14}

Rule rS12 applies to you if:
\nl\item you are authorised and currently permitted to carry on reserved legal
activities by another \emph{Approved Regulator}; and
\item you hold yourself out as a \emph{barrister} or a \emph{registered
European lawyer} (as appropriate) other than as a \emph{manager} or
employee of a \emph{BSB entity}; and
\item when supplying \emph{legal services} to any \emph{person} or
\emph{employer} for the first time, you inform them clearly in writing
at the earliest opportunity that you are not practising as a
\emph{barrister} or a \emph{registered European lawyer}.
\ln
\rulesubsubsection{rS15}

Rule rS12 applies to you provided that:
\nl\item you supplied \emph{legal services} prior to 31 March 2012 pursuant to
paragraph 206.1 or 206.2 of the 8th Edition of the Code; and
\item if you supply any \emph{legal services} in England and Wales, you
were called to the \emph{Bar} before 31 July 2000; and
\item before 31 March in each year, and promptly after any change in the
details previously supplied to the \emph{Bar Standards Board}, you
provide in writing to the \emph{Bar Standards Board}, details of the
current address(es) with telephone number(s) of the office or premises
from which you do so, and:
\al
\item if you are employed, the name, address, telephone number and nature
of the \emph{practice} of your \emph{employer}; or

\item if you are an employee or \emph{manager} of, or you have an
ownership interest in, a \emph{regulated entity}, the name, address,
email address, telephone number and the name of the \emph{regulated
entity} and its \emph{Approved Regulator}; and\la
\item unless you only offer services to your \emph{employer} or to the
\emph{regulated entity} of which you are a \emph{manager} or an employee
or which you have an ownership interest in, you are (or, if you are
supplying \emph{legal services} to \emph{clients} of your
\emph{employer} or \emph{regulated entity} of which you are an
\emph{owner}, \emph{manager} or an employee, your \emph{employer} or
such body is) currently insured in accordance with the requirements of
Rule C76r and you comply with the requirements of Section 2.D4.
\ln
\rulesection{Part 3
- B2. Provision of reserved legal activities and of legal services
{Rules}}


\rulesubsubsection{rS16}

You may only carry on \emph{reserved legal activities} or supply other
\emph{legal services} in the following capacities:
\nl\item as a \emph{self-employed barrister}, subject to the limitations
imposed by Section 3.B3;
\item as a \emph{BSB entity} subject to the limitations imposed by Section
3.B4;
\item as a \emph{manager} of a \emph{BSB entity} or as an \emph{employed
barrister (BSB entity)}, subject to the limitations imposed by Section
3.B5;
\item as a \emph{manager} of an \emph{authorised (non-BSB) body} or as an
\emph{employed barrister (authorised non-BSB body),} subject to the
limitations imposed by Section 3.B6;
\item as an \emph{employed barrister (non authorised body)}, subject to the
limitations imposed by Section 3.B7; or
\item as a \emph{registered European lawyer} in any of the above
capacities, in which case the equivalent limitations that would have
applied if you were practising as a \emph{barrister} shall apply to your
\emph{practice} as a \emph{registered European lawyer}.
\ln
\rulesubsubsection{rS17}

Where you carry on \emph{reserved legal activities} in one of the
capacities set out at Rule rS16, so as to be subject to regulation by
the \emph{Bar Standards Board} in respect of those \emph{reserved legal
activities}, any other \emph{legal services} you may supply in that same
capacity will also be subject to regulation by the \emph{Bar Standards
Board}, even if unreserved.

\rulesubsubsection{rS18}

You may only \emph{practise} or be involved with the supply of
\emph{legal services} (whether \emph{reserved legal activities} or
otherwise) in more than one of the capacities listed in Rule rS16 after:
\nl\item having obtained an amended \emph{practising certificate} from the
\emph{Bar Standards Board} which recognises the capacities in respect of
which you are intending to practise; and
\item having agreed with each \emph{employer} or \emph{regulated entity}
with which you are involved a protocol that enables you to avoid or
resolve any conflict of interests or duties arising from your
\emph{practice} and/or involvement in those capacities,
\ln
and provided always that you do not work in more than one capacity in
relation to the same case or issue for the same \emph{client}, at the
same time.

\rulesubsubsection{rS19}

If you are a \emph{pupil} with a provisional practising certificate, you
may only supply \emph{legal services} to \emph{the public} or exercise
any right which you have by reason of being a \emph{barrister}, if you
have the permission of your \emph{pupil supervisor}, or head of
\emph{chambers} or \emph{HOLP} (as appropriate).

\rulesubsubsection{rS20}

Subject to Rule rS21, if you are a \emph{barrister} of less than three
\emph{years' standing}, you may:
\nl\item only supply \emph{legal services} to the public or exercise any
\emph{right of audience} by virtue of authorisation by the \emph{Bar
Standards Board}; or
\item only \emph{conduct litigation} by virtue of authorisation by the
\emph{Bar Standards Board},

if your principal place of \emph{practice} (or if you are
\emph{practising} in a dual capacity, each of your principal places of
\emph{practice}) is either:

\al\item a \emph{chambers} or an annex of \emph{chambers} which is also the
principal place of \emph{practice} of a relevant qualified \emph{person}
who is readily available to provide guidance to you; or

\item an office of an organisation of which an employee, \emph{partner},
\emph{manager} or \emph{director} is a relevant qualified \emph{person}
who is readily available to provide guidance to you.
\la\ln
\rulesubsubsection{rS21}

If you are an \emph{employed barrister (non-authorised body)} and you
are only exercising a \emph{right of audience} or conducting litigation
for those \emph{persons} listed at Rule rS39.1 to rS39.6, then the place
of \emph{practice} from which you perform such duties is only required
to be an office of an organisation of which an employee, \emph{partner},
\emph{manager} or \emph{director} is a relevant qualified \emph{person}
who is readily available to provide guidance to you if you are of less
than one year's standing.

\rulesubsubsection{rS22}

In Rule rS20 and Rule rS21 above, the references to ``qualified
\emph{person}'' mean the following:

\subsection{Supply of legal services to the public -- qualified person}
\nl\item Where you are a \emph{barrister} intending to supply \emph{legal
services} to \emph{the public}, a \emph{person} shall be a qualified
\emph{person} for the purpose of Rule rS20 if they:
\all have been entitled to \emph{practise} and have \emph{practised} as a
\emph{barrister} (other than as a \emph{pupil} who has not completed
\emph{pupillage} in accordance with the Bar Qualification Rules) or as a
\emph{person} authorised by another \emph{Approved Regulator} for a
period (which need not have been as a \emph{person} authorised by the
same \emph{Approved Regulator}) for at least six years in the previous
eight years; and

\item for the previous two years have made such \emph{practice} their
primary occupation; and

\item are not acting as a qualified \emph{person} in relation to more than
two other people; and

\item has not been designated by the \emph{Bar Standards Board} as
unsuitable to be a qualified \emph{person}.\la

\subsection{The exercise of a right of audience -- qualified person}
\item Where:
\al
\item you are a \emph{barrister} exercising a \emph{right of audience} in
England and Wales, a \emph{person} is a qualified \emph{person} for the
purpose of Rule rS20 if they:
\begin{romlist}
\item have been entitled to \emph{practise} and have \emph{practised} as a
\emph{barrister} (other than as a \emph{pupil} who has not completed
\emph{pupillage} in accordance with the \emph{Bar Qualification Rules})
or as a \emph{person} authorised by another \emph{Approved Regulator}
for a period (which need not have been as a \emph{person} authorised by
the same \emph{Approved Regulator}) for at least six years in the
previous eight years; and

\item for the previous two years:
\begin{numlist}
\item have made such \emph{practice} their primary occupation; and

\item have been entitled to exercise a \emph{right of audience} before
every \emph{court} in relation to all proceedings; and

\item are not acting as a qualified \emph{person} in relation to more
than two other people; and \end{numlist}

\item have not been designated by the \emph{Bar Standards Board} as
unsuitable to be a qualified \emph{person.}
\end{romlist}\la
\subsection{The exercise of a right to conduct litigation -- qualified
person}
\item Where:
\all you are a \emph{barrister} exercising a \emph{right to conduct
litigation} in England and Wales, a \emph{person} is a qualified
\emph{person} for the purpose of Rule rS20 if they:

\rl\item have been entitled to \emph{practise} and have \emph{practised} as a
\emph{barrister} (other than as a \emph{pupil} who has not completed
\emph{pupillage} in accordance with the Bar Qualification Rules) or as a
\emph{person} authorised by another \emph{Approved Regulator} for a
period (which need not have been as a \emph{person} authorised by the
same \emph{Approved Regulator}) for at least six years in the previous
eight years; and

\item for the previous two years have made such \emph{practice} their
primary occupation; and

\item are entitled to \emph{conduct litigation} before every \emph{court}
in relation to all proceedings; and

\item are not acting as a qualified \emph{person} in relation to more than
two other people; and

\item have not been designated by the \emph{Bar Standards Board} as
unsuitable to be a qualified \emph{person.}\lr\la\ln

\guidancesection{Guidance
to Rules S20-S22}


\guidancesubsubsection{gS2}

If you are a \emph{practising barrister} of less than three \emph{years'
standing} and you are authorised to \emph{conduct litigation}, you will
need to work with a qualified \emph{person} who is authorised to do
litigation as well as with someone who meets the criteria for being a
qualified \emph{person} for the purpose of providing services to
\emph{the public} and exercising \emph{rights of audience}. This may be,
but is not necessarily, the same \emph{person}.

\rulesection{Part 3
- B3. Scope of practice as a self-employed barrister (Rules S23-S24)
{Rules}}


\rulesubsubsection{rS23}

Rules rS24 and rS25 below apply to you where you are acting in your
capacity as a \emph{self-employed barrister}, whether or not you are
acting for a fee.

\rulesubsubsection{rS24}

You may only supply \emph{legal services} if you are appointed or
instructed by the \emph{court} or instructed:\nl\item by a \emph{professional client} (who may be an employee of the
\emph{client}); or
\item by a \emph{licensed access client}, in which case you \textcolor{myred}{\textbf{must }}comply
with the \emph{licensed access rules}; or
\item by or on behalf of any other \emph{client}, provided that:
\al\item the matter is \emph{public access instructions} and:
\rl\item you are entitled to provide public access work and the
\emph{instructions} are relevant to such entitlement; and

\item you have notified the \emph{Bar Standards Board} that you are
willing to accept \emph{instructions} from lay \emph{clients}; and

\item you comply with the \emph{public access rules}; or
\lr
\item the matter relates to the \emph{conduct of litigation} and
\rl
\item you have a litigation extension to your \emph{practising
certificate}; and

\item you have notified the \emph{Bar Standards Board} that you are
willing to accept \emph{instructions} from lay \emph{clients}.\lr\la\ln

\guidancesection{Guidance
to Rule S24}


\guidancesubsubsection{gS3}

References to professional \emph{client} in Rule rS24.1 include
\emph{foreign lawyers} and references to \emph{client} in Rule rS24.3
include \emph{foreign clients}.

\guidancesubsubsection{gS4}

If you are instructed by a \emph{foreign lawyer} to provide advocacy
services in relation to \emph{court} proceedings in England and Wales,
you should advise the \emph{foreign lawyer} of any limitation on the
services you can provide. In particular, if \emph{conduct of litigation}
will be required, and you are not authorised to \emph{conduct
litigation} or have not been instructed to do so, you should advise the
\emph{foreign lawyer} to take appropriate steps to instruct a
\emph{person} authorised to \emph{conduct litigation} and, if requested,
assist the \emph{foreign lawyer} to do so. If it appears to you that the
\emph{foreign lawyer} is not taking reasonable steps to instruct someone
authorised to \emph{conduct litigation}, then you should consider
whether to return your \emph{instructions} under rules C25 and C26.

\rulesection{Part 3
- B3. Scope of practice as a self-employed barrister (Rules S25-S26)
{Rules}}


\rulesubsubsection{rS25}

Subject to Rule rS26, you \textcolor{myred}{\textbf{must not }}in the course of your \emph{practice}
undertake the management, administration or general conduct of a
\emph{client's} affairs.

\rulesubsubsection{rS26}

Nothing in Rule rS25 prevents you from undertaking the management,
administration or general conduct of a client's affairs where such work
is \emph{foreign work} performed by you at or from an office outside
England and Wales which you have established or joined primarily for the
purposes of carrying out that particular \emph{foreign work} or
\emph{foreign work} in general.

\rulesection{Part 3
- B4. Scope of practice as a BSB entity (Rules S27-S28)}


\rulesubsubsection{rS27}

Rules rS28 and rS29 apply to you where you are acting in your capacity
as a \emph{BSB entity}.

\rulesubsubsection{rS28}

You may only supply \emph{legal services} if you are appointed or
instructed by the \emph{court} or instructed:\nl\item by a professional \emph{client} (who may be an employee of the
\emph{client});
\item by a \emph{licensed access client}, in which case you \textcolor{myred}{\textbf{must }}comply
with the \emph{licensed access rules}; or
\item by or on behalf of any other \emph{client}, provided that:

\al\item at least one manager or employee is suitably qualified and
experienced to undertake public access work; and

\item you have notified the \emph{Bar Standards Board} that you are willing
to accept \emph{instructions} from \emph{lay clients}.\la
\ln

\guidancesection{Guidance
to Rule S28}


\guidancesubsubsection{gS5}

References to professional client in Rule rS28.1 include foreign lawyers
and references to client in Rule rS28.3 include foreign clients.

\guidancesubsubsection{gS6}

If you are instructed to provide advocacy services in relation to
\emph{court} proceedings in England and Wales by a \emph{foreign lawyer}
or other professional \emph{client} who does not have a \emph{right to
conduct litigation} pursuant to Rule rS28.1 and you are not authorised
to \emph{conduct litigation} yourself or you are otherwise not
instructed to conduct the litigation in the particular matter, then you
must:\nl\item advise the \emph{foreign lawyer} to take appropriate steps to
instruct a \emph{solicitor} or other authorised litigator to conduct the
litigation and, if requested, take reasonable steps to assist the
\emph{foreign lawyer} to do so;
\item cease to act and return your \emph{instructions} if it appears to you
that the \emph{foreign lawyer} is not taking reasonable steps to
instruct a \emph{solicitor} or other authorised litigator to conduct the
litigation; and
\item not appear in \emph{court} unless a \emph{solicitor} or other
authorised litigator has been instructed to conduct the litigation.
\ln
\guidancesubsubsection{gS7}

The public access and licensed access rules do not apply to \emph{BSB
entities} as their circumstances will vary considerably. Nevertheless
those rules provide guidance on best practice. In the case of a
barrister, ``suitably qualified and experienced to undertake public
access work'' will mean successful completion of the public access
training required by the BSB or an exemption for the requirement to do
the training. If you are a \emph{BSB entity}, you will also need to have
regard to relevant provisions in the Code of Conduct (Part 2 of this
Handbook), especially C17, C21.7, C21.8 and C22. You will therefore need
to consider whether:\nl\item You have the necessary skills and experience to do the work,
including, where relevant, the ability to work with a vulnerable client;
\item The employees who will be dealing with the \emph{client} are either
authorised to \emph{conduct litigation} or entitled to do public access
work or have had other relevant training and experience;
\item it would be in the best interests of the client or of the interests
of justice for the client to instruct a solicitor or other professional
client if you are not able to provide such services;
\item If the matter involves the \emph{conduct of litigation} and you are
not able or instructed to \emph{conduct litigation}, whether the client
will be able to undertake the tasks that you cannot perform for them;
\item The \emph{client} is clear about the services which you will and will
not provide and any limitations on what you can do, and what will be
expected of them;
\item If you are not able to act in legal aid cases, the \emph{client} is
in a position to take an informed decision as to whether to seek legal
aid or proceed with public access.
\ln
\guidancesubsubsection{gS8}

You will also need to ensure that you keep proper records.

\rulesection{Part 3
- B4. Scope of practice as a BSB entity (Rules S29-S30)}


\rulesubsubsection{rS29}

Subject to Rule rS30, you \textcolor{myred}{\textbf{must not }}in the course of your \emph{practice}
undertake the management, administration or general conduct of a
\emph{client's} affairs.

\rulesubsubsection{rS30}

Nothing in Rule rS29 prevents you from undertaking the management,
administration or general conduct of a client's affairs where such work
is foreign work performed by you at or from an office outside England
and Wales which you have established or joined primarily for the
purposes of carrying out that particular foreign work or foreign work in
general.

\rulesection{Part 3
- B5. Scope of practice as a manager of a BSB entity or as an employed
barrister (BSB entity)}


\rulesubsubsection{rS31}

Rules rS32 and rS33 below apply to you where you are acting in your
capacity as a \emph{manager} of a \emph{BSB entity} or as an
\emph{employed barrister (BSB entity)}.

\rulesubsubsection{rS32}

You may only supply \emph{legal services} to the following
\emph{persons}:\nl\item the \emph{BSB entity}; or
\item any employee, \emph{director}, or company secretary of the \emph{BSB
entity} in a matter arising out of or relating to that \emph{person's}
employment;
\item any \emph{client} of the \emph{BSB entity};
\item if you supply \emph{legal services} at a \emph{Legal Advice Centre},
\emph{clients} of the \emph{Legal Advice Centre}; or~
\item if you supply \emph{legal services} free of charge, members of the
public.\ln

\rulesubsubsection{rS33}

Subject to Rule rS34, you \textcolor{myred}{\textbf{must not }}in the course of your practice
undertake the management, administration or general conduct of a
\emph{client's} affairs.

\rulesubsubsection{rS34}

Nothing in Rule rS33 prevents you from undertaking the management,
administration or general conduct of a client's affairs where such work
is foreign work performed by you at or from an office outside England
and Wales which you have established or joined primarily for the
purposes of carrying out that particular foreign work or foreign work in
general.

\rulesection{Part 3
- B6. Scope of practice as a manager of an authorised (non-BSB) body or
as an employed barrister (authorised non-BSB body)}


\rulesubsubsection{rS35}

Rules rS36 and rS37 apply to you where you are acting in your capacity
as a \emph{manager} of an \emph{authorised (non-BSB) body} or as an
\emph{employed barrister (authorised non-BSB body)}.

\rulesubsubsection{rS36}

You may only supply legal services to the following persons:\nl\item the \emph{authorised (non-BSB) body};
\item any employee, \emph{director} or company secretary of the
\emph{authorised (non-BSB) body} in a matter arising out of or relating
to that \emph{person's} employment;
\item any \emph{client} of the \emph{authorised (non-BSB) body};
\item if you provide \emph{legal services} at a \emph{Legal Advice Centre},
\emph{clients} of the \emph{Legal Advice Centre}; or
\item if you supply \emph{legal services} free of charge, members of the
public.\ln

\rulesubsubsection{rS37}

You \textcolor{myred}{\textbf{must }}comply with the rules of the \emph{Approved Regulator} or
\emph{licensing authority} of the \emph{authorised (non-BSB) body}.

\rulesection{Part 3
- B7. Scope of practice as an employed barrister (non authorised body)
{Rules}}


\rulesubsubsection{rS38}

Rule rS39 applies to you where you are acting in your capacity as an
\emph{employed barrister (non authorised body)}.

\rulesubsubsection{rS39}

Subject to s. 15(4) of the Legal Services Act 2007, you may only supply
\emph{legal services} to the following \emph{persons}:\nl\item your \emph{employer};
\item any employee, \emph{director} or company secretary of your
\emph{employer} in a matter arising out of or relating to that
\emph{person's} employment;
\item if your \emph{employer} is a public authority (including the Crown or
a Government department or agency or a local authority), another public
authority on behalf of which your \emph{employer} has made arrangements
under statute or otherwise to supply any \emph{legal services} or to
perform any of that other public authority's functions as agent or
otherwise;
\item if you are employed by or in a Government department or agency, any
Minister or Officer of the Crown;
\item if you are employed by a \emph{trade association}, any individual
member of the association;
\item if you are, or are performing the functions of, a \emph{Justices'
clerk}, the Justices whom you serve;
\item if you are employed by the \emph{Legal Aid Agency}, members of the
public;
\item if you are employed by or at a \emph{Legal Advice Centre},
\emph{clients} of the \emph{Legal Advice Centre};
\item if you supply \emph{legal services} free of charge, members of the
public; or
\item if your \emph{employer} is a \emph{foreign lawyer} and the
\emph{legal services} consist of foreign work, any \emph{client} of your
\emph{employer}.
\ln

\guidancesection{Guidance
to Rule S39}


\guidancesubsubsection{gS8A}

If you provide services through a \emph{non-authorised body} (A) whose
purpose is to facilitate the provision by you of in-house \emph{legal
services} to another \emph{non-authorised body} (B) then for the
purposes of rS39 you will be treated as if you are employed by B and you
should comply with your duties under this \emph{Handbook} as if you are
employed by B.

\guidancesubsubsection{gS8B}

If you provide services through a \emph{non-authorised body} (C) whose
purpose is to facilitate the provision by you of \emph{legal services}
to an authorised body (D) or clients of D (where those services are
provided by D and regulated by D's \emph{Approved Regulator}) then you
will be treated as if you are employed by D and you should comply with
your duties under this \emph{Handbook} as if you are employed by D.

\guidancesubsubsection{gS8C}

\emph{Reserved legal activities} may only be provided in a way that is
permitted by s15 of the \emph{Legal Services Act 2007}. S15 details when
an employer needs to be authorised to carry on \emph{reserved legal
activities} and prevents those activities from being provided to the
public, or a section of the public, by a \emph{non-authorised body}.

\rulesection{Part 3
- B8. Scope of practice of a barrister called to undertake a particular
case}


\rulesubsubsection{rS40}

If you are called to the \emph{Bar} under rQ25 (temporary call of QFLs),
you may not \emph{practise} as a \emph{barrister} other than to conduct
the case or cases specified in the certificate referred to in rQ26.

\rulesection{Part 3
- B9. Legal Advice Centres}


\rulesubsubsection{rS41}

You may supply \emph{legal services} at a \emph{Legal Advice Centre} on
a voluntary or part time basis and, if you do so, you will be treated
for the purposes of this \emph{Handbook} as if you were employed by the
\emph{Legal Advice Centre}.

\rulesubsubsection{rS42}

If you supply legal services at a \emph{Legal Advice Centre} to clients
of a \emph{Legal Advice Centre} in accordance with Rule rS41:\nl\item you \textcolor{myred}{\textbf{must not }}in any circumstances receive either directly or
indirectly any fee or reward for the supply of any \emph{legal services}
to any \emph{client} of the \emph{Legal Advice Centre} other than a
salary paid by the \emph{Legal Advice Centre};
\item you \textcolor{myred}{\textbf{must }}ensure that any fees in respect of \emph{legal services}
supplied by you to any \emph{client} of the \emph{Legal Advice Centre}
accrue and are paid to the \emph{Legal Advice Centre}, or to the Access
to Justice Foundation or other such charity as prescribed by order made
by the Lord Chancellor under s.194(8) of the Legal Services Act 2007;
and
\item you \textcolor{myred}{\textbf{must not }}have any financial interest in the \emph{Legal Advice
Centre}.
\ln

\guidancesection{Guidance
to Rules S41-S42}


\guidancesubsubsection{gS9}

You may provide \emph{legal services} at a \emph{Legal Advice Centre} on
an unpaid basis irrespective of the capacity in which you normally work.

\guidancesubsubsection{gS10}

If you are a \emph{self-employed barrister}, you do not need to inform
the Bar Standards Board that you are also working for a \emph{Legal
Advice Centre}.

\guidancesubsubsection{gS11}

Transitional arrangements under the LSA allow \emph{Legal Advice
Centres} to provide \emph{reserved legal activities} without being
authorised. When this transitional period comes to an end, the Rules
relating to providing services at \emph{Legal Advice Centres} will be
reviewed.

\rulesection{Part 3
- B10. Barristers authorised by other approved regulators}


\rulesubsubsection{rS43}

If you are authorised by another \emph{Approved Regulator} to carry on a
\emph{reserved legal activity} and currently permitted to
\emph{practise} by that \emph{Approved Regulator}, you \textcolor{myred}{\textbf{must }}not
\emph{practise} as a \emph{barrister} and you are not eligible for a
\emph{practising certificate}.

\chap{Part 3
- C. Practising Certificate Rules}



\rulesection{Part 3
- C1. Eligibility for practising certificates and litigation extensions
{Rules}}


\rulesubsubsection{rS44}

In this Section 3.C, references to ``you'' and ``your'' are references
to \emph{barristers} and \emph{registered European lawyers} who are
intending to apply for authorisation to \emph{practise} as a
\emph{barrister} or a \emph{registered European lawyer} (as the case may
be) or who are otherwise intending to apply for a \emph{litigation
extension} to their existing \emph{practising certificate}.

\rulesubsubsection{rS45}

You are eligible for a \emph{practising certificate} if:\nl\item you are a \emph{barrister} or \emph{registered European lawyer} and
you are not currently \emph{suspended} from \emph{practice} and have not
been disbarred; and
\item you meet the requirements of Rules rS46.1, rS46.2, rS46.3 or rS46.4;
and
\item either:
\al
\item  within the last 5 years either \rl 
\item  you have held a \emph{practising
certificate}; or 
\item you have satisfactorily completed (or have been
exempted from the requirement to complete) the pupillage component of
training; or\lr

\item if not, you have complied with such training requirements as may be
imposed by the \emph{Bar Standards Board}.\la
\ln
\rulesubsubsection{rS46}

You are eligible for:\nl\item a \emph{full practising certificate} if either:
\al
\item you have satisfactorily completed \emph{pupillage}; or

\item you have been exempted from the requirement to complete pupillage; or

\item on 30 July 2000, you were entitled to exercise full \emph{rights of
audience} by reason of being a \emph{barrister}; or

\item you were called to the \emph{Bar} before 1 January 2002 and:
\rl
\item you notified the \emph{Bar Council} that you wished to exercise a
\emph{right of audience} before every \emph{court} and in relation to
all proceedings; and

\item you have complied with such training requirements as the \emph{Bar
Council} or the \emph{Bar Standards Board} may require or you have been
informed by the \emph{Bar Council} or the \emph{Bar Standards Board}
that you do not need to comply with any such further requirements;\lr\la

in each case, before 31 March 2012;
\item a \emph{provisional practising certificate} if you have
satisfactorily completed (or have been exempted from the requirement to
complete) a period of pupillage satisfactory to the \emph{BSB} for the
purposes of Rule Q4 and at the time when you apply for a
\emph{practising certificate} you are registered as a \emph{Pupil};
\item a \emph{limited practising certificate} if you were called to the
\emph{Bar} before 1 January 2002 but you are not otherwise eligible for
a \emph{full practising certificate} in accordance with Rule rS46.1
above; or
\item a \emph{registered European lawyer's practising certificate} if you
are a \emph{registered European lawyer}.
\ln
\rulesubsubsection{rS47}

You are eligible for a litigation extension:\nl\item where you have or are due to be granted a \emph{practising
certificate} (other than a \emph{provisional practising certificate});
and
\item where you are:
\al
\item more than three \emph{years' standing}; or

\item less than three \emph{years' standing}, but your principal place of
\emph{practice} (or if you are \emph{practising} in a dual capacity,
each of your principal places of \emph{practice}) is either:
\rl
\item a \emph{chambers} or an annex of \emph{chambers} which is also the
principal place of \emph{practice} of a qualified \emph{person} (as that
term is defined in Rule rS22.3) who is readily available to provide
guidance to you; or

\item an office of an organisation of which an employee, \emph{partner},
\emph{manager} or \emph{director} is a qualified \emph{person} (as that
term is defined in Rule rS22.3) who is readily available to provide
guidance to you;
\lr\la
\item you have the relevant administrative systems in place to be able to
provide \emph{legal services} direct to \emph{clients} and to administer
the conduct of litigation; and
\item you have the procedural knowledge to enable you to \emph{conduct
litigation} competently.

\ln
\guidancesection{Guidance
to Rule S47.3}


\guidancesubsubsection{gS12}

You should refer to the more detailed guidance published by the
\emph{Bar Standards Board} from time to time which can be found on its
website. This provides more information about the evidence you may be
asked for to show that you have procedural knowledge to enable you to
\emph{conduct litigation} competently

\rulesection{Part 3
- C2. Applications for practising certificates and litigation extensions
by barristers and registered European lawyers}


\rulesubsubsection{rS48}

You may apply for a \emph{practising certificate} by:\nl\item completing the relevant application form (in such form as may be
designated by the \emph{Bar Standards Board}) and submitting it to the
\emph{Bar Standards Board}; and
\item submitting such information in support of the application as may be
prescribed by the \emph{Bar Standards Board}; and
\item paying (or undertaking to pay in a manner determined by the \emph{Bar
Council}) the appropriate \emph{practising certificate fee}~to the
\emph{Bar Council} in the amount determined in accordance with Rule rS50
(subject to any reduction pursuant to Rule rS53).\ln

\rulesubsubsection{rS49}

You may apply for a litigation extension to a \emph{practising
certificate} (other than a \emph{provisional practising certificate})
by:\nl\item completing the relevant application form supplied by the \emph{Bar
Standards Board}~and submitting it to the \emph{Bar Standards Board};
and
\item confirming that you meet the relevant requirements of Rule rS47.1;
\item paying (or undertaking to pay in a manner determined by the \emph{Bar
Standards Board}) the \emph{application fee} (if any) and the
\emph{litigation extension fee} (if any) to the \emph{Bar Standards
Board};
\item confirming, in~such form as the \emph{Bar Standards Board} may
require from time to time, that you have the relevant administrative
systems in place to be able to provide \emph{legal services} direct to
\emph{clients} and to administer the \emph{conduct of litigation} in
accordance with Rule rS47.3; and
\item confirming, in such form as the \emph{Bar Standards Board} may
require from time to time, that you have the procedural knowledge to
enable you to \emph{conduct litigation} competently in accordance with
Rule rS47.4.
\ln
\rulesubsubsection{rS50}

An application will only have been made under either Rule rS48 or rS49
once the~\emph{Bar Standards Board}~has received, in respect of the
relevant application, the application form in full, together with the
\emph{application fee}, the \emph{litigation extension fee} (if any, or
an undertaking to pay such a fee~in a manner determined by the \emph{Bar
Standards Board}), all the information required in support of the
application, confirmation from you, in the form of a declaration, that
the information contained in, or submitted in support of, the
application is full and accurate, and (in the case of Rule S48) once the
\emph{Bar Council} has received the \emph{practising certificate fee}
(if any, or an undertaking to pay such a fee in a manner determined by
the \emph{Bar Council}).

\rulesubsubsection{rS51}

On receipt of the application, the \emph{Bar Standards Board}~may
require, from you or a third party (including, for the avoidance of
doubt, any \emph{BSB entity}), such additional information, documents or
references as it considers appropriate to the consideration of your
application.

\rulesubsubsection{rS52}

You are personally responsible for the contents of your application and
any information submitted to the \emph{Bar Standards Board}~by you or on
your behalf and you \textcolor{myred}{\textbf{must not }}submit (or cause or permit to be submitted
on your behalf) information to the \emph{Bar Standards Board}~which you
do not believe is full and accurate.

\rulesubsubsection{rS53}

When applying for a \emph{practising certificate} you may apply to the
\emph{Bar Standards Board}~for a reduction in the \emph{practising
certificate fee} payable by you if your gross fee income or salary is
less than such amount as the \emph{Bar Council} may decide from time to
time. Such an application must be submitted by completing the form
supplied for that purpose by the \emph{Bar Standards Board}.

\rulesection{Part 3
- C3. Practising certificate fees and litigation extension fees}


\rulesubsubsection{rS54}

The \emph{practising certificate fee} shall be the amount or amounts
prescribed in the Schedule of \emph{Practising Certificate Fees} issued
by the \emph{Bar Council} from time to time, and any reference in these
Rules to the ``appropriate \emph{practising certificate fee}'' or the
``\emph{practising certificate fee} payable by you'' refers to the
\emph{practising certificate fee} payable by you pursuant to that
Schedule, having regard, amongst other things, to:\nl\item the different annual \emph{practising certificate fees} which may be
prescribed by the \emph{Bar Council} for different categories of
\emph{barristers}, e.g. for Queen's Counsel and junior counsel, for
\emph{barristers} of different levels of seniority, and/or for
\emph{barristers} \emph{practising} in different capacities and/or
according to different levels of income (i.e. \emph{self-employed
barristers}, \emph{employed barristers}, \emph{managers} or employees of
\emph{BSB entities} or \emph{barristers practising} with dual capacity);
\item any reductions in the annual \emph{practising certificate fees} which
may be permitted by the \emph{Bar Council} in the case of
\emph{practising certificates} which are valid for only part of a
\emph{practising certificate year};
\item any discounts from the annual \emph{practising certificate fee} which
may be permitted by the \emph{Bar Council} in the event of payment by
specified methods;
\item any reduction in, or rebate from, the annual \emph{practising
certificate fee} which may be permitted by the \emph{Bar Council} on the
grounds of low income, change of category or otherwise; and
\item any surcharge or surcharges to the annual \emph{practising
certificate fee} which may be prescribed by the Bar Council in the event
of an application for renewal of a \emph{practising certificate} being
made after the end of the \emph{practising certificate year}.
\ln
\rulesubsubsection{rS55}

The \emph{litigation extension fee} shall be the amount or amounts
prescribed by the \emph{Bar Standards Board}~from time to time, and in
these Rules the ``appropriate \emph{litigation extension fee}'' or the
``\emph{litigation extension fee} payable by you'' is the
\emph{litigation extension fee} payable by you having regard to, among
other things:\nl\item any reductions in the annual \emph{litigation extension fees} which
may be permitted by the \emph{Bar Standards Board}~in the case of
\emph{litigation extensions} which are valid for only part of a
\emph{practising certificate year};
\item any discounts from the annual \emph{litigation extension fee} which
may be permitted by the \emph{Bar Standards Board}~in the event of
payment by specified methods;
\item any reduction in, or rebate from, the annual \emph{litigation
extension fee} which may be permitted by the \emph{Bar Standards
Board}~on the grounds of low income, change of category, or otherwise;
and
\item any surcharge or surcharges to the annual \emph{litigation extension
fee} which may be prescribed by the \emph{Bar Standards Board}~in the
event of an application for a \emph{litigation extension} being made at
a time different from the time of your application for a
\emph{practising certificate}.
\ln
\rulesubsubsection{rS56}

If you have given an undertaking to pay the \emph{practising certificate
fee}~to the \emph{Bar Council} or the \emph{litigation extension fee~}to
the \emph{Bar Standards Board}, you \textcolor{myred}{\textbf{must }}comply with that undertaking in
accordance with its terms.

\rulesection{Part 3
- C4. Issue of practising certificates and litigation extensions
{Rules}}


\rulesubsubsection{rS57}

The \emph{Bar Standards Board}~shall not issue a \emph{practising
certificate} to a \emph{barrister} or \emph{registered European lawyer}:\nl\item who is not eligible for a \emph{practising certificate}, or for a
\emph{practising certificate} of the relevant type; or
\item who has not applied for a \emph{practising certificate}; or
\item who has not paid or not otherwise undertaken to pay in a manner
determined by the \emph{Bar Council}, the appropriate \emph{practising
certificate fee}; or
\item who is not insured against claims for professional negligence as
provided for in Rule C76.
\ln
\rulesubsubsection{rS58}

The \emph{Bar Standards Board}~shall not grant a \emph{litigation
extension} to a \emph{barrister} or \emph{registered European lawyer}:\nl\item in circumstances where the \emph{Bar Standards Board}~is not
satisfied that the requirements of \emph{litigation extension} are met;
or
\item who has not applied for a \emph{litigation extension}; or
\item who has not paid or not otherwise undertaken to pay in a manner
determined by the \emph{Bar Standards Board}, the appropriate
\emph{application fee} (if any) and the \emph{litigation extension fee}
(if any).
\ln
\rulesubsubsection{rS59}

The \emph{Bar Standards Board}~may refuse to issue a \emph{practising
certificate} or to grant a \emph{litigation extension}, or may revoke a
\emph{practising certificate} or a \emph{litigation extension} in
accordance with Section 3.C5, if it is satisfied that the information
submitted in support of the application for the \emph{practising
certificate} or \emph{litigation extension} (as the case may be) is (or
was when submitted) incomplete, inaccurate or incapable of verification,
or that the relevant \emph{barrister} or \emph{registered European
lawyer}:\nl\item does not hold adequate insurance in accordance with Rule C76;
\item has failed and continues to fail to pay the appropriate
\emph{practising certificate fee}~to the \emph{Bar Council} or
\emph{litigation extension fee}~to the \emph{Bar Standards Board} when
due;
\item would be, or is, \emph{practising} in breach of the provisions of
Section 3.B;
\item has not complied with any of the requirements of the Continuing
Professional Development Regulations applicable to them;
\item has not declared information on type and area of practice in a form
determined by the BSB;
\item has not made the declarations required by the BSB in relation to
Youth Court work;
\item has not made the declarations required by the BSB in relation to the
Money Laundering, Terrorist Financing and Transfer of Funds (Information
on the Payer) Regulations 2017;
\item has not provided the BSB with a unique email address.
\ln
\rulesubsubsection{rS60}

When the~\emph{Bar Standards Board}~issues a \emph{practising
certificate} or a \emph{litigation extension}, it shall:\nl\item inform the relevant \emph{barrister} or \emph{registered European
lawyer} of that fact; and
\item in the case of a \emph{practising certificate}, publish that fact,
together with the name and \emph{practising address} of the
\emph{barrister} and \emph{registered European lawyer} and the other
details specified in Rule rS61 in the register on the \emph{Bar
Standards Board's} website; or
\item in the case of a litigation extension:
\al
\item issue a revised and updated \emph{practising certificate} to
incorporate an express reference to such litigation extension in
accordance with Rule rS66; and

\item amend the register maintained on the Bar Standards Board's website to
show that the relevant \emph{barrister} or \emph{registered European
lawyer} (as the case may be) is now authorised to \emph{conduct
litigation}.\la\ln
\rulesubsubsection{rS61}

A \emph{practising certificate} must state:\nl\item the name of the \emph{barrister} or \emph{registered European lawyer}
(as the case may be);
\item the period for which the \emph{practising certificate} is valid;
\item the \emph{reserved legal activities} which the \emph{barrister} or
\emph{registered European lawyer} (as the case may be) to whom it is
issued is thereby authorised to carry on;
\item the capacity (or capacities) in which the \emph{barrister} or
\emph{registered European lawyer} (as the case may be) practises; and
\item whether the \emph{barrister} or \emph{registered European lawyer} (as
the case may be) is registered with the \emph{Bar Standards Board}~as a
\emph{Public Access} practitioner.
\ln
\rulesubsubsection{rS62}

A \emph{practising certificate} may be valid for a \emph{practising
certificate year} or part thereof and for one month after the end of the
\emph{practising certificate year}.

\rulesubsubsection{rS63}

A \emph{full practising certificate} shall authorise a \emph{barrister}
to exercise a \emph{right of audience} before every \emph{court} in
relation to all proceedings.

\rulesubsubsection{rS64}

A \emph{provisional practising certificate} shall authorise a
\emph{pupil} to exercise a \emph{right of audience} before every
\emph{court} in relation to all proceedings.

\rulesubsubsection{rS65}

A \emph{limited practising certificate} shall not authorise a
\emph{barrister} to exercise a \emph{right of audience}, save that it
shall authorise a \emph{barrister} to exercise any \emph{right of
audience} which they had by reason of being a \emph{barrister} and was
entitled to exercise on 30 July 2000.

\rulesubsubsection{rS66}

A \emph{practising certificate} shall authorise a \emph{barrister} to
\emph{conduct litigation} in relation to every \emph{court} and all
proceedings if the \emph{practising certificate} specifies a
\emph{litigation extension}.

\rulesubsubsection{rS67}

Every \emph{practising certificate} issued to a \emph{barrister} shall
authorise the \emph{barrister}:\nl\item to undertake:
\al
\item \emph{reserved instrument activities};

\item \emph{probate activities};

\item \emph{the administration of oaths}; and

\item \emph{immigration work}.\la
\ln
\rulesubsubsection{rS68}

A \emph{registered European lawyer's practising certificate} shall
authorise a \emph{registered European lawyer} to carry on the same
\emph{reserved legal activities} as a \emph{full practising certificate}
issued to a \emph{barrister}, save that:\nl\item a \emph{registered European lawyer} is only authorised to exercise a
\emph{right of audience} or \emph{conduct litigation} in proceedings
which can lawfully only be provided by a \emph{solicitor},
\emph{barrister} or other qualified \emph{person}, if they act in
conjunction with a \emph{solicitor} or \emph{barrister} authorised to
\emph{practise} before the \emph{court}, tribunal or public authority
concerned and who could lawfully exercise that right; and
\item a \emph{registered European lawyer} is not authorised to prepare for
remuneration any instrument creating or transferring an interest in land
unless they have a \emph{home professional title} obtained in Denmark,
the Republic of Ireland, Finland, Sweden, Iceland, Liechtenstein,
Norway, the Czech Republic, Cyprus, Hungary or Slovakia.
\ln
\rulesection{Part 3
- C5. Amendment and revocation of practising certificates and litigation
extensions}


\rulesubsubsection{rS69}

You \textcolor{myred}{\textbf{must }}inform the \emph{Bar Standards Board}~as soon as reasonably
practicable, and in any event within 28 days, if any of the information
submitted in support of your \emph{practising certificate} application
form or \emph{litigation extension} application form:\nl\item was incomplete or inaccurate when the application form was submitted;
or
\item changes before the expiry of your \emph{practising certificate}.
\ln
\rulesubsubsection{rS70}

If you wish to:\nl\item change the capacity in which you \emph{practise} (e.g. if you change
from being an \emph{employed barrister} or a \emph{manager} or employee
of a \emph{BSB entity} or an \emph{authorised (non-BSB) body} to a
\emph{self-employed barrister}, or vice versa, or if you commence or
cease \emph{practice} in a dual capacity); or
\item cease to be authorised to \emph{conduct litigation},

before the expiry of your \emph{practising certificate}, you \textcolor{red}{\textbf{must}}:
\al
\item notify the \emph{Bar Standards Board~}of such requested amendment to
your \emph{practising certificate}; and

\item submit to the \emph{Bar Standards Board}~such further information as
the \emph{Bar Standards Board}~may reasonably require in order for them
to be able to determine whether or not to grant such proposed amendment
to your \emph{practising certificate}; and

\item within 14 days of demand by the \emph{Bar Council} pay to the
\emph{Bar Council} the amount (if any) by which the annual
\emph{practising certificate fee} which would apply to you in respect of
your amended \emph{practising certificate} exceeds the annual
\emph{practising certificate fee} which you have already paid (or
undertaken to pay) to the \emph{Bar Council}. In the event that the
revised annual \emph{practising certificate fee} is less than the amount
originally paid to the \emph{Bar Council}~or in circumstances where you
wish to cease to be authorised to \emph{conduct litigation}, the
\emph{Bar Council}~is not under any obligation to refund any part of the
annual \emph{practising certificate fee} already paid although it may in
its absolute discretion elect to do so in the circumstances contemplated
by the Schedule of \emph{Practising Certificate} Fees issued by the
\emph{Bar Council} from time to time. In circumstances where you wish to
cease to be authorised to \emph{conduct litigation}, the \emph{Bar
Standards Board} is not under any obligation to refund any part of the
\emph{litigation extension fee} already paid although it may in its
absolute discretion elect to do so.\la\ln
\rulesubsubsection{rS71}

The \emph{Bar Standards Board}~may amend a \emph{practising certificate}
if it is satisfied that any of the information contained in the relevant
application form was inaccurate or incomplete or has changed, but may
not amend a \emph{practising certificate} (except in response to a
request from the \emph{barrister} or a \emph{registered European
lawyer}) without first:\nl\item giving written notice to the \emph{barrister} or \emph{registered
European lawyer} of the grounds on which the \emph{practising
certificate} may be amended; and
\item giving the \emph{barrister} or \emph{registered European lawyer} a
reasonable opportunity to make representations.
\ln
\rulesubsubsection{rS72}

The \emph{Bar Standards Board}~shall endorse a \emph{practising
certificate} to reflect any qualification restriction or condition
imposed on the \emph{barrister} or \emph{registered European lawyer} by
the \emph{Bar Standards Board}~or by a \emph{Disciplinary Tribunal},
\emph{Interim Suspension or Disqualification Panel}, \emph{Fitness to
Practise Panel}~or the High Court.

\rulesubsubsection{rS73}

The \emph{Bar Standards Board}:\nl\item shall revoke a \emph{practising certificate}:
\al
\item if the \emph{barrister} becomes authorised to practise by another
\emph{approved regulator};

\item if the \emph{barrister} or \emph{registered European lawyer} is
disbarred or \emph{suspended} from \emph{practice} as a \emph{barrister}
or \emph{registered European lawyer} whether on an interim basis under
section D of Part 5 or otherwise under section B of Part 5;

\item if the \emph{barrister} or \emph{registered European lawyer} has
notified the \emph{Bar Standards Board} that they no longer wish to have
a \emph{practising certificate};

\item in the case of a \emph{Registered European Lawyer}, where the
individual no longer meets the eligibility requirements;~and\la
\item may revoke a \emph{practising certificate}:
\al
\item in the circumstances set out in Rule rS59; or

\item if the \emph{barrister} or \emph{registered European lawyer} has
given an undertaking to pay the appropriate \emph{practising certificate
fee} and fails to comply with that undertaking in accordance with its
terms, but in either case only after:

\rl \item giving written notice to the relevant \emph{barrister} or
\emph{registered European lawyer} of the grounds on which the
\emph{practising certificate} may be revoked; and

\item giving the relevant \emph{barrister} or \emph{registered European
lawyer} a reasonable opportunity to make representations.\lr\la
\ln
\rulesubsubsection{rS74}

The \emph{Bar Standards Board}:\nl\item shall revoke a \emph{litigation extension} if the \emph{barrister} or
\emph{registered European lawyer} has notified the \emph{Bar Standards
Board} that they no longer wish to have the \emph{litigation extension};
and
\item may revoke a \emph{litigation extension}:

\al\item in the circumstances set out in Rule rS59; or

\item if the \emph{barrister} or \emph{registered European lawyer} has
given an undertaking to pay the appropriate \emph{litigation extension
fee} and fails to comply with that undertaking in accordance with its
terms, but in either case only after:

\rl\item giving written notice to the relevant \emph{barrister} or
\emph{registered European lawyer} of the grounds on which the
\emph{litigation extension} may be revoked; and

\item giving the relevant \emph{barrister} or \emph{registered European
lawyer} a reasonable opportunity to make representations.
\lr\la\ln
\rulesection{Part 3
- C6. Applications for review}


\rulesubsubsection{rS75}

If you contend that the \emph{Bar Standards Board}~has:\nl\item wrongly failed or refused to issue or amend a \emph{practising
certificate}; or
\item wrongly amended or revoked a \emph{practising certificate}; or
\item wrongly failed or refused to issue a \emph{litigation extension}; or
\item wrongly revoked a \emph{litigation extension},

in each case in accordance with this Section 3.C, then you may lodge an
application for review using the form supplied for that purpose by the
\emph{Bar Standards Board} which can be found on its website. For the
avoidance of doubt, this Section 3.C6 does not apply to any amendment or
revocation of a \emph{practising certificate} or \emph{litigation
extension} made by order of a \emph{Disciplinary Tribunal},
\emph{Interim Suspension} or \emph{Disqualification Panel},
\emph{Fitness to Practise Panel}~or the High Court.

\rulesubsubsection{rS76}

The decision of the~\emph{Bar Standards Board}~shall take effect
notwithstanding any application for review being submitted in accordance
with Rule S75. However, the \emph{Bar Standards Board}~may, in its
absolute discretion, issue a temporary \emph{practising certificate} or
\emph{litigation extension} to a \emph{barrister} or \emph{registered
European lawyer} who has lodged an application for review.
\ln
\rulesubsubsection{rS77}

If the review finds that the \emph{Bar Standards Board}:\nl\item has wrongly failed or refused to issue a \emph{practising
certificate}, then the \emph{Bar Standards Board}~must issue such
\emph{practising certificate} as ought to have been issued; or
\item has wrongly failed or refused to amend a \emph{practising
certificate}, then the \emph{Bar Standards Board}~must make such
amendment to the \emph{practising certificate} as ought to have been
made; or
\item has wrongly amended a \emph{practising certificate}, then the
\emph{Bar Standards Board}~must cancel the amendment; or
\item has wrongly revoked a \emph{practising certificate}, then the
\emph{Bar Standards Board}~must re-issue the \emph{practising
certificate}; or
\item has wrongly failed or refused to grant a \emph{litigation extension},
then the \emph{Bar Standards Board}~must grant such \emph{litigation
extension} as ought to have been granted; or
\item has wrongly revoked a \emph{litigation extension}, then the \emph{Bar
Standards Board}~must re-grant the \emph{litigation extension}.
\ln
\chap{Part 3
- D. The Registration of European Lawyers Rules}


\rulesubsubsection{rS78}

If you are a \emph{Qualified Swiss~lawyer} and wish to \emph{practise}
in England and Wales under a \emph{home professional title}, you may
apply to the \emph{Bar Standards Board} to be registered as a
\emph{registered European lawyer}.~Such an application will be valid if
it was made before 1 January 2025 and in accordance with the \emph{Swiss
Citizens' Rights Agreement.}

\rulesubsubsection{rS79}

An application for registration must be made before 1 January 2025 in
such form as may be prescribed by the \emph{Bar Standards Board} and be
accompanied by:\nl\item a certificate, not more than three months old at the date of receipt
of the application by the \emph{Bar Standards Board}, that you are
registered with the Competent Authority in Switzerland~as a lawyer
qualified to \emph{practise} in that \emph{Member State} under a
relevant Swiss~professional title;
\item a declaration that:

\al\item you have not on the grounds of misconduct or of the commission of a
\emph{criminal offence} been prohibited from practising in
Switzerland~and are not currently \emph{suspended} from so practising;

\item no \emph{bankruptcy order} or \emph{directors disqualification order}
has been made against you and you have not entered into an individual
voluntary arrangement with your creditors;

\item you are not aware of any other circumstances relevant to your fitness
to \emph{practise} under your \emph{home professional title} in England
and Wales; and

\item you are not registered with the Law Society of England and Wales, of
Scotland or of Northern Ireland; and\la
\item the prescribed fee.
\ln
\rulesubsubsection{rS80}

Provided that it is satisfied that the application complies with the
requirements of Rule rS79, the \emph{Bar Standards Board} will:\nl\item register you as a \emph{registered European lawyer}; and
\item so inform you and the competent authority in your \emph{Member State}
which has issued the certificate referred to in Rule rS79.1.
\ln
\rulesubsubsection{rS81}

The \emph{Bar Standards Board} will:\nl\item remove a \emph{registered European lawyer} from the register:
\al
\item pursuant to a sentence of a \emph{Disciplinary Tribunal}; or

\item if the \emph{registered European lawyer} ceases to be a
\emph{European lawyer};\la
\item suspend a \emph{registered European lawyer} from the register:
\al
\item pursuant to a sentence of either a \emph{Disciplinary Tribunal} or an
\emph{Interim Suspension Panel}; or

\item if the \emph{registered European lawyer's} authorisation in their
\emph{home State} to pursue professional activities under their
\emph{home professional title} is \emph{suspended};

and in each case, notify the \emph{European lawyer's home professional
body}:

\item of their removal or suspension from the register; and

\item of any criminal conviction or \emph{bankruptcy order} of which it
becomes aware against a \emph{registered European lawyer}.\la
\ln
\chap{Part 3
- E. Entity Application and Authorisation}



\rulesection{Part 3
- E1. Eligibility for authorisation to practise as a BSB entity}


\rulesubsubsection{rS82}

In this Section 3.E, ``you'' and ``your'' refer to the
\emph{partnership}, \emph{LLP} or \emph{company} which is applying for,
or has applied for (in accordance with this Section 3.E) authorisation
or (if a licensable body) a licence to practise as a \emph{BSB entity},
and references in these Rules to ``authorisation to practise'' mean the
grant by the \emph{Bar Standards Board}~of an authorisation or a licence
(as the case may be) under this Section 3.E (distinguishing between the
two only where the context so requires).

\rulesubsubsection{rS83}

To be eligible for authorisation to \emph{practise} as a \emph{BSB
entity}, you:\nl\item must have arrangements in place designed to ensure at all times that
any obligations imposed from time to time on the \emph{BSB entity}, its
\emph{managers}, \emph{owners} or employees by or under the \emph{Bar
Standards Board's} regulatory arrangements, including its rules and
disciplinary arrangements, are complied with and confirm that the
\emph{BSB entity} and all \emph{owners} and \emph{managers} expressly
consent to be bound by the \emph{Bar Standards Board's} regulatory
arrangements (including disciplinary arrangements);
\item must have arrangements in place designed to ensure at all times that
any other statutory obligations imposed on the \emph{BSB entity}, its
\emph{managers}, \emph{owners} or employees, in relation to the
activities it carries on, are complied with;
\item must confirm that, subject to the provisions of rS131, you will have
in place, at all times, individuals appointed to act as a \emph{HOLP}
(who must also be a \emph{manager}) and a \emph{HOFA} of the \emph{BSB
entity};
\item must confirm that you have or will have appropriate insurance
arrangements in place at all times in accordance with Rule C76 and you
must be able to provide evidence of those insurance arrangements if
required to do so by the \emph{Bar Standards Board};
\item must confirm that, in connection with your proposed \emph{practice},
you will not directly or indirectly hold \emph{client money} in
accordance with Rule C73 or have someone else hold \emph{client money}
on your behalf other than in those circumstances permitted by Rule C74;
\item must confirm that no individual that has been appointed or will be
appointed as a \emph{HOLP}, \emph{HOFA}, \emph{manager} or employee of
the \emph{BSB entity} is disqualified from acting as such by the
\emph{Bar Standards Board} or any \emph{Approved Regulator} pursuant to
section 99 of the \emph{LSA} or otherwise as a result of its regulatory
arrangements;
\item must confirm that you will at all times have a \emph{practising
address} in England or Wales;
\item must confirm that:
\al
\item if you are an \emph{LLP}, you are incorporated and registered in
England and Wales, Scotland or Northern Ireland under the Limited
Liability Partnerships Act 2000;

\item if you are a \emph{Company}, you are:
\rl
\item incorporated and registered in England and Wales, Scotland or
Northern Ireland under Parts 1 and 2 of the Companies Act 2006;\lr\la
\item must confirm that at least one \emph{manager} or employee is an
\emph{authorised individual} in respect of each \emph{reserved legal
activity} which you wish to provide;

\item must confirm that you will pay annual fees as and when they become
due.
\ln
\rulesubsubsection{rS84}

In addition to the requirements set out at Rule rS83:\nl\item to be eligible for authorisation to \emph{practise} as a \emph{BSB
entity}:

\al\item all of the \emph{managers} of the \emph{partnership}, \emph{LLP} or
\emph{company} (as the case may be) must be \emph{BSB authorised
individuals} or \emph{authorised (non-BSB) individuals}; and

\item all of the owners (whether or not the ownership interest is material)
of the \emph{partnership}, \emph{LLP} or \emph{company} (as the case may
be) must be \emph{BSB authorised individuals} or \emph{authorised
(non-BSB) individuals};\la
\item to be licensed to \emph{practise} as a \emph{BSB licensed body}:
\al
\item the body must be a \emph{licensable body}, as defined by section 72
of the \emph{LSA} but must also meet the eligibility requirements set
out at Rule rS83; and

\item all of the non-authorised owners in the \emph{partnership},
\emph{LLP} or \emph{company} (as the case may be) must be approved by
the \emph{Bar Standards Board} as being able to hold such interest
taking into account the relevant \emph{suitability criteria}.\la
\ln
\rulesubsubsection{rS85}

In the event that you meet the eligibility criteria set out in Rule
rS83, you may submit an application in accordance with Section 3.E2 and
the \emph{Bar Standards Board} will review that application in
accordance with Section 3.E3 and 3.E4 to determine whether or not to
authorise you or to grant you a licence (as appropriate) to
\emph{practise} as a \emph{BSB entity}. In the event that the \emph{Bar
Standards Board} determines that you should be authorised or licensed
(as appropriate) to practise as a \emph{BSB entity} then it may either:

\rulesubsubsection{rS86}

Authorise you to \emph{practise} as a \emph{BSB entity} in the event
that you also meet the eligibility criteria set out in Rule rS84.1 and
you have applied to be authorised as such in your relevant application
form; or\nl\item license you to \emph{practise} as a \emph{BSB licensed body}, in the
event that you also meet the eligibility criteria set out in Rule rS84.2
and you have applied to be authorised as such in your relevant
application form.\ln

\rulesubsubsection{rS87}

Such authorisation or licence (as appropriate) will entitle you to:\nl\item to exercise a \emph{right of audience} before every \emph{court} in
relation to all proceedings;
\item to carry on:
\al
\item \emph{reserved instrument activities};

\item \emph{probate activities};

\item the \emph{administration of oaths};\la
\item to do \emph{immigration work}; and
\item if you have been granted a \emph{litigation extension}, to
\emph{conduct litigation}.\ln

\guidancesection{Guidance
to Rules S82-S85}


\guidancesubsubsection{gS13}

Single person entities are permitted under these arrangements.
Therefore, a \emph{BSB entity} may (subject to any structural
requirements imposed by general law for the particular type of entity)
comprise just one barrister who is both the owner and manager of that
entity.

\guidancesubsubsection{gS14}

These are mandatory eligibility requirements. The \emph{Bar Standards
Board} has a discretion to take other factors into account in deciding
whether an \emph{applicant body} is one which it would be appropriate
for it to regulate (see Section 3.E3 and 3.E4 below).

\rulesection{Part 3
- E2. Applications for authorisation (Rules S88-S89)}


\subsection{Application to be authorised or licensed as a BSB entity}

\rulesubsubsection{rS88}

To apply for authorisation to \emph{practise} as a \emph{BSB entity} you
must:\nl\item complete the application form supplied by the \emph{Bar Standards
Board} and submit it to the \emph{Bar Standards Board}; and
\item submit such other information, documents and references in support of
the application as may be required by the application form or by the
\emph{Bar Standards Board} from time to time; and
\item pay the \emph{application fee} in the amount determined in accordance
with Rule rS94 and the \emph{authorisation or licence fee} for the first
year.
\ln
\subsection{Application for a litigation extension}

\rulesubsubsection{rS89}

To apply for a \emph{litigation extension} you \textcolor{red}{\textbf{must}}:\nl\item make this clear on your application form submitted in accordance with
rS88 (where appropriate) or otherwise submit the relevant application
form made available by the \emph{Bar Standards Board} on its website for
this purpose; and
\item pay (or undertake to pay in a manner prescribed by the Bar Standards
Board) the \emph{application fee} (if any) and the relevant
\emph{litigation extension fee} (if any) in the amount determined in
accordance with Rule rS94; and
\item provide such other information to the \emph{Bar Standards Board} as
it may require in order to satisfy itself that:
\al
\item you have the relevant administrative systems in place to be able to
provide \emph{legal services} direct to \emph{clients} and to administer
the \emph{conduct of litigation}; and

\item you have a sufficient number of \emph{persons} who are authorised to
\emph{conduct litigation} and to provide guidance to any \emph{managers}
or employees that may be involved in assisting in the \emph{conduct of
litigation} who are not themselves authorised and that you have an
adequate number of qualified \emph{persons} to provide guidance to any
persons authorised to \emph{conduct litigation} who are of less than
three years' standing.
\la 
\ln
\guidancesection{Guidance
to Rules S88-S89}


\guidancesubsubsection{gS15}

In the event that your application is rejected, the \emph{authorisation
fee} and/or \emph{litigation fee} (as appropriate) will be reimbursed to
you but the \emph{application fee(s)} shall be retained by the Bar
Standards Board.

\guidancesubsubsection{gS16}

A qualified \emph{person} referred to in Rule rS89.3 shall be defined in
accordance with Rule S22.3.

\rulesection{Part 3
- E2. Applications for authorisation (Rules S90-S94)}


\subsection{Approval applications for any new HOLPs, HOFAs, owners and/or
managers}

\rulesubsubsection{rS90}

If, following authorisation or the grant of a licence (as appropriate),
a \emph{BSB entity} wishes to appoint a new \emph{HOLP, HOFA, owner} or
\emph{manager}, the \emph{BSB entity} must:\nl\item notify the \emph{Bar Standards Board} of such a proposed appointment
before it is made; and
\item make an application to the \emph{Bar Standards Board} for approval of
the new \emph{HOLP, HOFA, owner} or \emph{manager} (as appropriate); and
\item ensure that the new \emph{HOLP, HOFA, owner} or \emph{manager} (as
appropriate) has expressly consented to be bound by the \emph{Bar
Standards Board's} regulatory arrangements (including disciplinary
arrangements); and
\item pay any fees set by the \emph{Bar Standards Board} in respect of such
approval applications.
\ln
\subsection{Application Process}

\rulesubsubsection{rS91}

An application for authorisation and/or a \emph{litigation extension} is
only made once the \emph{Bar Standards Board} has received the
application form in full, together with the appropriate fees, all the
information required in support of the application and confirmation from
you in the form of a declaration that the information contained in, or
submitted in support of, the application is full and accurate.

\rulesubsubsection{rS92}

On receipt of the application, the \emph{Bar Standards Board} may
require, from you or from a third party, such additional information,
documents or references as it considers appropriate to the consideration
of your application.

\rulesubsubsection{rS93}

You are responsible for the contents of your application and any
information submitted to the \emph{Bar Standards Board} by you, or on
your behalf, and you \textcolor{myred}{\textbf{must not }}submit (or cause or permit to be submitted
on your behalf) information to the \emph{Bar Standards Board} which you
do not believe is full and accurate.

\rulesubsubsection{rS94}

The \emph{application fee} and the \emph{litigation extension fee} shall
be the amount or amounts prescribed by the \emph{Bar Standards Board}
from time to time. The \emph{authorisation fee} and \emph{litigation
fee} shall also be payable and shall be the amount or amounts prescribed
by the \emph{Bar Standards Board} from time to time.


\guidancesection{Guidance
to Rules S91-S93}


\guidancesubsubsection{gS17}

Application forms and guidance notes for completion can be found on the
\emph{Bar Standard Board's} website.

\guidancesubsubsection{gS18}

Once you have submitted an application, if you fail to disclose to the
\emph{Bar Standards Board} any information of which you later become
aware and which you would have been required to supply if it had been
known by you at the time of the original application the Bar Standards
Board may refuse your application in accordance with rS101.5.

\guidancesubsubsection{gS19}

Details of the relevant \emph{application fee}, \emph{litigation
extension fee}, \emph{authorisation fee}, \emph{licence fee} and
\emph{litigation fee} can be found on the \emph{Bar Standards Board's}
website.

\rulesection{Part 3
- E3. Decision process}


\rulesubsubsection{rS95}

Subject to Rules rS96 and rS97, the \emph{Bar Standards Board} must make
a decision in respect of each valid and complete application within the
\emph{decision period}.

\rulesubsubsection{rS96}

In the event that the \emph{Bar Standards Board} is not able to reach a
decision within the \emph{decision period}, it must notify you and must
confirm to you the latest date by which you will have received a
response to your application from the \emph{Bar Standards Board}.

\rulesubsubsection{rS97}

The \emph{Bar Standards Board} may issue more than one notice to extend
the \emph{decision period} except that:\nl\item any notice to extend must always be issued before the decision period
expires on the first occasion, and before any such extended
\emph{decision period} expires on any second and subsequent occasions;
and
\item no notice to extend can result in the total \emph{decision period}
exceeding more than 9 months.
\ln
\rulesubsubsection{rS98}

During its consideration of your application form, the \emph{Bar
Standards Board} may identify further information or documentation which
it needs in order to be able to reach its decision. If this is the case,
you \textcolor{myred}{\textbf{must }}provide such additional information or documentation as soon as
possible after you receive the relevant request from the \emph{Bar
Standards Board}. Any delay in providing this information shall further
entitle the \emph{Bar Standards Board} to issue an extension notice in
accordance with Rule rS96 and rS97 (as the case may be) or to treat the
application as having been withdrawn.

\rulesection{Part 3
- E4. Issues to be considered by the Bar Standards Board (Rules
S99-S100)}


\subsection{Applications for authorisation or the grant of a licence}

\rulesubsubsection{rS99}

In circumstances where the mandatory conditions in Rules rS83 and rS84
have been met, the \emph{Bar Standards Board} must then consider whether
to exercise its discretion to grant the authorisation or licence (as
appropriate). In exercising this discretion, the \emph{Bar Standards
Board} will consider whether the entity is one which it would be
appropriate for the \emph{Bar Standards Board} to regulate, taking into
account its analysis of the risks posed by you, the \emph{regulatory
objectives} of the \emph{LSA} and the Entity Regulation Policy Statement
of the \emph{Bar Standards Board} as published from time to time.

\rulesubsubsection{rS100}

In circumstances where the mandatory conditions set out at Rules S83 and
S84 have not been met, the \emph{Bar Standards Board} must refuse to
grant the authorisation or licence (as appropriate).


\guidancesection{Guidance
to Rules S99-S100}


\guidancesubsubsection{gS20}

In exercising its discretion whether to grant the authorisation or
licence the \emph{Bar Standards Board} will have regard to its current
Entity Regulation Policy Statement.

\rulesection{Part 3
- E4. Issues to be considered by the Bar Standards Board (Rule S101)
{Rules}}


\rulesubsubsection{rS101}

Where the \emph{Bar Standards Board} concludes that you are an entity
which it is appropriate for it to regulate the \emph{Bar Standards
Board} may nonetheless in its discretion refuse your application for
authorisation if:\nl\item it is not satisfied that your \emph{managers} and \emph{owners} are
suitable as a group to operate or control a \emph{practice} providing
services regulated by the \emph{Bar Standards Board};
\item if it is not satisfied that your proposed \emph{HOLP} and \emph{HOFA}
meet the relevant \emph{suitability criteria};
\item it is not satisfied that your management or governance arrangements
are adequate to safeguard the \emph{regulatory objectives} of the
\emph{LSA} or the policy objectives of the \emph{Bar Standards Board} as
set out in the Entity Regulation Policy Statement;
\item it is not satisfied that, if the authorisation is granted, you will
comply with the \emph{Bar Standards Board's} regulatory arrangements
including this \emph{Handbook} and any conditions imposed on the
authorisation;
\item you have provided inaccurate or misleading information in your
application or in response to any requests by the \emph{Bar Standards
Board} for information;
\item you have failed to notify the \emph{Bar Standards Board} of any
changes in the information provided in the application;
\item removed;
\item for any other reason, the \emph{Bar Standards Board} considers that
it would be inappropriate for the \emph{Bar Standards Board} to grant
authorisation to you, having regard to its analysis of the risk posed by
you, the regulatory objectives of the \emph{LSA} or the Entity
Regulation Policy Statement of the Bar Standards Board.
\ln

\guidancesection{Guidance
to Rule S101}


\guidancesubsubsection{gS21}

In circumstances where the \emph{Bar Standards Board} rejects your
application on the basis of Rule rS101, you will have the opportunity to
make the necessary adjustments to your composition and to re-apply to
become a \emph{BSB entity}.

\rulesection{Part 3
- E4. Issues to be considered by the Bar Standards Board (Rules
S102-S103)}


\subsection{Applications for authorisation to conduct litigation}

\rulesubsubsection{rS102}

If the \emph{Bar Standards Board} is unable to satisfy itself that the
\emph{BSB entity} meets the requirements set out in Rule rS89, it can
refuse to grant the litigation extension.

\subsection{Approval applications for any new HOLPs, HOFAs, owners and/or
managers}

\rulesubsubsection{rS103}

The \emph{Bar Standards Board} must consider any approval applications
for any new \emph{HOLPs, HOFAs, owners} and/or \emph{managers} made in
accordance with Rule rS90 and must determine any application by deciding
whether the relevant individual meets the \emph{suitability criteria}
which apply relevant to such a proposed appointment.

\rulesection{Part 3
- E5. Suitability criteria in respect of HOLPs, HOFAs, owners and
managers}


\rulesubsubsection{rS104}

The \emph{Bar Standards Board} must conclude that an individual does not
meet the suitability criteria to undertake the role of a \emph{HOLP} if:\nl\item they are not an \emph{authorised individual};
\item they are disqualified from acting as a \emph{HOLP} by the \emph{Bar
Standards Board} or an \emph{Approved Regulator} or \emph{licensing
authority} pursuant to section 99 of the \emph{LSA} or otherwise as a
result of its regulatory arrangements; or
\item It determines that the individual is not able effectively to carry
out the duties imposed on a HOLP by section 91 of the LSA.
\ln
\rulesubsubsection{rS105}

The \emph{Bar Standards Board} may conclude that an individual does not
meet the suitability criteria to undertake the role of a \emph{HOLP} if
any of the circumstances listed in Rule rS110 apply to the individual
designated as the \emph{HOLP}.

\rulesubsubsection{rS106}

The \emph{Bar Standards Board} must conclude that an individual does not
meet the suitability criteria for acting as a \emph{HOFA} if:\nl\item they are disqualified from acting as a \emph{HOFA} by the \emph{Bar
Standards Board} or by an \emph{Approved Regulator} or \emph{licensing
authority} pursuant to section 99 of the \emph{LSA} or otherwise as a
result of its regulatory arrangements; or
\item the \emph{Bar Standards Board} determines that they are not able
effectively to carry out the duties imposed on a \emph{HOFA} by section
92 of the \emph{LSA}.
\ln
\rulesubsubsection{rS107}

The \emph{Bar Standards Board} may conclude that an individual does not
meet the suitability criteria for acting as a \emph{HOFA} if any of the
circumstances listed in Rule rS110 apply to them.

\rulesubsubsection{rS108}

If an \emph{owner} is also a \emph{non-authorised individual}, the
\emph{Bar Standards Board} must approve them as an \emph{owner}. The
\emph{Bar Standards Board} shall approve a \emph{non-authorised
individual} to be an \emph{owner} of a \emph{BSB licensed body} if:\nl\item their holding of an ownership interest does not compromise the
\emph{regulatory objectives}; and
\item their holding of an ownership interest does not compromise compliance
with the duties imposed pursuant to section 176 of the \emph{LSA} by the
\emph{licensed body} or by any authorised individuals who are to be
employees or \emph{managers} of that \emph{licensed body}; and
\item they otherwise meet the \emph{suitability criteria} to hold that
ownership interest taking into account:
\al
\item their probity and financial position;

\item  whether they are disqualified pursuant to section 100(1) of
\emph{LSA} or included in the list maintained by the \emph{Legal
Services Board} pursuant to paragraph 51 of Schedule 13 of the
\emph{LSA}; and

\item their \emph{associates}; and

\item the \emph{suitability criteria} in Rule rS110 which apply to
\emph{managers} and employees.\la
\ln
\rulesubsubsection{rS109}

If a \emph{manager} is a \emph{non-authorised individual}, the \emph{Bar
Standards Board} must approve them as a \emph{manager}. The \emph{Bar
Standards Board} must approve a \emph{non-authorised individual} to be a
\emph{manager} of a \emph{BSB licensed body} if they meet the
\emph{suitability criteria} to hold that interest taking into account:\nl\item their probity;
\item whether they are disqualified pursuant to section 100(1) of the
\emph{LSA} or included in the list maintained by the \emph{Legal
Services Board} pursuant to paragraph 51 of Schedule 13 of the
\emph{LSA}; and
\item the \emph{suitability criteria} in Rule rS110 which apply to
\emph{managers} and employees.
\ln
\rulesubsubsection{rS110}

The \emph{Bar Standards Board} may reject an application if it is not
satisfied that:\nl\item an individual identified in an application for authorisation or the
grant of a licence as a proposed \emph{owner, manager, HOLP or HOFA} of
the relevant \emph{applicant body}; or
\item any individual identified as a replacement owner, manager,
\emph{HOLP} or \emph{HOFA},

meets the \emph{suitability criteria} to act as an \emph{owner, manager,
HOLP} or \emph{HOFA} of a \emph{BSB entity}. Reasons why the \emph{Bar
Standards Board} may conclude that an individual does not meet the
\emph{suitability criteria} include where an individual:
\item has been committed to prison in civil or criminal proceedings (unless
the Rehabilitation of Offenders Act 1974 (Exceptions) Order 1975 (SI
1975/1023) applies, this is subject to any conviction being unspent
under the Rehabilitation of Offenders Act 1974 (as amended));
\item has been disqualified from being a \emph{director};
\item has been removed from the office of charity trustee or trustee for a
charity by an order under section 72(1)(d) of the Charities Act 1993;
\item is an undischarged bankrupt;
\item has been adjudged bankrupt and discharged;
\item has entered into an individual voluntary arrangement or a
\emph{partnership} voluntary arrangement under the Insolvency Act 1986;
\item has been a \emph{manager} of a \emph{regulated entity} or a \emph{BSB
entity} which has entered into a voluntary arrangement under the
Insolvency Act 1986;
\item has been a \emph{director} of a \emph{company} or a \emph{member} of
an \emph{LLP} (as defined by section 4 of the Limited Liability
Partnerships Act 2000) which has been the subject of a winding up order,
an administration order or administrative receivership; or has entered
into a voluntary arrangement under the Insolvency Act 1986; or has been
otherwise wound up or put into administration in circumstances of
insolvency;
\item lacks capacity (within the meaning of the Mental Capacity Act 2005)
and powers under sections 15 to 20 or section 48 of that Act are
exercisable in relation to that individual;\item is the subject of an outstanding judgment or judgments involving the
payment of money;

\item is currently charged with an \emph{indictable offence}, or has been
convicted of an \emph{indictable offence}, any offence of dishonesty, or
any offence under the Financial Services and Markets Act 2000, the
Immigration and Asylum Act 1999 or the Compensation Act 2006 (unless the
Rehabilitation of Offenders Act 1974 (Exceptions) Order 1975 (SI
1975/1023) applies, this is subject to the Rehabilitation of Offenders
Act 1974 (as amended));

\item has been disqualified from being appointed to act as a \emph{HOLP}
or a \emph{HOFA} or from being a \emph{manager} or employed by an
\emph{authorised or licensed body} (as appropriate) by the \emph{Bar
Standards Board} or another \emph{Approved Regulator} or \emph{licensing
authority} pursuant to its or their powers under section 99 of the
\emph{LSA} or otherwise as a result of its regulatory arrangements;\item has been the subject in another jurisdiction of circumstances
equivalent to those listed in Rules rS110.1 to rS110.14;\item has an investigation or disciplinary proceedings pending against
them and/or has professional conduct findings against them either under
the disciplinary scheme for \emph{barristers} or otherwise; or\item has been involved in other conduct which calls into question their
honesty, integrity, or respect for the law;
\item has not consented to be bound by the regulatory arrangements
(including disciplinary arrangements) of the \emph{Bar Standards Board}.
\ln

\guidancesection{Guidance
to Rule S110}


\guidancesubsubsection{gS21.1}

For the avoidance of doubt rS110 does not oblige you to disclose
cautions or criminal convictions that are ``spent'' under the
Rehabilitation of Offenders Act 1974 unless the Rehabilitation of
Offenders Act 1974 (Exceptions) Order 1975 (SI 1975/1023) applies. The
latter entitles the BSB to ask for disclosure of unprotected cautions or
criminal convictions that are ``spent'' in relation to \emph{HOLPs} and
\emph{HOFAs} of \emph{licensed bodies} when seeking authorisation and
owners who require approval under Schedule 13 to the LSA.

\rulesection{Part 3
- E6. Notification of the authorisation decision}


\rulesubsubsection{rS111}

The \emph{Bar Standards Board} will notify you of its decision in
writing within the \emph{decision period} or by such later date as may
have been notified to the \emph{applicant body} in accordance with Rules
rS96 or rS97. In the event that the \emph{Bar Standards Board} decides
to refuse to grant the application, it must give the reasons for such
refusal.

\rulesection{Part 3
- E7. Terms of authorisation}


\rulesubsubsection{rS112}

Any authorisation given by the \emph{Bar Standards Board} to a \emph{BSB
entity}, and the terms of any licence granted by the \emph{Bar Standards
Board} to a \emph{BSB licensed body} in accordance with this Section 3.E
must specify:\nl\item the activities which are \emph{reserved legal activities} and which
the \emph{BSB entity} is authorised to carry on by virtue of the
authorisation or the licence (as the case may be); and
\item any conditions subject to which the authorisation or the licence (as
the case may be) is given (which may include those in Rule rS114).
\ln
\rulesubsubsection{rS113}

Authorisations and licences must, in all cases, be given on the
conditions that:\nl\item any obligation which may from time to time be imposed on you (or your
\emph{managers}, employees, or \emph{owners}) by the \emph{Bar Standards
Board} is complied with; and
\item any other obligation imposed on you (or your \emph{managers},
employees or \emph{owners}) by or under the \emph{LSA} or any other
enactment is complied with.
\item you (and your \emph{managers}, employees, and \emph{owners}) consent
to be bound by the regulatory arrangements (including the disciplinary
arrangements) of the \emph{Bar Standards Board}; and
\item if the conditions outlined at rS113.5 apply, the \emph{Bar Standards
Board} may without notice:
\al
\item modify an authorisation granted under rS116;

\item revoke an authorisation under rS117;

\item require specific co-operation with the \emph{Bar Standards Board} as
provided for in rC64 and rC70;

\item take such action as may be necessary in the public or \emph{clients'}
interests and in the interests of the regulatory objectives; and

\item recover from the \emph{BSB entity} any reasonable costs that were
necessarily incurred in the exercise of its regulatory functions.\la
\item The conditions referred to in rS113.4 are that:
\al
\item one or more of the terms of the \emph{BSB entity's} authorisation
have not been complied with;

\item a person has been appointed receiver or manager of the property of
the \emph{BSB entity};

\item a relevant insolvency event has occurred in relation to the \emph{BSB
entity};

\item the \emph{Bar Standards Board} has reason to suspect dishonesty on
the part of any \emph{manager} or employee of the \emph{BSB entity} in
connection with either that \emph{BSB entity's} business or the business
of another body of which the person was a manager or employee, or the
\emph{practice} or former \emph{practice} of the \emph{manager} or
employee;

\item the \emph{Bar Standards Board} is satisfied that it is necessary to
exercise any of the powers listed in rS113.4 in relation to the
\emph{BSB entity} to protect the interests of \emph{clients} (or former
or potential \emph{clients}) of the \emph{BSB entity}.\la
\ln
\rulesubsubsection{rS114}

In addition to the provisions in Rule rS113, an authorisation or a
licence may be given subject to such other terms as the \emph{Bar
Standards Board} considers appropriate including terms as to:\nl\item the \emph{non-reserved activities} which you may or may not carry on;
and/or
\item in the case of \emph{licensed bodies}:
\al
\item the nature of any interest held by a non-authorised \emph{owner}
provided always that the \emph{Bar Standards Board} complies with its
obligations under paragraph 17 of Schedule 13 to the \emph{LSA}; and/or

\item any limitations on the shareholdings or voting controls which may be
held by non-authorised \emph{owners} in accordance with paragraph 33 of
Schedule 13 to the \emph{LSA}.\la
\ln
\rulesection{Part 3
- E8. Duration of the authorisation/licence granted}


\rulesubsubsection{rS115}

Except where indicated otherwise in the authorisation or licence, any
authorisation or licence granted in accordance with this Section 3.E
will be of unlimited duration except that the authorisation or licence:\nl\item the authorisation or licence shall cease to have effect on the
occurrence of any of the following:
\al
\item if you have your authorisation/licence withdrawn in accordance with
Rule rS117; or

\item if you obtain authorisation/licence from an \emph{Approved
Regulator} or \emph{licensing authority};\la
\item the authorisation or licence may cease to have effect on the
occurrence of any of the following:
\al
\item if you fail to provide the relevant monitoring information or fail
to pay any relevant fees in circumstances where the \emph{Bar Standards
Board} has notified you \rl
\item  that such information or payment is required
within a particular time; and \item that failure to provide such
information or payment within that time may result in the withdrawal of
your authorisation or licence in accordance with this Rule rS115; or\lr

\item if you fail to replace your \emph{HOLP/HOFA} in accordance with the
requirements of this \emph{Handbook}.\la
\item The licence of a partnership or other unincorporated body (``the
existing body'') may continue where the existing body ceases to exist
and another body succeeds to the whole or substantially the whole of its
business subject to the following in rS115.3(a)-(b):
\al
\item you have notified the \emph{Bar Standards Board} of such a change
within 28 days;

\item if there is no remaining \emph{partner} who was a \emph{partner}
before the existing body ceased to exist the licence shall cease to have
effect from the date the existing body ceased to exist.\la
\ln
\rulesection{Part 3
- E9. Modification of an authorisation/licence}


\rulesubsubsection{rS116}

In addition to any powers which the \emph{Bar Standards Board} may have
in accordance with Part 5, the \emph{Bar Standards Board} may modify the
terms of an authorisation or licence granted by it:\nl\item if you apply to the \emph{Bar Standards Board} for the terms of such
authorisation or licence (as the case may be) to be modified; or
\item if it is satisfied that any of the information contained in the
relevant application form was inaccurate or incomplete or has changed;
or
\item if such modification is required in accordance with the provisions of
this \emph{Handbook}; or
\item where the \emph{Bar Standards Board} reasonably considers that such
modification is appropriate and in accordance with the \emph{regulatory
objectives} under the \emph{LSA} or the policy objectives of the
\emph{Bar Standards Board}; or
\item where the conditions in rS113.5 are met, but, in the circumstances set out in Rules rS116.2 to rS116.4 above,
shall only be entitled to do so after:
\al
\item giving notice to you in writing of the modifications which the
\emph{Bar Standards Board} is intending to make to your authorisation or
licence (as the case may be); and

\item giving you a reasonable opportunity to make representations about
such proposed modifications.\la
\ln
\rulesection{Part 3
- E10. Revocation or suspension of an authorisation/licence}


\rulesubsubsection{rS117}

In addition to any powers which the Bar Standards Board may have in
accordance with Part 5, the \emph{Bar Standards Board} may:\nl\item revoke an authorisation or licence granted by it:
\al
\item subject to Section 3.F, in the event that you no longer comply with
the mandatory requirements set out in Rules rS83 and rS84; or

\item if your circumstances have changed in relation to the issues
considered by the \emph{Bar Standards Board} in Section 3.E4; or

\item if revocation otherwise appears appropriate taking into account the
\emph{regulatory objectives} of the \emph{Bar Standards Board}; or

\item where the conditions in rS113.5 are met.\la
\item suspend an authorisation or licence granted by it to give it an
opportunity to investigate whether or not your authorisation or licence
should be revoked in accordance with Rule rS117 (for the avoidance of
doubt a \emph{BSB entity} whose authorisation has been suspended remains
a \emph{BSB regulated person}),

but (except for when the conditions in rS113.5 are met) in either case
only after:

\rl \item  giving written notice to the relevant \emph{BSB entity} of the
grounds on which the authorisation or licence may be revoked; and

\item giving the relevant \emph{BSB entity} a reasonable opportunity to
make representations.\lr
\ln
\rulesection{Part 3
- E11. Applications for review}


\rulesubsubsection{rS118}

If you consider that the \emph{Bar Standards Board} has (other than
pursuant to {[}Section 5{]}):\nl\item wrongly refused an application for authorisation or licence; or
\item wrongly imposed a term or condition on an authorisation or licence;
or
\item wrongly modified the terms of your authorisation or licence; or
\item wrongly refused to modify the terms of your authorisation or licence;
or
\item wrongly revoked or \emph{suspended} your authorisation or licence; or
\item wrongly done any of these things in relation to a litigation
extension to your authorisation or licence; or
\item failed to provide to you notice of a decision in accordance with this
Section 3.E, then you may lodge an application for review of that
decision using the form supplied for that purpose by the \emph{Bar
Standards Board}. Such application for review will only have been made
once the \emph{Bar Standards Board} has received the relevant fee in
respect of such application for review.
\ln
\rulesubsubsection{rS119}

Any individual:\nl\item designated to act as a \emph{HOLP} or a \emph{HOFA}; or
\item identified as a non-authorised \emph{owner} or \emph{manager} of the
\emph{applicant body},
\ln
who considers that the \emph{Bar Standards Board} has wrongly concluded
that they do not meet the \emph{suitability criteria} which apply to
their proposed position in the entity, may lodge an application for a
review of that decision using the form supplied for that purpose by the
\emph{Bar Standards Board}. Alternatively, you may lodge an application
for review on their behalf whether or not they have asked you to. In
either case, such an application for a review will only have been made
once the \emph{Bar Standards Board} has received the relevant fee for
it.

\rulesubsubsection{rS120}

Any application for a review of the decision must be made within 28 days
from the date when the decision is notified to you.

\rulesubsubsection{rS121}

The decision of the \emph{Bar Standards Board} will take effect
notwithstanding the making of any application for a review in accordance
with Rule rS118 or rS119. However, the \emph{Bar Standards Board} may,
in its absolute discretion, issue a temporary authorisation, licence or
litigation extension to a \emph{BSB entity} which has lodged an
application for a review in accordance with this Section 3.E11.

\rulesubsubsection{rS122}

If the review finds that the \emph{Bar Standards Board}:\nl\item has wrongly failed or refused to grant an authorisation or licence;
or
\item has wrongly imposed a term or condition on an authorisation or
licence;

then in each case the \emph{Bar Standards Board} must issue such
authorisation or licence as ought to have been issued.
\ln
\rulesubsubsection{rS123}

If the review finds that the \emph{Bar Standards Board}:\nl\item finds that the \emph{Bar Standards Board} has wrongly modified an
authorisation or licence; or
\item finds that the \emph{Bar Standards Board} has wrongly refused to
modify an authorisation or licence,

then in each case the Bar Standards Board shall make such modification
to the authorisation or licence as ought to have been made.
\ln
\rulesubsubsection{rS124}

If the review finds that the \emph{Bar Standards Board} has wrongly
revoked or \emph{suspended} an authorisation or licence, then the
\emph{Bar Standards Board} shall re-issue such authorisation or licence.\nl\item If the review finds that the \emph{Bar Standards Board} has wrongly
done any of the things described in rS122 or--rS123 in relation to your
\emph{litigation extension}, then the \emph{Bar Standards Board~}shall
grant such \emph{litigation extension} as ought to have been granted.
\ln
\rulesubsubsection{rS125}

If the review finds that the \emph{Bar Standards Board} has wrongly
concluded that an individual does not meet the \emph{suitability
criteria} relevant to their proposed position, the \emph{Bar Standards
Board} shall amend its decision and confirm that they do meet the
suitability criteria which apply to their proposed position.

\rulesubsubsection{rS126}

If, after such a review, you or the relevant individual(s) (as the case
may be) do not agree with the decision you or the relevant individual(s)
may appeal to the \emph{First Tier Tribunal} against the decision.

\rulesubsubsection{rS127}

Any appeal to the \emph{First Tier Tribunal} against a decision of the
BSB must be lodged within 28 days from the date that the decision is
notified to you.

\rulesubsubsection{rS127A}

Where a BSB decision is appealed to the \emph{First Tier Tribunal}, the
\emph{First Tier Tribunal} may suspend the effect of that decision until
the conclusion of the appeal.

\rulesection{Part 3
- E12. Register}


\rulesubsubsection{rS128}

The \emph{Bar Standards Board} must keep a public register containing
the names and places of practice of all \emph{BSB entities} (together
with details of the \emph{reserved legal activities} which such
\emph{BSB entities} are able to undertake) as well as details of any
bodies which have in the past been granted authorisation or obtained a
licence from the \emph{Bar Standards Board} but where such licence
and/or authorisation is no longer current.

\rulesubsubsection{rS129}

If an authorisation or licence is, at any time, suspended or made
subject to conditions, this must be noted on the register of \emph{BSB
entities} by the Bar Standards Board.

\chap{Part 3
- F. Continuing Compliance with the Authorisation and Licensing
Requirements}



\rulesection{Part 3
- F1. Non-compliance with the mandatory conditions}


\rulesubsubsection{rS130}

If, at any time, and for whatever reason, you fail to meet the mandatory
conditions in Rules rS83 and rS84 which apply to the type of \emph{BSB
entity} which you are, then you \textcolor{myred}{\textbf{must }}notify the \emph{Bar Standards
Board} of your failure to comply with the mandatory conditions within
seven days of your failure to comply and, at the same time, you \textcolor{red}{\textbf{must}}
submit your proposals for rectifying that non-compliance which, for the
avoidance of doubt, must include your proposed timetable for rectifying
them. If the \emph{Bar Standards Board} considers that your proposals
for rectifying them are not sufficient, the \emph{Bar Standards Board}
may issue a notice suspending or revoking your authorisation or licence
(as appropriate) in accordance with Section 3.E10.


\guidancesection{Guidance
to Rule S130}


\guidancesubsubsection{gS22}

Examples of non-compliance include:\nl\item where your last remaining \emph{authorised person}:

\al\item dies; or

\item abandons, retires or resigns from the \emph{practice}; or\la
\item where you are a \emph{BSB entity} (other than a BSB licensed body) a
\emph{non-authorised individual} is appointed as a \emph{manager} of or
otherwise acquires an ownership interest in such a practice;
\item where you cease to have available at least one employee who is
authorised to carry on a particular reserved activity which you are
authorised to provide. Examples of situations where an individual should
be considered to be unavailable to a \emph{BSB entity} include where:
\al
\item they are committed to prison;

\item they are unable to attend to the \emph{practice} because of
incapacity caused by illness, accident or age;

\item they become and continue to lack capacity under Part 1 of the Mental
Capacity Act 2005;

\item they are made subject to a condition on their \emph{practising
certificate} or registration which would be breached if they continue to
be an \emph{owner} and/or \emph{manager} of the body; or

\item they are no longer authorised to perform the particular
\emph{reserved legal activity}.\la
\item you cease to have a \emph{HOLP} or a \emph{HOFA} appointed;
\item your \emph{HOLP, HOFA}, any \emph{manager} or \emph{owner} ceases to
meet the relevant \emph{suitability criteria}; or
\item where you are a \emph{licensed body}, your last remaining
\emph{owner} and/or \emph{manager} who is a \emph{non-authorised
individual} dies or otherwise leaves the \emph{practice}.
\ln
\guidancesubsubsection{gS23}

Examples of proposals that you may submit in order to rectify such
non-compliance include:\nl\item In the case of Guidance gS22.1, that you are seeking to appoint a
different \emph{authorised person} to be an \emph{owner} and/or a
\emph{manager} of a \emph{BSB entity};
\item In the case of Guidance gS22.2, confirmation that you will take the
necessary steps to rectify your status, whether by submitting an
application to the \emph{Bar Standards Board} for authorisation to
\emph{practise} as a \emph{licensed body} and/or for approval of the
\emph{non-authorised individual} as a \emph{manager} or by ensuring that
the \emph{non-authorised person} divest themselves of their interest as
soon as is reasonably practicable, or by seeking a licence from another
\emph{licensing authority}, as the case may be {[}but note Guidance
gS24{]};
\item in the case of Guidance gS22.4, that you are seeking to appoint a
replacement \emph{HOLP} or \emph{HOFA} (as appropriate) in accordance
with the relevant procedure in Rule sS90;
\item in the case of Guidance gS22.5, that you are taking the necessary
steps to exclude the relevant individual from the \emph{practice} and,
where necessary, you are taking steps to replace them; and
\item in the case of Guidance gS22.6, you confirm whether or not you are
likely to appoint a replacement \emph{non-authorised individual} or, if
not, whether you will be seeking authorisation from the \emph{Bar
Standards Board} to practise as a \emph{BSB authorised body}.
\ln
\guidancesubsubsection{gS24}

In respect of Guidance gS23.2, it may be the case that a
\emph{non-authorised individual} obtains an ownership interest in a
\emph{BSB entity} following the death of a \emph{barrister} or a
\emph{non-authorised person}. Similarly, a \emph{non-authorised person}
who has not been approved pursuant to the \emph{suitability criteria}
may acquire an ownership interest in a \emph{licensed body}. In these
cases, it may be that the \emph{BSB entity} will not need to apply for
authorisation to \emph{practise} as a \emph{licensed body} or for
approval of such \emph{non-authorised individual} (as appropriate) if
the \emph{BSB entity} instead satisfies the \emph{Bar Standards Board}
that it is taking steps to ensure that such \emph{non-authorised
individual} divest themselves of their interest as soon as is reasonably
practicable (for example, on completion of the relevant probate).

\rulesection{Part 3
- F2. Temporary emergency approvals for HOLPs and HOFAs}


\rulesubsubsection{rS131}

If a \emph{BSB entity} ceases to have a \emph{HOLP} or \emph{HOFA} whose
designation has been approved by the \emph{Bar Standards Board}, the
\emph{BSB entity} must immediately and in any event within seven days:\nl\item notify the \emph{Bar Standards Board};
\item designate another \emph{manager} or employee to replace its previous
\emph{HOLP} or \emph{HOFA}, as appropriate; and
\item make an application to the \emph{Bar Standards Board} for temporary
approval of the new \emph{HOLP} or \emph{HOFA}, as appropriate.
\ln
\rulesubsubsection{rS132}

The \emph{Bar Standards Board} may grant a temporary approval under this
Section 3.F2 if on the face of the application and any other information
immediately before the \emph{Bar Standards Board}, there is no evidence
suggesting that the new \emph{HOLP} or \emph{HOFA} is not suitable to
carry out the duties imposed on them under this \emph{Handbook}.

\rulesubsubsection{rS133}

If granted temporary approval under Rule rS132 for its designation of a
new \emph{HOLP} or \emph{HOFA}, the \emph{BSB entity} must:
\begin{numlist}\item designate a permanent \emph{HOLP} or \emph{HOFA}, as appropriate; and
\item submit a substantive application for approval of that designation in
accordance with Rule rS90,
\end{numlist}
before the expiry of the temporary approval or any extension of that
approval by the \emph{Bar Standards Board}, otherwise the \emph{Bar
Standards Board} may be entitled to suspend or revoke the authorisation
or licence in accordance with Section 3.E10.


