\section*{Version 4.6 (31 December 2020) - changes}
Version 4.6 of the BSB Handbook came into force at the end of the transition period following the UK’s exit from the EU. The changes affect Registered European Lawyers (RELs) practising at the Bar of England and Wales, and European lawyers seeking to be admitted to the Bar. They also implement provisions relating to the Swiss Citizens’ Rights Agreement, which was agreed by the UK and Switzerland in 2018. You can read more about the BSB’s regulation and the UK’s exit from the EU on our website here.

\section*{Version 4.5 (September 2020) - changes}
The new Internal Governance Rules, set by the Legal Services Board (LSB), aim to further enhance the BSB’s regulatory independence and Version 4.5 of the Handbook therefore reflects the requirements of the new rules.

Version 4.5 also includes amendments to the Disciplinary Tribunal Rules which enable Disciplinary Tribunals to rely on wasted cost orders as proof of conduct occurring, and clarify that Directions Judges have the discretion to make cost orders.

In addition, version 4.5 includes updated guidance on Core Duty 9, making clear that barristers’ duty to co-operate with their regulators includes all relevant regulators and ombudsman schemes.

\section*{Version 4.4 (February 2020) - Changes}
A new rule (Rule C85A) has been introduced in Version 4.4 to prevent barristers from supervising immigration advisers who have been subject to serious sanctions by the Office of the Immigration Services Commissioner (OISC) or a legal regulator. Barristers must also have regard to the updated guidance on supervising immigration advisers.

\section*{Version 4.3 (October 2019) - Changes}
Version 4.3 of the Handbook includes the new Enforcement Decision Regulations, contained in Part 5:A, which replace the old Complaints Regulations. The new Regulations were designed to improve the way that the BSB assesses and handles reports about persons we regulate.

The Regulations remove the distinction between “complaints” and other types of information received, to allow for a more holistic approach to addressing concerns about those whom we regulate. Information that requires assessment is referred to as a “report”, regardless of its source. A report can be treated as an allegation and formally investigated if it reveals a potential breach of the BSB Handbook.

Two new roles have been created to take regulatory decisions – the Commissioner and the Independent Decision-Making Body – following the disestablishment of the Professional Conduct Committee. The powers and functions of each role are set out in the Regulations, including the decisions available to each role at the end of the investigation of an allegation. The Commissioner’s powers include the power to authorise other persons to exercise any powers, and these authorisations are recorded in the Scheme of Delegations, contained in the Bar Standards Board Governance Manual.

\section*{Version 4.2 (September 2019) - Changes}
\subsection*{Scope of Practice - Employed Barristers in Non-Authorised Bodies}

rS39 sets out who employed barristers in non-authorised bodies are permitted to supply legal services to. They are normally only permitted to supply services to their employer i.e. the body itself, and not clients of their employer.

New guidance (gS8A - gS8C) clarifies that the following arrangements are permissible, and barristers seeking to work in these ways do not need to obtain waivers from rS39:
\begin{dotlist}
\item Barristers providing services through a non-authorised body (e.g. an agency or corporate vehicle) whose purpose is to facilitate the supply of in-house legal services to another non-authorised body (e.g. a local authority or corporate body); and
\item Barristers providing services through a non-authorised body (e.g. an agency or corporate vehicle) whose purpose is to facilitate the supply of legal services to an authorised body, or clients of an authorised body. Where services are being provided to clients of an authorised body, those services must be provided by the authorised body and regulated by an approved regulator under the Legal Services Act 2007 (LSA) e.g. the Solicitors Regulation Authority.\end{dotlist}


Barristers working in these ways are reminded that reserved legal activities (e.g. exercising rights of audience and conducting litigation) may only be provided in a way that is permitted by s15 of the LSA. s15 details when an employer needs to be authorised to carry on reserved legal activities and prevents those activities from being provided to the public, or a section of the public, by a non-authorised body.

\subsection*{Publication of Sexual Orientation and Religion or Belief Data by Chambers and BSB Entities}

rC110.3.s.i has been removed. This removes the restriction on the reporting of diversity data relating to sexual orientation and religion or belief unless all members of the workforce provide consent.

\subsection*{Guidance Changes}
\begin{dotlist}
\item The separate 'Media Comment Guidance' has been removed. The substantive provisions are now at gC22;

\item The separate 'Guidance on Self-Employed Practice' has been removed and replaced by 'Guidance on Investigating and Collecting Evidence and Taking Witness Statements'. See Code Guidance. The substantive provisions on attendance at police stations are now at gC39;

\item The separate 'Cab Rank Rule Guidance' has been removed. The substantive provisions are now at gC91B; and

\item The separate 'Guidance on Insurance and Limitation of Liability' has been removed. The substantive provisions are now at gC114.\end{dotlist}
\section*{Version 4.1 – July 2019}
\subsection*{New Price, Service and Redress Transparency Rules}

Version 4.1 of the Handbook includes the new price, service and redress transparency rules for self-employed barristers, chambers, and BSB entities. The new rules are designed to improve the information available to the public before they engage the services of a barrister, and follow recommendations from the Competition and Markets Authority that legal regulators should introduce new requirements in this area. 

The rules require all self-employed barristers, chambers, and BSB-regulated entities to publish specified information about their services, including which types of legal service they provide, their most commonly used pricing models (such as fixed fee or hourly rate), and details of their clients' rights of redress. As set out in the BSB's price transparency policy statement, Public Access barristers providing certain types of services are also required to publish additional price and service information. The BSB has also published guidance to help the profession comply with the new rules.


\section*{Version 4.0 – April 2019}
\subsection*{New Bar Qualification Rules}

The fourth edition of the Handbook includes the new Bar Qualification Rules. Our recent reforms to education and training for the Bar aim to ensure that training to become a barrister is more accessible, affordable, and flexible, and that it maintains the high standards of entry expected at the Bar. At their core, the Bar Qualification Rules set out the minimum training requirements to qualify and practise as a barrister. The rules also bring into force a new system for Authorised Education and Training Organisations (AETOs) to deliver one or more of the three components of training (academic, vocational, and pupillage or work-based learning) in accordance with our Authorisation Framework. To coincide with the publication of the Bar Qualification Rules, the BSB has also published a new Bar Qualification Manual. This provides further guidance on the new rules for AETOs, students, pupils, and transferring lawyers.

\subsection*{Standard of Proof Change}

The fourth edition of the Handbook also includes the change to the standard of proof used when determining allegations of professional misconduct.  As of 1 April 2019, the standard of proof for such allegations has changed from the criminal standard to the civil standard.  Any conduct occurring on or after 1 April 2019 will be considered applying the civil standard of proof. However, the criminal standard of proof will continue to apply to any conduct that occurred before that date. This change has been made following a public consultation in 2017 and brings the Bar's disciplinary arrangements in line with those of other professional regulators.

