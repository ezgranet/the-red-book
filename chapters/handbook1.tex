\part{Introduction}
\mysec{A General}
\mysec{A1. The
Bar Standards Board}

\subsection{I1}

The \emph{Bar Standards Board} is a specialist regulator focussing
primarily on the regulation of advocacy, litigation and legal advisory
services. These legal services have a close relationship to access to
justice and the rule of law. Our society is based on a rule of law.
Everyone needs to be able to seek expert advice on their legal rights
and obligations and to have access to skilled representation in the
event of a dispute or litigation. Our system of justice depends on those
who provide such services acting fearlessly, independently and
competently, so as to further their clients' best interests, subject
always to their duty to the Court.

\subsection{I2}

The regulatory objectives of the \emph{Bar Standards Board} derive from
the Legal Services Act 2007 and can be summarised as follows:
\begin{numlist}\item protecting and promoting the public interest;
\item supporting the constitutional principles of the rule of law;
\item improving access to justice;
\item protecting and promoting the interests of consumers;
\item promoting competition in the provision of the services;
\item encouraging an independent, strong, diverse and effective legal
profession;
\item increasing public understanding of the citizen's legal rights and
duties; and
\item promoting and maintaining adherence to the following professional
principles:
\begin{alphlist}
\item that \emph{authorised persons} act with independence and integrity;

\item that \emph{authorised persons} maintain proper standards of work;

\item that \emph{authorised persons} act in the best interests of their
clients;

\item that \emph{authorised persons} comply with their duty to the court to
act with independence in the interests of justice; and

\item that the affairs of clients are kept confidential.
\end{alphlist}
\end{numlist}
\subsection{I3}

The BSB Handbook (``\emph{this Handbook}'' or ``\emph{the Handbook}'')
sets out the standards that the \emph{Bar \emph{Standards Board}}
requires the \emph{persons} it regulates to comply with in order for it
to be able to meet its \emph{regulatory objectives}.

\subsection{I4}

Although the \emph{Handbook} is drafted with specific reference to those
regulated by the BSB and for use by them, the \emph{Handbook} should
also act as a useful reference tool for all consumers of legal services
regulated by the \emph{Bar Standards Board}. In particular, the Core
Duties and the outcomes set out in Part 2 of this Handbook should give
consumers a useful indication of what they should expect from the
\emph{Bar \emph{Standard Board's}} regulatory framework and those
subject to it.

\mysec{A2.
Structure of the Handbook}


\subsection{I5}

The \emph{Handbook} consists of the following parts:
\begin{numlist}\item Part 1 \subsection{-- Introduction};
\item Part 2 \subsection{-- The Code of Conduct} -- this part includes the ten
Core Duties which underpin the \emph{Bar Standards Board's} entire
regulatory framework, as well as the rules which supplement those Core
Duties. Compliance with both the Core Duties and the rules is mandatory.
The Code of Conduct also contains details of the outcomes which
compliance with the Core Duties and the rules is designed to achieve.
The \emph{Bar Standards Board's} approach to regulation is risk-focused
and so these outcomes have been defined by considering the risks which
the profession needs to manage if the~\emph{regulatory objectives} are
to be achieved;
\item Part 3 \subsection{-- Scope of Practice and Authorisation and Licensing
Rules} -- this part includes the requirements that must be met to become
entitled to practise as a \emph{barrister~}or a \emph{registered
\emph{European lawyer}} and the process that must be followed in order
to obtain~authorisation to practise as a \emph{BSB entity}. It also
provides a summary of the scope of activities that each type of
\emph{BSB authorised person} is permitted to undertake;
\item Part 4 \subsection{-- Bar Qualification Rules} -- this part sets out the
training which a person must complete, and other requirements which a
person must satisfy, in order to be called to the Bar by an \emph{Inn}
and become qualified to practise as a \emph{barrister}. It also includes
details of the training requirements that \emph{BSB authorised persons}
are required to meet, and provides for the regulation of
Authorised~Education and Training Organisations (AETOs);
\item Part 5 \subsection{-- Enforcement Regulations} -- this part sets out the
enforcement procedures that apply if \emph{applicable persons} fail to
act in accordance with the requirements of this~\emph{Handbook};
\item Part 6 \subsection{-- Definitions} -- this part defines all the
italicised terms used in this \emph{Handbook}.
\end{numlist}
\subsection{I6}

The \emph{Handbook} includes Core Duties, Outcomes, Guidance, Rules and
Regulations. ``CD'' refers to Core Duties, ``o'' to Outcomes, ``g'' to
Guidance, ``r'' to Rules and Regulations. The Regulations form the basis
upon which enforcement action may be taken and are set out in Part E of
this Handbook. The effect of something being classified as a Core Duty,
Outcome, Guidance, Rule or Regulations is as follows:
\begin{numlist}\item \cdsubsection{Core Duties --} these underpin the entire regulatory
framework and set the mandatory standards that all \emph{BSB regulated
persons} or \emph{unregistered barristers} are required to meet. They
also define the core elements of professional conduct. Disciplinary
proceedings may be taken against a \emph{BSB regulated person} or
\emph{unregistered barrister} if the \emph{Bar Standards Board} believes
there has been a breach by that person of the Core Duties set out in
this \emph{Handbook} and that such action would be in accordance with
the \emph{Enforcement Strategy}.
\item \ocsubsection{The Outcomes --} these explain the reasons for the regulatory
scheme and what it is designed to achieve. They are derived from the
\emph{regulatory objectives} as defined in the LSA and the risks which
must be managed if those objectives are to be achieved. They are not
themselves mandatory rules, but they are factors which \emph{BSB
regulated persons~}or \emph{unregistered barristers} should have in mind
when considering how the Core Duties, Conduct Rules or Bar Qualification
Rules (as appropriate) should be applied in particular circumstances.
The \emph{Bar Standards Board} will take into account whether or not an
Outcome has, or might have been, adversely affected when considering how
to respond to alleged breaches of the Core Duties, Conduct Rules or Bar
Qualification Rules.
 \item \rulesubsection{The Rules --} The Rules serve three purposes:
\begin{alphlist}
\item the Conduct Rules supplement the Core Duties and are mandatory.
Disciplinary proceedings may be taken against a \emph{BSB regulated
person} or \emph{unregistered barrister} if the \emph{Bar Standards
\emph{Board}} believes there has been a breach by that person of the
Conduct Rules set out as applying to them in Part 2 of this
\emph{Handbook} and that it would be in accordance with the
\emph{Enforcement strategy}~to take such action. However, the Conduct
Rules are not intended to be exhaustive. In any situation where no
specific Rule applies, reference should be made to the Core Duties. In
situations where specific Rules do apply, it is still necessary to
consider the Core Duties, since compliance with the Rules alone will not
necessarily be sufficient to comply with the Core Duties;

\item the Rules contained within ``Scope of Practice Rules'' set out the
requirements for authorisation and the scope of practice for different
kinds of \emph{BSB authorised person} and include some rules relevant to
\emph{unregistered barristers}. These rules are mandatory;

\item the rest of Part 3 and Part 4 set out the requirements which must be
met by a~\emph{person~}before they may undertake a specific role within
those regulated by the \emph{Bar Standards Board.} If a person fails to
meet those requirements, they will not be permitted to undertake that
role by the \emph{Bar Standards Board}. Where requirements are
continuing and a \emph{BSB regulated \emph{person}} or
\emph{unregistered barrister} fails to meet such requirements which are
relevant to that \emph{BSB regulated person} or \emph{unregistered
barrister}, the \emph{Bar Standards Board} may take steps in accordance
with Part 3 or Part 5 to have that \emph{BSB regulated person} or
\emph{unregistered barrister} prevented from continuing within that
role.
\end{alphlist}
\guidancesubsection{Guidance --}
\begin{alphlist}
\item Guidance serves a number of purposes:
\begin{romlist}
\item to assist in the interpretation and application of the Core Duties or
Rules to which such Guidance relates.

\item to provide examples of the types of conduct or behaviour that the
Rules are intended to encourage or which would likely indicate
compliance with the relevant Rule or, conversely, which may constitute
non-compliance with the Rule to which such Guidance relates.

\item to explain how the Rule applies to a particular type of
\emph{person} or \emph{unregistered barrister} and how that particular
\emph{person} could comply with that Rule.

\item to act as a signpost to other rules or to guidance on the \emph{Bar
Standards Board} website or elsewhere which may be relevant when
considering the scope of the Rule.

\item in Part 3, to give further information about the process of applying
for authorisation and about how the \emph{Bar Standards Board} intends
to exercise its discretionary powers in relation to the authorisation of
entities.
\end{romlist}
\item The Guidance set out in this Handbook is not the only guidance which
is relevant to \emph{BSB \emph{regulated persons and unregistered
barristers}.} In addition to the Guidance, the \emph{Bar Standards
Board} has published and will publish from time to time various guidance
on its website which supplements this \emph{Handbook}, including (but
not limited to):
\begin{romlist}
\item the Bar Qualification Manual; and

\item the BSB's Supporting Information on the BSB Handbook Equality Rules.
\end{romlist}
\item In carrying out their obligations or meeting the requirements of this
\emph{Handbook}, \emph{BSB regulated persons} and \emph{unregistered
barristers} must have regard to any relevant guidance issued by the
\emph{Bar Standards Board} which will be taken into account by the
\emph{Bar Standards Board} if there is an alleged breach of or otherwise
non-compliance with of the obligations imposed on a \emph{BSB regulated
person} or \emph{unregistered barrister} under this \emph{Handbook}.
Failure to comply with the guidance will not of itself be proof of such
breach or non-compliance but the \emph{BSB regulated person} or
\emph{unregistered barrister} will need to be able to show how the
obligation has been met notwithstanding the departure from the relevant
guidance.\end{alphlist}
\item \rulesubsection{Regulations} -- Part 5 of this \emph{Handbook} sets out the
regulations which bind the \emph{Bar Standards \emph{Board}} when it
considers alleged breaches of the \emph{Handbook} and subsequent
enforcement action. These Regulations also bind the various Tribunals
and panels referred to in that Part and all persons who are subject to
the enforcement process. When considering enforcement action under Part
5, the \emph{Bar Standards Board's} response to any alleged breach of or
non-compliance with the Core Duties or the Rules will be informed by the
impact of the alleged breach or non-compliance on the achievement of the
relevant Outcomes, as well by as its own \emph{Supervision and
Enforcement Strategies}~and any other policies published from time to
time which the \emph{Bar Standards Board} regards as relevant (taking
into account the nature of the alleged breach or non-compliance).
\end{numlist}
\rulesection{A3.
Amendments to the Handbook}


\rulesubsection{rI1}

Subject to Rules r1 and r2, the \emph{Bar Standards Board} may make
amendments and/or additions to this \emph{Handbook} by resolution and
any such amendments and/or additions will take effect on such date as
the \emph{Bar Standards Board} appoints or, if no such date is
appointed, on the date when notice of the amendment is first published
on the \emph{Bar Standard Board's} website following approval under
Schedule 4 of the Legal Services Act 2007.

\rulesubsection{rI2}

The \emph{Bar Standards Board} shall not without the unanimous consent
of the Inns amend or waive any rule so as to permit a person who has not
been called to the Bar by an Inn to practise as a barrister.

\rulesubsection{rI3}

Removed from 1 November 2017.

\rulesubsection{rI4}

Amendments and additions will be published on the \emph{Bar Standards
Board's} website.

\rulesection{A4.
Waivers}


\rulesubsection{rI5}

Subject to rI2, the \emph{Bar Standards Board} shall have the power to
waive or modify:
\begin{numlist}
\item the duty imposed on a \emph{BSB regulated person or unregistered
barrister} to comply with the provisions of this \emph{Handbook}; or
\item any other requirement of this \emph{Handbook}
\item in such circumstances and to such extent as the \emph{Bar Standards
Board} may think fit and either conditionally or unconditionally.
\end{numlist}
\rulesubsection{rI6}

Any application to the \emph{Bar Standards Board} for a waiver of any of
the mandatory requirements or to extend the time within which to
complete any of the mandatory requirements must be made in writing,
setting out all relevant circumstances relied on and supported by all
relevant documentary evidence.

\rulesection{B.
Application}


\rulesubsection{rI7}

Subject to paragraphs rI8 to rI11 below, this \emph{Handbook} applies to
the following categories of person:
\begin{numlist}\item all \emph{barristers}, that is to say:
\begin{alphlist}
\item barristers who hold a practising certificate in accordance with
Section 3.C (``\emph{practising~\emph{barristers}'');}

\item barristers who are undertaking \emph{pupillage}, or a part thereof
and who are registered with the \emph{Bar Standards Board} as a
\emph{pupil} (``\emph{pupils}''); and

\item all \emph{unregistered barristers}.
\item European lawyers registered as such by the \emph{Bar Standards
Board~}and by an \emph{Inn} in accordance with Section 3.D but only in
connection with professional work undertaken by them in England and
Wales (``\emph{registered European lawyers}'');\end{alphlist}
\item bodies which have been authorised or licensed by the \emph{Bar
Standards Board} in accordance with Section 3.E of this Handbook
(``\emph{BSB entities}'');
\item individuals who are authorised to provide \emph{reserved legal
activities} by another~\emph{Approved Regulator where such individuals
are employed by a \emph{BSB authorised person} (``\emph{authorised
(non-BSB)} \emph{individuals}'');}
\item all managers of \emph{BSB entities};
\item to the extent that this \emph{Handbook} is expressed to apply to them
in their capacity as such, owners of a \emph{BSB entity};
\item solely as regards provisions in this \emph{Handbook} relating to
disqualification from performing a \emph{relevant activity} or
\emph{relevant activities} and not otherwise, any \emph{non-authorised
individuals} who are employed by a \emph{BSB authorised person}; and
\item solely as regards Section 4.B of the \emph{Handbook}, individuals who
wish to be called to the Bar and to become qualified to practise as a
barrister and authorised education and training organisations.~Until 1
January 2020, for the purposes of any proceedings of the Inns Conduct
Committee, Part 4 applies as if version 3.5 of the BSB Handbook were in
force;
\item and persons within paragraphs rI7.1 to 7 (with the exception of
pupils without a provisional practising certificate, unregistered
barristers and owners) are referred to as \emph{``BSB regulated
persons''} throughout this \emph{Handbook}. For the purposes of Part 5
of the \emph{Handbook} these persons (and those who are no longer
\emph{BSB regulated persons} or \emph{unregistered barristers} but who
were at the time when any conduct was complained of or reported) are
referred to as \emph{``applicable persons''}. For the avoidance of
doubt, the \emph{Handbook} continues to apply to those who are subject
to suspension.
\end{numlist}
\rulesubsection{rI8}

If you are a \emph{BSB authorised individual} who is employed by or a
\emph{manager} of an~\emph{authorised (non-BSB) body} and is subject to
the regulatory arrangements of the~\emph{Approved Regulator} of that
body, and the requirements of that other \emph{Approved Regulator}
conflict with a provision within this \emph{Handbook} then the
conflicting provision within this \emph{Handbook} shall not apply to
you. You will instead be expected to comply with the requirements of
that other \emph{Approved Regulator} and, if you do so, you will not be
considered to be in breach of the relevant provision of this
\emph{Handbook}.

\rulesubsection{rI9}

If you are a \emph{pupil} and are:
\begin{numlist}\item the \emph{pupil} of an \emph{employed barrister (non-authorised
body)}; or
\item the \emph{pupil} of a manager or employee \emph{of a BSB entity}; or
\item the \emph{pupil} of a \emph{manager} or employee of an
\emph{authorised (non-BSB) body}; or
\item spending a period of external training with a \emph{BSB entity} or an
\emph{authorised (non-BSB) body}
\item this \emph{Handbook} will apply to you as though you were an employee
of the \emph{barrister's~}employer or the body concerned.
\end{numlist}
\rulesubsection{rI10}

If you are a \emph{registered European lawyer}, then, except where
otherwise provided, the provisions of this \emph{Handbook} which apply
to \emph{barristers} shall apply to you, in connection with all
professional work undertaken by you in England and Wales, as if you were
a \emph{self-employed barrister} or an \emph{employed \emph{barrister
(non-authorised body)}} or a \emph{manager} or employee of an
\emph{authorised (non BSB) body} or a manager or employee of a \emph{BSB
entity} (as the case may be) depending on the way in which you practise.

\rulesubsection{rI11}

In addition to the above, each Part to this Handbook has its own
application section which sets out the more detailed application of that
particular Part. In the event of any inconsistency, the application
section specific to the particular Part shall prevail over these general
provisions.

\mysec{C.
Commencement and Transitional Provisions}


\rulesubsection{rI12}

This fourth edition of the \emph{Handbook} came into force on 1 April
2019 and replaced the third edition of the \emph{Handbook} (which came
into effect from 3 April 2017).

\rulesubsection{rI13}

Subject to rI14 below, in respect of anything done or omitted to be done
or otherwise arising before 6 January 2014:
\begin{numlist}
\item Parts 2 and 3 of this Handbook shall not apply;
\item the edition of the Code of Conduct or relevant Annexe in force at the
relevant time shall apply; and
\item any reference to Part 2, Part 3 or Part 5 of this \emph{Handbook}
shall include reference to the corresponding Part of the edition of the
Code of Conduct or relevant Annexe which was in force at the relevant
time.
\end{numlist}

\rulesubsection{rI14}

Where:
\begin{numlist}
\item a matter is being dealt with under the Complaints Regulations prior
to 15 October 2019; or the Disciplinary Tribunal Regulations 2014 prior
to 1 November 2017; or Annexe J (The Complaints Rules 2011), Annexe K
(The Disciplinary Tribunals Regulations (2009) (Reissued 1 February
2012)), Annexe M (Hearings before the Visitors Rules), Annexe N (Interim
Suspension Rules) or Annexe O (Fitness to Practise Rules) prior to~6
January 2014; and that matter has not concluded or been disposed of;~or
\item anything done or omitted to be done or otherwise arising before 6
January 2014 required referral for consideration in accordance with any
of the above Annexes, then Part 5 of this \emph{Handbook} shall apply to
all such cases and any step taken pursuant to the Annexes then applying
(if any) shall be regarded, unless otherwise decided, as having been
taken pursuant to the equivalent provisions of Part 5 of this
\emph{Handbook}, save that no fine in excess of £15,000 may be imposed
by a \emph{Disciplinary Tribunal} in respect of conduct before 6 January
2014 and no financial \emph{administrative sanction} in excess of £300
may be imposed by the \emph{Commissioner} or an \emph{Independent
Decision-Making Panel} in respect of conduct before 6 January 2014.
\end{numlist}
\rulesection{D.
Interpretation}


\rulesubsection{rI15}

In this \emph{Handbook}:
\begin{numlist}\item words and phrases in italics shall have the meaning given to them in
Part 6;
\item any reference to the singular shall include the plural and vice
versa;
\item any reference to another provision in this \emph{Handbook} shall be a
reference to that provision as amended from time to time; and
\item where references are made to an enactment, it is a reference to that
enactment as amended, and includes a reference to that provision as
extended or applied by or under any other enactment.
\end{numlist}

