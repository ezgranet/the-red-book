\chap{Part 2A—Application }
%\section{Application}
\rulesection{rc1}
\rulesubsection{Who?}
\begin{numlist}\item Section 2(b) (Core Duties): applies to all \emph{BSB regulated
persons} and \emph{unregistered barristers} except where stated
otherwise, and references to ``you'' and ``your'' in Section 2(b) shall
be construed accordingly.

\item Section 2(c) (Conduct Rules):
\begin{alphlist}\item Applies to all \emph{BSB regulated persons}.
\item Rules rC3(5), rC4, rC8, rC16, rC19 and rC64 to rC70 (and associated
guidance to those rules) and the guidance on Core Duties also apply to
\emph{unregistered barristers}. If an \emph{unregistered barrister}
practises as a \emph{barrister} as set out in rS9 then those rules which
apply to practising barristers shall also apply.\\
References to ``you'' and ``your'' in Section 2(c) shall be construed
accordingly\end{alphlist}
\item Section 2(d) (Specific Rules): applies to specific groups as defined
in each sub-section and references to ``you'' and ``your'' shall be
construed accordingly.
\end{numlist}
\rulesection{rc2}
\rulesubsection{When?}

\begin{numlist}\item Section 2(b) applies when practising or otherwise providing
\emph{legal services}. In addition,  \textbf{\textcolor{mygold}{CD5}} and  \textbf{\textcolor{mygold}{CD9}} apply at all times.
\item Section 2(c) applies when practising or otherwise providing
\emph{legal services}. In addition, rules rC8, rC16 and rC64 to rC70 and
the associated guidance apply at all times.
\item Section 2(d) applies when practising or otherwise providing
\emph{legal services}.
\item Sections 2(b), 2(c) and 2(d) only apply to \emph{registered European
lawyers} in connection with professional work undertaken by them in that
capacity in England and Wales.
\end{numlist}


\color{mygold}
\chap{Part 2
- B. The Core Duties}\color{black}

\cdsubsection{CD1}

 You \textcolor{myred}{\textbf{must}} observe your duty to the court in the
administration of justice

\cdsubsection{CD2}

 You \textcolor{myred}{\textbf{must}} act in the best interests of each client


\cdsubsection{ \textbf{\textcolor{mygold}{CD3}}}

 You \textcolor{myred}{\textbf{must}} act with honesty, and with integrity 

\cdsubsection{ \textbf{\textcolor{mygold}{CD4}}}

 You \textcolor{myred}{\textbf{must}} maintain your independence 
 
\cdsubsection{ \textbf{\textcolor{mygold}{CD5}}}
 You \textcolor{myred}{\textbf{must not}} behave in a way which is likely to diminish the
trust and confidence which the public places in you or in the profession

\cdsubsection{ \textbf{\textcolor{mygold}{CD6}}}
 You \textcolor{myred}{\textbf{must}} keep the affairs of each client confidential


\cdsubsection{ \textbf{\textcolor{mygold}{CD7}}}

You \textcolor{myred}{\textbf{must}} provide a competent standard of work and service to
each client
\cdsubsection{ \textbf{\textcolor{mygold}{CD8}}}


 You \textcolor{myred}{\textbf{must not}} discriminate unlawfully against any person

\cdsubsection{ \textbf{\textcolor{mygold}{CD9}}}

 You \textcolor{myred}{\textbf{must}} be open and co-operative with your regulators

\cdsubsection{CD10}

 You \textcolor{myred}{\textbf{must}} take reasonable steps to manage your practice, or
carry out your role within your practice, competently and in such a way
as to achieve compliance with your legal and regulatory obligations


\guidancesection{Guidance
to the Core Duties}


\subsubsection{\color{darkgrey}gC1}

The Core Duties are not presented in order of precedence, subject to the
following:
\begin{numlist}\item \textbf{\textcolor{mygold}{CD1}} overrides any other core duty, if and to the extent the two are
inconsistent. Rules rC3(5) and rC4 deal specifically with the
relationship between \textbf{\textcolor{mygold}{CD1}}, \textbf{\textcolor{mygold}{CD2}} and  \textbf{\textcolor{mygold}{CD4}}~and you should refer to those
rules and to the related Guidance;
\item in certain other circumstances set out in this Code of Conduct one
Core Duty overrides another. Specifically, Rule rC16 provides that \textbf{\textcolor{mygold}{CD2}}
(as well as being subject to \textbf{\textcolor{mygold}{CD1}}) is subject to your obligations under
 \textbf{\textcolor{mygold}{CD3}},  \textbf{\textcolor{mygold}{CD4}} and  \textbf{\textcolor{mygold}{CD8}}.
\end{numlist}

\subsubsection{\color{darkgrey}gC2}

Your obligation to take reasonable steps to manage your \emph{practice},
or carry out your role within your \emph{practice}, competently and in
such a way as to achieve compliance with your legal and regulatory
obligations (CD10) includes an obligation to take all reasonable steps
to mitigate the effects of any breach of those legal and regulatory
obligations once you become aware of the same.

\subsubsection{\color{darkgrey}gC2A}

Your obligation to be open and co-operative with your regulators ( \textbf{\textcolor{mygold}{CD9}})
includes being open and co-operative with all relevant regulators and
ombudsman schemes, including but not limited to approved regulators
under the Legal Services Act 2007 and the Legal Ombudsman.

\chap{Part 2
- C. The Conduct Rules }
\chap{Part 2-C1. You and the court }

\ocsection{Outcomes C1-C5 }


\subsection{\color{bleu}oC1}

The \emph{court} is able to rely on information provided to it by those
conducting litigation and by advocates who appear before it.

\subsection{\color{bleu}oC2}

The proper administration of justice is served.

\subsection{\color{bleu}oC3}

The interests of \emph{clients} are protected to the extent compatible
with outcomes oC1 and oC2 and the Core Duties.

\subsection{\color{bleu}oC4}

Both those who appear before the \emph{court} and \emph{clients}
understand clearly the extent of the duties owed to the \emph{court} by
advocates and those conducting litigation and the circumstances in which
duties owed to \emph{clients} will be overridden by the duty owed to the
\emph{court}.

\subsection{\color{bleu}oC5}

The public has confidence in the administration of justice and in those
who serve it.

\rulesection{Rules
C3-C6 }


\rulesubsection{rc3}

You owe a duty to the \emph{court} to act with independence in the
interests of justice. This duty overrides any inconsistent obligations
which you may have (other than obligations under the criminal law). It
includes the following specific obligations which apply whether you are
acting as an advocate or are otherwise involved in the conduct of
litigation in whatever role (with the exception of Rule C3(1) below,
which applies when acting as an advocate):

\begin{numlist}\item You \textcolor{myred}{\textbf{must not}} knowingly or recklessly mislead or attempt to mislead
the \emph{court};
\item You \textcolor{myred}{\textbf{must not}} abuse your role as an advocate;
\item You \textcolor{myred}{\textbf{must}} take reasonable steps to avoid wasting the \emph{court's}
time;
\item You \textcolor{myred}{\textbf{must}} take reasonable steps to ensure that the \emph{court} has
before it all relevant decisions and legislative provisions;
\item You \textcolor{myred}{\textbf{must}} ensure that your ability to act independently is not
compromised.
\end{numlist}

\rulesubsection{rc4}

Your duty to act in the best interests of each \emph{client} is subject
to your duty to the \emph{court}.

\rulesubsection{rc5}

Your duty to the \emph{court} \textcolor{myred}{\textbf{does not}} require you to act in breach of
your duty to keep the affairs of each \emph{client} confidential.

\subsection{Not misleading the court}

\rulesubsection{rc6}

Your duty not to mislead the \emph{court} will include the following
obligations:
\begin{numlist}\item You \textcolor{myred}{\textbf{must not}}:
\begin{alphlist}\item make submissions, representations or any other statement; or
\item ask questions which suggest facts to witnesses
which you know, or are instructed, are untrue or misleading.\end{alphlist}
\item You \textcolor{myred}{\textbf{must not}} call witnesses to give evidence or put affidavits or
witness statements to the \emph{court} which you know, or are
\emph{instructed}, are untrue or misleading, unless you make clear to
the \emph{court} the true position as known by or instructed to you.
\end{numlist}

\guidancesection{Guidance
to Rules C3-C6 and relationship to \textbf{\textcolor{mygold}{CD1}}-CD2 }


\subsubsection{\color{darkgrey}gC3}

Rules rC3 -- rC6 set out some specific aspects of your duty to the
\emph{court} (\textcolor{mygold}{\textbf{CD1}}). See \textbf{\textcolor{mygold}{CD1}} and associated Guidance at gC1.

\subsubsection{\color{darkgrey}gC4}

As to your duty not to mislead the court:
\begin{numlist}\item knowingly misleading the \emph{court} includes being complicit in
another person misleading the court;
\item knowingly misleading the \emph{court} also includes inadvertently
misleading the court if you later realise that you have misled the
\emph{court}, and you fail to correct the position;
\item recklessly means being indifferent to the truth, or not caring
whether something is true or false; and
\item the duty continues to apply for the duration of the case.
\end{numlist}
\subsubsection{\color{darkgrey}gC5}

Your duty under Rule rC3(4)~includes drawing to the attention of the
\emph{court} any decision or provision which may be adverse to the
interests of your \emph{client}. It is particularly important where you
are appearing against a litigant who is not legally represented.

\subsubsection{\color{darkgrey}gC6}

You are obliged by \textbf{\textcolor{mygold}{CD2}} to promote and to protect your \emph{client's}
interests so far as that is consistent with the law and with your
overriding duty to the \emph{court} under \textbf{\textcolor{mygold}{CD1}}. Your duty to the
\emph{court} \textcolor{myred}{\textbf{does not}} prevent you from putting forward your
\emph{client's} case simply because you do not believe that the facts
are as your \emph{client} states them to be (or as you, on your
\emph{client's} behalf, state them to be), as long as any positive case
you put forward accords with your \emph{instructions} and you do not
mislead the \emph{court}. Your role when acting as an advocate or
conducting litigation is to present your \emph{client's} case, and it is
not for you to decide whether your \emph{client's} case is to be
believed.

\subsubsection{\color{darkgrey}gC7}

For example, you are entitled and it may often be appropriate to draw to
the witness's attention other evidence which appears to conflict with
what the witness is saying and you are entitled to indicate that a
\emph{court} may find a particular piece of evidence difficult to
accept. But if the witness maintains that the evidence is true, it
should be recorded in the witness statement and you will not be
misleading the \emph{court} if you call the witness to confirm their
witness statement. Equally, there may be circumstances where you call a
hostile witness whose evidence you are instructed is untrue. You will
not be in breach of Rule rC6 if you make the position clear to the
\emph{court}. See, further, the guidance at gC14.

\subsubsection{\color{darkgrey}gC8}

As set out in Rule rC5, your duty to the \emph{court} \textcolor{myred}{\textbf{does not}} permit or
require you to disclose confidential information which you have obtained
in the course of your \emph{instructions} and which your client has not
authorised you to disclose to the \emph{court}. However, Rule rC6
requires you not knowingly to mislead the \emph{court}. There may be
situations where you have obligations under both these rules.

\subsubsection{\color{darkgrey}gC9}

Rule rC4 makes it clear that your duty to act in the best interests of
your \emph{client} is subject to your duty to the \emph{court}. For
example, if your \emph{client} were to tell you that they have committed
the crime with which they were charged, in order to be able to ensure
compliance with Rule rC4 on the one hand and Rule rC3 and Rule rC6 on
the other:
\begin{numlist}\item you would not be entitled to disclose that information to the
\emph{court} without your \emph{client's} consent; and
\item you would not be misleading the \emph{court} if, after your
\emph{client} had entered a plea of `not guilty', you were to test in
cross-examination the reliability of the evidence of the prosecution
witnesses and then address the jury to the effect that the prosecution
had not succeeded in making them sure of your \emph{client's} guilt.
\end{numlist}

\subsubsection{\color{darkgrey}gC10}

However, you would be misleading the \emph{court} and would therefore be
in breach of Rules rC5 and rC6 if you were to set up a positive case
inconsistent with the confession, as for example by:
\begin{numlist}\item suggesting to prosecution witnesses, calling your \emph{client} or
your witnesses to show; or submitting to the jury, that your
\emph{client} did not commit the crime; or
\item suggesting that someone else had done so; or
\item putting forward an alibi.
\end{numlist}

\subsubsection{\color{darkgrey}gC11}

If there is a risk that the \emph{court} will be misled unless you
disclose confidential information which you have learned in the course
of your \emph{instructions}, you should ask the \emph{client} for
permission to disclose it to the \emph{court}. If your \emph{client}
refuses to allow you to make the disclosure You \textcolor{myred}{\textbf{must}} cease to act, and
return your \emph{instructions}: see Rules rC25 to rC27 below. In these
circumstances You \textcolor{myred}{\textbf{must not}} reveal the information to the \emph{court}.

\subsubsection{\color{darkgrey}gC12}

For example, if your \emph{client} tells you that they have previous
\emph{convictions} of which the prosecution is not aware, you \textcolor{myred}{\textbf{may not}}
disclose this without their consent. However, in a case where mandatory
sentences apply, the non-disclosure of the previous \emph{convictions}
will result in the \emph{court} failing to pass the sentence that is
required by law. In that situation, You \textcolor{myred}{\textbf{must}} advise your \emph{client}
that if consent is refused to your revealing the information you will
have to cease to act. In situations where mandatory sentences do not
apply, and your \emph{client} \textcolor{myred}{\textbf{does not}} agree to disclose the previous
\emph{convictions}, you can continue to represent your \emph{client} but
in doing so \textcolor{myred}{\textbf{must}} not say anything that misleads the court. This will
constrain what you can say in mitigation. For example, you could not
advance a positive case of previous good character knowing that there
are undisclosed prior \emph{convictions}. Moreover, if the \emph{court}
asks you a direct question You \textcolor{myred}{\textbf{must not}} give an untruthful answer and
therefore you would have to withdraw if, on your being asked such a
question, your \emph{client} still refuses to allow you to answer the
question truthfully. You should explain this to your \emph{client}.

\subsubsection{\color{darkgrey}gC13}

Similarly, if you become aware that your \emph{client} has a document
which should be disclosed but has not been disclosed, you cannot
continue to act unless your \emph{client} agrees to the disclosure of
the document. In these circumstances You \textcolor{myred}{\textbf{must not}} reveal the existence
or contents of the document to the \emph{court}.

\rulesection{Rule C7
- Not abusing your role as an advocate }


\rulesubsection{rc7}

Where you are acting as an advocate, your duty not to abuse your role
includes the following obligations:
\begin{numlist}\item You \textcolor{myred}{\textbf{must not}} make statements or ask questions merely to insult,
humiliate or annoy a witness or any other person;
\item You \textcolor{myred}{\textbf{must not}} make a serious allegation against a witness whom you
have had an opportunity to cross-examine unless you have given that
witness a chance to answer the allegation in cross-examination;
\item You \textcolor{myred}{\textbf{must not}} make a serious allegation against any person, or suggest
that a person is guilty of a crime with which your \emph{client} is
charged unless:
\begin{alphlist}
\item you have reasonable grounds for the allegation; and

\item the allegation is relevant to your \emph{client's} case or the
credibility of a witness; and

\item where the allegation relates to a third party, you avoid naming them
in open \emph{court} unless this is reasonably necessary.
\end{alphlist}
\item You \textcolor{myred}{\textbf{must not}} put forward to the \emph{court} a personal opinion of
the facts or the law unless you are invited or required to do so by the
\emph{court} or by law.
\end{numlist}

\chap{Part 2
- C2. Behaving ethically }



\ocsection{Outcomes
C6-C9 }


\subsection{\color{bleu}oC6}

Those regulated by the \emph{Bar Standards Board} maintain standards of
honesty, integrity and independence, and are seen as so doing.

\subsection{\color{bleu}oC7}

The proper administration of justice, access to justice and the best
interests of \emph{clients} are served.

\subsection{\color{bleu}oC8}

Those regulated by the \emph{Bar Standards Board} do not discriminate
unlawfully and take appropriate steps to prevent \emph{discrimination}
occurring in their practices.

\subsection{\color{bleu}oC9}

Those regulated by the \emph{Bar Standards Board} and \emph{clients}
understand the obligations of honesty, integrity and independence.

\rulesection{Rules
C8-C9 - Honesty, integrity and independence }


\rulesubsection{rc8}

You \textcolor{myred}{\textbf{must not}} do anything which could reasonably be seen by the public to
undermine your honesty, integrity ( \textbf{\textcolor{mygold}{CD3}}) and independence ( \textbf{\textcolor{mygold}{CD4}}).

\rulesubsection{rc9}

Your duty to act with honesty and with integrity under  \textbf{\textcolor{mygold}{CD3}} includes the
following requirements:
\begin{numlist}\item You \textcolor{myred}{\textbf{must not}} knowingly or recklessly mislead or attempt to mislead
anyone;
\item You \textcolor{myred}{\textbf{must not}} draft any statement of case, witness statement,
affidavit or other document containing:
\begin{alphlist}\item any statement of fact or contention which is not supported by your
\emph{client} or by your \emph{instructions};
\item any contention which you do not consider to be properly arguable;
\item any allegation of fraud, unless you have clear instructions to allege
fraud and you have reasonably credible material which establishes an
arguable case of fraud;
\item (in the case of a witness statement or affidavit) any statement of
fact other than the evidence which you reasonably believe the witness
would give if the witness were giving evidence orally;\end{alphlist}
\item You \textcolor{myred}{\textbf{must not}} encourage a witness to give evidence which is misleading
or untruthful;
\item You \textcolor{myred}{\textbf{must not}} rehearse, practise with or coach a witness in respect of
their evidence;
\item unless you have the permission of the representative for the opposing
side or of the \emph{court}, You \textcolor{myred}{\textbf{must not}} communicate with any witness
(including your \emph{client}) about the case while the witness is
giving evidence;
\item You \textcolor{myred}{\textbf{must not}} make, or offer to make, payments to any witness which
are contingent on their evidence or on the outcome of the case;
\item You \textcolor{myred}{\textbf{must}} only propose, or accept, fee arrangements which are legal.

\end{numlist}

\guidancesection{Guidance
to Rules C8-C9 and relationship to \textbf{\textcolor{mygold}{CD1}}- \textbf{\textcolor{mygold}{CD5}} }


\subsubsection{\color{darkgrey}gC14}

Your honesty, integrity and independence are fundamental. The interests
of justice (\textcolor{mygold}{\textbf{CD1}}) and the \emph{client's} best interests (\textcolor{mygold}{\textbf{CD2}}) can only
be properly served, and any conflicts between the two properly resolved,
if you conduct yourself honestly and maintain your independence from
external pressures, as required by  \textbf{\textcolor{mygold}{CD3}} and  \textbf{\textcolor{mygold}{CD4}}. You should also refer to
Rule rC16 which subjects your duty to act in the best interests of your
\emph{client} (\textcolor{mygold}{\textbf{CD2}}) to your observance of  \textbf{\textcolor{mygold}{CD3}} and  \textbf{\textcolor{mygold}{CD4}}, as well as to
your duty to the \emph{court} (\textcolor{mygold}{\textbf{CD1}}).

\subsubsection{\color{darkgrey}gC15}

Other rules deal with specific aspects of your obligation to act in your
\emph{client's} best interests (\textcolor{mygold}{\textbf{CD2}}) while maintaining honesty,
integrity ( \textbf{\textcolor{mygold}{CD3}}) and independence ( \textbf{\textcolor{mygold}{CD4}}), such as rule rC21(1)0 (not acting
where your independence is compromised), rule rC10 (not paying or
accepting \emph{referral fees}) and rC21 (not acting in circumstances of
a conflict of interest or where you risk breaching one \emph{client's}
confidentiality in favour of another's).

\subsubsection{\color{darkgrey}gC16}

Rule C8 addresses how your conduct is perceived by the public. Conduct
on your part which the public may reasonably perceive as undermining
your honesty, integrity or independence is likely to diminish the trust
and confidence which the public places in you or in the profession, in
breach of  \textbf{\textcolor{mygold}{CD5}}. Rule C9 is not exhaustive of the ways in which  \textbf{\textcolor{mygold}{CD5}} may be
breached.

\subsubsection{\color{darkgrey}gC17}

In addition to your obligation to only propose, or accept, fee
arrangements which are legal in Rule C9(7), You \textcolor{myred}{\textbf{must}} also have regard to
your obligations in relation to referral fees in Rule rC10 and the
associated guidance.

\subsection{Examples of how you may be seen as compromising your
independence}

\subsubsection{\color{darkgrey}gC18}

The following may reasonably be seen as compromising your independence
in breach of Rule 8 (whether or not the circumstances are such that Rule
rC10 is also breached):
\begin{numlist}\item offering, promising or giving:
\begin{alphlist}\item any commission or referral fee (of whatever size) — note that these
are in any case prohibited by Rule rC10 and associated guidance; or
\item a gift (apart from items of modest value),\end{alphlist}

to any \emph{client}, \emph{professional client} or other
\emph{intermediary}; or
\item lending money to any such \emph{client}, \emph{professional client}
or other \emph{intermediary}; or
\item accepting any money (whether as a loan or otherwise) from any
\emph{client}, \emph{professional client} or other \emph{intermediary},
unless it is a payment for your professional services or reimbursement
of expenses or of disbursements made on behalf of the \emph{client};
\end{numlist}

\subsubsection{\color{darkgrey}gC19}

If you are offered a gift by a current, prospective or former
\emph{client}, \emph{professional client} or other \emph{intermediary},
you should consider carefully whether the circumstances and size of the
gift would reasonably lead others to think that your independence had
been compromised. If this would be the case, you should refuse to accept
the gift.

\subsubsection{\color{darkgrey}gC20}

The giving or receiving of entertainment at a disproportionate level may
also give rise to a similar issue and so should not be offered or
accepted if it would lead others reasonably to think that your
independence had been compromised.

\subsubsection{\color{darkgrey}gC21}

Guidance gC18 to gC20 above is likely to be more relevant where you are
a \emph{self-employed barrister}, a \emph{BSB entity}, an
\emph{authorised (non-BSB) individual}, an \emph{employed barrister (BSB
entity)} or a \emph{manager} of a \emph{BSB entity}. If you are a
\emph{BSB authorised individual} who is a an employee or \emph{manager}
of an \emph{authorised (non-BSB) body} or you are an \emph{employed
barrister (non-authorised body)} and your \emph{approved regulator} or
\emph{employer} (as appropriate) permits payments to which Rule rC10
applies, you may make or receive such payments only in your capacity as
such and as permitted by the rules of your \emph{approved regulator} or
\emph{employer} (as appropriate). For further information on referral
fees, see the guidance at gC32).

\subsection{Media comment}

\subsubsection{\color{darkgrey}gC22}

The ethical obligations that apply in relation to your professional
practice generally continue to apply in relation to media comment. In
particular, \emph{barristers} should be aware of the following:
\begin{enumerate}[label=•]
\item Client's best interests: Core Duty 2 and Rules C15(1)-(2) require a
barrister to promote fearlessly and by all proper and lawful means the
lay \emph{client's} best interests and to do so without regard to their
own interests.

\item Independence: Core Duties 3 and 4 provide that You \textcolor{myred}{\textbf{must not}} permit
your absolute independence, integrity and freedom from external
pressures to be compromised.

\item Trust and confidence: Core Duty 5 provides that You \textcolor{myred}{\textbf{must not}} behave in
a way which is likely to diminish the trust and confidence which the
public places in you or the profession.

\item Confidentiality: Core Duty 6 and Rule C15(5) require you to preserve
the confidentiality of your lay client's affairs and You \textcolor{myred}{\textbf{must not}}
undermine this unless permitted to do so by law or with the express
consent of the lay client.

\end{enumerate}
\subsection{Examples of what your duty to act with honesty and integrity may
require}

\subsubsection{\color{darkgrey}gC23}

Rule rC9 sets out some specific aspects of your duty under  \textbf{\textcolor{mygold}{CD3}} to act
with honesty and also with integrity.

\subsubsection{\color{darkgrey}gC24}

In addition to the above, where the other side is legally represented
and you are conducting correspondence in respect of the particular
matter, you are expected to correspond at all times with that other
party's legal representative — otherwise you may be regarded as
breaching  \textbf{\textcolor{mygold}{CD3}} or Rule C9.

\subsection{Other possible breaches of  \textbf{\textcolor{mygold}{CD3}} and/or  \textbf{\textcolor{mygold}{CD5}}}

\subsubsection{\color{darkgrey}gC25}

A breach of Rule rC9 may also constitute a breach of  \textbf{\textcolor{mygold}{CD3}} and/or  \textbf{\textcolor{mygold}{CD5}}.
Other conduct which is likely to be treated as a breach of  \textbf{\textcolor{mygold}{CD3}} and/or
 \textbf{\textcolor{mygold}{CD5}} includes (but is not limited to):
\begin{numlist}\item subject to Guidance C27~below, breaches of Rule rC8;
\item breaches of Rule rC10;
\item criminal conduct, other than \emph{minor criminal offences} (see
Guidance C27);
\item seriously offensive or discreditable conduct towards third parties;
\item dishonesty;
\item unlawful \emph{victimisation} or \emph{harassment}; or
\item abuse of your professional position.
\end{numlist}
\subsubsection{\color{darkgrey}gC26}

For the purposes of Guidance gC25(7) above, referring to your status as a
\emph{barrister}, for example on professional notepaper, in a context
where it is irrelevant, such as in a private dispute, may well
constitute abuse of your professional position and thus involve a breach
of  \textbf{\textcolor{mygold}{CD3}} and/or  \textbf{\textcolor{mygold}{CD5}}.

\subsubsection{\color{darkgrey}gC27}

Conduct which is not likely to be treated as a breach of Rules rC8 or
rC9, or  \textbf{\textcolor{mygold}{CD3}} or  \textbf{\textcolor{mygold}{CD5}}, includes (but is not limited to):
\begin{numlist}\item \emph{minor criminal offences};
\item your conduct in your private or personal life, unless this involves:
\begin{alphlist}\item abuse of your professional position; or
\item committing a \emph{criminal offence}, other than a \emph{minor
criminal offence}.\end{alphlist}
\end{numlist}

\subsubsection{\color{darkgrey}gC28}

For the purpose of Guidance C27 above, \emph{minor criminal offences}
include:
\begin{numlist}\item an offence committed in the United Kingdom which is a fixed-penalty
offence under the Road Traffic Offenders Act 1988; or
\item an offence committed in the United Kingdom or abroad which is dealt
with by a procedure substantially similar to that for such a
fixed-penalty offence; or
\item an offence whose main ingredient is the unlawful parking of a motor
vehicle.
\end{numlist}

\rulesection{Rule
C10 - Referral fees }


\rulesubsection{rc10}

You \textcolor{myred}{\textbf{must not}} pay or receive \emph{referral fees}.


\guidancesection{Guidance
to Rule C10 and relationship to \textbf{\textcolor{mygold}{CD2}}- \textbf{\textcolor{mygold}{CD5}} }


\subsubsection{\color{darkgrey}gC29}

Making or receiving payments in order to procure or reward the referral
to you by an intermediary of professional \emph{instructions} is
inconsistent with your obligations under \textbf{\textcolor{mygold}{CD2}} and/or  \textbf{\textcolor{mygold}{CD3}} and/or  \textbf{\textcolor{mygold}{CD4}} and
may also breach  \textbf{\textcolor{mygold}{CD5}}.

\subsubsection{\color{darkgrey}gC30}

Moreover:
\begin{numlist}\item where public funding is in place, the \emph{Legal Aid Agency's}
Unified Contract Standard Terms explicitly prohibit contract-holders
from making or receiving any payment (or any other benefit) for the
referral or introduction of a \emph{client}, whether or not the lay
\emph{client} knows of, and consents to, the payment;
\item whether in a private or publicly funded case, a \emph{referral fee}
to which the \emph{client} has not consented may constitute a bribe and
therefore a \emph{criminal offence} under the Bribery Act 2010;
\end{numlist}

\subsubsection{\color{darkgrey}gC31}

\emph{Referral fees} and inducements (as defined in the Criminal Justice
and Courts Act 2015) are prohibited where they relate to a claim or
potential claim for damages for personal injury or death or arise out of
circumstances involving personal injury or death personal injury claims:
section 56 Legal Aid, Sentencing and Punishment of Offenders Act 2012
and section 58 Criminal Justice and Courts Act 2015. Rule rC10 \textcolor{myred}{\textbf{does not}}
prohibit proper expenses that are not a reward for referring work, such
as genuine and reasonable payments for:
\begin{numlist}\item clerking and administrative services (including where these are
outsourced);
\item membership subscriptions to ADR bodies that appoint or recommend a
person to provide mediation, arbitration or adjudication services; or
\item advertising and publicity, which are payable whether or not any work
is referred. However, the fact that a fee varies with the amount of work
received \textcolor{myred}{\textbf{does not}} necessarily mean that that it is a \emph{referral
fee}, if it is genuinely for a marketing service from someone who is not
directing work to one provider rather than another, depending on who
pays more.
\end{numlist}

\subsubsection{\color{darkgrey}gC32}

Further guidance is available on the BSB's website.

\rulesection{Rule
C11 - Undertakings }


\rulesubsection{rc11}

You \textcolor{myred}{\textbf{must}} within an agreed timescale or within a reasonable period of
time comply with any undertaking you give in the course of conducting
litigation.


\guidancesection{Guidance
to Rule C11 }


\subsubsection{\color{darkgrey}gC33}

You should ensure your insurance covers you in respect of any liability
incurred in giving an undertaking.

\rulesection{Rule
C12 - Discrimination }


\rulesubsection{rc12}

You \textcolor{myred}{\textbf{must not}} discriminate unlawfully against, victimise or harass any
other person on the grounds of race, colour, ethnic or national origin,
nationality, citizenship, sex, gender re-assignment, sexual orientation,
marital or civil partnership status, disability, age, religion or
belief, or pregnancy and maternity.


\guidancesection{Guidance
to Rule C12 }


\subsubsection{\color{darkgrey}gC34}

Rules rC110 and associated guidance are also relevant to equality and
diversity. The BSB's Supporting Information on the BSB Handbook Equality
Rules is available on the BSB website.

\rulesection{Rules
C13-C14 - Foreign work }


\rulesubsection{rc13}

In connection with any \emph{foreign work} You \textcolor{myred}{\textbf{must}} comply with any
applicable rule of conduct prescribed by the law or by any national or
local Bar of:
\begin{numlist}\item the place where the work is or is to be performed; and
\item the place where any proceedings or matters to which the work relates
are taking place or contemplated; unless such rule is inconsistent with
any requirement of the Core Duties.
\end{numlist}
\rulesubsection{rc14}

If you solicit work in any jurisdiction outside England and Wales, you
must not do so in a manner which would be prohibited if you were a
member of the local Bar.


\guidancesection{Guidance
to Rules C13-14 }


\subsubsection{\color{darkgrey}gC35}

When you are engaged in \emph{cross border activities} within a
\emph{CCBE State} other than the UK, You \textcolor{myred}{\textbf{must}} comply with the rules at
2(d)5 which implement the part of the \emph{Code of Conduct for European
Lawyers} not otherwise covered by this Handbook as well as with any
other applicable rules of conduct relevant to that particular \emph{CCBE
State}. It is your responsibility to inform yourself as to any
applicable rules of conduct.

\chap{Part 2
- C3. You and your client }



\ocsection{Outcomes
C10-C20 }


\subsection{\color{bleu}oC10}

\emph{Clients} receive a competent standard of work and service.

\subsection{\color{bleu}oC11}

\emph{Clients'} best interests are protected and promoted by those
acting for them.

\subsection{\color{bleu}oC12}

\emph{BSB authorised persons} do not accept instructions from
\emph{clients} where there is a conflict between their own interests and
the \emph{clients'} or where there is a conflict between one or more
\emph{clients} except when permitted in this \emph{Handbook}.

\subsection{\color{bleu}oC13}

\emph{Clients} know what to expect and understand the advice they are
given.

\subsection{\color{bleu}oC14}

Care is given to ensure that the interests of vulnerable \emph{clients}
are taken into account and their needs are met.

\subsection{\color{bleu}oC15}

\emph{Clients} have confidence in those who are instructed to act on
their behalf.

\subsection{\color{bleu}oC16}

\emph{Instructions} are not accepted, refused, or returned in
circumstances which adversely affect the administration of justice,
access to justice or (so far as compatible with these) the best
interests of the \emph{client}.

\subsection{\color{bleu}oC17}

\emph{Clients} and \emph{BSB authorised persons} and \emph{authorised
(non-BSB) individuals} and \emph{managers} of \emph{BSB entities} are
clear about the circumstances in which \emph{instructions} \textcolor{myred}{\textbf{may not}} be
accepted or may or \textcolor{myred}{\textbf{must}} be returned.

\subsection{\color{bleu}oC18}

\emph{Clients} are adequately informed as to the terms on which work is
to be done.

\subsection{\color{bleu}oC19}

\emph{Clients} understand how to bring a \emph{complaint} and
\emph{complaints} are dealt with promptly, fairly, openly and
effectively.

\subsection{\color{bleu}oC20}

\emph{Clients} understand who is responsible for work done for them.

\rulesection{Rules
C15-C16 - Best interests of each client, provision of a competent
standard of work and confidentiality }


\rulesubsection{rc15}

Your duty to act in the best interests of each \emph{client} (\textcolor{mygold}{\textbf{CD2}}), to
provide a competent standard of work and service to each \emph{client}
( \textbf{\textcolor{mygold}{CD7}}) and to keep the affairs of each \emph{client} confidential ( \textbf{\textcolor{mygold}{CD6}})
includes the following obligations:
\begin{numlist}\item You \textcolor{myred}{\textbf{must}} promote fearlessly and by all proper and lawful means the
\emph{client's} best interests;
\item You \textcolor{myred}{\textbf{must}} do so without regard to your own interests or to any
consequences to you (which may include, for the avoidance of doubt, you
being required to take reasonable steps to mitigate the effects of any
breach of this \emph{Handbook});
\item You \textcolor{myred}{\textbf{must}} do so without regard to the consequences to any other person
(whether to your \emph{professional client}, \emph{employer} or any
other person);
\item You \textcolor{myred}{\textbf{must not}} permit your \emph{professional client}, \emph{employer}
or any other person to limit your discretion as to how the interests of
the \emph{client} can best be served; and
\item You \textcolor{myred}{\textbf{must}} protect the confidentiality of each \emph{client's} affairs,
except for such disclosures as are required or permitted by law or to
which your \emph{client} gives informed consent.
\end{numlist}

\rulesubsection{rc16}

Your duty to act in the best interests of each \emph{client} (\textcolor{mygold}{\textbf{CD2}}) is
subject to your duty to the \emph{court} (\textcolor{mygold}{\textbf{CD1}}) and to your obligations
to act with honesty, and with integrity ( \textbf{\textcolor{mygold}{CD3}}) and to maintain your
independence ( \textbf{\textcolor{mygold}{CD4}}).


\guidancesection{Guidance
to Rules C15-C16 and relationship to \textbf{\textcolor{mygold}{CD2}} and  \textbf{\textcolor{mygold}{CD6}}- \textbf{\textcolor{mygold}{CD7}} }


\subsubsection{\color{darkgrey}gC36}

Your duty is to your \emph{client}, not to your \emph{professional
client} or other \emph{intermediary} (if any).

\subsubsection{\color{darkgrey}gC37}

Rules rC15 and rC16 are expressed in terms of the interests of each
\emph{client}. This is because you may only accept \emph{instructions}
to act for more than one \emph{client} if you are able to act in the
best interests of each \emph{client} as if that \emph{client} were your
only \emph{client}, as \textbf{\textcolor{mygold}{CD2}} requires of you. See further Rule rC17 on the
circumstances when you are obliged to advise your \emph{client} to seek
other legal representation and Rules rC21(2) and rC21(3) on conflicts of
interest and the guidance to those rules at gC69.

\subsubsection{\color{darkgrey}gC38}

 \textbf{\textcolor{mygold}{CD7}} requires not only that you provide a competent standard of work but
also a competent standard of service to your \emph{client}. Rule rC15 is
not exhaustive of what You \textcolor{myred}{\textbf{must}} do to ensure your compliance with \textbf{\textcolor{mygold}{CD2}}
and  \textbf{\textcolor{mygold}{CD7}}. By way of example, a competent standard of work and of service
also includes:
\begin{numlist}\item treating each \emph{client} with courtesy and consideration; and
\item seeking to advise your \emph{client}, in terms they can understand;
and
\item taking all reasonable steps to avoid incurring unnecessary expense;
and
\item reading your instructions promptly. This may be important if there is
a time limit or limitation period. If you fail to read your instructions
promptly, it is possible that you will not be aware of the time limit
until it is too late.
\end{numlist}

\subsubsection{\color{darkgrey}gC39}

In order to be able to provide a competent standard of work, you should
keep your professional knowledge and skills up to date, regularly take
part in professional development and educational activities that
maintain and further develop your competence and performance and, where
you are a \emph{BSB entity} or a \emph{manager} of such body, you should
take reasonable steps to ensure that \emph{managers} and employees
within your organisation undertake such training. Merely complying with
the Continuing Professional Development requirements \textcolor{myred}{\textbf{may not}} be
sufficient to comply with Rule rC15. You should also ensure that you
comply with any specific training requirements of the \emph{Bar
Standards Board} before undertaking certain activities — for example,
you should not attend a police station to advise a suspect or
interviewee as to the handling and conduct of police interviews unless
you have complied with the following training requirements imposed by
the \emph{Bar Standards Board}: \emph{barristers} undertaking publicly
funded police station work under a criminal contract \textcolor{myred}{\textbf{must}} comply with
the training requirements specified by the Legal Aid Agency.
\emph{Barristers} undertaking privately funded police station work \textcolor{myred}{\textbf{must}}
complete the Police Station Qualification (``PSQ'') and (if they do not
hold higher rights of audience) the Magistrates Court Qualification.
Similarly, you should not undertake public access work without
successfully completing the required training specified by the \emph{Bar
Standards Board}.

\subsubsection{\color{darkgrey}gC40}

In addition to Guidance gC38 above, a \emph{BSB entity} or a
\emph{manager} of such body should ensure that work is allocated
appropriately, to \emph{managers} and/or employees with the appropriate
knowledge and expertise to undertake such work.

\subsubsection{\color{darkgrey}gC41}

You should remember that your \emph{client} \textcolor{myred}{\textbf{may not}} be familiar with
legal proceedings and may find them difficult and stressful. You should
do what you reasonably can to ensure that the \emph{client} understands
the process and what to expect from it and from you. You should also try
to avoid any unnecessary distress for your \emph{client}. This is
particularly important where you are dealing with a vulnerable
\emph{client}.

\subsubsection{\color{darkgrey}gC42}

The duty of confidentiality ( \textbf{\textcolor{mygold}{CD6}}) is central to the administration of
justice. \emph{Clients} who put their confidence in their legal advisers
must be able to do so in the knowledge that the information they give,
or which is given on their behalf, will stay confidential. In normal
circumstances, this information will be privileged and not disclosed to
a \emph{court}.  \textbf{\textcolor{mygold}{CD6}}, rC4 and Guidance gC8 and gC11 to gC13 provide
further information.

\subsubsection{\color{darkgrey}gC43}

Rule rC15(5) acknowledges that your duty of confidentiality is subject to
an exception if disclosure is required or permitted by law. For example,
you may be obliged to disclose certain matters by the Proceeds of Crime
Act 2002. Disclosure in those circumstances would not amount to a breach
of  \textbf{\textcolor{mygold}{CD6}} or Rule rC15(5) In other circumstances, you may only make
disclosure of confidential information where your \emph{client} gives
informed consent to the disclosure. See the Guidance to Rule rC21 at
gC69~for an example of circumstances where it may be appropriate for you
to seek such consent.

\subsubsection{\color{darkgrey}gC44}

There may be circumstances when your duty of confidentiality to your
\emph{client} conflicts with your duty to the \emph{court}. Rule rC4 and
Guidance gC8 and gC11 to gC13 provide further information.

\subsubsection{\color{darkgrey}gC45}

Similarly, there may be circumstances when your duty of confidentiality
to your \emph{client} conflicts with your duty to your regulator. Rule
rC64 and Guidance gC92 in respect of that rule provide further
information. In addition, Rule rC66 may also apply.

\subsubsection{\color{darkgrey}gC46}

If you are a \emph{pupil} of, or are \emph{devilling} work for, a
\emph{self-employed barrister}, Rule rC15(5) applies to you as if the
\emph{client} of the \emph{self-employed barrister} was your own
\emph{client}.

\subsubsection{\color{darkgrey}gC47}

The section You and Your Practice, at 2(c)5, provides for duties
regarding the systems and procedures You \textcolor{myred}{\textbf{must}} put in place and enforce
in order to ensure compliance with Rule rC15(5).

\subsubsection{\color{darkgrey}gC48}

If you are an \emph{authorised individual} or a \emph{manager} working
in a \emph{BSB entity} your personal duty to act in the best interests
of your \emph{client} requires you to assist in the redistribution of
\emph{client} files and otherwise assisting to ensure each
\emph{client's} interests are protected in the event that the \emph{BSB
entity} itself is unable to do so for whatever reason (for example,
insolvency).

\rulesection{Rule
C17}

\rulesubsection{rc17}

Your duty to act in the best interests of each \emph{client} (\textcolor{mygold}{\textbf{CD2}})
includes a duty to consider whether the \emph{client's} best interests
are served by different legal representation, and if so, to advise the
\emph{client} to that effect.


\guidancesection{Guidance
to Rule C17 }


\subsubsection{\color{darkgrey}gC49}

Your duty to comply with Rule rC17 may require you to advise your
\emph{client} that in their best interests they should be represented
by:
\begin{numlist}
\item a different advocate or legal representative, whether more senior or
more junior than you, or with different experience from yours;

\item more than one advocate or legal representative;

\item fewer advocates or legal representatives than have been instructed;
or

\item in the case where you are acting through a \emph{professional
client}, different \emph{solicitors}.
\end{numlist}
\subsubsection{\color{darkgrey}gC50}

Specific rules apply where you are acting on a public access basis,
which oblige you to consider whether \emph{solicitors} should also be
instructed. As to these see the public access rules at Section 2(d)2 and
further in respect of \emph{BSB entities} Rule S28 and the associated
guidance.

\subsubsection{\color{darkgrey}gC51}

CD2 and Rules rC15(5) and rC17 require you, subject to Rule rC16, to put
your \emph{client's} interests ahead of your own and those of any other
person. If you consider that your \emph{professional client}, another
\emph{solicitor} or \emph{intermediary}, another \emph{barrister}, or
any other person acting on behalf of your \emph{client} has been
negligent, you should ensure that your \emph{client} is advised of this.

\rulesection{Rule
C18 }


\rulesubsection{rc18}

Your duty to provide a competent standard of work and service to each
\emph{client} ( \textbf{\textcolor{mygold}{CD7}}) includes a duty to inform your \emph{professional
client}, or your \emph{client} if instructed by a \emph{client}, as far
as reasonably possible in sufficient time to enable appropriate steps to
be taken to protect the \emph{client's} interests, if:
\begin{numlist}\item it becomes apparent to you that you will not be able to carry out the
\emph{instructions} within the time requested, or within a reasonable
time after receipt of \emph{instructions}; or
\item there is an appreciable risk that you \textcolor{myred}{\textbf{may not}} be able to undertake
the \emph{instructions}.
\end{numlist}
\guidancesection{Guidance
to Rule C18 }


\subsubsection{\color{darkgrey}gC52}

For further information about what you should do in the event that you
have a clash of listings, please refer to our guidance which can be
accessed on the \emph{Bar Standards Board's} website.

\rulesection{Rule
C19 - Not misleading clients and potential clients }


\rulesubsection{rc19}

If you supply, or offer to supply, \emph{legal services}, You \textcolor{myred}{\textbf{must not}}
mislead, or cause or permit to be misled, any person to whom you supply,
or offer to supply, \emph{legal services} about:
\begin{numlist}\item the nature and scope of the \emph{legal services} which you are
offering or agreeing to supply;
\item the terms on which the \emph{legal services} will be supplied, who
will carry out the work and the basis of charging;
\item who is legally responsible for the provision of the services;
\item whether you are entitled to supply those services and the extent to
which you are regulated when providing those services and by whom; or
\item the extent to which you are covered by insurance against claims for
professional negligence.
\end{numlist}
\guidancesection{Guidance
to Rule C19 }


\subsubsection{\color{darkgrey}gC53}

The best interests of \emph{clients} (\textcolor{mygold}{\textbf{CD2}}) and public confidence in the
profession ( \textbf{\textcolor{mygold}{CD5}}) are undermined if there is a lack of clarity as to
whether services are regulated, who is supplying them, on what terms,
and what redress \emph{clients} have and against whom if things go
wrong. Rule rC19 may potentially be infringed in a broad variety of
situations. You \textcolor{myred}{\textbf{must}} consider how matters will appear to the
\emph{client}.

\subsubsection{\color{darkgrey}gC54}

\emph{Clients} may, by way of example, be misled if \emph{self-employed
barristers} were to share premises with \emph{solicitors} or other
professionals without making sufficiently clear to \emph{clients} that
they remain separate and independent from one another and are not
responsible for one another's work.

\subsubsection{\color{darkgrey}gC55}

Likewise, it is likely to be necessary to make clear to \emph{clients}
that any entity established as a ``ProcureCo'' is not itself able to
supply \emph{reserved legal activities} and is not subject to regulation
by the \emph{Bar Standards Board}.

\subsubsection{\color{darkgrey}gC56}

A set of \emph{chambers} dealing directly with unsophisticated lay
\emph{clients} might breach Rule rC19 if its branding created the
appearance of an entity or \emph{partnership} and it failed to explain
that the members of \emph{chambers} are, in fact, self-employed
individuals who are not responsible for one another's work.

\subsubsection{\color{darkgrey}gC57}

Knowingly or recklessly publishing advertising material which is
inaccurate or likely to mislead could also result in you being in breach
of Rule rC19. You should be particularly careful about making
comparisons with other persons as these may often be regarded as
misleading.

\subsubsection{\color{darkgrey}gC58}

If you carry out public access work but are not authorised to
\emph{conduct litigation}, you would breach Rule rC19 if you caused or
permitted your \emph{client} to be misled into believing that you are
entitled to, or will, provide services that include the \emph{conduct of
litigation} on behalf of your \emph{client}.

\subsubsection{\color{darkgrey}gC59}

If you are a \emph{self-employed barrister}, you would, for example,
likely be regarded as having breached Rule rC19 if you charged at your
own hourly rate for work done by a \emph{devil} or \emph{pupil}.
Moreover, such conduct may well breach your duty to act with honesty and
also with integrity ( \textbf{\textcolor{mygold}{CD3}}).

\subsubsection{\color{darkgrey}gC60}

If you are an \emph{unregistered barrister}, you would breach Rule rC19
if you misled your \emph{client} into thinking that you were providing
\emph{legal services} to them as a \emph{barrister} or that you were
subject to the same regulation as a \emph{practising barrister}. You
would also breach the rule if you implied that you were covered by
insurance if you were not, or if you suggested that your \emph{clients}
could seek a remedy from the \emph{Bar Standards Board} or the
\emph{Legal Ombudsman} if they were dissatisfied with the services you
provided. You should also be aware of the rules set out in Section D4~of
this Code of Conduct and the additional guidance for \emph{unregistered
barristers} available on the \emph{Bar Standards Board} website.

\subsubsection{\color{darkgrey}gC61}

Rule C19(3) is particularly relevant where you act in more than one
capacity, for example as a \emph{BSB authorised individual} as well as a
manager or employee of an \emph{authorised (non BSB) body}. This is
because you should make it clear to each \emph{client} in what capacity
you are acting and, therefore, who has legal responsibility for the
provision of the services.

\subsubsection{\color{darkgrey}gC62}

If you are a \emph{pupil}, you should not hold yourself out as a member
of \emph{chambers} or permit your name to appear as such. You should
ensure the \emph{client} understands your status.

\subsubsection{\color{darkgrey}gC63}

A number of other rules impose positive obligations on you, in
particular circumstances, to make clear your regulatory status and the
basis and terms on which you are acting. See, for example, Rule rC23 and
guidance gC74.

\rulesection{Rule
C20 - Personal responsibility }


\rulesubsection{rc20}

Where you are a \emph{BSB authorised individual}, you are personally
responsible for your own conduct and for your professional work. You
must use your own professional judgment in relation to those matters on
which you are instructed and be able to justify your decisions and
actions. You \textcolor{myred}{\textbf{must}} do this notwithstanding the views of your
\emph{client}, \emph{professional client}, \emph{employer} or any other
person.


\guidancesection{Guidance
to Rule C20 }


\subsubsection{\color{darkgrey}gC64}

It is fundamental that \emph{BSB authorised individuals} and
\emph{authorised (non-BSB) individuals} are personally responsible for
their own conduct and for their own professional work, whether they are
acting in a self-employed or employed capacity (in the case of \emph{BSB
authorised individuals}) or as an employee or \emph{manager} of a
\emph{BSB entity} (in the case of \emph{authorised (non-BSB)
individuals}).

\subsubsection{\color{darkgrey}gC65}

Nothing in Rule rC20 is intended to prevent you from delegating or
outsourcing to any other person discrete tasks (for example, research)
which such other person is well-equipped to provide. However, where such
tasks are delegated or outsourced, you remain personally responsible for
such work. Further, in circumstances where such tasks are being
outsourced, Rule rC86 which deals with outsourcing, \textcolor{myred}{\textbf{must}} be complied
with.

\subsubsection{\color{darkgrey}gC66}

You are responsible for the service provided by all those who represent
you in your dealings with your \emph{client}, including your clerks or
any other employees or agents.

\subsubsection{\color{darkgrey}gC67}

Nothing in this rule or guidance prevents a \emph{BSB entity} from
contracting on the basis that any civil liability for the services
provided by a \emph{BSB regulated individual} lies with the \emph{BSB
entity} and the \emph{BSB regulated individual} is not to be liable.
However, any such stipulation as to civil liability \textcolor{myred}{\textbf{does not}} affect the
regulatory obligations of the \emph{BSB regulated individual} including
(but not limited to) that of being personally responsible under Rule
rC20 for the professional judgments made.

\subsubsection{\color{darkgrey}gC68}

See, further, guidance to Rule rC19, as regards work by \emph{pupils}
and \emph{devils} Rule rC15, gC124 and Rule rC86 (on outsourcing).

\rulesection{Rule
C21 - Accepting instructions }


\rulesubsection{rc21}

You \textcolor{myred}{\textbf{must not}} accept \emph{instructions} to act in a particular matter
if:
\begin{enumerate}[label=(\arabic*)]
\item due to any existing or previous \emph{instructions} you are not able
to fulfil your obligation to act in the best interests of the
prospective \emph{client}; or

\item there is a conflict of interest, or real risk of conflict of
interest, between your own personal interests and the interests of the
prospective \emph{client} in respect of the particular matter; or

\item there is a conflict of interest, or real risk of conflict of
interest, between the prospective \emph{client} and one or more of your
former or existing \emph{clients} in respect of the particular matter
unless all of the \emph{clients} who have an interest in the particular
matter give their informed consent to your acting in such circumstances;
or

\item there is a real risk that information confidential to another former
or existing \emph{client}, or any other person to whom you owe duties of
confidence, may be relevant to the matter, such that if, obliged to
maintain confidentiality, you could not act in the best interests of the
prospective \emph{client}, and the former or existing \emph{client} or
person to whom you owe that duty \textcolor{myred}{\textbf{does not}} give informed consent to
disclosure of that confidential information; or

\item your instructions seek to limit your ordinary authority or discretion
in the conduct of proceedings in \emph{court}; or

\item your instructions require you to act other than in accordance with
law or with the provisions of this \emph{Handbook}; or

\item you are not authorised and/or otherwise accredited to perform the
work required by the relevant \emph{instruction}; or

\item you are not competent to handle the particular matter or otherwise do
not have enough experience to handle the matter; or

\item you do not have enough time to deal with the particular matter,
unless the circumstances are such that it would nevertheless be in the
\emph{client's} best interests for you to accept; or

\item there is a real prospect that you are not going to be able to
maintain your independence.
\end{enumerate}


\guidancesection{Guidance
to Rule C21 }


\subsubsection{\color{darkgrey}gC69}

Rules rC21(2), rC21(3) and rC21(4) are intended to reflect the law on
conflict of interests and confidentiality and what is required of you by
your duty to act in the \emph{client's} best interests (\textcolor{mygold}{\textbf{CD2}}),
independently ( \textbf{\textcolor{mygold}{CD4}}), and maintaining \emph{client} confidentiality
( \textbf{\textcolor{mygold}{CD6}}). You are prohibited from acting where there is a conflict of
interest between your own personal interests and the interests of a
prospective \emph{client}. However, where there is a conflict of
interest between an existing \emph{client} or \emph{clients} and a
prospective \emph{client} or \emph{clients} or two or more prospective
\emph{clients}, you may be entitled to accept instructions or to
continue to act on a particular matter where you have fully disclosed to
the relevant \emph{clients} and prospective \emph{clients} (as
appropriate) the extent and nature of the conflict; they have each
provided their informed consent to you acting; and you are able to act
in the best interests of each \emph{client} and independently as
required by \textbf{\textcolor{mygold}{CD2}} and  \textbf{\textcolor{mygold}{CD4}}.

\subsubsection{\color{darkgrey}gC70}

Examples of where you may be required to refuse to accept
\emph{instructions} in accordance with Rule rC21(7) include:
\begin{numlist}\item where the \emph{instructions} relate to the provision of litigation
services and you have not been authorised to \emph{conduct litigation}
in accordance with the requirements of this \emph{Handbook};
\item removed
\item where the matter would require you to conduct correspondence with
parties other than your \emph{client} (in the form of letters, faxes,
emails or the like), you do not have adequate systems, experience or
resources for managing appropriately such correspondence and/or you do
not have adequate insurance in place in accordance with Rule rC75 which
covers, amongst other things, any loss suffered by the \emph{client} as
a result of the conduct of such correspondence.
\end{numlist}

\subsubsection{\color{darkgrey}gC71}

Competency and experience under Rule rC21(8) includes your ability to
work with vulnerable \emph{clients}.

\subsubsection{\color{darkgrey}gC72}

Rule rC21(9) recognises that there may be exceptional circumstances when
\emph{instructions} are delivered so late that no suitable, competent
advocate would have adequate time to prepare. In those cases you~are not
required to refuse \emph{instructions} as it will be in the
\emph{client's} best interests that you accept. Indeed, if you are
obliged under the cab rank rule to accept the \emph{instructions}, you
must do so.

\subsubsection{\color{darkgrey}gC73}

Rule rC21(1)0 is an aspect of your broader obligation to maintain your
independence ( \textbf{\textcolor{mygold}{CD4}}). Your ability to perform your duty to the
\emph{court} (\textcolor{mygold}{\textbf{CD1}}) and act in the best interests of your \emph{client}
(\textcolor{mygold}{\textbf{CD2}}) may be put at risk if you act in circumstances where your
independence is compromised. Examples of when you \textcolor{myred}{\textbf{may not}} be able to
maintain your independence include appearing as an advocate in a matter
in which you are likely to be called as a witness (unless the matter on
which you are likely to be called as a witness is peripheral or minor in
the context of the litigation as a whole and is unlikely to lead to your
involvement in the matter being challenged at a later date). If it
appears that you are likely to be a witness on a material question of
fact, and therefore \textcolor{myred}{\textbf{must}} withdraw from a case as there is a real
prospect that you are not going to be able to maintain your independence
(Rules C21(1)0 and C25), You \textcolor{myred}{\textbf{must}} also comply with Rule C27.

\subsubsection{\color{darkgrey}gC74}

Where the \emph{instructions} relate to public access or licensed access
work and you are a \emph{self-employed barrister} you will also need to
have regard to the relevant rules at 2(d)2. If you are a \emph{BSB
entity}, you should have regard to the guidance to Rule S28.

\rulesection{Rules
C22-C24 - Defining terms or basis on which instructions are accepted
}


\rulesubsection{rc22}

Where you first accept \emph{instructions} to act in a matter:
\begin{numlist}\item you \textcolor{myred}{\textbf{must}}, subject to Rule rC23, confirm in writing acceptance of the
\emph{instructions} and the terms and/or basis on which you will be
acting, including the basis of charging;
\item where your instructions are from a \emph{professional client}, the
confirmation required by rC22(1) \textcolor{myred}{\textbf{must}} be sent to the \emph{professional
client};
\item where your instructions are from a \emph{client}, the confirmation
required by rC22(1) \textcolor{myred}{\textbf{must}} be sent to the \emph{client}.
\item if you are a \emph{BSB entity}, You \textcolor{myred}{\textbf{must}} ensure that the terms under
which you accept instructions from \emph{clients} include consent from
clients to disclose and give control of files to the \emph{Bar Standards
Board} or its agent in circumstances where the conditions in rS113(5) are
met.\end{numlist}


\rulesubsection{rc23}

In the event that, following your acceptance of the \emph{instructions}
in accordance with Rule rC22, the scope of the \emph{instructions} is
varied by the relevant \emph{client} (including where the \emph{client}
instructs you on additional aspects relating to the same matter), you
are not required to confirm again in writing acceptance of the
instructions or the terms and/or basis upon which you will be acting. In
these circumstances, you will be deemed to have accepted the
instructions when you begin the work, on the same terms or basis as
before, unless otherwise specified.

\rulesubsection{rc24}

You \textcolor{myred}{\textbf{must}} comply with the requirements set out in Rules rC22 and rC23
before doing the work unless that is not reasonably practicable, in
which case you should do so as soon as reasonably practicable.


\guidancesection{Guidance
to Rules C22-C24 }


\subsubsection{\color{darkgrey}gC75}

Compliance with the requirement in Rule rC22 to set out the terms and/or
basis upon which you will be acting may be achieved by including a
reference or link to the relevant terms in your written communication of
acceptance. You may, for example, refer the \emph{client} or
\emph{professional client} (as the case may be) to the terms of service
set out on your website or to standard terms of service set out on
the~\emph{Bar Council's} website (in which regard, please also refer to
the guidance on the use of the standard terms of service). Where you
agree to do your work on terms and conditions that have been proposed to
you by the \emph{client} or by the \emph{professional client}, you
should confirm in writing that that is the basis on which your work is
done. Where there are competing sets of terms and conditions, which
terms have been agreed and are the basis of your retainer will be a
matter to be determined in accordance with the law of contract.

\subsubsection{\color{darkgrey}gC76}

Your obligation under Rule rC23 is to ensure that the basis on which you
act has been defined, which \textcolor{myred}{\textbf{does not}} necessarily mean governed by your
own contractual terms. In circumstances where Rule rC23 applies, you
should take particular care to ensure that the \emph{client} is clear
about the basis for charging for any variation to the work where it may
be unclear. You \textcolor{myred}{\textbf{must}} also ensure that you comply with the requirements
of the Provision of Services Regulations 2009. See, further Rule rC19
(not misleading \emph{clients} or prospective \emph{clients}) and the
guidance to that rule at gC52 to gC62.

\subsubsection{\color{darkgrey}gC77}

If you are a \emph{self-employed barrister} a clerk may confirm on your
behalf your acceptance of \emph{instructions} in accordance with Rules
rC22 and rC23 above.

\subsubsection{\color{darkgrey}gC78}

When accepting instructions, You \textcolor{myred}{\textbf{must}} also ensure that you comply with
the complaints handling rules set out in Section 2(d).

\subsubsection{\color{darkgrey}gC79}

When accepting instructions in accordance with Rule rC22, confirmation
by email will satisfy any requirement for written acceptance.

\subsubsection{\color{darkgrey}gC80}

You may have been instructed in relation to a discrete and finite task,
such as to provide an opinion on a particular issue, or to provide
ongoing services, for example, to conduct particular litigation. Your
confirmation of acceptance of instructions under Rule rC22 should make
clear the scope of the \emph{instructions} you are accepting, whether by
cross-referring to the \emph{instructions}, where these are in writing
or by summarising your understanding of the scope of work you are
instructed to undertake.

\subsubsection{\color{darkgrey}gC81}

Disputes about costs are one of the most frequent \emph{complaints}. The
provision of clear information before work starts is the best way of
avoiding such \emph{complaints}. The \emph{Legal Ombudsman} has produced
a useful guide ``An Ombudsman's view of good costs service'' which can
be found on its website.

\subsubsection{\color{darkgrey}gC82}

Where the \emph{instructions} relate to public access or licensed access
work and you are a \emph{self-employed barrister}, you will also need to
have regard to the relevant rules at 2(d)2. If you are a \emph{BSB
entity}, you should have regard to the guidance to Rule S28.

\rulesection{Rules
C25-C26 - Returning instructions }


\rulesubsection{rc25}

Where you have accepted \emph{instructions} to act but one or more of
the circumstances set out in Rules rC21(1) to rC21(1)0 above then arises,
You \textcolor{myred}{\textbf{must}} cease to act and return your \emph{instructions} promptly. In
addition, You \textcolor{myred}{\textbf{must}} cease to act and return your \emph{instructions} if:
\begin{numlist}\item in a case funded by the \emph{Legal Aid Agency} as part of Criminal
Legal Aid or Civil Legal Aid it has become apparent to you that this
funding has been wrongly obtained by false or inaccurate information and
action to remedy the situation is not immediately taken by your
\emph{client}; or
\item the \emph{client} refuses to authorise you to make some disclosure to
the \emph{court} which your duty to the \emph{court} requires you to
make; or
\item you become aware during the course of a case of the existence of a
document which should have been but has not been disclosed, and the
\emph{client} fails to disclose it or fails to permit you to disclose
it, contrary to your advice.
\end{numlist}

\rulesubsection{rc26}

You may cease to act on a matter on which you are instructed and return
your \emph{instructions} if:
\begin{numlist}\item your professional conduct is being called into question; or
\item the \emph{client} consents; or
\item you are a \emph{self-employed barrister} and:
\begin{alphlist}\item despite all reasonable efforts to prevent it, a hearing becomes fixed
for a date on which you have already entered in your professional diary
that you will not be available; or
\item illness, injury, pregnancy, childbirth, a bereavement or a similar
matter makes you unable reasonably to perform the services required in
the \emph{instructions}; or
\item you are unavoidably required to attend on jury service;\end{alphlist}
\item you are a \emph{BSB entity} and the only appropriate \emph{authorised
individual(s)} are unable to continue acting on the particular matter
due to one or more of the grounds referred to at Rules rC26(3)(a) to
rC26(3)(c) above occurring;
\item you do not receive payment when due in accordance with terms agreed,
subject to Rule rC26(7) (if you are conducting litigation) and in any
other case subject to your giving reasonable notice requiring the
non-payment to be remedied and making it clear to the \emph{client} in
that notice that failure to remedy the non-payment may result in you
ceasing to act and returning your \emph{instructions} in respect of the
particular matter; or
\item you become aware of confidential or privileged information or
documents of another person which relate to the matter on which you are
instructed; or
\item if you are conducting litigation, and your \emph{client} \textcolor{myred}{\textbf{does not}}
consent to your ceasing to act, your application to come off the record
has been granted; or
\item there is some other substantial reason for doing so (subject to Rules
rC27 to rC29 below).
\end{numlist}


\guidancesection{Guidance
to Rule C26 }


\subsubsection{\color{darkgrey}gC83}

In deciding whether to cease to act and to return existing instructions
in accordance with Rule rC26, you should, where possible and subject to
your overriding duty to the \emph{court}, ensure that the \emph{client}
is not adversely affected because there is not enough time to engage
other adequate legal assistance.

\subsubsection{\color{darkgrey}gC84}

If you are working on a referral basis and your \emph{professional
client} withdraws, you are no longer instructed and cannot continue to
act unless appointed by the \emph{court}, or you otherwise receive new
instructions. You will not be bound by the cab rank rule if appointed by
the court. For these purposes working on a ``referral basis'' means
where a \emph{professional client} instructs a \emph{BSB authorised
individual} to provide \emph{legal services} on behalf of one of that
\emph{professional client's} own clients

\subsubsection{\color{darkgrey}gC85}

You should not rely on Rule rC26(3) to break an engagement to supply
legal services so that you can attend or fulfil a non-professional
engagement of any kind other than those indicated in Rule rC26(3).

\subsubsection{\color{darkgrey}gC86}

When considering whether or not you are required to return instructions
in accordance with Rule rC26(6) you should have regard to relevant case
law including: English \& American Insurance Co Ltd \& Others -v-
Herbert Smith; ChD 1987; (1987) NLJ 148 and Ablitt -v- Mills \& Reeve (A
Firm) and Another; ChD (Times, 24-Oct-1995).

\subsubsection{\color{darkgrey}gC87}

If a fundamental change is made to the basis of your remuneration, you
should treat such a change as though your original instructions have
been withdrawn by the \emph{client} and replaced by an offer of new
\emph{instructions} on different terms. Accordingly:
\begin{numlist}\item You \textcolor{myred}{\textbf{must}} decide whether you are obliged by Rule rC29 to accept the
new \emph{instructions};
\item if you are obliged under Rule rC29 to accept the new
\emph{instructions}, You \textcolor{myred}{\textbf{must}} do so;
\item if you are not obliged to accept the new \emph{instructions}, you may
decline them;
\item if you decline to accept the new \emph{instructions} in such
circumstances, you are not to be regarded as returning your
\emph{instructions}, nor as withdrawing from the matter, nor as ceasing
to act, for the purposes of Rules rC25 to rC26, because the previous
\emph{instructions} have been withdrawn by the \emph{client}.
\end{numlist}

\rulesection{Rule
C27 }


\rulesubsection{rc27}

Notwithstanding the provisions of Rules rC25 and rC26, You \textcolor{myred}{\textbf{must not}}:
\begin{numlist}\item cease to act or return \emph{instructions} without either:
\begin{alphlist}\item obtaining your \emph{client's} consent; or
\item clearly explaining to your \emph{client} or your \emph{professional
client} the reasons for doing so; or\end{alphlist}
\item return instructions to another person without the consent of your
\emph{client} or your \emph{professional client}.
\end{numlist}

\rulesection{Rule
C28 - Requirement not to discriminate }


\rulesubsection{rc28}

You \textcolor{myred}{\textbf{must not}} withhold your services or permit your services to be
withheld:
\begin{numlist}\item on the ground that the nature of the case is objectionable to you or
to any section of the public;
\item on the ground that the conduct, opinions or beliefs of the
prospective client are unacceptable to you or to any section of the
public;
\item on any ground relating to the source of any financial support which
may properly be given to the prospective \emph{client} for the
proceedings in question.
\end{numlist}


\guidancesection{Guidance
to Rule C28 }


\subsubsection{\color{darkgrey}gC88}

As a matter of general law you have an obligation not to discriminate
unlawfully as to those to whom you make your services available on any
of the statutorily prohibited grounds such as gender or race. See the
Equality Rules in the BSB Handbook and the BSB's website for guidance as
to your obligations in respect of equality and diversity. This rule of
conduct is concerned with a broader obligation not to withhold your
services on grounds that are inherently inconsistent with your role in
upholding access to justice and the rule of law and therefore in this
rule ``discriminate'' is used in this broader sense. This obligation
applies whether or not the \emph{client} is a member of any protected
group for the purposes of the Equality Act 2010. For example, you \textcolor{myred}{\textbf{must}}
not withhold services on the ground that any financial support which may
properly be given to the prospective \emph{client} for the proceedings
in question will be available as part of Criminal Legal Aid and Civil
Legal Aid.

\rulesection{Rules
C29-C30 - The cab rank rule }


\rulesubsection{rc29}

If you receive \emph{instructions} from a \emph{professional client},
and you are:
\begin{numlist}
\item a \emph{self-employed barrister} instructed by a \emph{professional
client}; or

\item an \emph{authorised individual} working within a \emph{BSB entity};
or

\item a \emph{BSB entity} and the \emph{instructions} seek the services of
a named \emph{authorised individual} working for you,

and the \emph{instructions} are appropriate taking into account the
experience, seniority and/or field of practice of yourself or (as
appropriate) of the named \emph{authorised individual} you \textcolor{myred}{\textbf{must}}, subject
to Rule rC30 below, accept the \emph{instructions} addressed
specifically to you, irrespective of:
\begin{alphlist}
\item the identity of the \emph{client};

\item the nature of the case to which the \emph{instructions} relate;

\item whether the \emph{client} is paying privately or is publicly funded;
and

\item any belief or opinion which you may have formed as to the character,
reputation, cause, conduct, guilt or innocence of the \emph{client}.
\end{alphlist}
\end{numlist}
\rulesubsection{rc30}

The cab rank Rule rC29 \textcolor{myred}{\textbf{does not}} apply if:
\begin{numlist}
\item you are required to refuse to accept the \emph{instructions} pursuant
to Rule rC21; or

\item accepting the \emph{instructions} would require you or the named
\emph{authorised individual} to do something other than in the course of
their ordinary working time or to cancel a commitment already in their
diary; or

\item the potential liability for professional negligence in respect of the
particular matter could exceed the level of professional indemnity
insurance which is reasonably available and likely to be available in
the market for you to accept; or

\item you are a Queen's Counsel, and the acceptance of the
\emph{instructions} would require you to act without a junior in
circumstances where you reasonably consider that the interests of the
\emph{client} require that a junior should also be instructed; or

\item accepting the \emph{instructions} would require you to do any
\emph{foreign work}; or

\item accepting the \emph{instructions} would require you to act for a
\emph{foreign lawyer} (other than a \emph{European lawyer}, a lawyer
from a country that is a member of EFTA, a \emph{solicitor} or
\emph{barrister} of Northern Ireland or a \emph{solicitor} or advocate
under the law of Scotland); or

\item the \emph{professional client}:
\begin{alphlist}
\item is not accepting liability for your fees; or

\item represents, in your reasonable opinion, an unacceptable credit risk;
or

\item is instructing you as a lay \emph{client} and not in their capacity as a \emph{professional client}; or
\end{alphlist}
\item you have not been offered a proper fee for your services (except that
you shall not be entitled to refuse to accept \emph{instructions} on
this ground if you have not made or responded to any fee proposal within
a reasonable time after receiving the \emph{instructions}); or

\item except where you are to be paid directly by 
\begin{romlist}
\item the \emph{Legal Aid
Agency} as part of the Community Legal Service or the Criminal Defence
Service or 
\item the Crown Prosecution Service:\end{romlist}
\begin{alphlist}
\item your fees have not been agreed (except that you shall not be entitled
to refuse to accept \emph{instructions} on this ground if you have not
taken reasonable steps to agree fees within a reasonable time after
receiving the \emph{instructions});

\item having required your fees to be paid before you accept the
\emph{instructions}, those fees have not been paid;

\item accepting the \emph{instructions} would require you to act other than
on 
\begin{Alphlist}
\item the Standard Contractual Terms for the Supply of Legal Services
by Barristers to Authorised Persons 2020~as published on the \emph{Bar
Council's} website; or \item if you publish standard terms of work, on
those standard terms of work.
\end{Alphlist}\end{alphlist}
\end{numlist}
\guidancesection{Guidance
to Rules C29-C30 }


\subsubsection{\color{darkgrey}gC89}

Rule rC30 means that you would not be required to accept
\emph{instructions} to, for example, \emph{conduct litigation} or attend
a police station in circumstances where you do not normally undertake
such work or, in the case of litigation, are not authorised to undertake
such work.

\subsubsection{\color{darkgrey}gC90}

In determining whether or not a fee is proper for the purposes of Rule
C30(8), regard shall be had to the following:
\begin{numlist}\item the complexity length and difficulty of the case;
\item your ability, experience and seniority; and
\item the expenses which you will incur.
\end{numlist}
\subsubsection{\color{darkgrey}gC91}

Further, you may refuse to accept instructions on the basis that the fee
is not proper if the instructions are on the basis that you will do the
work under a \emph{conditional fee agreement} or damages based
agreement.

\subsubsection{\color{darkgrey}gC91A}

Examples of when you might reasonably conclude (subject to the following
paragraph) that a \emph{professional client} represents an unacceptable
credit risk for the purposes of Rule C30(7)(b) include:
\begin{numlist}\item Where they are included on the \emph{Bar Council's} List of
Defaulting Solicitors;
\item Where to your knowledge a \emph{barrister} has obtained a judgment
against a \emph{professional client}, which remains unpaid;
\item Where a firm or sole practitioner is subject to insolvency
proceedings, an individual voluntary arrangement or partnership
voluntary arrangement; or
\item Where there is evidence of other unsatisfied judgments that
reasonably call into question the \emph{professional client's} ability
to pay your fees.\end{numlist}


Even where you consider that there is a serious credit risk, you should
not conclude that the \emph{professional client} represents an
unacceptable credit risk without first considering alternatives. This
will include considering whether the credit risk could be mitigated in
other ways, for example by seeking payment of the fee in advance or
payment into a third party payment service as permitted by rC74, rC75
and associated guidance.

\subsubsection{\color{darkgrey}gC91B}

The standard terms referred to in Rule C30(9)(c) may be drafted as if the
\emph{professional client} were an \emph{authorised person} regulated by
the Solicitors Regulation Authority (SRA). However, the cab rank rule
applies (subject to the various exceptions in Rule C30) to instructions
from any \emph{professional client}, therefore you may be instructed
under the cab rank rule by \emph{authorised persons} who are regulated
by another \emph{approved regulator}.

The BSB expects all \emph{authorised persons} to be able to access the
cab rank rule on behalf of their \emph{clients} in the same way.
Therefore, if you are instructed by an \emph{authorised person} who is
not regulated by the SRA, you are obliged to act on the same terms. You
should therefore apply the standard terms referred to in Rule C30(9)(c) as
if \begin{romlist}
\item the definition of \emph{professional client} includes that
\emph{authorised person}, and \item any references to the SRA or its
regulatory arrangements~are references to that person's \emph{approved
regulator} and its regulatory arrangements.\end{romlist}

\rulesection{Rules
C31-C63 }


Removed.

\chap{Part 2
- C4. You and your regulator }

\ocsection{Outcomes
C21-C23 }


\subsection{\color{bleu}oC21}

\emph{BSB regulated persons} are effectively regulated.

\subsection{\color{bleu}oC22}

The public have confidence in the proper regulation of \emph{persons}
regulated by the \emph{Bar Standards Board}.

\subsection{\color{bleu}oC23}

The \emph{Bar Standards Board} has the information that it needs in
order to be able to assess risks and regulate effectively and in
accordance with the \emph{regulatory objectives}.

\rulesection{Rule
C64 - Provision of information to the Bar Standards Board }


\rulesubsection{rc64}

You \textcolor{myred}{\textbf{must}}:
\begin{numlist}\item promptly provide all such information to the \emph{Bar Standards
Board} as it may, for the purpose of its regulatory functions, from time
to time require of you, and notify it of any material changes to that
information;
\item comply in due time with any decision or sentence imposed by the
\emph{Bar Standards Board}, a \emph{Disciplinary Tribunal}, the High
Court, the \emph{First Tier Tribunal}, an \emph{interim panel}, a
\emph{review panel}, an \emph{appeal panel}, an \emph{Independent
Decision-Making Panel}~or a \emph{Fitness to Practise Panel};
\item if you are a \emph{BSB entity} or an \emph{owner} or \emph{manager}
of a \emph{BSB entity} and the conditions outlined in rS113(5) apply,
give the \emph{Bar Standards Board} whatever co-operation is necessary:
\item comply with any notice sent by the \emph{Bar Standards Board} or its
agent; and
\item register within 28 days if you undertake work in the Youth Court if
you did not register when applying for a practising certificate.
\end{numlist}


\guidancesection{Guidance
to Rule C64 }


\subsubsection{\color{darkgrey}gC92}

Your obligations under Rule rC64 include, for example, responding
promptly to any request from the \emph{Bar Standards Board} for comments
or information relating to any matter whether or not the matter relates
to you, or to another \emph{BSB regulated person} or \emph{unregistered
barrister}.

\subsubsection{\color{darkgrey}gC92A}

A notice under rC64(4) refers to a notice under any part of the Legal
Services Act 2007, or the Legal Services Act 2007 (General Council of
the Bar) (Modification of Functions) Order 2018.

\subsubsection{\color{darkgrey}gC93}

The documents that you are required to disclose pursuant to Rule C64 may
include client information that is subject to legal professional
privilege. Pursuant to R (Morgan Grenfell \& Co Ltd) v Special
Commissioner {[}2003{]} 1 A(c). 563, referred to in R (Lumsdon) v Legal
Services Board {[}2014{]} EWHC 28 (Admin) at {[}73{]}, the BSB is
entitled to serve you with a notice for production of those documents.

Where you are being required to report serious misconduct by others and
legal professional privilege applies, this will override the requirement
to report serious misconduct by another. However, the BSB may
subsequently serve you with a notice for production of documents in
which case the same principles set out above apply.

For the avoidance of doubt, none of this casts any doubt on your
entitlement to withhold from the BSB any material that is subject to
your own legal privilege (such as legal advice given to you about your
own position during a BSB investigation).

\rulesection{Rule
C65 - Duty to report certain matters to the Bar Standards Board }


\rulesubsection{rc65}

You \textcolor{myred}{\textbf{must}} report promptly to the \emph{Bar Standards Board} if:
\begin{numlist}\item you are charged with an \emph{indictable offence}; in the
jurisdiction of England and Wales or with a \emph{criminal offence} of
comparable seriousness in any other jurisdiction;
\item subject to the Rehabilitation of Offenders Act 1974 (as amended) you
are convicted of, or accept a caution, for any \emph{criminal offence},
in any jurisdiction, other than a \emph{minor criminal offence};
\item you (or an entity of which you are a \emph{manager}) to your
knowledge are the subject of any disciplinary or other regulatory or
enforcement action by another \emph{Approved Regulator} or other
regulator, including being the subject of disciplinary proceedings;
\item you are a \emph{manager} of a \emph{regulated entity} (other than a
\emph{BSB entity}) which is the subject of an intervention by the
\emph{approved regulator} of that body;
\item you are a \emph{registered European lawyer} and:
\begin{alphlist}\item to your knowledge any investigation into your conduct is commenced by
your home regulator; or
\item any finding of professional misconduct is made by your home
regulator; or
\item your authorisation in your \emph{home state} to pursue professional
activities under your \emph{home professional title} is withdrawn or
\emph{suspended}; or
\item you are charged with a disciplinary offence.\end{alphlist}
\item any of the following occur:
\begin{alphlist}\item bankruptcy proceedings are initiated in respect of or against you;
\item \emph{director's disqualification} proceedings are initiated against
you;
\item a \emph{bankruptcy order} or \emph{director's disqualification order}
is made against you;
\item you have made a composition or arrangement with, or granted a trust
deed for, your creditors;
\item winding up proceedings are initiated in respect of or against you;

\item you have had an administrator, administrative receiver, receiver or
liquidator appointed in respect of you;

\item administration proceedings are initiated in respect of or against
you;\end{alphlist}
\item you have committed serious misconduct;
\item you become authorised to \emph{practise} by another \emph{approved
regulator}.
\end{numlist}


\guidancesection{Guidance
to Rule C65 }


\subsubsection{\color{darkgrey}gC94}

In circumstances where you have committed serious misconduct you should
take all reasonable steps to mitigate the effects of such serious
misconduct.

\subsubsection{\color{darkgrey}gC94(1)}

For the avoidance of doubt rC65(2) \textcolor{myred}{\textbf{does not}} oblige you to disclose
cautions or criminal convictions that are ``spent'' under the
Rehabilitation of Offenders Act 1974 unless the Rehabilitation of
Offenders Act 1974 (Exceptions) Order 1975 (SI 1975/1023) applies.
However, unless the caution or conviction is immediately spent, you \textcolor{myred}{\textbf{must}}
notify the BSB before it becomes spent.

\rulesection{Rules
C66-C69 - Reporting serious misconduct by others }


\rulesubsection{rc66}

Subject to your duty to keep the affairs of each \emph{client}
confidential and subject also to Rules rC67 and rC68, You \textcolor{myred}{\textbf{must}} report to
the \emph{Bar Standards Board} if you have reasonable grounds to believe
that there has been serious misconduct by a \emph{barrister} or a
\emph{registered European lawyer}, a \emph{BSB entity}, \emph{manager}
of a \emph{BSB entity} or an \emph{authorised (non-BSB) individual} who
is working as a \emph{manager} or an employee of a \emph{BSB entity}.

\rulesubsection{rc67}

You \textcolor{myred}{\textbf{must}} never make, or threaten to make, a report under Rule rC66
without a genuine and reasonably held belief that Rule rC66 applies.

\rulesubsection{rc68}

You are not under a duty to report serious misconduct by others if:
\begin{numlist}\item you become aware of the facts giving rise to the belief that there
has serious misconduct from matters that are in the public domain and
the circumstances are such that you reasonably consider it likely that
the facts will have come to the attention of the \emph{Bar Standards
Board}; or
\item you are aware that the person that committed the serious misconduct
has already reported the serious misconduct to the Bar Standards Board;
or
\item the information or documents which led to you becoming aware of that
other person's serious misconduct are subject to legal professional
privilege; or
\item you become aware of such serious misconduct as a result of your work
on a \emph{Bar Council} advice line.
\end{numlist}

\rulesubsection{rc69}

You \textcolor{myred}{\textbf{must not}} victimise anyone for making in good faith a report under
Rule C66.

\guidancesection{Guidance
to Rules C65(7)-C68 }


\subsubsection{\color{darkgrey}gC95}

It is in the public interest that the \emph{Bar Standards Board}, as an
\emph{Approved Regulator}, is made aware of, and is able to investigate,
potential instances of serious misconduct. The purpose of Rules rC65(7)
to rC69, therefore, is to assist the \emph{Bar Standards Board} in
undertaking this regulatory function.

\subsubsection{\color{darkgrey}gC96}

Serious misconduct includes, without being limited to:
\begin{numlist}\item dishonesty ( \textbf{\textcolor{mygold}{CD3}});
\item assault or harassment ( \textbf{\textcolor{mygold}{CD3}} and/or  \textbf{\textcolor{mygold}{CD5}} and/or  \textbf{\textcolor{mygold}{CD8}});
\item seeking to gain access without consent to \emph{instructions} or
other confidential information relating to the opposing party's case
( \textbf{\textcolor{mygold}{CD3}} and/or  \textbf{\textcolor{mygold}{CD5}}); or
\item seeking to gain access without consent to confidential information
relating to another member of \emph{chambers}, member of staff or
\emph{pupil} ( \textbf{\textcolor{mygold}{CD3}} and/or  \textbf{\textcolor{mygold}{CD5}});
\item encouraging a witness to give evidence which is untruthful or
misleading (CD1 and/or  \textbf{\textcolor{mygold}{CD3}});
\item knowingly or recklessly misleading, or attempting to mislead, the
\emph{court} or an opponent (CD1 and/ or  \textbf{\textcolor{mygold}{CD3}}); or
\item being drunk or under the influence of drugs in \emph{court} (CD2
and/or  \textbf{\textcolor{mygold}{CD7}}); or
\item failure to report promptly to the \emph{Bar Standards Board} pursuant
to rC65(1)-rC65(5) and/or rC66 above or if;
\begin{enumerate}[label=•]
\item \emph{director's disqualification} proceedings are initiated against
you;

\item a \emph{director's disqualification} order is made against you;

\item winding up proceedings are initiated in respect of or against you;

\item you have had an administrator, administrative receiver, receiver or
liquidator appointed in respect of you;

\item administration proceedings are initiated in respect of or against you;

\end{enumerate}
\item a breach of rC67 above; for example, reporting, or threatening to
report, another \emph{person} as a litigation tactic or otherwise
abusively; or merely to please a \emph{client} or any other
\emph{person} or otherwise for an improper motive;~
\item conduct that poses a serious risk to the public.
\end{numlist}
\subsubsection{\color{darkgrey}gC97}

If you believe (or suspect) that there has been serious misconduct, then
the first step is to carefully consider all of the circumstances. The
circumstances include:
\begin{numlist}\item whether that person's \emph{instructions} or other confidential
matters might have a bearing on the assessment of their conduct;
\item whether that person has been offered an opportunity to explain their
conduct, and if not, why not;
\item any explanation which has been or could be offered for that person's
conduct;
\item whether the matter has been raised, or will be raised, in the
litigation in which it occurred, and if not, why not.
\end{numlist}
\subsubsection{\color{darkgrey}gC98}

Having considered all of the circumstances, the duty to report arises if
you have reasonable grounds to believe there has been serious
misconduct. This will be so where, having given due consideration to the
circumstances, including the matters identified at Guidance gC97, you
have material before you which as it stands establishes a reasonably
credible case of serious misconduct. Your duty under Rule rC66 is then
to report the potential instance of serious misconduct so that the
\emph{Bar Standards Board} can investigate whether or not there has in
fact been misconduct.

\subsubsection{\color{darkgrey}gC99}

Circumstances which may give rise to the exception from the general
requirement to report serious misconduct set out in Rule rC68(1) include
for example where misconduct has been widely reported~in the national
media. In these circumstances it would not be in the public interest for
every \emph{BSB regulated person} and \emph{unregistered barrister} to
have an obligation to report such serious misconduct.

\subsubsection{\color{darkgrey}gC100}

In Rule rC68(4) ``work on the \emph{Bar Council} advice line'' means:
\begin{numlist}
\item dealing with queries from \emph{BSB regulated persons} and
\emph{unregistered barristers} who contact an advice line operated by
the \emph{Bar Council} for the purposes of providing advice to those
persons; and

\item either providing advice to \emph{BSB regulated persons} and
\emph{unregistered barristers} in the course of working for an advice
line or to any individual working for an advice line where \begin{romlist}\item you are
identified on the list of \emph{BSB regulated persons} maintained by the
\emph{Bar Council} as being permitted to provide such advice (the
``approved list''); and \item the advice which you are being asked to
provide to the individual working for an advice line arises from a query
which originated from their work for that service; and\end{romlist}

\item providing advice to \emph{BSB regulated persons} and
\emph{unregistered barristers} where any individual working for an
advice line arranges for you to give such advice and you are on the
approved list.

\item for the purposes of Rule C68, the relevant advice lines are:
\begin{enumerate}[label=--]
	\item  the Ethical Queries Helpline;

\item the Equality and Diversity Helpline;

\item the Remuneration Helpline; and

\item the Pupillage Helpline.
\end{enumerate}
\end{numlist}
\subsubsection{\color{darkgrey}gC101}

Rule rC68(4) has been carved out of the general requirement to report
serious misconduct of others because it is not in the public interest
that the duty to report misconduct should constrain \emph{BSB regulated
persons} or \emph{unregistered barristers} appointed by or on behalf of
the \emph{Bar Council} to offer ethical advice to others from doing so
or inhibit \emph{BSB regulated persons} or \emph{unregistered
barristers} needing advice from seeking it. Consequently, \emph{BSB
regulated persons} or \emph{unregistered barristers} appointed by or on
behalf of the \emph{Bar Council} to offer ethical advice to \emph{BSB
regulated persons} or \emph{unregistered barristers} through a specified
advice service will not be under a duty to report information received
by them in confidence from persons seeking such advice, subject only to
the requirements of the general law. However, in circumstances where
Rule C68(4) applies, the relevant \emph{BSB regulated person} or
\emph{unregistered barrister} will still be expected to encourage the
relevant \emph{BSB regulated person} or \emph{unregistered barrister}
who has committed serious misconduct to disclose such serious misconduct
to the \emph{Bar Standards Board} in accordance with Rule C65(7).

\subsubsection{\color{darkgrey}gC102}

Misconduct which falls short of serious misconduct should, where
applicable, be reported to your HOLP so that they can keep a record of
non-compliance in accordance with Rule rC96(4).

\rulesection{Rule
C70 - Access to premises }


\rulesubsection{rc70}

You \textcolor{myred}{\textbf{must}} permit the \emph{Bar Standards Board}, or any person appointed
by them, reasonable access, on request, to inspect:
\begin{numlist}
\item any premises from which you provide, or are believed to provide,
\emph{legal services}; and

\item any documents or records relating to those premises and your
\emph{practice}, or \emph{BSB entity},~and the \emph{Bar Standards
Board}, or any person appointed by them, shall be entitled to take
copies of such documents or records as may be required by them for the
purposes of their functions.
\end{numlist}
\rulesection{Rule
C71 - Co-operation with the Legal Ombudsman }


\rulesubsection{rc71}

You \textcolor{myred}{\textbf{must}} give the \emph{Legal Ombudsman} all reasonable assistance
requested of you, in connection with the investigation, consideration,
and determination, of \emph{complaints} made under the Ombudsman scheme.

\rulesection{Rule
C72 - Ceasing to practise }


\rulesubsection{rc72}

Once you are aware that you (if you are a \emph{self-employed barrister}
or a \emph{BSB entity}) or the \emph{BSB entity} within which you work
(if you are an authorised individual or \emph{manager} of such \emph{BSB
entity}) will cease to practise, you shall effect the orderly wind-down
of activities, including:
\begin{numlist}
\item informing the \emph{Bar Standards Board} and providing them with a
contact address;

\item notifying those \emph{clients} for whom you have current matters and
liaising with them in respect of the arrangements that they would like
to be put in place in respect of those matters;

\item providing such information to the \emph{Bar Standards Board} in
respect of your practice and your proposed arrangements in respect of
the winding down of your activities as the \emph{Bar Standards Board}
may require.
\end{numlist}
\chap{Part 2
- C5. You and your practice }



\ocsection{Outcomes
C24-C25 }


\subsection{\color{bleu}oC24}

Your \emph{practice} is run competently in a way that achieves
compliance with the Core Duties and your other obligations under this
\emph{Handbook}. Your employees, \emph{pupils} and trainees understand,
and do, what is required of them in order that you meet your obligations
under this \emph{Handbook}.

\subsection{\color{bleu}oC25}

\emph{Clients} are clear about the extent to which your services are
regulated and by whom, and who is responsible for providing those
services.

\rulesection{Rules
C73-C75 - Client money }


\rulesubsection{rc73}

Except where you are acting in your capacity as a \emph{manager} or
employee of an \emph{authorised (non-BSB) body}, You \textcolor{myred}{\textbf{must not}} receive,
control or handle \emph{client money} apart from what the client pays
you for your services.

\rulesubsection{rc74}

If you make use of a third party payment service for making payments to
or from or on behalf of your \emph{client} you \textcolor{myred}{\textbf{must}}:
\begin{numlist}
\item Ensure that the service you use will not result in your receiving,
controlling or handling \emph{client money}; and

\item Only use the service for payments to or from or on behalf of your
\emph{client} that are made in respect of legal services, such as fees,
disbursements or settlement monies; and

\item Take reasonable steps to check that making use of the service is
consistent with your duty to act competently and in your \emph{client's}
best interests.
\end{numlist}
\rulesubsection{rc75}

The \emph{Bar Standards Board} may give notice under this rule that
(effective from the date of that notice) you may only use third party
payment services approved by the \emph{Bar Standards Board} or which
satisfy criteria set by the \emph{Bar Standards Board}


\guidancesection{Guidance
to Rules C73-C74 }


\subsubsection{\color{darkgrey}gC103}

The prohibition in Rule rC73 applies to you and to anyone acting on your
behalf, including any ``ProcureCo'' being a company established as a
vehicle to enable the provision of \emph{legal services} but \textcolor{myred}{\textbf{does not}} in
itself supply or provide those \emph{legal services}. Rule rC73
prohibits you from holding \emph{client money} or other \emph{client}
assets yourself, or through any agent, third party or nominee.

\subsubsection{\color{darkgrey}gC104}

Receiving, controlling or handling \emph{client money} includes entering
into any arrangement which gives you de facto control over the use
and/or destination of funds provided by or for the benefit of your
\emph{client} or intended by another party to be transmitted to your
\emph{client}, whether or not those funds are beneficially owned by your
client and whether or not held in an account of yours.

\subsubsection{\color{darkgrey}gC105}

The circumstances in which you will have de facto control within the
meaning of Rule rC73 include when you can cause money to be transferred
from a balance standing to the credit of your \emph{client} without that
\emph{client's} consent to such a withdrawal. For large withdrawals,
explicit consent should usually be required. However, the
\emph{client's} consent may be deemed to be given if:
\begin{numlist}\item the \emph{client} has given informed consent to an arrangement which
enables withdrawals to be made after the \emph{client} has received an
invoice; and
\item the \emph{client} has not objected to the withdrawal within a
pre-agreed reasonable period (which should not normally be less than one
week from receipt of the invoice).
\end{numlist}

\subsubsection{\color{darkgrey}gC106}

A fixed fee paid in advance is not \emph{client money} for the purposes
of Rule rC73.

\subsubsection{\color{darkgrey}gC107}

If you have decided in principle to take a particular case you may
request an `upfront' fixed fee from your prospective \emph{client}
before finally agreeing to work on their behalf. This should only be
done having regard to the following principles:

\begin{dotlist}
 \item You should take care to estimate accurately the likely time commitment
and only take payment when you are satisfied that:
\begin{dashlist}
\item it is a reasonable payment for the work being done; and

\item in the case of public access work, that it is suitable for you to
undertake.
\end{dashlist}
\item If the amount of work required is unclear, you should consider staged
payments rather than a fixed fee in advance.

\item You should never accept an upfront fee in advance of considering
whether it is appropriate for you to take the case and considering
whether you will be able to undertake the work within a reasonable
timescale.

\item If the \emph{client} can reasonably be expected to understand such an
arrangement, you may agree that when the work has been done, you will
pay the \emph{client} any difference between that fixed fee and (if
lower) the fee which has actually been earned based on the time spent,
provided that it is clear that you will not hold the difference between
the fixed fee and the fee which has been earned on trust for the
\emph{client}. That difference will not be \emph{client money} if you
can demonstrate that this was expressly agreed in writing, on clear
terms understood by the \emph{client}, and before payment of the fixed
fee. You should also consider carefully whether such an arrangement is
in the \emph{client's} interest, taking into account the nature of the
instructions, the \emph{client} and whether the \emph{client} fully
understands the implications. Any abuse of an agreement to pay a fixed
fee subject to reimbursement, the effect of which is that you receive
more money than is reasonable for the case at the outset, will be
considered to be holding \emph{client money} and a breach of rC73. For
this reason, you should take extreme care if contracting with a
\emph{client} in this way.

\item In any case, rC22 requires you to confirm in writing the acceptance of
any instructions and the terms or basis on which you are acting,
including the basis of charging.
	
\end{dotlist}

\subsubsection{\color{darkgrey}gC108}

Acting in the following ways may demonstrate compliance with Rules rC73,
rC74 and rC75:

\subsubsection{\color{darkgrey}gC109}

Checking that any third party payment service you may use is not
structured in such a way that the service provider is holding, as your
agent, money to which the \emph{client} is beneficially entitled. If
this is so you will be in breach of Rule rC73.

\subsubsection{\color{darkgrey}gC110}

Considering whether your \emph{client} will be safe in using the third
party payment service as a means of transmitting or receiving funds. The
steps you should take in order to satisfy yourself will depend on what
would be expected in all the circumstances of a reasonably competent
legal adviser acting in their \emph{client's} best interests. However,
you are unlikely to demonstrate that you have acted competently and in
your \emph{client's} best interests if you have not:
\begin{numlist}\item ensured that the payment service is authorised or regulated as a
payment service by the Financial Conduct Authority (FCA) and taken
reasonable steps to satisfy yourself that it is in good standing with
the FCA;
\item if the payment service is classified as a small payment institution,
ensured that it has arrangements to safeguard \emph{clients'} funds or
adequate insurance arrangements;
\item ensured that the payment service segregates \emph{client money} from
its own funds;
\item satisfied yourself that the terms of the service are such as to
ensure that any money paid in by or on behalf of the \emph{client} can
only be paid out with the \emph{client's} consent;
\item informed your \emph{client} that moneys held by the payment service
provider are not covered by the Financial Services Compensation Scheme.
\end{numlist}
\subsubsection{\color{darkgrey}gC111}

Unless you are reasonably satisfied that it is safe for your client to
use the third party payment service (see rC74(3), gC109 and gC110 above),
advising your \emph{client} against using the third party payment
service and not making use of it yourself.

\subsubsection{\color{darkgrey}gC112}

The \emph{Bar Standards Board} has not yet given notice under rule rC75.

\rulesection{Rules
C76-C78 - Insurance }


\rulesubsection{rc76}

You \textcolor{myred}{\textbf{must}}:
\begin{numlist}
\item ensure that you have adequate insurance (taking into account the
nature of your practice) which covers all the \emph{legal services} you
supply to the public; and

\item if you are a \emph{BSB authorised person} or a \emph{manager} of a
\emph{BSB entity} then in the event that the \emph{Bar Standards Board},
by any notice it may from time to time issue under this Rule C76,
stipulates a minimum level of insurance and/or minimum terms for the
insurance which \textcolor{myred}{\textbf{must}} be taken out by \emph{BSB authorised persons}, you
must ensure that you have or put in place within the time specified in
such notice, insurance meeting such requirements as apply to you.
\end{numlist}
\rulesubsection{rc77}

Where you are acting as a \emph{self-employed barrister}, You \textcolor{myred}{\textbf{must}} be a
member of \emph{BMIF}, unless:

\begin{numlist}\item you are a \emph{pupil} who is covered by your \emph{pupil
supervisor's} insurance; or

\item you were called to the \emph{Bar} under Rule Q25, in which case you
must either be insured with \emph{BMIF} or be covered by insurance
against claims for professional negligence arising out of the supply of
your services in England and Wales in such amount and on such terms as
are currently required by the \emph{Bar Standards Board}, and have
delivered to the \emph{Bar Standards Board} a copy of the current
insurance policy, or the current certificate of insurance, issued by the
insurer.
\end{numlist}
\rulesubsection{rc78}

If you are a member of \emph{BMIF}, you \textcolor{myred}{\textbf{must}}:

\begin{numlist}\item pay promptly the insurance premium required by \emph{BMIF}; and

\item supply promptly such information as \emph{BMIF} may from time to time
require pursuant to its rules.\end{numlist}


\guidancesection{Guidance
to Rules C76-C78 }


\subsubsection{\color{darkgrey}gC113}

Where you are working in a \emph{BSB entity}, you will satisfy the
requirements of Rule rC76(1) so long as the \emph{BSB entity} has taken
out insurance, which covers your activities. A \emph{BSB entity} will
have to confirm each year that it has reviewed the adequacy of its
insurance cover on the basis of a risk analysis and that they have
complied with this rule.

\subsubsection{\color{darkgrey}gC114}

Any notice issued under Rule rC76 will be posted on the \emph{Bar
Standards Board's} website and may also be publicised by such other
means as the \emph{Bar Standards Board} may judge appropriate. Notices
issued under Rule C76, which stipulate minimum terms of cover for
\emph{self-employed barristers} and \emph{BSB entities}, are currently
in force and available on the \emph{Bar Standards Board's} website.

The \emph{Bar Standards Board's} requirements in respect of professional
indemnity insurance, including the minimum terms, are concerned with
ensuring consumer protection, specifically that there is adequate cover
for liabilities which \emph{BSB regulated persons} may incur to their
\emph{clients} or other parties to whom they may owe duties when
performing their \emph{legal services}. This includes claims for
contribution which third parties, such as instructing \emph{solicitors},
may make on the basis that the \emph{BSB regulated person} has such a
liability to a mutual \emph{client}. However, Rule C76(1) of the
\emph{Handbook} \textcolor{myred}{\textbf{does not}} require \emph{BSB regulated persons} to carry
insurance for other types of liability, which do not relate to their
liabilities towards consumers, such as a contractual liability to
instructing \emph{solicitors} in respect of losses incurred by the
\emph{solicitor} that are not based on any liability the
\emph{solicitor} has in turn incurred to the \emph{client}. Nor are the
minimum terms concerned with the latter type of liability and whether
and on what terms to seek to insure against such exposure is a
commercial judgment for \emph{BSB regulated persons} to make. You should
however ensure that you are aware of and comply with any general legal
requirements for you to carry other types of insurance than professional
indemnity cover.

\emph{BSB regulated persons} considering excluding or limiting liability
should consider carefully the ramifications of the Unfair Contract Terms
Act 1977 and other legislation and case law. If a \emph{BSB regulated
person} is found by the court to have limited liability in a way which
is in breach of the Unfair Contract Terms Act, that might amount to
professional misconduct.

\emph{BSB regulated persons} should regularly review the amount of their
professional indemnity insurance cover, taking into account the type of
work which they undertake and the likely liability for negligence. They
should be aware that claims can arise many years after the work was
undertaken and that they would be prudent to maintain adequate insurance
cover for that time since cover operates on a ``claims made'' basis and
as such it is the policy and the limits in force at the time a claim is
made that are relevant, not the policy and limits in force when the work
was undertaken. They should also bear in mind the need to arrange
run-off cover if they cease practice.

\subsubsection{\color{darkgrey}gC115}

Where you are working in an \emph{authorised (non-BSB) body}, the rules
of the \emph{approved regulator} of that body will determine what
insurance the \emph{authorised (non-BSB) body} \textcolor{myred}{\textbf{must}} have.

\subsubsection{\color{darkgrey}gC116}

Where you are working as an \emph{employed barrister (non-authorised
body)}, the rule \textcolor{myred}{\textbf{does not}} require you to have your own insurance if you
provide \emph{legal services} only to your \emph{employer}. If you
supply \emph{legal services} to other people (to the extent permitted by
the Scope of Practice and Authorisation, and Licensing Rules set out at
Section S(b) you should consider whether you need insurance yourself
having regard to the arrangements made by your \emph{employer} for
insuring against claims made in respect of your services. If your
\emph{employer} already has adequate insurance for this purpose, you
need not take out any insurance of your own. You should ensure that your
\emph{employer's} policy covers you, for example, for any pro-bono work
you may do.

\subsubsection{\color{darkgrey}gC117}

Where you are a \emph{registered European lawyer}, the rule \textcolor{myred}{\textbf{does not}}
require you to have your own insurance if:
\begin{numlist}\item you provide to the \emph{Bar Standards Board} evidence to show that
you are covered by insurance taken out or a guarantee provided in
accordance with the rules of your \emph{home State}; and
\item the \emph{Bar Standards Board} is satisfied that such insurance or
guarantee is fully equivalent in terms of conditions and extent of cover
to the cover required pursuant to Rule rC76. However, where the
\emph{Bar Standards Board} is satisfied that the equivalence is only
partial, the \emph{Bar Standards Board} may require you to arrange
additional insurance or an additional guarantee to cover the elements
which are not already covered by the insurance or guarantee contracted
by you in accordance with the rules of your \emph{home state}
\end{numlist}
\rulesection{Rules
C79-C85 - Associations with others }


\rulesubsection{rc79}

You \textcolor{myred}{\textbf{may not}} do anything, practising in \emph{an association}, which you
are otherwise prohibited from doing.

\rulesubsection{rc80}

Where you are in \emph{an association} on more than a one-off basis, you
must notify the \emph{Bar Standards Board} that you are in \emph{an
association}, and provide such details of that association as are
required by the \emph{Bar Standards Board}.

\rulesubsection{rc81}

If you have a material commercial interest in an organisation to which
you plan to refer a \emph{client}, you \textcolor{myred}{\textbf{must}}:
\begin{numlist}\item tell the \emph{client} in writing about your interest in that
organisation before you refer the \emph{client}; and
\item keep a record of your referrals to any such organisation for review
by the \emph{Bar Standards Board} on request.
\end{numlist}
\rulesubsection{rc82}

If you have a material commercial interest in an organisation which is
proposing to refer a matter to you, you \textcolor{myred}{\textbf{must}}:

\begin{numlist}\item tell the \emph{client} in writing about your interest in that
organisation before you accept such \emph{instructions};

\item make a clear agreement with that organisation or other public
statement about how relevant issues, such as conflicts of interest, will
be dealt with; and

\item keep a record of referrals received from any such organisation for
review by the \emph{Bar Standards Board} on reasonable request.\end{numlist}

\rulesubsection{rc83}

If you refer a \emph{client} to a third party which is not a \emph{BSB
authorised person} or an \emph{authorised (non-BSB) person}, you \textcolor{myred}{\textbf{must}}
take reasonable steps to ensure that the \emph{client} is not wrongly
led to believe that the third party is subject to regulation by the
\emph{Bar Standards Board} or by another \emph{approved regulator}.

\rulesubsection{rc84}

You \textcolor{myred}{\textbf{must not}} have a material commercial interest in any organisation
which gives the impression of being, or may be reasonably perceived as
being, subject to the regulation of the \emph{Bar Standards Board} or of
another \emph{approved regulator}, in circumstances where it is not so
regulated.

\rulesubsection{rc85}

A material commercial interest for the purposes of Rules rC78 to rC84 is
an interest which an objective observer with knowledge of the salient
facts would reasonably consider might potentially influence your
judgment.

\rulesubsection{rc85A}

You \textcolor{myred}{\textbf{must not}} act as a supervisor of an immigration adviser for the
purposes of section 84(2) of the Immigration and Asylum Act 1999 (as
amended) (IAA 1999) where the Office of the Immigration Services
Commissioner has refused or cancelled the adviser's registration, or
where the adviser is:

1. disqualified in accordance with paragraph 4 of Schedule 6 to the IAA
1999; or

2. prohibited or suspended by the First-tier Tribunal (Immigration
Services); or

3. permanently prohibited from practising by an \emph{approved
regulator}, or a designated professional body under the IAA 1999,
pursuant to its powers as such, and removed from the relevant register;
or

4. currently suspended from practising by an \emph{approved regulator},
or a designated professional body under the IAA 1999, pursuant to its
powers as such.


\guidancesection{Guidance
to Rules C79-C85 (and  \textbf{\textcolor{mygold}{CD5}}) }


\subsubsection{\color{darkgrey}gC118}

You \textcolor{myred}{\textbf{may not}} use an association with the purpose of, or in order to evade
rules which would otherwise apply to you. You \textcolor{myred}{\textbf{may not}} do anything,
practising in \emph{an association}, which you are individually
prohibited from doing.

\subsubsection{\color{darkgrey}gC119}

You will bring yourself and your profession into disrepute ( \textbf{\textcolor{mygold}{CD5}}) if you
are personally involved in arrangements which breach the restrictions
imposed by the Legal Services Act 2007 on those who can provide reserved
legal activities. For example, You \textcolor{myred}{\textbf{must not}} remain a member of any
``ProcureCo'' arrangement where you know or are reckless as to whether
the ProcureCo is itself carrying on reserved legal activities without a
licence or where you have failed to take reasonable steps to ensure this
is not so before joining or continuing your involvement with the
Procureco.

\subsubsection{\color{darkgrey}gC120}

The purpose of Rules rC79 to rC85 is to ensure that \emph{clients} and
members of the public are not confused by any such association. In
particular, the public should be clear who is responsible for doing
work, and about the extent to which that person is regulated in doing
it: see Rules rC79-85.

\subsubsection{\color{darkgrey}gC121}

This \emph{Handbook} applies in full whether or not you are practising
in an association. You are particularly reminded of the need to ensure
that, notwithstanding any such association, you continue to comply with
Rules C8, C9, C10, C12, C15, C19, C20, C28, C73, C75, C79, C82 and C86
(and, where relevant C80, C81, C83, C74 and C110).

\subsubsection{\color{darkgrey}gC122}

References to ``organisation'' in Rules rC81 and C82 include \emph{BSB
entities} and \emph{authorised (non-BSB) bodies}, as well as
non-authorised bodies. So, if you have an interest, as owner, or
manager, in any such body, your relationship with any such organisation
is caught by these rules.

\subsubsection{\color{darkgrey}gC123}

These rules do not permit you to accept \emph{instructions} from a third
party in any case where that would give rise to a potential conflict of
interest contrary to \textbf{\textcolor{mygold}{CD2}} or any relevant part of Rule rC21.

\subsubsection{\color{darkgrey}gC124}

You should only refer a \emph{client} to an organisation in which you
have a material commercial interest if it is in the \emph{client's} best
interest to be referred to that organisation. This is one aspect of what
is required of you by \textbf{\textcolor{mygold}{CD2}}. Your obligations of honesty and integrity, in
 \textbf{\textcolor{mygold}{CD3}}, require you to be open with \emph{clients} about any interest you
have in, or arrangement you have with, any organisation to which you
properly refer the \emph{client}, or from which the \emph{client} is
referred to you. It is inherently unlikely that a general referral
arrangement obliging you (whether or not you have an interest in such
organisation) to refer to that organisation, without the option to refer
elsewhere if the \emph{client's} circumstances make that more
appropriate, could be justified as being in the best interests of each
individual \emph{client} (\textcolor{mygold}{\textbf{CD2}}) and it may well also be contrary to your
obligations of honesty and integrity ( \textbf{\textcolor{mygold}{CD3}}) and compromise your
independence ( \textbf{\textcolor{mygold}{CD4}}).

\subsubsection{\color{darkgrey}gC125}

The \emph{Bar Standards Board} may require you to provide copies of any
protocols that you may have in order to ensure compliance with these
rules.

\subsubsection{\color{darkgrey}gC126}

Your obligations under  \textbf{\textcolor{mygold}{CD5}} require you not to act in \emph{an
association} with a person where, merely by being associated with such
person, you may reasonably be considered as bringing the profession into
disrepute or otherwise diminishing the trust that the public places in
you and your profession.

\rulesection{Rule
C86 - Outsourcing }


\rulesubsection{rc86}

Where you outsource to a third party any support services that are
critical to the delivery of any \emph{legal services} in respect of
which you are instructed:
\begin{numlist}\item any outsourcing \textcolor{myred}{\textbf{does not}} alter your obligations to your
\emph{client};
\item you remain responsible for compliance with your obligations under
this \emph{Handbook} in respect of the \emph{legal services};
\item You \textcolor{myred}{\textbf{must}} ensure that such outsourcing is subject to contractual
arrangements which ensure that such third party:
\begin{alphlist}\item is subject to confidentiality obligations similar to the
confidentiality obligations placed on you in accordance with this
\emph{Handbook};
\item complies with any other obligations set out in this Code of Conduct
which may be relevant to or affected by such outsourcing;
\item processes any personal data in accordance with your
\emph{instructions};
\item is required to allow the \emph{Bar Standards Board} or its agent to
obtain information from, inspect the records (including electronic
records) of, or enter the premises of such third party in relation to
the outsourced activities or functions, and;
\item processes any personal data in accordance with those arrangements,
and for the avoidance of doubt, those arrangements are compliant with
any relevant data protection laws.\end{alphlist}
\end{numlist}

\guidancesection{Guidance
to Rule C86 }


\subsubsection{\color{darkgrey}gC127}

Rule C86 applies to the outsourcing of clerking services.

\subsubsection{\color{darkgrey}gC128}

Rule C86 \textcolor{myred}{\textbf{does not}} apply where the \emph{client} enters into a separate
agreement with the third party for the services in question.

\subsubsection{\color{darkgrey}gC129}

Rule C86 \textcolor{myred}{\textbf{does not}} apply where you are instructing a \emph{pupil} or a
\emph{devil} to undertake work on your behalf. Instead rC15 will apply
in those circumstances.

\subsubsection{\color{darkgrey}gC130}

Removed from 11 June 2018.

\rulesection{Rules
C87-C88 - Administration and conduct of self-employed practice }


\rulesubsection{rc87}

You \textcolor{myred}{\textbf{must}} take reasonable steps to ensure that:
\begin{numlist}\item your practice is efficiently and properly administered having regard
to the nature of your practice; and
\item proper records of your practice are kept.

When deciding how long records need to be kept, you will need to take
into consideration various requirements, such as those of this
\emph{Handbook} (see, for example, Rules C108, C129 and C141), any
relevant data protection law and HM Revenue and Customs. You may want to
consider drawing up a Records Keeping policy to ensure that you have
identified the specific compliance and other needs of your
\emph{practice}.
\end{numlist}

\rulesubsection{rc88}

You \textcolor{myred}{\textbf{must}}:

\begin{numlist}\item ensure that adequate records supporting the fees charged or claimed
in a case are kept at least until the later of the following:
\begin{alphlist}
\item your fees have been paid; and

\item any determination or assessment of costs in the case has been
completed and the time for lodging an appeal against that assessment or
determination has expired without any such appeal being lodged, or any
such appeal has been finally determined;

\item provide your \emph{client} with such records or details of the work
you have done as may reasonably be required for the purposes of
verifying your charges.\end{alphlist}\end{numlist}

\rulesection{Rules
C89-C90 - Administration of chambers}


\rulesubsection{rc89}

Taking into account the provisions of Rule rC90, You \textcolor{myred}{\textbf{must}} take
reasonable steps to ensure that:
\begin{numlist}
\item your \emph{chambers} is administered competently and efficiently;

\item your \emph{chambers} has appointed an individual or individuals to
liaise with the \emph{Bar Standards Board} in respect of any regulatory
requirements and has notified the \emph{Bar Standards Board};

\item \emph{barristers} in your \emph{chambers} do not employ any person
who has been disqualified from being employed by an authorised person
\begin{romlist}
\item by the \emph{Bar Standards Board} and included on the \emph{Bar
Standards Board's} list of disqualified persons, or 
\item by another
approved regulator or \emph{licensing authority} pursuant to its powers
as such, and such disqualification is continuing in force. This shall
not apply where the \emph{barrister} obtains the express written consent
of the \emph{Bar Standards Board} to the appointment of a person who has
been disqualified before they are appointed;\end{romlist}
\item proper arrangements are made in your \emph{chambers} for dealing with
\emph{pupils} and pupillage;

\item proper arrangements are made in \emph{chambers} for the management of
conflicts of interest and for ensuring the confidentiality of
\emph{clients'} affairs;

\item all non-authorised persons working in your \emph{chambers}
(irrespective of the identity of their \emph{employer}):
\begin{alphlist}
\item are competent to carry out their duties;

\item carry out their duties in a correct and efficient manner;

\item are made clearly aware of such provisions of this \emph{Handbook} as
may affect or be relevant to the performance of their duties;

\item do nothing which causes or substantially contributes to a breach of
this \emph{Handbook} by any \emph{BSB authorised individual} or
\emph{authorised (non-BSB) individual} within \emph{chambers},

and all \emph{complaints} against them are dealt with in accordance with
the complaints rules;\end{alphlist}

\item all \emph{registered European lawyers} and all \emph{foreign lawyers}
in your \emph{chambers} comply with this \emph{Handbook} insofar as
applicable to them;

\item appropriate risk management procedures are in place and are being
complied with; and

\item there are systems in place to check that:
\begin{alphlist}
\item all persons practising from your \emph{chambers} whether they are
members of the \emph{chambers} or not have insurance in place in
accordance with Rules rC75 to rC77 above (other than any \emph{pupil}
who is covered under their \emph{pupil supervisor's} insurance); and

\item every \emph{BSB authorised individual} practising from your
\emph{chambers} has a current \emph{practising certificate} and every
other \emph{authorised (non-BSB) individual} providing \emph{reserved
legal activities} is currently authorised by their \emph{Approved
Regulator}.
\end{alphlist}
\end{numlist}
\rulesubsection{rc90}

For the purposes of Rule rC89 the steps which it is reasonable for you
to take will depend on all the circumstances, which include, but are not
limited to:
\begin{numlist}
\item the arrangements in place in your \emph{chambers} for the management
of \emph{chambers};

\item any role which you play in those arrangements; and

\item the independence of individual members of \emph{chambers} from one
another.
\end{numlist}


\guidancesection{Guidance
to Rules C89-C90 }


\subsubsection{\color{darkgrey}gC131}

Members of \emph{chambers} are not in partnership but are independent of
one another and are not responsible for the conduct of other members.
However, each individual member of \emph{chambers} is responsible for
their own conduct and the constitution of \emph{chambers} enables, or
should enable, each individual member of \emph{chambers} to take steps
to terminate another person's membership in specified circumstances.
Rule C89 \textcolor{myred}{\textbf{does not}} require you to sever connection with a member of
\emph{chambers} solely because to your knowledge they are found to
breach this \emph{Handbook}, provided that they are not disbarred and
comply with such sanctions as may be imposed for such breach; however,
your chambers \emph{constitution} should be drafted so as to allow you
to exclude from chambers a member whose conduct is reasonably considered
such as to diminish the trust the public places in you and your
profession and you should take such steps as are reasonably available to
you under your constitution to exclude any such member.

\subsubsection{\color{darkgrey}gC132}

The \emph{Supervision Team} of the \emph{Bar Standards Board} reviews
the key controls that are in place in \emph{chambers} and \emph{BSB
entities} to manage the risks in relation to key processes. These key
processes are shown in guidance that is published on the Supervision
section of the \emph{Bar Standards Board's} website. You should retain
relevant policies, procedures, monitoring reports and other records of
your practice so that they are available to view if a Supervision visit
is arranged.

\subsubsection{\color{darkgrey}gC133}

Your duty under Rule rC89(4) to have proper arrangements in place for
dealing with pupils includes ensuring:
\begin{numlist}\item that all \emph{pupillage} vacancies are advertised in the manner set
out in the Bar Qualification Manual;
\item that arrangements are made for the funding of \emph{pupils} by
\emph{chambers} which comply with the Pupillage Funding Rules (rC113 to
rC118); and
\item the chambers meets the mandatory requirements set out in the
\emph{Authorisation Framework} and complies with conditions imposed upon
its authorisation as an \emph{Authorised Education and Training
Organisation (AETO)}.
\end{numlist}
\subsubsection{\color{darkgrey}gC134}

Your duty under Rule rC89(5) to have proper arrangements in place for
ensuring the confidentiality of each \emph{client's} affairs includes:
\begin{numlist}
\item putting in place and enforcing adequate procedures for the purpose of
protecting confidential information;

\item complying with data protection obligations imposed by law;

\item taking reasonable steps to ensure that anyone who has access to such
information or data in the course of their work for you complies with
these obligations; and \item taking into account any further guidance on
confidentiality which is available on the Bar Standards Board's website.\end{numlist}

\subsubsection{\color{darkgrey}gC135}

In order to ensure compliance with Rule rC89(6)(d), you may want to
consider incorporating an obligation along these lines in all new
employment contracts entered into after the date of this
\emph{Handbook}.

\subsubsection{\color{darkgrey}gC136}

For further guidance on what may constitute appropriate risk management
procedures in accordance with Rule rC89(8) please refer to the further
guidance published by the \emph{Bar Standards Board} which can be
accessed on the Supervision section of its website.

\subsubsection{\color{darkgrey}gC137}

Rule rC90(3) means that you should consider, in particular, the
obligation of each individual members of \emph{chambers} to act in the
best interests of their own \emph{client} (\textcolor{mygold}{\textbf{CD2}}) and to preserve the
confidentiality of their own \emph{client's} affairs ( \textbf{\textcolor{mygold}{CD6}}), in
circumstances where other members of \emph{chambers} are free (and,
indeed, may be obliged by the cab rank rule (rC29) to act for
\emph{clients} with conflicting interests.

\rulesection{Rules
C91-C95 - Administration of BSB entities }


\subsection{Duties of the BSB entity, authorised (non-BSB) individuals and
managers of BSB entities}

\rulesubsection{rc91}

If you are a \emph{BSB entity}, You \textcolor{myred}{\textbf{must}} ensure that (or, if you are a
\emph{BSB regulated individual} working within such \emph{BSB entity}
You \textcolor{myred}{\textbf{must}} use reasonable endeavours (taking into account the provisions
of Rule rC95) to procure that the \emph{BSB entity} ensures that):
\begin{numlist}\item the \emph{BSB entity} has at all times a person appointed by it to
act as its \emph{HOLP}, who shall be a \emph{manager};
\item the \emph{BSB entity} has at all times a person appointed by it to
act as its \emph{HOFA}; and
\item subject to rC92, the \emph{BSB entity} \textcolor{myred}{\textbf{does not}} appoint any
individual to act as a \emph{HOLP} or a \emph{HOFA}, or to be a
\emph{manager} or employee of that \emph{BSB entity}, in circumstances
where that individual has been disqualified from being appointed to act
as a \emph{HOLP} or a \emph{HOFA} or from being a \emph{manager} or
employed by an \emph{authorised person} (as appropriate) 

\begin{romlist}\item by the
\emph{Bar Standards Board} and included on the \emph{Bar Standards
Board's} list of disqualified persons, or \item by another \emph{Approved
Regulator} or \emph{licensing authority} \end{romlist}pursuant to its powers as such
and such disqualification is continuing in force.

\end{numlist}

\rulesubsection{rc92}

Rule rC91(3) shall not apply where the \emph{BSB entity} obtains the
express written consent of the \emph{Bar Standards Board} to the
appointment of a person who has been disqualified before they are
appointed.

\rulesubsection{rc93}

If you are a \emph{manager} or employee, You \textcolor{myred}{\textbf{must not}} do anything to
cause (or substantially to contribute to) a breach by the \emph{BSB
entity} or by any \emph{BSB authorised individual} in it of their duties
under this \emph{Handbook}.

\rulesubsection{rc94}

If you are a \emph{BSB entity}, You \textcolor{myred}{\textbf{must}} at all times have (or, if you
are a \emph{BSB regulated individual} working in such \emph{BSB entity}
You \textcolor{myred}{\textbf{must}} use reasonable endeavours (taking into account the provisions
of Rule rC95 to procure that the \emph{BSB entity} shall have) suitable
arrangements to ensure that:
\begin{numlist}
\item the \emph{managers} and other \emph{BSB regulated individuals}
working as employees of the \emph{BSB entity} comply with the \emph{Bar
Standards Board's} regulatory arrangements as they apply to them, as
required under section 176 of the LSA;

\item all employees:
\begin{alphlist}
\item are competent to carry out their duties;

\item carry out their duties in a correct and efficient manner;

\item are made clearly aware of such provisions of this \emph{Handbook} as
may affect or be relevant to the performance of their duties;

\item do nothing which causes or substantially contributes to, a breach of
this \emph{Handbook} by the \emph{BSB entity} or any of the \emph{BSB
regulated individuals} employed by it; and

\item co-operate with the \emph{Bar Standards Board} in the exercise of its
regulatory functions, in particular in relation to any notice under rC64
or any request under rC70;
\end{alphlist}
\item the \emph{BSB entity} is administered in a correct and efficient
manner, is properly staffed and keeps proper records of its practice;

\item \emph{pupils} and \emph{pupillages} are dealt with properly;

\item conflicts of interest are managed appropriately and that the
confidentiality of \emph{clients'} affairs is maintained at all times;

\item all \emph{registered European lawyers} and all \emph{foreign lawyers}
employed by or working for you comply with this \emph{Handbook} insofar
as it applies to them;

\item every \emph{BSB authorised individual} employed by, or working for,
the \emph{BSB entity} has a current \emph{practising certificate} (and
where a \emph{barrister} is working as an \emph{unregistered barrister},
there \textcolor{myred}{\textbf{must}} be appropriate systems to ensure that they are complying with
the provisions of this \emph{Handbook} which apply to \emph{unregistered
barristers}) and every other \emph{authorised (non-BSB) individual}
providing \emph{reserved legal activities} is currently authorised by
their \emph{Approved Regulator}; and

\item adequate records supporting the fees charged or claimed in a case are
kept at least until the later of the following:
\begin{alphlist}
\item your fees have been paid; and

\item any determination or assessment of costs in the case has been
completed and the time for lodging an appeal against that assessment or
determination has expired without any such appeal being lodged, or any
such appeal has been finally determined;
\end{alphlist}
\item your \emph{client} is provided with such records or details of the
work you have done as may reasonably be required for the purpose of
verifying your charges;

\item appropriate procedures are in place requiring all \emph{managers}
and employees to work with the \emph{HOLP} with a view to ensuring that
the \emph{HOLP} is able to comply with their obligations under Rule
rC96;

\item appropriate risk management procedures are in place and are being
complied with; and

\item appropriate financial management procedures are in place and are
being complied with.
\end{numlist}
\rulesubsection{rc95}

For the purposes of Rule rC91 and rC94 the steps which it is reasonable
for you to take will depend on all the circumstances, which include, but
are not limited to:
\begin{numlist}\item the arrangements in place in your \emph{BSB entity} for the
management of it; and
\item any role which you play in those arrangements.

\end{numlist}
\guidancesection{Guidance
to Rule C94 }


\subsubsection{\color{darkgrey}gC138}

Section 90 of the \emph{LSA} places obligations on \emph{non-authorised
individuals} who are employees and \emph{managers} of \emph{licensed
bodies}, as well as on \emph{non-authorised individuals} who hold an
ownership interest in such a \emph{licensed body} (whether by means of a
shareholding or voting powers in respect of the same) to do nothing
which causes, or substantially contributes to a breach by the
\emph{licensed body} or by its employees or \emph{managers}, of this
\emph{Handbook}. Rule C94 extends this obligation to \emph{BSB entities}
other than \emph{licensed bodies}.

\subsubsection{\color{darkgrey}gC139}

Your duty under Rule rC94(4) to have proper arrangements for dealing with
pupils includes ensuring:
\begin{numlist}
\item that all \emph{pupillage} vacancies are advertised in the manner
manner set out in the Bar Qualification Manual;

\item that arrangements are made for the funding of \emph{pupils} by the
\emph{BSB entity} which comply with the Pupillage Funding Rules (rC113
to rC118); and

\item the \emph{BSB entity} meets the mandatory requirements set out in the
\emph{Authorisation Framework} and complies with conditions imposed upon
its authorisation as an \emph{Authorised Education and Training
Organisation (AETO)}.\end{numlist}

\rulesection{Rules
C96-C97 - Duties of the HOLP and HOFA }


\rulesubsection{rc96}

If you are a \emph{HOLP}, in addition to complying with the more general
duties placed on the \emph{BSB entity} and on the \emph{BSB regulated
individuals} employed by it, you \textcolor{myred}{\textbf{must}}:
\begin{numlist}
\item take all reasonable steps to ensure compliance with the terms of your
\emph{BSB entity's} authorisation;

\item take all reasonable steps to ensure that the \emph{BSB entity} and
its employees and \emph{managers} comply with the duties imposed by
section 176 of the LSA;

\item take all reasonable steps to ensure that \emph{non-authorised
individuals} subject to the duty imposed by section 90 of the LSA comply
with that duty;

\item keep a record of all incidents of non-compliance with the Core Duties
and this \emph{Handbook} of which you become aware and to report such
incidents to the \emph{Bar Standards Board} as soon as reasonably
practicable (where such failures are material in nature) or otherwise on
request by the \emph{Bar Standards Board} or during the next monitoring
visit or review by the \emph{Bar Standards Board}.
\end{numlist}
\rulesubsection{rc97}

If you are a \emph{HOFA}, in addition to complying with the more general
duties placed on the \emph{BSB entity} and its \emph{BSB regulated
individuals}, You \textcolor{myred}{\textbf{must}} ensure compliance with Rules rC73 and rC74.

\rulesection{Rule
C98 - New managers, HOLPs and HOFAs }


\rulesubsection{rc98}

A \emph{BSB entity} \textcolor{myred}{\textbf{must}} not take on a new \emph{manager}, \emph{HOLP}
or \emph{HOFA} without first submitting an application to the \emph{Bar
Standards Board} for approval in accordance with the requirements of
Section S(d).

\chap{Part 2
- D. Rules Applying to Particular Groups of Regulated Persons}

\chap{Part 2
- D1. Self-employed barristers, chambers and BSB entities }

\ocsection{Outcomes
C26-C29 }


\subsection{\color{bleu}oC26}

\emph{Clients} are provided with appropriate information about redress,
know that they can make a \emph{complaint} if dissatisfied, and know how
to do so.~

\subsection{\color{bleu}oC27}

\emph{Complaints} are dealt with promptly and the \emph{client} is kept
informed about the process.

\subsection{\color{bleu}oC28}

\emph{Self-employed barristers}, \emph{chambers} and \emph{BSB entities}
run their practices without \emph{discrimination}.

\subsection{\color{bleu}oC29}

\emph{Pupils} are treated fairly and paid in accordance with the
Pupillage Funding Rules.

\rulesection{Rules
C99-C109 - Complaints rules }


\subsection{Provision of information}

\rulesubsection{rc99}

You \textcolor{myred}{\textbf{must}} notify \emph{clients} in writing when you are
\emph{instructed}, or, if that is if not practicable, at the next
appropriate opportunity:
\begin{numlist}\item of their right to make a \emph{complaint}, including their right to
complain to the \emph{Legal Ombudsman} (if they have such a right), how,
and to whom, they can complain, and of any time limits for making a
\emph{complaint};
\item if you are doing referral work, that the lay \emph{client} may
complain directly to \emph{chambers} or the \emph{BSB entity} without
going through \emph{solicitors}.
\end{numlist}
\rulesubsection{rc100}

If you are doing public access, or licensed access work using an
\emph{intermediary}, the \emph{intermediary} \textcolor{myred}{\textbf{must}} similarly be informed.

\rulesubsection{rc101}

If you are doing referral work, you do not need to give a
\emph{professional client} the information set out in Rules rC99(1) and
rC99(2), in a separate, specific letter. It is enough to provide it in
the ordinary terms of reference letter (or equivalent letter) which you
send when you accept \emph{instructions} in accordance with Rule rC21.

\rulesubsection{rc102}

If you do not send a letter of engagement to a lay \emph{client} in
which this information can be included, a specific letter \textcolor{myred}{\textbf{must}} be sent
to them giving them the information set out at Rules rC99(1) and rC99(2).

\rulesubsection{rc103}

Each website of \emph{self-employed barristers}, \emph{chambers} and
\emph{BSB entities} \textcolor{myred}{\textbf{must}} display:
\begin{numlist}\item on the homepage, the text ``regulated by the Bar Standards Board''
(for sole practitioners) or ``barristers regulated by the Bar Standards
Board'' (for \emph{chambers}) or ''authorised and regulated by the Bar
Standards Board'' (for \emph{BSB entities}); and
\item in a sufficiently accessible and prominent place:
\begin{alphlist}\item information about their complaints procedure, any right to complain
to the Legal Ombudsman, how to complain to the Legal Ombudsman and any
time limits for making a complaint;
\item a link to the decision data on the Legal Ombudsman's website; and
\item a link to the Barristers' Register on the BSB's website.\end{alphlist}
\item All e-mail and letterheads from \emph{self-employed barristers} and
\emph{BSB entities}, their \emph{managers} and employees \textcolor{myred}{\textbf{must}} state
``regulated by the Bar Standards Board'' (for \emph{self-employed
barristers}) or ``authorised and regulated by the Bar Standards Board''
(for \emph{BSB entities}).
\item \emph{Self-employed barristers}, \emph{chambers} and \emph{BSB
entities} \textcolor{myred}{\textbf{must}} have regard to guidance published from time to time by
the \emph{Bar Standards Board} in relation to redress transparency.
\end{numlist}
\subsection{Response to complaints}

\rulesubsection{rc104}

All \emph{complaints} \textcolor{myred}{\textbf{must}} be acknowledged promptly. When you
acknowledge a \emph{complaint}, You \textcolor{myred}{\textbf{must}} give the complainant:
\begin{numlist}\item the name of the person who will deal with the \emph{complaint} and a
description of that person's role in \emph{chambers} or in the \emph{BSB
entity} (as appropriate);
\item a copy of the \emph{chambers'} complaints procedure or the \emph{BSB
entity's} Complaints Procedure (as appropriate);
\item the date by which the complainant will next hear from \emph{chambers}
or the \emph{BSB entity} (as appropriate).
\end{numlist}
\rulesubsection{rc105}

When \emph{chambers} or a \emph{BSB entity} (as appropriate) has dealt
with the \emph{complaint}, complainants \textcolor{myred}{\textbf{must}} be told in writing of their
right to complain to the \emph{Legal Ombudsman} (where applicable), of
the time limit for doing so, and how to contact them.

\subsection{Documents and record keeping}

\rulesubsection{rc106}

All communications and documents relating to complaints \textcolor{myred}{\textbf{must}} be kept
confidential. They \textcolor{myred}{\textbf{must}} be disclosed only so far as is necessary for:
\begin{numlist}\item the investigation and resolution of the \emph{complaint};
\item internal review in order to improve \emph{chambers'} or the \emph{BSB
entity's} (as appropriate) handling of complaints;
\item complying with requests from the \emph{Bar Standards Board} in the
exercise of its monitoring and/or auditing functions.\end{numlist}

\rulesubsection{rc107}

The disclosure to the \emph{Bar Standards Board} of internal documents
relating to the handling of the \emph{complaint} (such as the minutes of
any meeting held to discuss a particular \emph{complaint}) for the
further resolution or investigation of the \emph{complaint} is not
required.

\rulesubsection{rc108}

A record \textcolor{myred}{\textbf{must}} be kept of each \emph{complaint}, of all steps taken in
response to it, and of the outcome of the \emph{complaint}. Copies of
all correspondence, including electronic mail, and all other documents
generated in response to the \emph{complaint} \textcolor{myred}{\textbf{must}} also be kept. The
records and copies should be kept for 6 years from resolution of the
\emph{complaint}.

\rulesubsection{rc109}

The person responsible for the administration of the procedure \textcolor{myred}{\textbf{must}}
report at least annually to either:
\begin{numlist}\item the \emph{HOLP}; or
\item the appropriate member/committee of \emph{chambers},\end{numlist}
on the number of \emph{complaints} received, on the subject areas of the
\emph{complaints} and on the outcomes. The \emph{complaints} should be
reviewed for trends and possible training issues.

\rulesection{Rules
C110-C112 - Equality and diversity }


\rulesubsection{rc110}

You \textcolor{myred}{\textbf{must}} take reasonable steps to ensure that in relation to your
\emph{chambers} or \emph{BSB entity}:
\begin{numlist}
\item there is in force a written statement of policy on equality and
diversity; and

\item there is in force a written plan implementing that policy;

\item the following requirements are complied with:
\subsection{Equality and Diversity Officer}
\begin{alphlist}

\item \emph{chambers} or \emph{BSB entity} has at least one \emph{Equality
and Diversity Officer};

\subsection{Training}

\item removed.

\item save in exceptional circumstances, every member of all selection
panels \textcolor{myred}{\textbf{must}} be trained in fair recruitment and selection processes;

\subsection{Fair and objective criteria}

\item recruitment and selection processes use objective and fair criteria;

\subsection{Equality monitoring}

\item your \emph{chambers} or \emph{BSB entity}:
\begin{romlist}
\item conducts a regular review of its policy on equality and diversity and
of its implementation in order to ensure that it complies with the
requirements of this Rule rC110; and

\item takes any appropriate remedial action identified in the light of
that review;
\end{romlist}
\item subject to Rule rC110(3)(h) \emph{chambers} or \emph{BSB entity}
regularly reviews:
\begin{romlist}
\item the number and percentages of its \emph{workforce} from different
groups; and

\item applications to become a member of its \emph{workforce}; and

\item in the case of \emph{chambers}, the \emph{allocation of unassigned
work},
\end{romlist}
\item the reviews referred to in Rule rC110(3)(f) above include:
\begin{romlist}
\item collecting and analysing data broken down by race, disability and
gender;

\item investigating the reasons for any disparities in that data; and

\item taking appropriate remedial action;
\end{romlist}
\item the requirement to collect the information referred to in Rule
C110(3)(g) \textcolor{myred}{\textbf{does not}} apply to the extent that the people referred to in
Rule rC110(3)(f)(i) and Rule rC110(3)(f)(ii) refuse to disclose it.

\subsection{Fair access to work}

\item if you are a \emph{self-employed barrister}, the affairs of your
\emph{chambers} are conducted in a manner which is fair and equitable
for all members of \emph{chambers}, \emph{pupils} and/or employees (as
appropriate). This includes, but is not limited to, the fair
distribution of work opportunities among \emph{pupils} and members of
\emph{chambers};

\subsection{Harassment}

\item \emph{chambers} or \emph{BSB entity} has a written
anti-\emph{harassment} policy which, as a minimum:
\begin{romlist}
\item states that \emph{harassment} will not be tolerated or condoned and
that \emph{managers}, employees, members of \emph{chambers},
\emph{pupils} and others temporarily in your \emph{chambers} or
\emph{BSB entity} such as mini-pupils have a right to complain if it
occurs;

\item sets out how the policy will be communicated;

\item sets out the procedure for dealing with \emph{complaints} of
\emph{harassment};\end{romlist}

\subsection{Parental leave}

\item \emph{chambers} has a \emph{parental leave} policy which \textcolor{myred}{\textbf{must}} cover
as a minimum:
\begin{romlist}
\item the right of a member of \emph{chambers} to take \emph{parental
leave};

\item the right of a member of \emph{chambers} to return to
\emph{chambers} after a specified period, or number of separate periods,
of \emph{parental leave}, provided the total leave taken \textcolor{myred}{\textbf{does not}} exceed
a specified maximum duration (which \textcolor{myred}{\textbf{must}} be at least one year);

\item a provision that enables \emph{parental leave} to be taken flexibly
and allows the member of \emph{chambers} to maintain their
\emph{practice} while on \emph{parental leave}, including the ability to
carry out fee earning work while on \emph{parental leave} without giving
up other \emph{parental leave} rights;

\item the extent to which a member of chambers is or is not required to
contribute to chambers' rent and expenses during parental leave;

\item the method of calculation of any waiver, reduction or reimbursement
of \emph{chambers'} rent and expenses during \emph{parental leave};

\item where any element of rent is paid on a flat rate basis, the
\emph{chambers'} policy \textcolor{myred}{\textbf{must}} as a minimum provide that \emph{chambers}
will offer members taking a period of \emph{parental leave} a minimum of
6 months free of \emph{chambers'} rent;

\item the procedure for dealing with grievances under the policy;

\item \emph{chambers'} commitment to regularly review the effectiveness
of the policy;
\end{romlist}
\subsection{Flexible working}

\item \emph{chambers} or \emph{BSB entity} has a flexible working policy
which covers the right of a member of \emph{chambers}, \emph{manager} or
employee (as the case may be) to take a career break, to work part-time,
to work flexible hours, or to work from home, so as to enable them to
manage their family responsibilities or disability without giving up
work;

\subsection{Reasonable adjustments policy}

\item \emph{chambers} or \emph{BSB entity} has a reasonable adjustments
policy aimed at supporting disabled \emph{clients}, its \emph{workforce}
and others including temporary visitors;

\section{Appointment of Diversity Data Officer}

\item \emph{chambers} or \emph{BSB entity} has a Diversity Data Officer;

\item \emph{chambers} or \emph{BSB entity} \textcolor{myred}{\textbf{must}} provide the name and
contact details of the Diversity Data Officer to the \emph{Bar Standards
Board} and \textcolor{myred}{\textbf{must}} notify the \emph{Bar Standards Board} of any change to
the identity of the Diversity Data Officer, as soon as reasonably
practicable;

\subsection{Responsibilities of Diversity Data Officer}

\item The Diversity Data Officer shall comply with the requirements in
relation to the collection, processing and publication of
\emph{diversity data} set out in the paragraphs rC110(3).q to .t below;

\subsection{Collection and publication of diversity data}

\item The Diversity Data Officer shall invite members of the
\emph{workforce} to provide \emph{diversity data} in respect of
themselves to the Diversity Data Officer using the model questionnaire
in Section 7 of the BSB's Supporting Information on the BSB Handbook
Equality Rules, which is available on the BSB's website;

\item The Diversity Data Officer shall ensure that such data is anonymised
and that an accurate and updated summary of it is published on
\emph{chambers'} or \emph{BSB entity's} website every three years. If
\emph{chambers} or the \emph{BSB entity} \textcolor{myred}{\textbf{does not}} have a website, the
Diversity Data Officer shall make such data available to the public on
request;

\item The published summary of anonymised data shall:
\begin{romlist}
\item removed;

\item exclude diversity data in relation to any characteristic where there
is a real risk that individuals could be identified, unless all affected
individuals consent; and

\item subject to the foregoing, include anonymised data in relation to
each characteristic, categorised by reference to the job title and
seniority of the \emph{workforce}.\end{romlist}

\item The Diversity Data Officer shall:

\begin{romlist}\item ensure that \emph{chambers} or \emph{BSB entity} has in place a
written policy statement on the collection, publication, retention and
destruction of \emph{diversity data} which shall include an explanation
that the provision of \emph{diversity data} is voluntary;

\item notify the \emph{workforce} of the contents of the written policy
statement; and

\item ask for explicit consent from the \emph{workforce} to the provision
and processing of their \emph{diversity data} in accordance with the
written policy statement and these rules, in advance of collecting their
\emph{diversity data}.\end{romlist}
\end{alphlist}
\end{numlist}
\rulesubsection{rc111}

For the purposes of Rule rC110 above, the steps which it is reasonable
for you to take will depend on all the circumstances, which include, but
are not limited to:
\begin{numlist}
\item the arrangements in place in your \emph{chambers} or \emph{BSB
entity} for the management of \emph{chambers} or the \emph{BSB entity};
and

\item any role which you play in those arrangements.
\end{numlist}
\rulesubsection{rc112}

For the purposes Rule rC110 above ``allocation of unassigned work''
includes, but is not limited to work allocated to:
\begin{numlist}\item \emph{pupils};
\item \emph{barristers} of fewer than four \emph{years' standing}; and
\item \emph{barristers} returning from \emph{parental leave};
\end{numlist}

\guidancesection{Guidance
to Rules C110-C112 }


\subsubsection{\color{darkgrey}gC140}

Rule rC110 places a personal obligation on all \emph{self-employed
barristers}, however they practise, and on the \emph{managers} of
\emph{BSB entities}, as well as on the entity itself, to take reasonable
steps to ensure that they have appropriate policies which are enforced.

\subsubsection{\color{darkgrey}gC141}

In relation to Rule rC110, if you are a Head of \emph{chambers} or a
\emph{HOLP} it is likely to be reasonable for you to ensure that you
have the policies required by Rule rC110, that an \emph{Equality and
Diversity Officer} is appointed to monitor compliance, and that any
breaches are appropriately punished. If you are a member of a
\emph{chambers} you are expected to use the means available to you under
your constitution to take reasonable steps to ensure there are policies
and that they are enforced. If you are a \emph{manager} of a \emph{BSB
entity}, you are expected to take reasonable steps to ensure that there
are policies and that they are enforced.

\subsubsection{\color{darkgrey}gC142}

For the purpose of Rule rC110 training means any course of study
covering all the following areas:
\begin{alphlist}
\item Fair and effective selection \& avoiding unconscious bias

\item Attraction and advertising

\item Application processes

\item Shortlisting skills

\item Interviewing skills

\item Assessment and making a selection decision

\item Monitoring and evaluation
\end{alphlist}
\subsubsection{\color{darkgrey}gC143}

Training should ideally be undertaken via classroom sessions. However,
it is also permissible for training to be undertaken in the following
ways: online sessions, private study of relevant materials such as the
Bar Council's Fair Recruitment Guide and completion of CPD covering fair
recruitment and selection processes.

\subsubsection{\color{darkgrey}gC144}

The purpose of Rule rC110(3)(d) is to ensure that applicants with relevant
characteristics are not refused employment because of such
characteristics. In order to ensure compliance with this rule,
therefore, it is anticipated that the \emph{Equality and Diversity
Officer} will compile and retain data about the relevant characteristics
of all applicants for the purposes of reviewing the data in order to see
whether there are any apparent disparities in recruitment.

\subsubsection{\color{darkgrey}gC145}

For the purpose of Rule rC110 ``regular review'', means as often as is
necessary in order to ensure effective monitoring and review takes
place. In respect of data on pupils it is likely to be considered
reasonable that ``regularly'' should mean annually. In respect of
managers of a \emph{BSB entity} or tenants, it is likely to be
considered reasonable that ``regularly'' should mean every three years
unless the numbers change to such a degree as to make more frequent
monitoring appropriate.

\subsubsection{\color{darkgrey}gC146}

For the purposes of Rule rC110, ``remedial action'' means any action
aimed at removing or reducing the disadvantage experienced by particular
relevant groups. Remedial action cannot, however, include positive
discrimination in favour of members of relevant groups.

\subsubsection{\color{darkgrey}gC147}

Rule rC110(3)(f)(iii) places an obligation on \emph{practices} to take
reasonable steps to ensure the work opportunities are shared fairly
among its \emph{workforce}. In the case of \emph{chambers}, this
obligation includes work which has not been allocated by the solicitor
to a named \emph{barrister}. It includes fairness in presenting to
solicitors names for consideration and fairness in opportunities to
attract future named work (for example, fairness in arrangements for
marketing). These obligations apply even if individual members of
\emph{chambers} incorporate their practices, or use a ``ProcureCo'' to
obtain or distribute work, as long as their relationship between each
other remains one of independent service providers competing for the
same work while sharing clerking arrangements and costs.

\subsubsection{\color{darkgrey}gC148}

a) Rule rC110(3)(k) applies to all members of \emph{chambers},
irrespective of whether their partner or spouse takes \emph{parental
leave}.

b) A flexible policy might include for example: keeping in touch (KIT)
days; returns to practice in between periods of \emph{parental leave};
or allowing a carer to practise part time.

c) Any periods of leave/return should be arranged between
\emph{chambers} and members taking \emph{parental leave} in a way that
is mutually convenient.

\subsubsection{\color{darkgrey}gC149}

Rule rC110(3)(k)(vi) sets out the minimum requirements which \textcolor{myred}{\textbf{must}} be
included in a \emph{parental leave} policy if any element of rent is
paid on a flat rate. If rent is paid on any other basis, then the policy
should be drafted so as not to put any \emph{self-employed barrister} in
a worse position than they would have been in if any element of the rent
were paid on a flat rate.

\subsubsection{\color{darkgrey}gC150}

For the purposes of Rule rC110 above investigation means, considering
the reasons for disparities in data such as:
\begin{numlist}\item Under or overrepresentation of particular groups e(g). men, women,
different ethnic groups or disabled people
\item Absence of particular groups e(g). men, women, different ethnic groups
or disabled people

3 Success rates of particular groups
\item In the case of \emph{chambers}, over or under allocation of
unassigned work to particular groups
\end{numlist}
\subsubsection{\color{darkgrey}gC151}

These rules are supplemented by the BSB's Supporting Information on the
BSB Handbook Equality Rules (''the Supporting Information'') which is
available on the BSB's website. These describe the legal and regulatory
requirements relating to equality and diversity and provide guidance on
how they should be applied in \emph{chambers} and in \emph{BSB
entities}. If you are a \emph{self-employed barrister}, a \emph{BSB
entity}, or a \emph{manager} of a \emph{BSB entity}, you should seek to
comply with the Supporting Information as well as with the rules as set
out above.

\subsubsection{\color{darkgrey}gC152}

The Supporting Information is also relevant to all \emph{pupil
supervisors} and \emph{AETOs}. \emph{AETOs} will be expected to show how
they comply with the Supporting Information as a condition of
authorisation.

\subsubsection{\color{darkgrey}gC153}

Although the Supporting Information \textcolor{myred}{\textbf{does not}} apply directly to \emph{BSB
authorised persons} working as \emph{employed barristers (non-authorised
bodies)} or \emph{employed barristers (authorised non-BSB body)}, they
provide helpful guidance which you are encouraged to take into account
in your practice.

\rulesection{Rules
C113-C118 - Pupillage funding }


\subsection{Funding}

\rulesubsection{rc113}

The members of a set of \emph{chambers} or the \emph{BSB entity} \textcolor{myred}{\textbf{must}}
pay to each non-practising \emph{pupil} (as appropriate), by the end of
each month of the non-practising period of their \emph{pupillage} no
less than:
\begin{numlist}\item the \emph{specified amount}; and
\item such further sum as may be necessary to reimburse expenses reasonably
incurred by the \emph{pupil} on:
\item travel for the purposes of their \emph{pupillage} during that month;
and
\item attendance during that month at courses which they are required to
attend as part of their \emph{pupillage}.
\end{numlist}
\rulesubsection{rc114}

The members of a set of \emph{chambers}, or the \emph{BSB entity}, \textcolor{myred}{\textbf{must}}
pay to each practising \emph{pupil} by the end of each month of the
practising period of their \emph{pupillage} no less than:
\begin{numlist}
\item the \emph{specified amount}; plus

\item such further sum as may be necessary to reimburse expenses reasonably
incurred by the \emph{pupil} on:
\begin{alphlist}
\item travel for the purposes of their \emph{pupillage} during that month;
and

\item attendance during that month at courses which they are required to
attend as part of their \emph{pupillage}; less such amount, if any, as the \emph{pupil} may receive during that month
from their \emph{practice} as a \emph{barrister}; and less
\item such amounts, if any, as the \emph{pupil} may have received during
the preceding months of their practising \emph{pupillage} from their
\emph{practice} as a \emph{barrister}, save to the extent that the
amount paid to the \emph{pupil} in respect of any such month was less
than the total of the sums provided for in sub-paragraphs rC114(2)(a) and\item above.
\end{alphlist}
\end{numlist}

\rulesubsection{rc115}

The members of a set of \emph{chambers}, or the \emph{BSB entity}, may
not seek or accept repayment from a \emph{chambers pupil} or an entity
\emph{pupil} of any of the sums required to be paid under Rules rC113
and rC114 above, whether before or after they cease to be a chambers
pupil or an entity \emph{pupil}, save in the case of misconduct on their
part.

\rulesubsection{rc116}

If you are a \emph{self-employed barrister}, You \textcolor{myred}{\textbf{must}} pay any
\emph{chambers pupil} for any work done for you which because of its
value to you warrants payment, unless the \emph{pupil} is receiving an
award or remuneration which is paid on terms that it is in lieu of
payment for any individual item of work.

\subsection{Application}

\rulesubsection{rc117}

Removed.

\rulesubsection{rc118}

For the purposes of these requirements:
\begin{numlist}\item ``\emph{chambers pupil}'' means, in respect of any set of
\emph{chambers}, a \emph{pupil} doing the non-practising or practising
period of \emph{pupillage} with a \emph{pupil supervisor}, or
\emph{pupil supervisors}, who is or are a member, or members, of that
set of \emph{chambers};
\item ``entity \emph{pupil}'' means, in respect of a \emph{BSB entity} a
\emph{pupil} doing the non-practising or practising period of
\emph{pupillage} with a \emph{pupil supervisor} or \emph{pupil
supervisors} who are \emph{managers} or employees of such \emph{BSB
entity};
\item ``non-practising \emph{pupil}'' means a \emph{chambers pupil} or an
entity \emph{pupil} doing the non-practising period of \emph{pupillage};
\item ``practising \emph{pupil}'' means a \emph{chambers pupil} or an
entity \emph{pupil} doing the practising period of \emph{pupillage};
\item ``month'' means calendar month starting on the same day of the month
as that on which the \emph{pupil} began the non-practising, or
practising, period \emph{pupillage}, as the case may be;
\item any payment made to a \emph{pupil} by a \emph{barrister} pursuant to
Rule rC115 above shall constitute an amount received by the \emph{pupil}
from their \emph{practice} as a \emph{barrister}; and
\item the following travel by a \emph{pupil} shall not constitute travel
for the purposes of their \emph{pupillage}:
\begin{alphlist}\item travel between their home and \emph{chambers} or, for an entity
\emph{pupil}, their place of work; and
\item travel for the purposes of their \emph{practice} as a
\emph{barrister}.\end{alphlist}\end{numlist}

\chap{Part 2
- D2. Barristers undertaking public access and licensed access work
}



\ocsection{Outcomes
C30-C32 }


\subsection{\color{bleu}oC30}

\emph{Barristers} undertaking public access or licensed access work have
the necessary skills and experience required to do work on that basis.

\subsection{\color{bleu}oC31}

\emph{Barristers} undertaking public access or licensed access work
maintain appropriate records in respect of such work.

\subsection{\color{bleu}oC32}

\emph{Clients} only instruct via public access when it is in their
interests to do so and they fully understand what is expected of them.

\rulesection{Rules
C119-C131 - Public access rules }


\rulesubsection{rc119}

These rules apply to \emph{barristers} instructed by or on behalf of a
lay \emph{client} (other than a \emph{licensed access client}) who has
not also instructed a \emph{solicitor} or other \emph{professional
client} (public access clients). Guidance on public access rules is
available on the \emph{Bar Standards Board} website.

\rulesubsection{rc120}

Before accepting any \emph{public access instructions} from or on behalf
of a \emph{public access client}, you \textcolor{myred}{\textbf{must}}:
\begin{numlist}\item be properly qualified by having been issued with a full
\emph{practising certificate}, by having satisfactorily completed the
appropriate public access training, and by registering with
the~\emph{Bar Standards Board}~as a public access practitioner;
\item Removed from 1 February 2018.
\item take such steps as are reasonably necessary to ensure that the
\emph{client} is able to make an informed decision about whether to
apply for legal aid or whether to proceed with public access.
\end{numlist}
\rulesubsection{rc121}

As a barrister with less than three \emph{years' standing} who has
completed the necessary training You \textcolor{myred}{\textbf{must}} have a \emph{barrister} who is
a qualified person within Rule S22 and has registered with the \emph{Bar
Standards Board} as a public access practitioner readily available to
provide guidance to you.

\rulesubsection{rc122}

You \textcolor{myred}{\textbf{may not}} accept \emph{instructions} from or on behalf of a public
access \emph{client} if in all the circumstances, it would be in the
best interests of the public access \emph{client} or in the interests of
justice for the public access \emph{client} to instruct a
\emph{solicitor} or other \emph{professional client}.

\rulesubsection{rc123}

In any case where you are not prohibited from accepting
\emph{instructions}, You \textcolor{myred}{\textbf{must}} at all times consider the developing
circumstances of the case, and whether at any stage it is in the best
interests of the public access \emph{client} or in the interests of
justice for the public access \emph{client} to instruct a
\emph{solicitor} or other \emph{professional client}. If, after
accepting \emph{instructions} from a public access \emph{client} you
form the view that circumstances are such that it would be in the best
interests of the public access \emph{client}, or in the interests of
justice for the public access \emph{client} to instruct a
\emph{solicitor} or other \emph{professional client} you \textcolor{myred}{\textbf{must}}:
\begin{numlist}\item inform the public access \emph{client} of your view; and
\item withdraw from the case in accordance with the provisions of Rules
rC25 and rC26 and associated guidance unless the \emph{client} instructs
a \emph{solicitor} or other \emph{professional client} to act in the
case.
\end{numlist}
\rulesubsection{rc124}

You \textcolor{myred}{\textbf{must}} have regard to guidance published from time to time by the
\emph{Bar Standards Board} in considering whether to accept and in
carrying out any public access \emph{instructions}.

\rulesubsection{rc125}

Having accepted \emph{public access instructions}, You \textcolor{myred}{\textbf{must}} forthwith
notify your public access \emph{client} in writing, and in clear and
readily understandable terms, of:
\begin{numlist}\item the work which you have agreed to perform;
\item the fact that in performing your work you will be subject to the
requirements of Parts 2 and 3 of this \emph{Handbook} and, in
particular, Rules rC25 and rC26;
\item unless authorised to \emph{conduct litigation} by the \emph{Bar
Standards Board}, the fact that you cannot be expected to perform the
functions of a \emph{solicitor} or other \emph{person} who is authorised
to \emph{conduct litigation} and in particular to fulfil obligations
arising out of or related to the \emph{conduct of litigation};
\item the fact that you are self-employed, are not employed by a
\emph{regulated entity} and (subject to Rule S26) do not undertake the
management, administration or general conduct of a client's affairs;
\item in any case where you have been instructed by an \emph{intermediary}:
\begin{alphlist}\item the fact that you are independent of and have no liability for the
\emph{intermediary}; and
\item the fact that the \emph{intermediary} is the agent of the lay\end{alphlist}
\emph{client} and not your agent;
\item the fact that you may be prevented from completing the work by reason
of your professional duties or conflicting professional obligations, and
what the \emph{client} can expect of you in such a situation;
\item the fees which you propose to charge for that work, or the basis on
which your fee will be calculated;
\item your contact arrangements; and
\item the information about your complaints procedure required by D1(1) of
this Part 2.
\end{numlist}
\rulesubsection{rc126}

Save in exceptional circumstances, you will have complied with Rule
rC125 above if you have written promptly to the public access
\emph{client} in the terms of the model letter provided on the \emph{Bar
Standards Board} website.

\rulesubsection{rc127}

In any case where you have been instructed by an \emph{intermediary},
You \textcolor{myred}{\textbf{must}} give the notice required by Rule C125 above both:
\begin{numlist}\item directly to the public access \emph{client}; and
\item to the \emph{intermediary}.
\end{numlist}
\rulesubsection{rc128}

Having accepted \emph{public access instructions}, You \textcolor{myred}{\textbf{must}} keep a case
record which sets out:
\begin{numlist}\item the date of receipt of the \emph{instructions}, the name of the lay
\emph{client}, the name of the case, and any requirements of the
\emph{client} as to time limits;
\item the date on which the \emph{instructions} were accepted;
\item the dates of subsequent \emph{instructions}, of the despatch of
advices and other written work, of conferences and of telephone
conversations; and
\item when agreed, the fee.
\end{numlist}
\rulesubsection{rc129}

Having accepted \emph{public access instructions}, You \textcolor{myred}{\textbf{must}} either
yourself retain or take reasonable steps to ensure that the lay
\emph{client} will retain for at least seven years after the date of the
last item of work done:
\begin{numlist}\item copies of all \emph{instructions} (including supplemental
\emph{instructions});
\item copies of all advices given and documents drafted or approved;
\item the originals, copies or a list of all documents enclosed with any
\emph{instructions}; and
\item notes of all conferences and of all advice given on the telephone.
\end{numlist}
\rulesubsection{rc130}

Removed from 1 February 2018.

\rulesubsection{rc131}

Save where otherwise agreed:
\begin{numlist}\item you shall be entitled to copy all documents received from your lay
\emph{client}, and to retain such copies;
\item you shall return all documents received from your lay \emph{client}
on demand, whether or not you have been paid for any work done for the
lay \emph{client}; and
\item you shall not be required to deliver to your lay \emph{client} any
documents drafted by you in advance of receiving payment from the lay
\emph{client} for all work done for that \emph{client}.
\item Removed from 1 February 2018.
\end{numlist}
\rulesection{Rules
C132-C141 - Licensed access rules }


\rulesubsection{rc132}

Subject to these rules and to compliance with the Code of Conduct (and
to the \emph{Scope of Practice, Authorisation and Licensing Rules}) a
barrister in self-employed practice may accept \emph{instructions} from
a \emph{licensed access client} in circumstances authorised in relation
to that \emph{client} by the Licensed Access Recognition Regulations
(which are available on the BSB's website) whether that \emph{client} is
acting for themselves or another.

\rulesubsection{rc133}

These rules apply to every matter in which a \emph{barrister} in
self-employed \emph{practice} is instructed by a \emph{licensed access
client} save that Rules rC134(2) and rC139 do not apply to any matter in
which a \emph{licensed access client} is deemed to be a \emph{licensed
access client} by reason only of paragraph 7 or paragraph 8 of the
Licensed Access Recognition Regulations (which are available on the
BSB's website).

\rulesubsection{rc134}

You are only entitled to accept \emph{instructions} from a
\emph{licensed access client} if at the time of giving instructions the
\emph{licensed access client}:
\begin{numlist}\item is identified; and
\item you ensure that the \emph{licensed access client} holds a valid
Licence issued by the \emph{Bar Standards Board} (either by requiring
the \emph{licensed access client} to send you a copy of the Licence, or
referring to the list of \emph{licensed access clients} published on the
\emph{Bar Standards Board} website).
\end{numlist}
\rulesubsection{rc135}

You \textcolor{myred}{\textbf{must not}} accept any \emph{instructions} from a \emph{licensed access
client}:
\begin{numlist}\item unless you are able to provide the services required of you by that
\emph{licensed access client};
\item if you consider it in the interests of the lay \emph{client} or the
interests of justice that a \emph{solicitor} or other \emph{person} who
is authorised to \emph{conduct litigation} or some other appropriate
\emph{intermediary} (as the case may be) be instructed either together
with you or in your place.
\end{numlist}
\rulesubsection{rc136}

If you agree standard terms with a \emph{licensed access client}, you
must keep a copy of the agreement in writing with the \emph{licensed
access client} setting out the terms upon which you have agreed and the
basis upon which you are to be paid.

\rulesubsection{rc137}

Having accepted \emph{instructions} from a \emph{licensed access
client}, You \textcolor{myred}{\textbf{must}} promptly send the \emph{licensed access client}:
\begin{numlist}
\item a statement in writing that the \emph{instructions} have been
accepted (as the case may be) on the standard terms previously agreed in
writing with that \emph{licensed access client}; or

\item if you have accepted \emph{instructions} otherwise than on such
standard terms, a copy of the agreement in writing with the
\emph{licensed access client} setting out the terms upon which you have
agreed to do the work and the basis upon which you are to be paid; and

\item unless you have accepted \emph{instructions} on standard terms which
incorporate the following particulars \textcolor{myred}{\textbf{must}} at the same time advise the
\emph{licensed access client} in writing of:
\begin{alphlist}
\item the effect of rC21 as it relevantly applies in the circumstances;

\item unless authorised by the \emph{Bar Standards Board} to \emph{conduct
litigation}, the fact that you cannot be expected to perform the
functions of a \emph{solicitor} or other \emph{person} who is authorised
to \emph{conduct litigation} and in particular to fulfil obligations
arising out of or related to the conduct of litigation; and

\item the fact that circumstances may require the \emph{client} to retain a
\emph{solicitor} or other \emph{person} who is authorised to
\emph{conduct litigation} at short notice and possibly during the case.
\end{alphlist}\end{numlist}
\rulesubsection{rc138}

If at any stage you, being instructed by a \emph{licensed access
client}, consider it in the interests of the lay \emph{client} or the
interests of justice that a \emph{solicitor} or other \emph{person} who
is authorised to \emph{conduct litigation} or some other appropriate
\emph{intermediary} (as the case may be) be instructed either together
with you or in your place:
\begin{numlist}
\item You \textcolor{myred}{\textbf{must}} forthwith advise the \emph{licensed access client} in
writing to instruct a \emph{solicitor} or other \emph{person} who is
authorised to \emph{conduct litigation} or other appropriate
\emph{intermediary} (as the case may be); and

\item unless a \emph{solicitor} or other \emph{person} who is authorised to
\emph{conduct litigation} or other appropriate \emph{intermediary} (as
the case may be) is instructed as soon as reasonably practicable
thereafter You \textcolor{myred}{\textbf{must}} cease to act and \textcolor{myred}{\textbf{must}} return any
\emph{instructions}.
\end{numlist}
\rulesubsection{rc139}

If at any stage you, being instructed by a \emph{licensed access
client}, consider that there are substantial grounds for believing that
the \emph{licensed access client} has in some significant respect failed
to comply with the terms of the Licence granted by the \emph{Bar
Standards Board} You \textcolor{myred}{\textbf{must}} forthwith report the facts to the \emph{Bar
Standards Board}.

\rulesubsection{rc140}

Having accepted \emph{instructions} from a \emph{licensed access
client}, You \textcolor{myred}{\textbf{must}} keep a case record which sets out:
\begin{numlist}\item the date of receipt of the \emph{instructions}, the name of the
\emph{licensed access client}, the name of the case, and any
requirements of the \emph{licensed access client} as to time limits;
\item the date on which the \emph{instructions} were accepted;
\item the dates of subsequent \emph{instructions}, of the despatch of
advices and other written work, of conferences and of telephone
conversations; and
\item when agreed, the fee.
\end{numlist}
\rulesubsection{rc141}

Having accepted \emph{instructions} from a \emph{licensed access
client}, You \textcolor{myred}{\textbf{must}} either yourself retain or take reasonable steps to
ensure that the \emph{licensed access client} will retain for seven
years after the date of the last item of work done:
\begin{numlist}\item copies of \emph{instructions} (including supplemental
\emph{instructions});
\item copies of all advices given and documents drafted or approved;
\item a list of all documents enclosed with any \emph{instructions}; and
\item notes of all conferences and of all advice given on the telephone.
\end{numlist}
\chap{Part 2
- D3. Registered European lawyers }




\ocsection{Outcome%
 C33}


\subsection{\color{bleu}oC33}

\emph{Clients} are not confused about the qualifications and status of
\emph{registered European lawyers}.

\rulesection{Rules
C142-C143 }


\rulesubsection{rc142}

If you are a \emph{registered European lawyer} and not a
\emph{barrister}, You \textcolor{myred}{\textbf{must not}} hold yourself out to be a
\emph{barrister}.

\rulesubsection{rc143}

You \textcolor{myred}{\textbf{must}} in connection with all professional work undertaken in England
and Wales as a \emph{registered European lawyer}:
\begin{numlist}\item use your \emph{home professional title};
\item indicate the name of your \emph{home professional body} or the
\emph{court} before which you are entitled to practise in that
\emph{Member State}; and
\item indicate that you are registered with the \emph{Bar Standards Board}
as a \emph{European lawyer}.
\end{numlist}
\chap{Part 2
- D4. Unregistered barristers }



\ocsection{Outcome C34}


\subsection{\color{bleu}oC34}

\emph{Clients} who receive \emph{legal services} from \emph{unregistered
barristers} are aware that such \emph{unregistered barristers} are not
subject to the same regulatory safeguards that would apply if they
instructed a \emph{practising barrister}.

\rulesection{Rule
C144 }


\rulesubsection{rc144}

If you are an \emph{unregistered barrister} and you supply \emph{legal
services} (other than as provided for in Rule rC145) to any
inexperienced \emph{client} then, before supplying such services:
\begin{numlist}\item You \textcolor{myred}{\textbf{must}} explain to the \emph{client} that:
\begin{alphlist}\item (unless you are supplying \emph{legal services} pursuant to Rule S12)
you are not acting as a \emph{barrister};
\item you are not subject to those parts of the Code of Conduct and other
provisions of this \emph{Handbook} which apply only to \emph{BSB
authorised persons};
\item the \emph{Bar Standards Board} will only consider \emph{reports}
about you which concern the Core Duties or those parts of the Code of
Conduct and other provisions of this \emph{Handbook} which apply to you;
\item (unless you are covered by professional indemnity insurance) you are
not covered by professional indemnity insurance;
\item they have the right to make a \emph{complaint}, how they can
complain, to whom, of any time limits for making a \emph{complaint} but
that they have no right to complain to the \emph{Legal Ombudsman} about
the services you supply; and
\item in respect of any legal advice you provide, there is a substantial
risk that they will not be able to rely on legal professional privilege.\end{alphlist}

\item You \textcolor{myred}{\textbf{must}} get written confirmation from the \emph{client} that you
have given this explanation.

For the purposes of this Rule rC144, an inexperienced \emph{client}
includes any individual or other person who would, if you were a
\emph{BSB authorised person}, have a right to bring a complaint pursuant
to the \emph{Legal Ombudsman} Scheme Rules.

\end{numlist}
\guidancesection{Guidance
to Rule C144 }


\subsubsection{\color{darkgrey}gC154}

For the purposes of determining whether Rule rC144 applies, the people
who would be entitled to complain to the \emph{Legal Ombudsman} if you
were a \emph{BSB authorised person} are:
\begin{numlist}\item an individual; or
\item a business or enterprise that was a micro-enterprise within the
meaning of Article 1 and Article 2(1) and (3) of the Annex to Commission
Recommendation 2003/361/EC (broadly a business or enterprise with fewer
than 10 employees and turnover or assets not exceeding €2 million), when
it referred the \emph{complaint} to you; or

3 a charity with an annual income net of tax of less than £1 million at
the time at which the complainant refers the \emph{complaint} to you; or
\item a club, association or organisation, the affairs of which are managed
by its members or a committee of its members, with an annual income net
of tax of less than £1 million at the time at which the complainant
refers the \emph{complaint} to you; or
\item a trustee of a trust with an asset value of less than £1 million at
the time at which the complainant refers the \emph{complaint} to you; or
\item a personal representative or beneficiary of the estate of a person
who, before they died, had not referred the complaint to the \emph{Legal
Ombudsman}.\end{numlist}
\rulesection{Rule
C145 }


\rulesubsection{rc145}

rC144 \textcolor{myred}{\textbf{does not}} apply to you if you supply \emph{legal services}:
\begin{numlist}\item as an employee or \emph{manager} of a \emph{regulated entity};
\item as an employee or \emph{manager} of a body subject to regulation by a
professional body or regulator;
\item as provided for in Section S(b)9 (\emph{Legal Advice Centres});
\item pursuant to an authorisation that you have obtained from another
\emph{approved regulator}; or
\item in accordance with Rules S13 and S14.

\end{numlist}
\guidancesection{Guidance
to Rule C145 }


\subsubsection{\color{darkgrey}gC155}

Guidance on the disclosures which unregistered barristers should
consider making to \emph{clients} covered by Rule rC145, and other
\emph{clients} who are not inexperienced \emph{clients}, to ensure that
they comply with Rule rC19 and do not mislead those \emph{clients} is
available on BSB website.

\chap{Part 2
- D5. Cross-border activities between CCBE States }



\ocsection{Outcome C35}


\subsection{\color{bleu}oC35}

\emph{BSB regulated persons} who undertake \emph{cross-border
activities} comply with the terms of the \emph{Code of Conduct for
European Lawyers}.

\rulesection{Rule
C146 }


\rulesubsection{rc146}

If you are a \emph{BSB regulated person} undertaking \emph{cross-border
activities} then, in addition to complying with the other provisions of
this \emph{Handbook} which apply to you, You \textcolor{myred}{\textbf{must}} also comply with Rules
rC147 to rC158 below.


\guidancesection{Guidance
to Rule C146 }


\subsubsection{\color{darkgrey}gC156}

Where the \emph{cross-border activities} constitute \emph{foreign work}
(in other words, limb (a) of the definition of \emph{cross-border
activities}), you should note, in particular, Rules rC13 and rC14 and
the associated guidance.

\subsubsection{\color{darkgrey}gC157}

The purpose of this section D5 is to implement those provisions of the
\emph{Code of Conduct for European Lawyers} which are not otherwise
covered by the \emph{Handbook}. If a provision of the \emph{Code of
Conduct for European Lawyers} has not been included here then the
equivalent provisions of \emph{Handbook} need to be complied with in
respect of all \emph{cross-border activities} (including where they
place a higher burden on the \emph{BSB regulated person} than the
\emph{Code of Conduct for European Lawyers} itself which is the case,
for example, in respect of the handling of \emph{client money} (Rule
rC73 and rC74)).

\rulesection{Rules
C147-C158 }


\subsection{Incompatible occupations}

\rulesubsection{rc147}

If you act in legal proceedings or proceedings before public authorities
in a \emph{CCBE State} other than the \emph{UK}, you \textcolor{myred}{\textbf{must}}, in that
\emph{CCBE State}, observe the Rules regarding incompatible occupations
as they are applied to lawyers of that \emph{CCBE State}.

\rulesubsection{rc148}

If you are established in a \emph{CCBE State} other than the \emph{UK}
and you wish to participate directly in commercial or other activities
not connected with the practice of the law in that \emph{CCBE State},
You \textcolor{myred}{\textbf{must}} respect the Rules regarding forbidden or incompatible
occupations as they are applied to lawyers of that \emph{CCBE State}.

\subsection{Fee sharing with non-lawyers}

\rulesubsection{rc149}

You \textcolor{myred}{\textbf{must not}} share your fees with a person situated in a \emph{CCBE
State} other than the \emph{UK} who is not a lawyer except where
otherwise permitted by the terms of this \emph{Handbook} or Rule rC150
below.

\rulesubsection{rc150}

Rule rC149 shall not preclude you from paying a fee, commission or other
compensation to a deceased lawyer's heirs or to a retired lawyer in
respect of taking over the deceased or retired lawyer's practice.

\subsection{Co-operation among lawyers of different member states}

\rulesubsection{rc151}

If you are approached by a lawyer of a \emph{CCBE State} other than the
UK to undertake work which you are not competent to undertake, you \textcolor{myred}{\textbf{must}}
assist that lawyer to obtain the information necessary to find and
instruct a lawyer capable of providing the service asked for.

\rulesubsection{rc152}

When co-operating with a lawyer of a \emph{CCBE State} other than the UK
You \textcolor{myred}{\textbf{must}} take into account the differences which may exist between your
respective legal systems and the professional organisations,
competencies and obligations of lawyers in your respective states.

\subsection{Correspondence between lawyers in different CCBE states}

\rulesubsection{rc153}

If you want to send to a lawyer in a \emph{CCBE State} other than the UK
a communication which you wish to remain ``confidential'' or ``without
prejudice'', you \textcolor{myred}{\textbf{must}}, before sending the communication, clearly express
your intention in order to avoid misunderstanding, and ask if the lawyer
is able to accept the communication on that basis.

\rulesubsection{rc154}

If you are the intended recipient of a communication from a lawyer in
another \emph{CCBE State} which is stated to be ``confidential'' or
``without prejudice'', but which you are unable to accept on the basis
intended by that lawyer, You \textcolor{myred}{\textbf{must}} inform that lawyer accordingly without
delay.

\subsection{Responsibility for fees}

\rulesubsection{rc155}

If in the course of practice you instruct a lawyer of a \emph{CCBE
State} other than the UK to provide \emph{legal services} on your
behalf, You \textcolor{myred}{\textbf{must}} pay the fees, costs and outlays which are properly
incurred by that lawyer (even where the \emph{client} is insolvent)
unless:
\begin{numlist}\item you were simply introducing the \emph{client} to them and the lawyer
of the \emph{CCBE State} other than the UK has since had a direct
contractual relationship with the \emph{client}; or
\item you have expressly disclaimed that responsibility at the outset, or
at a later date you have expressly disclaimed responsibility for any
fees incurred after that date; or the lawyer of the \emph{CCBE State}
other than the UK is, in the particular matter, practising as a lawyer
in England or Wales (whether authorised by the \emph{BSB} or any other
\emph{Approved Regulator}).\end{numlist}

\subsection{Disputes amongst lawyers in different member states}

\rulesubsection{rc156}

If you consider that a lawyer in a \emph{CCBE State} other than the UK
has acted in breach of a rule of professional conduct You \textcolor{myred}{\textbf{must}} draw the
breach to the other lawyer's attention.

\rulesubsection{rc157}

If any personal dispute of a professional nature arises between you and
a lawyer in a \emph{CCBE State} other than the UK You \textcolor{myred}{\textbf{must}} first try to
settle it in a friendly way.

\rulesubsection{rc158}

You \textcolor{myred}{\textbf{must not}} commence any form of proceedings against a lawyer in a
\emph{CCBE State} other than the UK on matters referred to in Rules
rC156 or rC157 without first informing the \emph{Bar Council} and the
other lawyer's bar or law society in order to allow them an opportunity
to assist in resolving the matter.

\chap{Part 2
- D6. Price and service transparency rules for self-employed barristers,
chambers and BSB entities }



\ocsection{Outcome 
C36 }


\subsection{\color{bleu}oC36}

\emph{Clients} are provided with appropriate information to help them
make informed choices and understand the price and service they will
receive.

\rulesection{Rules
C159-C163 - Self-employed barristers, chambers and BSB entities }


\subsection{Publication of information}

\rulesubsection{rc159}

Each website of \emph{self-employed barristers}, \emph{chambers} and
\emph{BSB entities} \textcolor{myred}{\textbf{must}}, in a sufficiently accessible and prominent
place:
\begin{numlist}\item state that professional, licensed access and/or lay clients (as
appropriate) may contact the \emph{barrister}, \emph{chambers} or
\emph{BSB entity} to obtain a quotation for \emph{legal services} and
provide contact details. Quotations \textcolor{myred}{\textbf{must}} be provided if sufficient
information has been provided by the \emph{client}, and the
\emph{barrister}, \emph{barristers} in \emph{chambers} or \emph{BSB
entity} would be willing to provide the legal services. Quotations \textcolor{myred}{\textbf{must}}
be provided within a reasonable time period, and in clear and readily
understandable terms;
\item state their most commonly used pricing models for \emph{legal
services}, such as fixed fee or hourly rate. Where different models are
typically used for different \emph{legal services}, this \textcolor{myred}{\textbf{must}} be
explained;
\item state the areas of law in which they most commonly provide
\emph{legal services}, and state and describe the \emph{legal services}
which they most commonly provide, in a way which enables clients to
sufficiently understand the expertise of the \emph{barrister},
\emph{chambers} or \emph{BSB entity}; and
\item provide information about the factors which might influence the
timescales of their most commonly provided \emph{legal services}.
\end{numlist}
\rulesubsection{rc160}

All \emph{self-employed barristers}, \emph{chambers} and \emph{BSB
entities} \textcolor{myred}{\textbf{must}} review their website content at least annually to ensure
that it is accurate and complies with the transparency requirements
referred to in Rules C103, C159 and where applicable, Rules C164 --
C168.

\rulesubsection{rc161}

\emph{Self-employed barristers}, \emph{chambers} and \emph{BSB entities}
must comply with the transparency requirements referred to in Rules
C103, C159 and where applicable, Rules C164 -- C168 by ensuring the
required information is readily available in alternative format. This
must be provided on request (for example, if they do not operate a
website, or a \emph{client} or prospective \emph{client} \textcolor{myred}{\textbf{does not}} have
Internet access).

\subsection{Provision of information to the Bar Standards Board}

\rulesubsection{rc162}

All \emph{self-employed barristers}, \emph{chambers} and \emph{BSB
entities} \textcolor{myred}{\textbf{must}} notify the \emph{Bar Standards Board} of their website
address(es) offering \emph{legal services}, and any changes to their
website address(es), within 28 days of the creation or change of the
same.

\guidancesubsection{Bar Standards Board guidance}

\rulesubsection{rc163}

When offering their services to \emph{clients} and prospective
\emph{clients}, all \emph{self-employed barristers}, \emph{chambers} and
\emph{BSB entities} \textcolor{myred}{\textbf{must}} have regard to guidance published from time to
time by the \emph{Bar Standards Board} in relation to price and service
transparency.

\rulesection{Rules
C164-C169 - Self-employed barristers undertaking public access work and
BSB entities supplying legal services directly to the public }


\subsection{Public Access Guidance for Lay Clients}

\rulesubsection{rc164}

Each website of \emph{self-employed barristers} undertaking public
access work and/or their chambers, and \emph{BSB entities} supplying
\emph{legal services} directly to the public, \textcolor{myred}{\textbf{must}} in a sufficiently
accessible and prominent place display a link to the Public Access
Guidance for Lay Clients on the BSB's website.

\subsection{Price transparency policy statement}

\rulesubsection{rc165}

\emph{Self-employed barristers} undertaking public access work and/or
their \emph{chambers}, and \emph{BSB entities} supplying \emph{legal
services} directly to the public, \textcolor{myred}{\textbf{must}} comply with the \emph{Bar
Standards Board's} price transparency policy statement insofar as it
applies to them.

\subsection{Publication of information}

\rulesubsection{rc166}

\emph{Self-employed barristers} undertaking public access work and/or
their \emph{chambers}, and \emph{BSB entities} supplying \emph{legal
services} directly to the public, are required by the \emph{Bar
Standards Board's} price transparency policy statement to provide price
information in relation to certain \emph{legal services} in certain
circumstances. In relation to those \emph{legal services} and in those
circumstances, each website of \emph{self-employed barristers}
undertaking public access work and/or their \emph{chambers}, and
\emph{BSB entities} supplying \emph{legal services} directly to the
public, \textcolor{myred}{\textbf{must}} in a sufficiently accessible and prominent place:
\begin{numlist}\item state their pricing model(s), such as fixed fee or hourly rate;
\item state their indicative fees and the circumstances in which they may
vary. For example, a fixed fee and the circumstances in which additional
fees may be charged, or an hourly rate by seniority of \emph{barrister};
\item state whether their fees include VAT (where applicable); and
\item state likely additional costs, what they cover and either the cost
or, if this can only be estimated, the typical range of costs.
\end{numlist}
\rulesubsection{rc167}

In compliance with the requirements of Rule C166 above:
\begin{numlist}\item a sole practitioner \textcolor{myred}{\textbf{must}} provide price information in relation to
them as an individual \emph{barrister};
\item a \emph{BSB entity} \textcolor{myred}{\textbf{must}} provide price information in relation to the
entity; and
\item a \emph{chambers} may provide price information either in relation to\begin{numlist}\item individual \emph{barristers}, or \item \emph{barristers} in
\emph{chambers} in the form of ranges or average fees.
\end{numlist}\end{numlist}


\rulesubsection{rc168}

\emph{Self-employed barristers} undertaking public access work and/or
their \emph{chambers}, and \emph{BSB entities} supplying \emph{legal
services} directly to the public, are required by the \emph{Bar
Standards Board's} price transparency policy statement to provide
service information in relation to certain \emph{legal services} in
certain circumstances. In relation to those legal services and in those
circumstances, each website of \emph{self-employed barristers}
undertaking public access work and/or their \emph{chambers}, and
\emph{BSB entities} supplying \emph{legal services} directly to the
public, \textcolor{myred}{\textbf{must}} in a sufficiently accessible and prominent place:
\begin{numlist}\item state and describe the \emph{legal services}, including a concise
statement of the key stages, in a way which enables \emph{clients} to
sufficiently understand the service of the sole practitioner,
\emph{barristers} in chambers or \emph{BSB entity}; and
\item provide an indicative timescale for the key stages of the \emph{legal
services}.
\end{numlist}

\rulesubsection{rc169}

\emph{Self-employed barristers} undertaking public access work, and BSB
entities supplying \emph{legal services} directly to the public, may be
asked to accept instructions to provide the \emph{legal services} listed
in the \emph{Bar Standards Board's} price transparency policy statement
at short notice. In these circumstances, you are not required to comply
with Rules C166 -- C168 above before accepting the \emph{instructions}.
However, You \textcolor{myred}{\textbf{must}} do so as soon as reasonably practicable after
accepting the \emph{instructions}.

