\part{Enforcement Regulations}


\rulesection{Regulation E1 }Regulations


\rulesubsection{rE1}
These regulations set out the powers and functions in relation to: the
assessment of \emph{reports} and the investigation
of \emph{allegations} which may indicate a potential breach of
the \emph{Handbook} or require regulatory action; the decisions
available to the \emph{Commissioner }and an\emph{ Independent
Decision-Making Panel }at the conclusion of an investigation; the
reconsideration of \emph{allegations} that have been disposed of; and
the disclosure of \emph{reports} or \emph{allegations} by the \emph{Bar
Standards Board}. These regulations also set out the operation of
the \emph{administrative sanction }appeal procedure and
the \emph{determination by consent procedure}\par
\rulesection{Part 5 - A1. The assessment of reports }\par
Powers of the \emph{Commissioner} in relation to the assessment
of \emph{reports}\par
\rulesubsection{rE2}\par
The powers of the \emph{Commissioner} include (but are not limited
to):\\\nl \item gathering information relating to \emph{applicable persons} from any
source for the purposes of assessing whether there has been a potential
breach of the \emph{Handbook}; and\item exercising the power under rE12 to determine that a \emph{report} or
part of a \emph{report} may be treated as an \emph{allegation.}\ln
\rulesubsection{rE3}\par
The \emph{Commissioner} shall have the power to authorise
any \emph{person}, group or body to fulfil any function or exercise any
power given to the \emph{Commissioner} by this Section 5.A. Any
authorisations given  \textcolor{myred}{\textbf{must}} be in writing and may be either or both
retrospective and prospective, and either or both general and for a
particular purpose.\\
\rulesubsection{reference to the Legal Ombudsman}\par
\rulesubsection{rE4}\par
If a \emph{report} is received by the \emph{Bar Standards Board} from a
person entitled to complain to the \emph{Legal Ombudsman} about the
subject of the \emph{report}, the \emph{Commissioner}  \textcolor{myred}{\textbf{must}} refer
the \emph{report} without further consideration to the \emph{Legal
Ombudsman} or signpost the provider of the \emph{report} to
the \emph{Legal Ombudsman}.\\
\rulesubsection{reference where an \emph{applicable person} acting in judicial
or \emph{quasi-judicial} capacity}\par
\rulesubsection{rE5}\par
If it appears to the \emph{Commissioner} that a \emph{report} relates to
an \emph{applicable person}'s actions in a part-time or temporary
judicial or \emph{quasi-judicial} capacity, the \emph{Commissioner}  \textcolor{myred}{\textbf{must}}
refer the \emph{report} to the person or body responsible for the
appointment of the \emph{applicable person} to the judicial
or \emph{quasi-judicial} office concerned or another  person or body
responsible for considering such reports (``the appropriate body''),
where it appears that the appropriate body should consider
the \emph{report}, requesting notification of the outcome of the
appropriate body's consideration as soon as it has been dealt with,
subject to rE6 to rE8 below.\\
\rulesubsection{rE6}\par
Where:\\\nl \item the appropriate body refuses to deal with the \emph{report; }or\item it appears there is no appropriate body\ln
the \emph{Commissioner} may consider the \emph{report} in accordance
with the provisions of this Section 5.A.\\
\rulesubsection{rE7}\par
When the appropriate body has dealt with the \emph{report}, or
the \emph{Commissioner} considers that the appropriate body has not
dealt with it within a reasonable time or fully or satisfactorily,
the \emph{Commissioner} may consider the \emph{report} in accordance
with the provisions of this Section 5.A. and may consider any finding
made and any action taken by the appropriate body.
\rulesubsection{rE8}\par
The \emph{Commissioner}  \textcolor{myred}{\textbf{must not}} consider or take action in relation to
a \emph{report} arising in substance from dissatisfaction or
disagreement with anything decided, done or said by the \emph{applicable
person} in the proper exercise of their judicial
or \emph{quasi-judicial} functions.\\
\rulesubsection{reference to the Lord Chancellor or other appropriate body}\par
\rulesubsection{rE9}\par
If it appears to the \emph{Commissioner} that the \emph{report} relates
to the conduct of an \emph{applicable person} who, since the events
giving rise to the \emph{report} took place, has been appointed to and
continues to hold full-time judicial office and has ceased to practise,
the \emph{Commissioner} shall not consider the \emph{report} further and
must direct the person from whom the \emph{report} is received to the
Lord Chancellor or the Judicial Conduct Investigations Office or to such
other person or appropriate body with responsibility for addressing
complaints about judges.\\
\rulesubsection{reference to any other person}\par
\rulesubsection{rE10}\par
If it appears to the \emph{Commissioner} that a \emph{report} in respect
of an \emph{applicable person} might more appropriately be dealt with by
another body (e.g. an \emph{Inn}, Circuit, employer, a complaint
handling body or any other professional or regulatory body),
the \emph{Commissioner} may refer the \emph{report} to such other
body.\\
\rulesubsection{rE11}\par
If, having referred a \emph{report} to another body under rE10,
the \emph{Commissioner} subsequently considers that
the \emph{report }has not been dealt with by that other body within a
reasonable time or fully or satisfactorily, the \emph{Commissioner} may
choose to exercise the powers set out in rE2.1 and rE2.2 above.\\
\textbf{Initial assessment of \emph{reports}}\par
\rulesubsection{rE12}\par
Where the \emph{Commissioner, }having regard to rE13, considers that
a \emph{report}:\\\nl \item discloses a potential breach of the \emph{Handbook }by
an \emph{applicable person}; and/or\item potentially satisfies the \emph{disqualification condition}\par
the \emph{Commissioner} may treat the \emph{report} as
an \emph{allegation}.\ln
\rulesubsection{rE13}\par
In determining whether to treat a \emph{report }as
an \emph{allegation} under rE12 the \emph{Commissioner}  \textcolor{myred}{\textbf{must}} have regard
to:\\\nl \item whether the conduct disclosed in the \emph{report} or its
consequences presents sufficient risk to the \emph{regulatory
objectives} to justify further action;
\item whether the conduct disclosed in the \emph{report} can be properly
and fairly investigated; and
\item whether the conduct disclosed in the \emph{report} could not more
appropriately be dealt with under one or more of the provisions set out
at rE4 to rE11 above.\ln
\rulesection{Part 5 - A2. Investigation of allegations }\par
Powers of the \emph{Commissioner} in relation to the investigation
of \emph{allegations}\par
\rulesubsection{rE14}\par
The powers of the \emph{Commissioner} include (but are not limited to)
the power at any time:\\\nl \item to carry out the investigation of \emph{allegations} as appropriate;
and\item to withdraw any \emph{allegation} and treat it as if a decision under
rE12 had not been made.\ln
\textbf{Investigating \emph{allegations}}\par
\rulesubsection{rE15}\par
The \emph{Commissioner}  \textcolor{myred}{\textbf{must not}} conclude any investigation of
an \emph{allegation} without taking reasonable steps to ensure that
the \emph{applicable person} has been informed of
the \emph{allegation} and given a reasonable opportunity to comment on
the \emph{allegation}.\\
\rulesubsection{rE16}\par
If a new \emph{report} comes to light during an investigation of
an \emph{allegation} that meets the criteria of rE12, it may be treated
as a new \emph{allegation} and investigated in accordance with the
provisions of Section 5.A.\\
\rulesubsection{rE17}\par
The \emph{Commissioner} may defer further consideration of the
original \emph{allegation} until a new \emph{allegation} has been
investigated.\\
\rulesubsection{rE18}\par
No further investigation or opportunity to respond is required where the
subject matter of a new \emph{allegation} has already been investigated
by the \emph{Commissioner} and the \emph{applicable person} has already
been given an opportunity to comment on it during the original
investigation.\\
\rulesection{Part 5 - A3. Possible outcomes of the investigation of an
allegation }\par
Powers of the \emph{Commissioner} in relation to the conclusion of
investigations\\
\rulesubsection{rE19}\par
At the conclusion of an investigation of
an \emph{allegation} the \emph{Commissioner} has the power to decide:\\\nl \item that the conduct alleged did not constitute a breach of
the \emph{Handbook,} or that there was insufficient evidence of a breach
of the \emph{Handbook} (on the civil standard of proof\emph{)};
\item that the conduct alleged did constitute a breach of
the \emph{Handbook} (on the civil standard of proof) but that, in all
the circumstances, no enforcement action should be taken in respect of
the breach;
\item that the conduct alleged did constitute a breach of
the \emph{Handbook} (on the civil standard of proof) and that the breach
should be dealt with by the imposition of an \emph{administrative
sanction};
\item that the conduct alleged may constitute a breach of
the \emph{Handbook} and, if the breach were to be proved, that
an \emph{administrative sanction} under rE19.3 would not be appropriate
in all the circumstances, and that the subject matter of
the \emph{allegation} against an \emph{applicable person} involves:\\
\al
\item a conviction for an offence of dishonesty or deception; or\item a conviction for an offence under Section 4, Section 5 or Section 5A
Road Traffic Act 1988 (Driving or being in charge of a motor vehicle
with alcohol concentration/concentration of a controlled drug above
prescribed limit); or\item a breach of Part 3 or 4 of the \emph{Handbook}; or\item any failure to pay an administrative fine within the relevant time;
or\item a failure to comply with any requirements of a sanction imposed
following \emph{Disciplinary Action;}\la
in which case the \emph{allegation} may form the subject matter of a
referral to \emph{Disciplinary Action}; or\item to refer an \emph{allegation} to an \emph{Independent Decision-Making
Panel }for a decision.\ln
\rulesubsection{rE20}\par
In conjunction with a decision under rE19, or following a recommendation
from an \emph{Independent Decision-Making Panel} under
rE23, the \emph{Commissioner} may refer any \emph{allegation} for
supervisory action.\\
\rulesubsection{rE21}\par
In conjunction with a decision under rE19.1 or rE19.2
the \emph{Commissioner} may issue the \emph{applicable person} with
advice.\\
\textbf{Powers of an \emph{Independent Decision-Making Panel}\emph{ }in
relation to \emph{allegations} referred to it}\par
\rulesubsection{rE22}\par
Where an \emph{allegation} has been referred to an \emph{Independent
Decision-Making Panel }under rE19.5 the \emph{Independent
Decision-Making Panel }has the power to decide:\\\nl \item that, on the evidence before it, the conduct alleged did not
constitute a breach of the \emph{Handbook, }or that there was
insufficient evidence of a breach of the \emph{Handbook} (on the civil
standard of proof); or\item that, on the evidence before it, the conduct alleged did constitute a
breach of the \emph{Handbook} (on the civil standard of proof) but that,
in all the circumstances, no enforcement action should be taken in
respect of the breach; or\item that, on the evidence before it, the conduct alleged did constitute a
breach of the \emph{Handbook} (on the civil standard of proof) and that
the breach should be dealt with by an \emph{administrative sanction};
or\item that
\al \item there is a \emph{realistic prospect of a finding of professional
misconduct being made} or there is \item a \emph{realistic prospect of the
disqualification condition being satisfied}, and\item  having regard to the \emph{regulatory objectives}, it is in the
public interest to pursue \emph{Disciplinary Action}\la
in which case the \emph{allegation}  \textcolor{myred}{\textbf{must}} form the subject matter
of \emph{Disciplinary Action}.\\
\rulesubsection{rE23}\par
In conjunction with a decision under rE22 the \emph{Independent
Decision-Making Panel} may recommend the matter be referred for
supervisory action.\\
\rulesubsection{rE24}\par
In conjunction with a decision under rE22.1 or rE22.2
the \emph{Independent Decision-Making Panel }may issue
the \emph{applicable person} with advice.\\
\textbf{\emph{Independent Decision-Making
Panel} and \emph{Commissioner} powers/requirements}\par
\rulesubsection{rE25}\par
In exercising its powers under Section 5.A, the \emph{Commissioner} or
an \emph{Independent Decision-Making Panel }must have regard to
the \emph{Bar Standards Board enforcement strategy} and any
published \emph{Bar Standards Board} policy and guidance that appear to
be relevant.\\
\textbf{\emph{Administrative sanction}}\par
\rulesubsection{rE26}\par
Pursuant to rE19.3 and rE22.3 above, the \emph{Commissioner }or
an \emph{Independent Decision-Making Panel }may impose
an \emph{administrative sanction} on an \emph{applicable person} where
there is sufficient evidence on the balance of probabilities of a breach
of the \emph{Handbook} by that \emph{applicable person}.\\
\rulesubsection{rE27}\par
The \emph{Commissioner} or an \emph{Independent Decision-Making
Panel }may only impose an \emph{administrative sanction} on
an \emph{applicable person} pursuant to rE26 where:\\\nl \item the \emph{Commissioner }or an \emph{Independent Decision-Making
Panel }considers that to impose an \emph{administrative sanction} is
proportionate and sufficient in the public interest; or\item where the matter falls to be considered under rE209 of Section 5.B of
the \emph{Handbook.}\ln
\rulesubsection{rE28}\par
In determining the level of \emph{administrative sanction} to be
imposed, the \emph{Commissioner }or an \emph{Independent Decision-Making
Panel }must have regard to any published \emph{Bar Standards
Board} policy that appears to the \emph{Commissioner }or
an \emph{Independent Decision-Making Panel }to be relevant.\\
\rulesubsection{rE29}\par
The maximum level of a fine which can be imposed by
the \emph{Commissioner }or an \emph{Independent Decision-Making
Panel }under rE19.3 and rE22.3 is:\\\nl \item £1,000 (one thousand pounds) where the fine is to be imposed on
an \emph{applicable person} who is not a BSB entity; or\item £1,500 (one thousand and five hundred pounds) where the fine is to be
imposed on a \emph{BSB entity}.\ln
\rulesubsection{rE30}\par
Any decision to impose an \emph{administrative sanction} will be
recorded and may, where appropriate, be considered for continued
monitoring and supervision but will not be disclosed to any third
parties except in accordance with Section A.7 of these regulations.\\
\rulesubsection{rE31}\par
The \emph{applicable person} may appeal a decision of
the \emph{Commissioner }or an \emph{Independent Decision-Making
Panel }to impose an \emph{administrative sanction} in accordance with
Section 5.A of the Handbook\\
\rulesubsection{rE32}\par
In the case of a \emph{non-authorised individual} (other than an
unregistered barrister, a manager of a BSB entity or a registered
European lawyer who does not have a current practising certificate) who
at the time of the alleged conduct was an employee of a \emph{BSB
authorised person} the \emph{Commissioner }or an \emph{Independent
Decision-Making Panel }may only:\\\nl \item decide that no further action should be taken in relation to
the \emph{allegation;} or\item make an application to the \emph{Disciplinary Tribunal} that
the \emph{non-authorised individual} be subject to
a \emph{disqualification order}.\ln
\rulesection{Part 5 - A4. Professional misconduct proceedings }\par
\emph{Disciplinary Action}\par
\rulesubsection{rE33}\par
Where rE19.4 or rE22.4 is applicable, the \emph{allegation} shall be
referred to \emph{Disciplinary Action} only where
the \emph{Commissioner }or an \emph{Independent Decision-Making
Panel }is satisfied that:\\\nl \item there is a \emph{realistic prospect of a finding of professional
misconduct being made} or there is a \emph{realistic prospect of the
disqualification condition being satisfied}; and
\item having regard to the \emph{regulatory objectives}, it is in the
public interest to pursue \emph{Disciplinary Action.}\ln
\rulesubsection{rE34}\par
Where the \emph{Commissioner }or an \emph{Independent Decision-Making
Panel }is satisfied that the requirements of rE33 are met,
an \emph{allegation} which the \emph{Commissioner }or
an \emph{Independent Decision-Making Panel }is otherwise intending to
refer to the \emph{Disciplinary Tribunal} may, with the consent of
the \emph{applicable person} against whom the \emph{allegation} is made,
be finally determined by an \emph{Independent Decision-Making Panel}.
This is referred to as the ``\emph{determination by consent
procedure}''.\\
\rulesubsection{rE35}\par
The \emph{Commissioner }or an \emph{Independent Decision-Making
Panel }must, in deciding whether to refer an \emph{allegation} to
the \emph{determination by consent procedure}, consider all the
circumstances. However, the \emph{Commissioner }or an \emph{Independent
Decision-Making Panel }may only make the \emph{allegation} subject to
the \emph{determination by consent procedure }if:\\\nl \item the \emph{applicable person} submits to the jurisdiction of
an \emph{Independent Decision-Making Panel}; and\item the \emph{Commissioner }or an \emph{Independent Decision-Making
Panel }considers that:\\
\al \item there are no substantial disputes of fact which can only fairly be
resolved by oral evidence being taken; and\item having regard to the \emph{regulatory objectives}, it is in the
public interest to resolve the \emph{allegation} under
the \emph{determination by consent procedure}; and\item the potential\emph{ professional
misconduct} or \emph{disqualification condition}, if proved, combined
with the \emph{applicable person}'s previous disciplinary history, does
not appear to be such as to warrant a period of \emph{suspension} or
disbarment, the withdrawal of
an \emph{authorisation} or \emph{licence} (as appropriate) or the
imposition of a \emph{disqualification order} (or equivalent by
another \emph{Approved Regulator}).\la\ln
\rulesubsection{rE36}\par
\emph{Disciplinary Action} will be conducted in accordance with such
procedures as the \emph{Bar Standards Board} may prescribe from time to
time, including in Section 5.B of the Handbook, and will apply the
appropriate standard of proof as described in rE37, rE38, rE164, and
rE261A. \\
\rulesubsection{rE37}\par
Where a matter is to be considered under the \emph{determination by
consent procedure} as per rE34, the standard of proof to be applied is
the civil standard of proof, except when rE38 applies.\\
\rulesubsection{rE38}\par
In considering \emph{allegations} under the \emph{determination by
consent procedure}, the \emph{Commissioner} or an \emph{Independent
Decision-Making Panel}  \textcolor{myred}{\textbf{must}} apply the criminal standard of proof when
deciding charges of \emph{professional misconduct} where the conduct
alleged within that charge occurred prior to 1 April 2019, including
where the same alleged conduct continued beyond 31 March 2019 and forms
the basis of a single charge of \emph{professional misconduct}.\\
\textbf{\emph{Determination by Consent}}\par
\rulesubsection{rE39}\par
Where the \emph{Commissioner} or an \emph{Independent Decision-Making
Panel }has decided to refer an \emph{allegation} to
the \emph{determination by consent} \emph{procedure} in accordance with
rE35, the \emph{Commissioner} or an \emph{Independent Decision-Making
Panel} (as the case may be) may terminate the \emph{determination by
consent procedure} at any time if it no longer considers that the
requirements of rE35 are satisfied, or for any other good reason.\\
\rulesubsection{rE40}\par
If the \emph{determination by consent procedure }ends other than by a
finding and sanction to which the \emph{applicable person} consents,
then an \emph{allegation} may be referred to a
three-person \emph{Disciplinary Tribunal}.\\
\rulesubsection{rE41}\par
An \emph{Independent Decision-Making Panel }may impose on
an \emph{applicable person} against whom a charge of \emph{professional
misconduct} has been found proved under the \emph{determination by
consent procedure} any one or more the following:\\\nl \item an order to pay a fine to the \emph{Bar Standards Board} (the amount
of such fine to be determined having regard to the relevant sanctions
guidance) on such terms as to payment as the \emph{Independent
Decision-Making Panel} thinks fit;\item the imposition of any conditions on their licence or authorisation
(where appropriate);\item a reprimand by the \emph{Bar Standards Board};\item advice by the \emph{Independent Decision-Making Panel} as to their
future conduct; and\item an order to complete (or, in the case of a \emph{BSB entity}, an
order to procure that any relevant \emph{managers} or employees
complete) continuing professional development of such nature and
duration as an \emph{Independent Decision-Making Panel }shall direct and
to provide satisfactory proof of compliance with this order to
the \emph{Commissioner}.\ln
\rulesubsection{rE42}\par
In determining what sanction, if any, to impose under
the \emph{determination by consent procedure,} an \emph{Independent
Decision-Making Panel }shall have regard to any relevant policy or
guidelines issued by the \emph{Bar Standards Board} and/or by
the \emph{Council of the Inns of Court} from time to time.\\
\rulesubsection{rE43}\par
An \emph{Independent Decision-Making Panel }may not make an award of
costs when dealing with an \emph{allegation} under
the \emph{determination by consent procedure}.\\
\rulesubsection{rE44}\par
The \emph{Commissioner}  \textcolor{myred}{\textbf{must}} publish any finding and sanction resulting
from the \emph{determination by consent procedure} to the same extent as
such publication would have taken place on a finding and sanction by
a \emph{Disciplinary Tribunal}, as provided for in the Disciplinary
Tribunal Regulations.\\
\rulesubsection{rE45}\par
If the \emph{applicable person} accepts the outcome of
the \emph{determination by consent procedure}, no one may appeal against
it.\\
\textbf{\emph{Disciplinary Tribunal}}\par
\rulesubsection{rE46}\par
At the same time as the \emph{Commissioner }or an \emph{Independent
Decision-Making Panel }directs that an \emph{allegation} shall form the
subject matter of a disciplinary charge
and/or \emph{disqualification} application before a \emph{Disciplinary
Tribunal}, the \emph{Commissioner }or an \emph{Independent
Decision-Making Panel }must also decide whether a three-person panel or
a five-person panel is to be constituted.\\
\rulesubsection{rE47}\par
In deciding whether to direct the constitution of a three-person or a
five-person panel, the \emph{Commissioner }or an \emph{Independent
Decision-Making Panel }shall consider the sanction which it considers is
likely to be imposed on the \emph{applicable person} if the charge or
application is proved, having regard to:\\\nl \item any indicative sanctions guidance published
by \emph{BTAS/COIC}/\emph{Bar Standards Board} from time to time; and\item the previous disciplinary record of the \emph{applicable person}.\ln
\rulesubsection{rE48}\par
The \emph{Commissioner }or an \emph{Independent Decision-Making
Panel }may direct that a five-person panel is to be constituted if it
considers that:\\\nl \item having regard to any indicative sanctions guidance published
by \emph{BTAS/COIC}/\emph{Bar Standards Board}, in all the
circumstances, a sanction of disbarment
or \emph{suspension} from \emph{practice} for more than twelve months
may be appropriate; or\item having regard to any indicative sanctions guidance published
by \emph{BTAS/COIC}/\emph{Bar Standards Board}, in all the
circumstances, a sanction of
indefinite \emph{disqualification} or \emph{disqualification }for a
defined term of more than twelve months may be appropriate; or\item having regard to any indicative sanctions guidance published
by \emph{BTAS/COIC}/\emph{Bar Standards Board}, in all the
circumstances, a sanction of a \emph{BSB entity} having its
authorisation or licence revoked or \emph{suspended} for a period of
more than twelve months may be appropriate; or\item the \emph{allegation} involves a conviction for dishonesty or
deception\\
otherwise the \emph{Commissioner }or an \emph{Independent
Decision-Making Panel }must direct that a three-person panel is to be
constituted.\ln
\rulesubsection{rE49}\par
The \emph{Commissioner }or an \emph{Independent Decision-Making
Panel }must inform the \emph{applicable person} of the direction that it
has made pursuant to rE46. There is no appeal against the decision to
refer a matter to a three or a five-person panel.\\
\rulesubsection{rE50}\par
The \emph{Commissioner }or an \emph{Independent Decision-Making
Panel }may:\\\nl \item refer to the same \emph{Disciplinary Tribunal} any charges
and/or \emph{disqualification} applications which they consider may
conveniently be dealt with together; and\item refer any additional charges or \emph{disqualification} applications
relating to the same \emph{applicable person} to the \emph{Disciplinary
Tribunal} which is dealing with the original disciplinary charge
or \emph{disqualification} application (as the case may be), even if the
additional charge or application, by itself, may be regarded as
insufficiently serious to merit disposal by a \emph{Disciplinary
Tribunal} of that level.\ln
\rulesubsection{rE51}\par
When the \emph{Commissioner }or an \emph{Independent Decision-Making
Panel }has directed that an \emph{allegation} shall form the subject
matter of a charge or application before a \emph{Disciplinary Tribunal},
the \emph{Commissioner} is responsible for bringing the charge or
application on behalf of the \emph{Bar Standards Board} and prosecuting
that charge before such \emph{Disciplinary Tribunal}. If so:\\\nl \item the \emph{Commissioner} may arrange for the appointment of a
representative to settle the charge and to present the case before
the \emph{Disciplinary Tribunal}; and\item any charges shall be brought in the name and on behalf of
the \emph{Bar Standards Board}.\ln
\rulesubsection{rE52}\par
Section 5.B applies in respect of the procedure to be followed by
the \emph{Disciplinary Tribunal}\par
\rulesubsection{rE53}\par
Where a \emph{Disciplinary Tribunal} directs that matter(s) be referred
to \emph{Commissioner }or an \emph{Independent Decision-Making
Panel }under rE209 to consider whether an \emph{administrative
sanction} should be imposed, the \emph{Commissioner }or
an \emph{Independent Decision-Making Panel }shall consider the matter in
accordance with rE26 to rE32 or take no enforcement action in accordance
with rE19.2 and rE22.2.\\
\rulesection{Part 5 - A5. Appeals }

\rulesubsubsection{rE54}
An \emph{applicable person} has a right to appeal from a decision to
impose an \emph{administrative sanction}. That appeal is to
an \emph{appeal panel} nominated by the \emph{President} in the same
composition as a \emph{three-person panel} constituted under rE141 of
the \emph{Disciplinary Tribunal Regulations}.

\rulesubsection{rE55}\par
An appeal, if made, shall be made by the \emph{applicable
person} sending to the \emph{Commissioner,} within 28 days of the
imposition of the \emph{administrative sanction}, a notice identifying
the decision appealed against, the decision the \emph{applicable
person} contends for, the grounds of such appeal and a statement whether
the \emph{applicable person} requires their appeal to be disposed of at
an oral hearing. If the \emph{applicable person} does not expressly
request an oral hearing, the appeal will be dealt with by a review of
the papers. The appeal is a review of the original decision, not a
re-hearing.\\
\rulesubsection{rE56}\par
The notice  \textcolor{myred}{\textbf{must}} be accompanied by a sum as prescribed by the \emph{Bar
Standards Board} from time to time. The sum will be payable to
the \emph{Bar Standards Board} to cover expenses.\\
\rulesubsection{rE57}\par
Where the appeal is to be dealt with at an oral hearing then:\\\nl \item at least 5 working days before the time set for the appeal,
the \emph{Bar Standards Board} will provide each member of
the \emph{appeal panel} and the \emph{applicable person} with a
paginated bundle of the correspondence and other documents on its files
relating to the original decision; and\item the \emph{applicable person} and \emph{Bar Standards Board} may be
represented at the hearing.\ln
\rulesubsection{rE58}\par
The decision of an \emph{appeal panel} is final and shall not be not
subject to any further appeal or reconsideration.\\
\rulesubsection{rE59}\par
The \emph{appeal panel}  \textcolor{myred}{\textbf{must}} decide whether to set aside or to vary the
original decision.\\
\rulesubsection{rE60}\par
If the \emph{appeal panel} allows the appeal in whole or in part,
the \emph{appeal panel} may direct that any administrative fine or
appeal fee already paid by the \emph{applicable person} be refunded
either in whole or in part, but the appeal panel has no power to award
costs.\\
\rulesection{Part 5 - A6. Reconsidering allegations which have been disposed
of }\par
\rulesubsubsection{rE61}
The \emph{Commissioner} or an \emph{Independent Decision-Making
Panel} may reconsider an \emph{allegation} which has been disposed of by
the \emph{Commissioner} or an \emph{Independent Decision-Making
Panel} respectively where:\\\nl \item new evidence becomes available which leads it to conclude that it
should do so, or\item for some other good reason.\ln
\rulesubsection{rE62}\par
Following such reconsideration, the \emph{Commissioner} may take any
further or different action the \emph{Commissioner} thinks fit, as if
any earlier decision had not been made.\\
\rulesection{Part 5 - A7. Confidentiality }\par
\rulesubsubsection{rE63}

The \emph{Bar Standards Board}  \textcolor{myred}{\textbf{must}}
keep \emph{reports} and \emph{allegations} assessed or investigated
under these regulations confidential. The \emph{Bar Standards
Board}  \textcolor{myred}{\textbf{must not}} disclose the fact that a \emph{report} exists, or
details of the \emph{report} or of its treatment as
an \emph{allegation} or otherwise, or of its disposal save as specified
in this Section 5.A, or as otherwise required by law.\\
\rulesubsection{rE64}\par
Disclosure may be made:\\\nl \item for the purpose of\emph{ }the \emph{Bar Standards Board's }regulatory
assurance, supervision or authorisations functions; or\item for the purpose of keeping the \emph{applicable person, }or any
source of information relating to the \emph{applicable person, }informed
of the progress of the consideration of
a \emph{report} or \emph{allegation}; or\item for the purpose of publicising any forthcoming public hearing of
charges arising from the \emph{allegation}; or\item where the \emph{applicable} \emph{person }consents; or\item in response to a request from the selection panel or a member of its
secretariat in respect of an application by a \emph{barrister }for silk;
or from any body responsible for the appointment of judges in respect of
an application for judicial appointment; or from some other body or
the \emph{BSB authorised individual }for a \emph{certificate of good
standing }in respect of a \emph{barrister}; or\item for the purposes of providing examples of the types of behaviour that
may constitute breaches of the \emph{Handbook} either externally or
internally within the \emph{Bar Standards Board}, provided that where
disclosure occurs in these circumstances although details of the
individual \emph{reports }or \emph{allegations} may be published, any
relevant party's identities will remain anonymous; or\item with the approval of the \emph{Commissioner}, where
the \emph{Commissioner} considers it is in the public interest to
disclose some or all of the details of
the \emph{report} or \emph{allegation}.\ln
\rulesection{Part 5 - A8. Interpretation }\par
\rulesubsubsection{rE65}
For the avoidance of doubt, this Section 5.A does not prevent the
immediate operation of the \emph{Interim Suspension and Disqualification
Regulations} or the \emph{Fitness to Practise Regulations}, where
appropriate.\\
\rulesection{Part 5 - A9. Commencement }\par
\rulesubsubsection{rE66}
This Section 5.A shall come into force in accordance with the provisions
of Part 1 of this Handbook.\\
\rulesubsection{rE67 -- rE100}\par
\rulesubsubsection{Removed.} 
\textbf{Schedule 1 - Proceedings and Composition of the Independent
Decision-Making Panels }\par
1) An \emph{Independent Decision-Making Panel} shall:\\
a) for the purposes of Part 5A, consist of five members of
the \emph{Independent Decision-Making Body};\\
b) for all other purposes, consist of three members of
the \emph{Independent Decision-Making Body;}\par
and shall have a lay majority of at least one.\\
2) \emph{Independent Decision-Making Panel} meetings shall be held in
private.\\
3) Decisions of \emph{Independent Decision-Making Panels}  \textcolor{myred}{\textbf{must}} be
recorded in writing.\\
4) \emph{Independent Decision-Making Panel} meetings shall be held at a
frequency to be determined by the \emph{Bar Standards Board.}\par
5) \emph{Independent Decision-Making Panels} may, at any time, invite
any person to attend an \emph{Independent Decision-Making Panel} meeting
in an advisory or consultative capacity.\\
\emph{6) Independent Decision-Making Panel }meetings may be held in
person, by email, by telephone or via videoconference.\\
\chap{Part 5 - B. The Disciplinary Tribunals
Regulations }
\rulesection{Part 5 - B1. The Regulations }

\rulesubsection{regulation E101 }\par

\rulesubsubsection{rE101}
These Regulations will apply following the referral of
an \emph{allegation} by the \emph{Commissioner }or an\emph{ Independent
Decision-Making Panel} to a \emph{Disciplinary Tribunal}, in accordance
with Part 5 Section A.\\
\rulesubsection{regulation E102 - Service of Charges and/or
Applications }\par
\rulesubsubsection{rE102}
The Bar Standards Board  \textcolor{myred}{\textbf{must}} ensure that a copy of the charge(s) and/or
application(s):\\\nl \item is served on the relevant \emph{respondent}(s), together with a copy
of these Regulations not later than ten weeks after the date on which
the \emph{Commissioner} or an \emph{Independent Decision-Making
Panel} decides to refer the matter to a \emph{Disciplinary Tribunal};
and\item at the same time, ensure that copies of the charge(s) and/or
application(s) are sent to \emph{BTAS}.\ln
\rulesubsection{regulations E103-E105 - Documents to be served on the
respondent }\par
\rulesubsubsection{rE103}
As soon as practicable after the issue of the charge(s) and/or
application(s) to the \emph{respondent(s)}, the Bar Standards Board  \textcolor{myred}{\textbf{must}}
serve on the \emph{respondent}(s) and file with \emph{BTAS}:\\\nl \item a copy of the evidence of any witness intended to be called in
support of any charge(s) or application(s) (which, for the avoidance of
doubt, may be a formal witness statement or an informal document such as
a letter or attendance note); and\item a copy of any other documents intended to be relied on by
the \emph{Bar Standards Board}; and\item the \emph{standard directions} and/or non-\emph{standard
directions,} which, subject to rE111, the Bar Standards Board proposes
to apply to the case and which  \textcolor{myred}{\textbf{must}} include such timetable as may be
considered reasonable by the Bar Standards Board, having regard to the
facts of that case.\ln
\hspace*{0.333em}\rulesubsection{rE104}\par
If the\emph{ }documents referred to in rE103.1 and/or rE103.2 are not
sent to the \emph{respondent(s)} within 28 days of the service of the
charges on \emph{the respondent(s) }in accordance with rE102 above, then
the \emph{Bar Standards Board}  \textcolor{myred}{\textbf{must}} provide to
the \emph{respondent(s)} within that period:\\\nl \item details of the \emph{evidence }that is \emph{s}till being sought;
and\item details of when it is believed that it will be practicable to supply
that evidence to the \emph{respondent(s)}.\ln
\rulesubsection{rE105}\par
Nothing in rE103 or rE104 above shall prevent a \emph{Disciplinary
Tribunal} from receiving the evidence of a witness which has not been
served on the \emph{respondent(s) }in accordance with \textbf{ } rE103
or rE104, provided that the \emph{Disciplinary Tribunal} is of the
opinion either that this does not materially prejudice
the \emph{respondent(s)}, or that the evidence is accepted on such terms
as are necessary to ensure that no such prejudice occurs.\\
\rulesubsection{regulations E106-E107 - Directions }\par
\rulesubsubsection{rE106}
Within 21 days of the date of service of the directions under rE103.3,
the \emph{respondent(s)}  \textcolor{myred}{\textbf{must}}:\\\nl \item agree the \emph{standard directions} and/or non-\emph{standard
directions}; or\item provide to the \emph{Bar Standards Board} written submissions
explaining why the directions sought by the \emph{Bar Standards Board},
should be amended, withdrawn or added to; and/or\item indicate to the \emph{Bar Standards Board} whether they intend to
make any of the applications referred to in rE127.\ln
\rulesubsection{rE107}\par
Within 14 days of the date when the \emph{Bar Standards Board} receives
any written submissions from a \emph{respondent} in accordance with
rE106.2, the \emph{Bar Standards Board}  \textcolor{myred}{\textbf{must}} consider them and  \textcolor{myred}{\textbf{must}}
during that 14 day period:\\\nl \item inform the \emph{respondent}(s) of those changes to the standard
directions or non-\emph{standard directions} (as appropriate) which
the \emph{Bar Standards Board} is able to agree; and\item seek to agree with the \emph{respondent}(s) such other changes to the
standard directions or non-\emph{standard directions} (as appropriate)
as may be acceptable to all parties.\ln
\rulesubsection{regulations E108-E109 - No reply from respondent }\par
\rulesubsubsection{rE108}
Where standard directions are sought by the Bar Standards Board and
the \emph{respondent} does not reply to a request to agree directions
within the relevant 21 day period referred to in rE106,
the \emph{respondent} will be deemed to have accepted the \emph{standard
directions} and they shall be deemed to apply to the particular matter,
save and in so far as they may have been modified on the application of
any other \emph{respondent} to the same proceedings which was made
within the relevant 21 day period. The \emph{Bar Standards Board}  \textcolor{myred}{\textbf{must}}
forthwith serve on the\emph{ respondent} and file with \emph{BTAS} any
directions which are deemed to apply to the matter.\\
\rulesubsection{rE109}\par
Where non-\emph{standard directions} are sought by the Bar Standards
Board and the \emph{respondent} does not reply within the 21 day period
referred to in rE106, the Bar Standards Board  \textcolor{myred}{\textbf{must}} send to the President
a copy of the non-\emph{standard directions} and invite them to appoint
a Directions Judge to endorse the directions in accordance with rE114 to
rE126.\\
\rulesubsection{regulations E110-E112 - Agreement of directions }\par
\rulesubsubsection{rE110}
Where \emph{standard directions} are sought in a case by the \emph{Bar
Standards Board} and the parties agree the directions within the
relevant 21 day period referred to in rE106, or within the 14 day period
referred to in rE107, those directions will apply to the case and the
Bar Standards Board  \textcolor{myred}{\textbf{must}} forthwith serve the agreed directions on
the \emph{respondent} and file them with \emph{BTAS}.\\
\rulesubsection{rE111}\par
The parties may agree non-\emph{standard directions}, save that where
any non-\emph{standard directions} would have the effect of
preventing \emph{BTAS} from carrying out any function given to it by
these Regulations, the said direction cannot be agreed without
endorsement of a Directions Judge. In these circumstances, the \emph{Bar
Standards Board}  \textcolor{myred}{\textbf{must}} send to the President a copy of the
non-\emph{standard directions} and invite them to appoint a Directions
Judge to endorse the directions in accordance with rE114 to rE126.\\
\rulesubsection{rE112}\par
Where non-\emph{standard directions}, which do not include matters under
rE111, are sought by the Bar Standards Board in a case and the parties
agree those directions within the relevant 21 day period referred to in
rE106, or within the 14 day period referred to in rE107, those
directions will apply to the case. The Bar Standards Board  \textcolor{myred}{\textbf{must}}
forthwith serve the agreed directions on the \emph{respondent} and file
them with \emph{BTAS}.\\
\rulesubsection{regulation E113 - Non-agreement of directions }\par
\rulesubsubsection{rE113}
Where \emph{standard directions} and/or non-\emph{standard
directions} are sought in a case by the Bar Standards Board and
the \emph{respondent} does not agree those directions within the
relevant 21 day period referred to in rE106, or within the 14 day period
referred to in rE107, the \emph{Bar Standards Board }must write to
the \emph{respondent} to confirm that the directions have not been
agreed and  \textcolor{myred}{\textbf{must}} send to the \emph{President} the following (where
relevant):\\\nl \item a copy of the directions, including any \emph{standard
directions} and/or non-\emph{standard directions} which have been
agreed;\item any written submissions received from the \emph{respondent(s)} in
accordance with rE106.2;\item any notice from the \emph{respondent(s)} that they may be intending
to make an application referred to at rE106.3; and\item the \emph{Bar Standards Board's} response to any such request(s)
and/or submissions.\ln
\rulesubsection{regulations E114-E126 - Agreement/endorsement of directions by a
Directions Judge }\par
\rulesubsubsection{rE114}
When the \emph{President }has received the documents referred to in
rE109,rE111 or rE113 above, the \emph{President}  \textcolor{myred}{\textbf{must}} designate either a
Queen's Counsel or \emph{Judge}, to be determined at
the \emph{President's} sole discretion ("the \emph{Directions judge}"),
to exercise the powers and functions conferred on the \emph{Directions
Judge} in these Regulations.\\
\rulesubsection{rE115}\par
The \emph{President}  \textcolor{myred}{\textbf{must}} ensure that copies of the charge(s) or
application(s), together with the documentation referred to at rE109,
rE111 or rE113 above, are sent to the \emph{Directions Judge} once
the \emph{Directions Judge} has been designated.\\
\rulesubsection{rE116}\par
When they receive the relevant documents, the \emph{Directions
Judge}  \textcolor{myred}{\textbf{must}} consider any submissions about the directions and will
determine whether an oral directions hearing is necessary.\\
\rulesubsection{rE117}\par
If the \emph{Directions Judge} considers that no oral hearing is
necessary, then:\\\nl \item they  \textcolor{myred}{\textbf{must}} make an order setting out those directions which are to
apply in the case taking into account all the relevant circumstances,
including any written submissions of the parties and
the \emph{Directions Judge's} own findings; and\item they may consider and decide any other issues which may be necessary
in accordance with rE129.\ln
\rulesubsection{rE118}\par
If the \emph{Directions Judge} considers that an oral hearing is
necessary, the \emph{Directions Judge}  \textcolor{myred}{\textbf{must}} give written notice to
the \emph{Bar Standards Board} and the \emph{respondent(s) }that an oral
hearing is to be held for the purpose of giving directions and taking
such other steps as the \emph{Directions Judge }considers suitable for
the clarification of the issues before the \emph{Disciplinary
Tribunal} and generally for the just and expeditious handling of the
proceedings. The \emph{Directions Judge} shall also provide
the \emph{Bar Standards Board} and the \emph{respondent(s)} with a time
estimate for the oral directions hearing.\\
\rulesubsection{rE119}\par
Within seven days of receiving the notice referred to in rE118 above,
the \emph{Bar Standards Board} and the \emph{respondent(s)}  \textcolor{myred}{\textbf{must notify}} the \emph{President} and the other party of their and, where relevant,
their representative's available dates and times during the six week
period immediately after the date of that notice.\\
\rulesubsection{rE120}\par
The \emph{Directions Judge}  \textcolor{myred}{\textbf{must}} try to find a date and time within that
six week period which are convenient for all parties. If that is not
possible, the \emph{Directions Judge }must fix a date and time for the
oral directions hearing within that six week period and  \textcolor{myred}{\textbf{must notify}}
the \emph{Bar Standards Board} and the \emph{respondent(s)} of that date
and time.\\
\rulesubsection{rE121}\par
Once the \emph{Directions Judge }has set a date for the oral
hearing, \emph{BTAS }must appoint a \emph{person(s)} in accordance with
rE136 to act as Clerk at the hearing to take a note of the proceedings;
draw up a record of the directions given and/or any admissions made at
it.\\
\rulesubsection{rE122}\par
\emph{BTAS}  \textcolor{myred}{\textbf{must}} arrange for a record of the oral hearing before
a \emph{Directions Judge }to be made.\\
\rulesubsection{rE123}\par
The oral hearing before a \emph{Directions Judge} will be in private.\\
\rulesubsection{rE124}\par
After the oral directions hearing (or, if one was not required, after
the review of the papers by the \emph{Directions
Judge}) \emph{BTAS }must ensure that copies of the directions order are
served on the \emph{Bar Standards Board} and on
the \emph{respondent(s)}.\\
\rulesubsection{rE125}\par
The directions order served under rE124 is final, and there is no appeal
against it.\\
\rulesubsection{rE126}\par
Any variation sought by a party to an order for \emph{standard
directions} made and served under rE108 or rE110, or to an order for
non-\emph{standard directions} made and served under rE112,  \textcolor{myred}{\textbf{must}} be
endorsed by a \emph{Directions Judge}, who shall be designated by
the \emph{President} in accordance with the requirements of rE114.\\
\rulesubsection{regulations E127-E128 - Applications }\par
\rulesubsubsection{rE127}
At any time before the hearing, either party can make any of the
following applications and thereafter file with \emph{BTAS} and serve on
the opposing party written submission in support of the applications,
namely:\\\nl \item an application to sever the charges and/or applications;\item an application to strike out the charges and/or applications which
relate to the \emph{respondent} who makes the application;\item an application to stay the proceedings;\item an application about the admissibility of documents;\item an application for disclosure of documents;\item an application to extend or abridge any relevant time limits;\item an application for the hearing to be held in private;\item an application for separate hearings or an application that
proceedings pending against separate \emph{respondent}s be dealt with at
the same hearing; or\item any other application to vary \emph{standard directions} or
non-\emph{standard directions} (which either party considers reasonable,
having regard to the facts of the case).\ln
\rulesubsection{rE128}\par
The \emph{Directions Judge} or Chair of the\emph{ Disciplinary
Tribunal} or the \emph{Disciplinary Tribunal} will consider how any of
the applications referred to rE127 are to be dealt with.\\
\rulesubsection{regulation E129 - Extent of powers to order
directions }\par
\rulesubsubsection{rE129}
The \emph{Directions Judge} or the Chair of the\emph{ Disciplinary
Tribunal} designated in the \emph{Convening Order} (or failing
the \emph{Directions Judge} or the Chair of the\emph{ Disciplinary
Tribunal}, any other \emph{Judge} nominated by the \emph{President})
may, at any stage, make such directions for the management of the case
or the hearing as they consider will expedite the just and efficient
conduct of the case.\\
\rulesubsection{regulations E130-E131 - Setting the hearing date }\par
\rulesubsubsection{rE130}
This regulation applies where, after the deemed acceptance, later
agreement of directions, or the service of a directions order
by \emph{the President}, the date of the hearing has not been fixed.
Where this Regulation applies, each party  \textcolor{myred}{\textbf{must}} submit details of their
availability for the substantive hearing to \emph{BTAS} in accordance
with the directions. After they receive such details, or, where no such
details are provided, once the time for providing such details has
expired, the \emph{President}  \textcolor{myred}{\textbf{must}} fix the date of the substantive
hearing, having regard to the availability of the parties (if provided)
and the need for the prompt determination of any charges and/or
application(s) made against the \emph{respondent(s)}, in accordance with
the provisions of these Regulations.\\
\rulesubsection{rE131}\par
\emph{BTAS}  \textcolor{myred}{\textbf{must}} inform all parties of the date fixed for the hearing as
soon as reasonably practicable after \emph{the} \emph{President} has
fixed the date.\\
\rulesubsection{regulations E132-E135 - Appointing a Disciplinary Tribunal and
issuing a Convening order }\par
\rulesubsubsection{rE132}

On\\\nl \item the deemed acceptance or later agreement of directions by the
parties; or\item the service of the directions order by \emph{BTAS}; or\item the fixing of the date of the hearing in accordance with rE130
above,\ln
\emph{the President}  \textcolor{myred}{\textbf{must}}, in all cases,\\
\al \item appoint an appropriate \emph{Disciplinary Tribunal} to sit on the
relevant date(s), taking into account the requirements of these
Regulations;\\
\item appoint a \emph{person} or \emph{persons} to act as Clerk or Clerks
to the \emph{Disciplinary }Tribunal in accordance with rE136;\\
\item not less than 14 days before the date of the substantive hearing,
serve an order on the \emph{respondent(s)} ("the \emph{Convening
order}") specifying:\\
\rl  \item the name of the \emph{respondent(s)} to the proceedings and such
other information as may be relevant to the \emph{respondent(s)}, for
example:
\nl \item where any \emph{respondent} is a \emph{barrister}, details of the \emph{barrister's} Inn, their date of call and (if appropriate) the date of their appointment as Queen's Counsel, and details of whether or not the \emph{barrister} was acting as a \emph{self-employed} \emph{barrister} or an\emph{ employed} \emph{barrister} (and, in the latter case, details of their \emph{employer}, including whether or not it is a \emph{BSB entity}) and if the \emph{barrister} was acting as a \emph{HOLP} or \emph{manager} of a \emph{regulated entity}, identifying this fact and identifying the \emph{regulated entity} and whether or not it is a \emph{BSB entity};\item where any \emph{respondent} is a \emph{BSB entity}, details of the date when that body was so authorised or licensed with a summary of the number of \emph{barristers} and other individuals working within that \emph{BSB authorised body};
\item where any \emph{respondent} is another type of \emph{BSB regulated
person}, details of whether or not the \emph{BSB regulated person} is
an \emph{authorised (}non-\emph{BSB) person} or is otherwise subject to
regulation by any other regulator and, if so, the identity of that
regulator, and the role of that individual, including whether they were
acting as a \emph{HOLP}, \emph{HOFA, manager} or employee of
a \emph{regulated entity} and identifying that \emph{regulated
entity }and its Approved Regulator; and
\item where any \emph{respondent} is a non-\emph{authorised }individual
employed by a \emph{BSB authorised person}, details of the role of  that
individual and identifying the \emph{BSB authorised person} who directly
or indirectly employs the \emph{respondent};
\rl  \setcounter{enumi}{1}

\item  the date and time of the sitting of the \emph{Disciplinary
Tribunal} at which it is proposed the charge(s) and/or application(s)
should be heard; and\\
\item  the names and status (that is, as Chair, as \emph{lay member},
as \emph{barrister} or other) of those \emph{persons} who it is proposed
should constitute the \emph{Disciplinary Tribunal} to hear the case;
and\\
\item  the name of the Clerk,\lr\ln
and send copies of that \emph{Convening Order} to the nominated members
of the \emph{Disciplinary Tribunal}, the\emph{ Bar Standards Board}, and
the Clerk. In the Order the attention of the \emph{respondent(s)} will
be drawn to:
\nl
\item their right to represent themselves or be represented
professionally, with or without instructing a \emph{solicitor}, as they
shall think fit; and
\item their right to inspect and be given copies of documents referred to
in the list served pursuant to rE103 above; and
\item their right (without prejudice to their right to appear and take
part in the proceedings) to deliver a written answer to the charge(s)
and/or application(s) if they think fit.\ln\lr\la

\rulesubsection{rE133}\par
The \emph{respondent(s)} may, when they receive the \emph{Convening
Order}, give notice to the \emph{President} objecting to any one or more
of the proposed members of the \emph{Disciplinary Tribunal}.
The \emph{respondent }must give this notice as soon as is reasonably
practicable and  \textcolor{myred}{\textbf{must}} specify the grounds for their objection.\\
\rulesubsection{rE134}\par
When the \emph{President} receives such an objection, they  \textcolor{myred}{\textbf{must}}, if
satisfied that it is justified (but subject to rE135), exercise the
power conferred on them by rE148 to nominate a substitute member or
members of the \emph{Disciplinary Tribunal}, and  \textcolor{myred}{\textbf{must notify}}
the \emph{respondent(s)} accordingly. When they receive that
notification, the \emph{respondent(s) }may object to any substitute
member or members, in the same way as they may object under rE133.\\
\rulesubsection{rE135}\par
No objection to any member of the \emph{Disciplinary Tribunal} may be
made, or if made, may be upheld, on the grounds only that they know, or
might have known, about a charge of \emph{professional misconduct}, or
of a breach of proper professional standards, or a previous application
to \emph{disqualify}, or a charge consisting of \emph{a legal
aid} \emph{complaint}, against the \emph{respondent(s)}, or any finding
on any such application or charge, or any sanction imposed on
the \emph{respondent(s)} in connection with any such application or
charge.\\
\rulesubsection{regulations E136-E138 - Appointment of Clerk(s) }\par
\rulesubsubsection{rE136}
\emph{BTAS} shall appoint a Clerk(s) to perform the functions specified
in these Regulations and such other functions as the \emph{President,
Directions Judge} or the Chair of any \emph{Disciplinary Tribunal} may
direct.\\
\rulesubsection{rE137}\par
The \emph{President} may publish qualifications or other requirements
for those appointed to be Clerks.\\
\rulesubsection{rE138}\par
No person who has been engaged in the investigation of
a \emph{allegation} or application against a \emph{respondent} in
accordance with the relevant procedure or otherwise shall act as Clerk
of proceedings under these Regulations arising out of
that \emph{allegation} or application.\\
\rulesubsection{regulations E139-E150 - The Disciplinary Tribunal - Composition
of Disciplinary Tribunals }\par
\rulesubsubsection{rE139}
A \emph{Disciplinary Tribunal}  \textcolor{myred}{\textbf{must}} consist of either three persons or
five persons.\\
\rulesubsection{rE140}\par
A five-person panel  \textcolor{myred}{\textbf{must}} include the following \emph{persons} nominated
by the \emph{President}:\\\nl \item as Chair, a \emph{Judge}; and\item two \emph{lay members}; and\item two \emph{practising barristers} of not less than seven years'
standing.\\
\rulesubsection{rE141}\par
A three-person panel shall include the
following \emph{persons} nominated by \emph{the President}:\\\nl \item as Chair, a Queen's Counsel or a \emph{Judge}; and\item one \emph{lay member}; and\item one \emph{practising barrister }of not less than seven years'
standing\emph{.}\ln
\rulesubsection{rE142}\par
With the exception of judicial Chairs, the persons nominated
by \emph{the President }to sit on a \emph{Disciplinary Tribunal }must be
selected from the pool appointed by the \emph{Tribunal Appointments
Body.}\par
\rulesubsection{rE143}\par
In deciding who will sit on the panel, the \emph{President} may have
regard to the nature of the charge(s) and/or application(s) being
determined and to the identity of the \emph{respondent}(s) against whom
the charges have been made. When constituting the panel, \emph{the
President }shall take into account the requirements of rE140 and rE141
above, and rE144 and rE145 below.\\
\rulesubsection{rE144}\par
A \emph{person}  \textcolor{myred}{\textbf{must not}} be nominated to serve on a \emph{Disciplinary
Tribunal} if they:\\\nl \item are a member of the \emph{Bar Council }or of any of its committees or
the \emph{Independent Decision-Making Body}; or\item are a member of the \emph{Bar Standards Board} or of any of its
committees or the \emph{Independent Decision-Making Body}; or\item were a member of the \emph{Bar Standards Board} or of any of its
committees or the \emph{Independent Decision-Making Body} at any time
when the matter was being considered by the \emph{Bar Standards
Board}.\ln
\rulesubsection{rE145}\par
The person nominated by \emph{the President, }in accordance with rE140
and rE141, to be Chair of the\emph{ Disciplinary Tribunal, }may be
the \emph{Directions Judge }as appointed under rE114, unless
the \emph{Directions Judge }considers there to be any reason why they
should not Chair the hearing.\\
\rulesubsection{rE146}\par
\emph{The President} may publish qualifications or other requirements
made for those appointed to serve on a \emph{Disciplinary Tribunal}.\\
\rulesubsection{rE147}\par
If a vacancy in the \emph{Disciplinary Tribunal} arises before the
substantive hearing of the charge, the \emph{President }must choose
another member of the relevant class to fill that vacancy.\\
\rulesubsection{rE148}\par
At any time before the substantive hearing of the charge starts,
the \emph{President} may cancel any or all of the nominations made
pursuant to these Regulations, and make such alternative nominations as,
in the exercise of their discretion, they deem necessary or expedient,
provided always that \emph{the President} notifies
the \emph{respondent(s)} of the identity of such substitutes as soon as
is reasonably practicable after they have chosen them.
The \emph{respondent(s) }may object to such substitute members in the
same way as they may object under rE133.\\
\rulesubsection{rE149}\par
The proceedings of a five-person panel will not be invalidated on the
sole ground that after the \emph{Convening Order} has been issued (in
accordance with rE132 above), one or more of the members becomes unable
to act or is disqualified from acting, provided that:\\\nl \item the Chair and at least one \emph{lay member} and one barrister member
are still able to act and are present throughout the substantive
hearing; and\item the number of members present throughout the substantive hearing of
the charge is not reduced below three.\ln
\rulesubsection{rE150}\par
A member of a \emph{Disciplinary Tribunal} who has been absent for any
time during a sitting shall take no further part in the proceedings.\\
\rulesubsection{regulations E151-E153 - Provision of documents to the
Disciplinary Tribunal }\par
\rulesubsubsection{rE151}
The \emph{Bar Standards Board} and the \emph{respondent}  \textcolor{myred}{\textbf{must}} send
to \emph{BTAS}, at least 14 days before the hearing:\\\nl \item in the case of a five-person \emph{Disciplinary Tribunal}, six copies
of the evidence they intend to rely on at the hearing;\item in the case of a three-person \emph{Disciplinary Tribunal}, four
copies of the evidence they intend to rely on at the hearing.\ln
\rulesubsection{rE152}\par
The evidence referred to in rE151  \textcolor{myred}{\textbf{must}} be indexed and paginated.\\
\rulesubsection{rE153}\par
\emph{BTAS} shall provide to each member of the \emph{Disciplinary
Tribunal} before the start of the substantive hearing copies of the
following documents:\\\nl \item the \emph{Covening Order};\item the charge(s) and/or application(s) and any particulars of them;\item any documents which the \emph{Bar Standards Board} or
the \emph{respondent(s)} propose to rely on, unless a direction has been
made that copies of such documents be withheld;\item any written answer to the charge(s) and/or application(s) submitted
by or on behalf of the \emph{respondent(s)};\item such other documents as have been agreed or directed to be laid
before the \emph{Disciplinary Tribunal} before the start of the hearing;
and\item all orders for directions which have been made in relation to the
case.\ln
\rulesubsection{regulations E154-E155 - Applications for adjournment before the
commencement of the hearing }\par
\rulesubsubsection{rE154}
Any application by a party for an adjournment of the substantive hearing
before the date on which the hearing is scheduled to commence  \textcolor{myred}{\textbf{must}} be in
writing and accompanied by any evidence upon which the party relies in
support of their application.\\
\rulesubsection{rE155}\par
An application under rE154  \textcolor{myred}{\textbf{must}} be submitted to the Chair of
the\emph{ Disciplinary Tribunal} which has been convened to hear the
case and served upon the other party. The Chair  \textcolor{myred}{\textbf{must}} make reasonable
attempts to seek any representations in response to the application from
the other party. The Chair  \textcolor{myred}{\textbf{must}} consider the application for adjournment
taking into account any response submitted by the other party and may:\\\nl \item grant the adjournment; or\item direct that the application  \textcolor{myred}{\textbf{must}} be renewed before
the \emph{Disciplinary Tribunal }on the\textbf{ }first day fixed for the
hearing; or\item refuse the application; and\item may make such directions as they consider appropriate for the further
conduct of the case.\ln
\rulesubsection{regulation E156 - Hearing in public }\par
\rulesubsubsection{rE156}
The hearing before a \emph{Disciplinary Tribunal }must be in public,
unless it has been directed that all or part of the hearing is not to be
held in public, and that direction has not been over-ruled by
the \emph{Disciplinary Tribunal.}\par
\rulesubsection{regulation E157 - Recording of proceedings }\par
\rulesubsubsection{rE157}
\emph{BTAS }must arrange for a verbatim record of the proceedings before
a \emph{Disciplinary Tribunal }to be made.\\
\rulesubsection{regulations E158-E160 - Joinder }\par
\rulesubsubsection{rE158}
Unless it is of the view that there is a risk of prejudice to the
fairness of the proceedings, the \emph{Disciplinary Tribunal }may
consider and determine charges against two or more \emph{respondent}s at
the same hearing where:\\\nl \item the charge(s) against each \emph{respondent} arises out of the same
circumstances; or\item in the view of the \emph{Disciplinary Tribunal, }a joint hearing is
necessary or desirable.\ln
\rulesubsection{rE159}\par
Where a joint hearing is held:\\\nl \item these Regulations are to have effect in relation to the hearing with
the necessary modifications as directed by the Chair; and\item each \emph{respondent }concerned is to be able to exercise any of the
rights granted to that \emph{respondent }under these Regulations whether
or not any other \emph{respondent }concerned wishes to exercise that
right.\ln
\rulesubsection{rE160}\par
Unless it is of the view that there is a risk of prejudice to the
fairness of the proceedings, the \emph{Disciplinary Tribunal }may
consider and determine at a single hearing two or more matters which
have been separately referred to the \emph{Disciplinary Tribunal }in
respect of the same \emph{respondent, }whether or not those matters
arise from the same circumstances.\\
\rulesubsection{regulation E161 - Amendment and addition of charge(s) and/or
application(s) }\par
\rulesubsubsection{rE161}
A \emph{Disciplinary Tribunal} may at any time before or during the
hearing grant permission to the Bar Standards Board to amend the
charge(s) and/or application(s) against any \emph{respondent}, or grant
permission for new charge(s) and/or application(s) be added, provided
that:\\\nl \item the \emph{Disciplinary Tribunal }is satisfied that
no \emph{respondent }will by reason of such an amendment or addition
suffer any substantial prejudice in the conduct of their defence; and\item the \emph{Disciplinary Tribunal }will, if so requested by
a \emph{respondent, }adjourn for such time as the \emph{Disciplinary
Tribunal considers }reasonably necessary to enable
that \emph{respondent }to meet the amended charge(s) or
application(s).\ln
\rulesubsection{regulations E162-E163 - Adjournment of the
hearing }\par
\rulesubsubsection{rE162}
Subject to rE163, the \emph{Disciplinary Tribunal}  \textcolor{myred}{\textbf{must}} sit from day to
day until it has made a finding and, if any charge or application is
found proved, until sanction has been determined.\\
\rulesubsection{rE163}\par
A \emph{Disciplinary Tribunal} may, if they decide an adjournment is
necessary for any reason, adjourn the hearing for such period or periods
as it may decide.\\
\rulesubsection{regulation E164 - Standard of proof }\par
\rulesubsubsection{rE164}
The \emph{Disciplinary Tribunal}  \textcolor{myred}{\textbf{must}} apply the civil standard of proof
when deciding charges of \emph{professional misconduct} and in deciding
whether the \emph{disqualification condition} has been established.\\
\rulesubsection{regulation E165 - Rules of natural justice }\par
\rulesubsubsection{rE165}
The rules of natural justice apply to proceedings of
a \emph{Disciplinary Tribunal}.\\
\rulesubsection{regulations E166-E168 - Evidence }\par
\rulesubsubsection{rE166}
The \emph{Disciplinary Tribunal} may:\\\nl \item (subject to rE167 below) admit any evidence, whether oral or written,
whether given in person, or over the telephone, or by video link, or by
such other means as the \emph{Disciplinary Tribunal} may deem
appropriate, whether direct or hearsay, and whether or not it would be
admissible in a \emph{court} of law;\item give such directions with regard to the admission of evidence at the
hearing as it considers appropriate, ensuring that
a \emph{respondent} has a proper opportunity of answering the charge(s)
and/or application(s) made against them;\item exclude any hearsay evidence if it is not satisfied that reasonable
steps have been taken to obtain direct evidence of the facts sought to
be proved by the hearsay evidence.\ln
\rulesubsection{rE167}\par
Any party may refer to the fact (if relevant) that
the \emph{determination by consent procedure} was used before
the \emph{allegation} was referred as a charge before
a \emph{Disciplinary Tribunal}. However, no reference may be made to the
substance of the procedure (including, without limitation, any reference
to the contents of any report produced in the course of such procedure,
or to the circumstances in which the determination by consent procedure
ended), unless and until the \emph{respondent} refers to the substance
of the procedure in the course of presenting their case, or when they
are being sanctioned.\\
\rulesubsection{rE168}\par
Where a party has previously failed to comply with any direction made by
the Directions Judge, or has failed to do any act, including the
submission of evidence, within the time period specified in a direction,
the \emph{Disciplinary Tribunal} may, at its discretion:\\\nl \item decide to exclude the relevant evidence; or\item draw an adverse inference against that party.\ln
\rulesubsection{regulations E169-E170 - Decisions of courts or
tribunals }\par
\rulesubsubsection{rE169}
In proceedings before a \emph{Disciplinary Tribunal} which involve the
decision of a \emph{court }or tribunal in previous proceedings to which
the \emph{respondent} was a party, or where a wasted costs order was
made against the \emph{respondent}, the following Regulations shall
apply:\\\nl \item a copy of the certificate or memorandum of conviction relating to the
offence shall be conclusive proof that the \emph{respondent} committed
the offence;\item any \emph{court }record of the findings of fact upon which
the \emph{conviction} was based (which may include any document prepared
by the sentencing judge or a transcript of the relevant proceedings)
shall be proof of those facts, unless proved to be inaccurate;\item the finding and sanction of any tribunal in or outside England and
Wales exercising a professional disciplinary jurisdiction may be proved
by producing an official copy of the finding and sanction and the
findings of fact upon which that finding or sanction was based shall be
proof of those facts, unless proved to be inaccurate; and\item the judgment of any civil \emph{court}, or a wasted costs order made
in any \emph{court}, may be proved by producing an official copy of the
judgment or order, and the findings of fact upon which that judgment or
order was based shall be proof of those facts, unless proved to be
inaccurate.\ln
\rulesubsection{rE170}\par
In proceedings before a \emph{Disciplinary Tribunal} which involve the
decision of a \emph{court }or tribunal in previous proceedings to which
the \emph{respondent} was not a party, with the exception of a wasted
costs order that was made against the \emph{respondent}, the provisions
of rE169 do not apply.\\
\rulesubsection{regulations E171-E175 - Witness evidence at the Disciplinary
Tribunal }\par
\rulesubsubsection{rE171}
Witnesses shall be required to take an oath, or to affirm, before giving
oral evidence at the hearing.\\
\rulesubsection{rE172}\par
Subject to rE176, witnesses:\\\nl \item if giving oral evidence-in-chief, shall first be examined by the
party calling them;\item may be cross-examined by the opposing party;\item may be re-examined by the party calling them; and\item may at any time be questioned by the \emph{Disciplinary Tribunal.}\ln
\rulesubsection{rE173}\par
Any further questioning of the witnesses by the parties shall be at the
discretion of the \emph{Disciplinary Tribunal}.\\
\rulesubsection{rE174}\par
The \emph{Disciplinary Tribunal} may, upon the application of a party,
agree that the identity of a witness should not be revealed in public.\ln
\rulesubsection{rE175}\par
A witness of fact shall be excluded from the hearing until they are
called to give evidence, failing which they will not be entitled to give
evidence without leave of the \emph{Disciplinary Tribunal}\par
\rulesubsection{regulations E176-E182 - Vulnerable Witnesses }\par
\rulesubsubsection{rE176}
For the purpose of Part 5: Section B, any person falling into one or
more of the following categories may be treated by
the \emph{Disciplinary Tribunal }as a vulnerable witness in proceedings
before it:\\\nl \item any witness under the age of 18 at the time of the hearing;\item any witness with a mental disorder within the meaning of the Mental
Health Act 1983;\item any witness who is significantly impaired in relation to intelligence
and social functioning;\item any witness with physical disabilities who requires assistance to
give evidence;\item any witness, where the allegation against the \emph{respondent }is of
a sexual or violent nature and the witness was the alleged victim; and\item any witness who complains of intimidation.\ln
\rulesubsection{rE177}\par
Subject to hearing representations from the parties, the Chair of
the\emph{ Disciplinary Tribunal} or the \emph{Disciplinary Tribunal} may
adopt such measures as it considers desirable to enable it to receive
evidence from a vulnerable witness.\\
\rulesubsection{rE178}\par
Any witness who is not regarded as a vulnerable witness under rE176 may
apply for one or more of the measures set out in rE179 to be put into
place on the ground that the measure(s) is desirable to enable
the \emph{Disciplinary Tribunal} to receive the witness's evidence.\\
\rulesubsection{rE179}\par
Measures adopted by the \emph{Disciplinary Tribunal} for receiving
evidence from a vulnerable witness may include, but are not to be
limited to:\\\nl \item use of video links;\item use of pre-recorded evidence as the evidence-in-chief of a witness,
provided always\textbf{ }that such a witness is available at the hearing
for cross-examination and\textbf{ }questioning by the \emph{Disciplinary
Tribunal;}\item use of interpreters (including signers and translators) or
intermediaries;\item use of screens or such other measures as the \emph{Disciplinary
Tribunal }consider necessary in the circumstances in order to prevent:\al
\item the identity of the witness being revealed to the press or the
general public; or\\
\item access to the witness by the \emph{respondent}\la
\item the hearing of evidence (either whole or in part) by
the \emph{Disciplinary Tribunal} in private.\ln
\rulesubsection{rE180}\par
No \emph{respondent }charged with an allegation of a sexual or violent
nature may cross-examine in person a witness who is the alleged victim,
either:\\\nl \item in connection with that allegation, or\item in connection with any other allegation (of whatever nature) with
which the said \emph{respondent }is charged in the proceedings.\ln
\rulesubsection{rE181}\par
In the circumstances set out in rE180, in the absence of
the \emph{respondent}'s written consent, \emph{BTAS}  \textcolor{myred}{\textbf{must}}, no less than
seven days before the hearing, appoint a legally qualified person to
cross-examine the witness on the \emph{respondent}'s behalf.\ln
\rulesubsection{rE182}\par
\rulesubsubsection{Removed from 1 November 2017}

\rulesubsection{regulations E183--E184 - Absence of Respondent }
\rulesubsubsection{rE183}
Where the \emph{respondent} has not attended at the time and place
appointed for the hearing, the \emph{Disciplinary Tribunal} may
nevertheless, subject to compliance with rE234.1 in respect of
that \emph{respondent}, proceed to hear and determine the charge(s) or
application(s) relating to that \emph{respondent,} if it considers it
just to do so and it is satisfied that the relevant procedure has been
complied with (that is, the \emph{respondent} has been duly served (in
accordance with rE249 of these Regulations) with the documents required
by rE102, rE103, and rE132.3.c (as appropriate)).\\
\rulesubsection{rE184}\par
If the relevant procedure has not been complied with, but
a \emph{Disciplinary Tribunal} is satisfied that it has not been
practicable to comply with the relevant procedure, the Tribunal may hear
and determine the charge(s) or application(s) in the absence of
that \emph{respondent}, if it considers it just to do so, subject to
compliance with rE234.2 in respect of that \emph{respondent} if
the \emph{Disciplinary Tribunal} finds any charge or application
proved.\\
\rulesubsection{regulations E185-E187 - Application for a fresh
hearing }\par
\rulesubsubsection{rE185}
Where the \emph{Disciplinary Tribunal} proceed in
the \emph{respondent}'s absence, in accordance with rE183 or rE184,
the \emph{respondent} may apply to \emph{BTAS} for
a \emph{Directions} \emph{Judge}, appointed by the \emph{President}, to
consider an application for a fresh hearing before a
new \emph{Disciplinary} \emph{Tribunal}.\\
\rulesubsection{rE186}\par
The \emph{respondent's }application under rE185  \textcolor{myred}{\textbf{must}} be supported by a
statement setting out the facts and/or circumstances upon which
the \emph{respondent }relies in support of their application.\\
\rulesubsection{rE187}\par
The \emph{Directions Judge }may grant a new hearing if they consider it
just to do so and if they are satisfied that:\\\nl \item the \emph{respondent }submitted their application for a new hearing
promptly upon\textbf{ }becoming aware of the decision of
the \emph{Disciplinary Tribunal; }and\item the \emph{respondent }had good reason for not attending the
hearing.\ln
\rulesubsection{regulations E188-E198 - Order of proceedings at a
hearing }\par
\rulesubsubsection{rE188}
The order of proceedings at a hearing shall be as set out in these
regulations unless the \emph{Disciplinary Tribunal} decides, having
considered the interests of justice and fairness to the parties, that
the procedure should be varied. The \emph{Disciplinary Tribunal} may
then give such directions with regard to the conduct of, and procedure
at, the hearing as it considers appropriate.\\
\rulesubsection{rE189}\par
At any time during the hearing when it considers it desirable,
the \emph{Disciplinary Tribunal} may retire into private to
deliberate.\\
\rulesubsection{rE190}\par
The \emph{Disciplinary Tribunal} shall consider any submissions from the
parties in relation to objection(s) to the charge(s) or preliminary
applications, following which the \emph{Disciplinary Tribunal} will
retire into private session to consider the submissions and shall
thereafter announce its determination.\\
\rulesubsection{rE191}\par
After the \emph{Disciplinary Tribunal} has dealt with any submissions or
applications under rE190, the Clerk shall read the charge(s) in
public.\\
\rulesubsection{rE192}\par
The Clerk shall ask the \emph{respondent}(s) whether the charge(s) is
admitted or denied. The \emph{respondent}(s) plea to the charge(s) will
be entered on the record.\\
\rulesubsection{rE193}\par
Where the \emph{respondent}(s) admit the charges(s), the Chair of
the\emph{ Disciplinary Tribunal} shall announce the charge(s) proved and
the \emph{Disciplinary Tribunal} shall record in writing its finding on
the charge(s) and its reasons. The matter shall then continue in
accordance with the procedure set out at paragraph rE199 onwards.\\
\rulesubsection{rE194}\par
Where the \emph{respondent}(s) denies the charge(s), the Bar Standards
Board will present the case against the \emph{respondent}(s), which may
include producing any evidence and calling any witness in person.\\
\rulesubsection{rE195}\par
After the evidence against the \emph{respondent} has been called,
the \emph{respondent} shall be entitled to submit that they have no case
to answer. The Bar Standards Board shall be entitled to respond to such
a submission. If such a submission is upheld the \emph{Disciplinary
Tribunal} shall dismiss the charge(s), either in whole or in part. If
the entirety of the case against the \emph{respondent} is not dismissed
and some charges remain the proceedings shall continue as set out at
rE196 to rE198.\\
\rulesubsection{rE196}\par
The \emph{respondent} shall then be entitled to call any witness, give
evidence on their own behalf and adduce any other evidence in support of
the \emph{respondent}'s defence.\\
\rulesubsection{rE197}\par
The \emph{Bar Standards Board }shall be entitled to call witnesses and
adduce evidence in rebuttal of any part of the defence case.\\
\rulesubsection{rE198}\par
After the \emph{respondent }has called any witness in person and adduced
any evidence, the \emph{Bar Standards Board }may address
the \emph{Disciplinary Tribunal, }and thereafter
the \emph{respondent.}\par
\rulesubsection{regulations E199-E202 - The finding }\par
\rulesubsubsection{rE199}
At the end of the hearing, the \emph{Disciplinary Tribunal}  \textcolor{myred}{\textbf{must}} record
in writing its finding(s) on each charge or application, and its
reasons. That record  \textcolor{myred}{\textbf{must}} be signed by the Chair and by all members of
the \emph{Disciplinary Tribunal}.\\
\rulesubsection{rE200}\par
If the members of the \emph{Disciplinary Tribunal} do not agree on any
charge or application, the finding to be recorded on that charge or
application  \textcolor{myred}{\textbf{must}} be that of the majority. If the members of
the \emph{Disciplinary Tribunal} are equally divided on any charge or
application, then, as the burden of proof is on the \emph{Bar Standards
Board}, the finding to be recorded on that charge or application  \textcolor{myred}{\textbf{must}} be
that which is the most favourable to the \emph{respondent}.\\
\rulesubsection{rE201}\par
The Chair of the\emph{ Disciplinary Tribunal}  \textcolor{myred}{\textbf{must}} then announce
the \emph{Disciplinary Tribunal}'s finding on the charge(s) or
application(s), and state whether each such finding was unanimous or by
a majority. The \emph{Disciplinary Tribunal} is free to reserve its
judgment.\\
\rulesubsection{rE202}\par
In any case where the \emph{Disciplinary Tribunal} dismisses the
charge(s) and/or application(s), it may give advice to
the \emph{respondent} about their future conduct\\
\rulesubsection{regulations E203-E219 - The sanction }\par
\rulesubsubsection{rE203}
If the \emph{Disciplinary Tribunal} finds any of the charges or
applications proved against a \emph{respondent}, it may hear evidence of
any previous:\\\nl \item finding of \emph{professional misconduct} by a \emph{Disciplinary
Tribunal} or under the \emph{determination by consent procedure}; or\item \emph{Disqualification Order}; or\item finding of a breach of proper professional standards by the \emph{Bar
Standards Board} or any other regulator\item adverse finding on a charge consisting of a \emph{legal
aid} \emph{complaint;}\par
made in respect of the \emph{respondent}, or, where the proved charge(s)
concerns a \emph{BSB entity}, in respect of that body or any person
employed in the \emph{BSB entity} directly implicated by the charges\ln
\rulesubsection{rE204}\par
After hearing any representations by or on behalf of
the \emph{respondent(s)}, the \emph{Disciplinary Tribunal}  \textcolor{myred}{\textbf{must}} decide
what sanction to impose on a \emph{respondent}, taking into account the
sentencing guidance and  \textcolor{myred}{\textbf{must}} record its sanction in writing, together
with its reasons.\\
\rulesubsection{rE205}\par
If the members of the \emph{Disciplinary Tribunal} do not agree on the
sanction to be imposed on a \emph{respondent}, the sanction to be
recorded  \textcolor{myred}{\textbf{must}} be that decided by the majority. If the members of
the \emph{Disciplinary Tribunal} are equally divided on the sanction to
be imposed on a \emph{respondent}, the sanction to be recorded  \textcolor{myred}{\textbf{must}} be
that which is the most favourable to the \emph{respondent}.\\
\rulesubsection{rE206}\par
The Chair of the \emph{Disciplinary Tribunal}  \textcolor{myred}{\textbf{must}} then announce
the \emph{Disciplinary Tribunal}'s decision on sanction and state
whether the decision was unanimous or by a majority.\\
\rulesubsection{rE207}\par
Subject to rE208 below:\\\nl \item a \emph{respondent} against whom a charge of \emph{professional
misconduct} has been found proved may be sanctioned by
the \emph{Disciplinary Tribunal} as follows:
\al
\item in the case of \emph{barristers}, in accordance with Annex 1 to these
Regulations;\\
\item in the case of a \emph{BSB authorised body}, in accordance with Annex
2 to these Regulations;
\item in the case of a \emph{BSB licensed body}, in accordance with Annex 3
to these Regulations;\\
\item in the case of \emph{registered European lawyers}, in accordance with
Annex 4 to these Regulations;\\
\item in the case of all other \emph{BSB regulated persons}, in accordance
with Annex 5 to these Regulations;\la\item in the case of a \emph{respondent} who is an \emph{applicable
person} in respect of whom the \emph{Disciplinary Tribunal} finds
the \emph{disqualification condition} to be established,
the \emph{Disciplinary Tribunal} may make a \emph{Disqualification
Order} if the \emph{Disciplinary Tribunal} considers that the making of
such a \emph{Disqualification Order} is a proportionate sanction and is
in the public interest (there being no other available sanction in
respect of a \emph{non-authorised individual }(other than an
unregistered barrister, manager of a BSB entity or a registered European
lawyer who does not have a current practicing
certificate\emph{)} directly or indirectly employed by a \emph{BSB
authorised person}).\ln
\rulesubsection{rE208}\par
In any case where a charge of \emph{professional misconduct} has been
found proved, the \emph{Disciplinary Tribunal} may decide that no
further action should be taken against the \emph{respondent}\par
\rulesubsection{rE209}\par
In any case where a charge of \emph{professional misconduct }has not
been found proved, the \emph{Disciplinary Tribunal }may direct that the
matter(s) be referred to the \emph{Bar Standards Board }for it to
consider whether an \emph{administrative sanction }should be imposed in
accordance with the provisions of rE19.3 or rE22.3 of the Enforcement
Decision Regulations, where:\\\nl \item The \emph{Disciplinary Tribunal }is satisfied there is sufficient
evidence on the balance of probabilities of a breach of
the \emph{Handbook }by the \emph{respondent}; and\item The \emph{Disciplinary Tribunal} considers that such referral to
the \emph{Bar Standards Board} is proportionate and in the public
interest\ln
\rulesubsection{rE209A}\par
A direction made under rE209 is not a disposal or a finding for the
purposes of the BSB Handbook.\\
\rulesubsection{rE210}\par
A three-person panel  \textcolor{myred}{\textbf{must not}}:\\\nl \item disbar a \emph{barrister} or suspend
a \emph{barrister's} \emph{practising certificate} for a period longer 
than twelve months; or\item revoke the authorisation or licence (as appropriate) of a \emph{BSB
entity} or suspend it for a period longer than twelve months; or\item remove a \emph{registered European lawyer} from the \emph{register of
European lawyers}; or\item impose a sanction of suspension on any \emph{BSB regulated
person} for a prescribed period longer than twelve months; or\item impose a \emph{Disqualification Order }for more than twelve months.\\
This Regulation does not prevent a three-person panel making an order in
accordance with rE211 below.\ln
\rulesubsection{rE211}\par
In the event that a three-person panel considers that a case before it
merits the imposition on a \emph{respondent }of any of the sentences
referred to in  rE210 or the three-person panel otherwise considers that
the case of a particular \emph{respondent} is complex enough to warrant
sentencing by a five-person panel:\\\nl \item the three-person panel  \textcolor{myred}{\textbf{must}} refer the case to a five-person panel for
it to sanction that \emph{respondent} (but may proceed to sanction any
other \emph{respondents} to the proceedings in respect of whom this
regulation does not apply); and\item the three-person panel  \textcolor{myred}{\textbf{must}}, in order to help the five-person panel,
prepare a statement of the facts as found (and, where relevant, the
sentences passed on any other \emph{respondents} to the proceedings).
The \emph{respondent} cannot challenge the facts found by the
three-person panel; and\item the three-person panel  \textcolor{myred}{\textbf{must}} direct within what period of time the
sentencing hearing before the five-person panel is to be held and make
appropriate directions for the parties to provide
the \emph{President }with their dates of availability.\ln
\rulesubsection{rE212}\par
Following a referral by a three-person panel under rE211, the
five-person panel  \textcolor{myred}{\textbf{must}} be constituted in accordance with
rE140. \emph{The President }must fix the date for the sentencing hearing
and in so doing shall have regard to the availability of the parties,
save that \emph{the President }may disregard the availability of any
party where that party has failed to provide any, or any reasonable
dates of availability. As soon as is reasonably practicable after they
have fixed the sentencing hearing, \emph{the President }must inform all
the parties of that date.\\
\rulesubsection{rE213}\par
The \emph{respondent }must be informed by \emph{BTAS }as soon as
practicable of the names and status (that is, as Chair, as \emph{lay
member, }as \emph{barrister }or other) of those \emph{persons }who it is
proposed will constitute the five-person\emph{ }panel.
The \emph{respondent }may, when they are so informed, give notice
to \emph{the President }objecting to any one or more of the proposed
members of the panel. That notice  \textcolor{myred}{\textbf{must}} be given as soon as is reasonably
practicable,  \textcolor{myred}{\textbf{must}} specify the ground of objection, and  \textcolor{myred}{\textbf{must}} be dealt
with in accordance with rE134 and rE135.\\
\rulesubsection{rE214}\par
If the five-person panel is satisfied that the requirements of rE212 and
rE213 above have been complied with, and the \emph{respondent} has not
attended at the time and place appointed for the sentencing hearing, the
five-person panel may nonetheless sanction the \emph{respondent},
provided that it complies with rE234.1.\\
\rulesubsection{rE215}\par
If the five-person panel is satisfied that it has not been practicable
to comply with the requirements of rE212 and rE213, above, and
the \emph{respondent} has not attended at the time and place appointed
for the sentencing hearing, the five-person panel may nonetheless
sanction the \emph{respondent}, provided that it complies with
rE234.2.\\
\rulesubsection{rE216}\par
If the procedure under rE215 has been followed,
the \emph{respondent} may apply to the \emph{Directions Judge} for an
order that there should be a new sentencing hearing before a fresh
five-person panel and the procedure for
the \emph{respondent's }application shall be as set out at rE185 to
rE187 in these Regulations.\\
\rulesubsection{rE217}\par
Sections 41 and 42 of the Administration of Justice Act 1985 (as
substituted by Section 33 of the Legal Aid Act 1988 and as amended by
Schedule 4 to the Access to Justice Act 1999) confer certain powers
(relating to the reduction or cancellation of fees otherwise payable by
the \emph{Legal Aid Agency }in connection with services provided as part
of the Criminal Legal Aid or Civil Legal Aid and to the exclusion from
providing representation funded by the \emph{Legal Aid Agency }as part
of the Criminal Legal Aid or Civil Legal Aid) on a \emph{Disciplinary
Tribunal }in the cases to which those Sections apply). Accordingly:\\\nl \item any \emph{Disciplinary Tribunal} which hears a charge consisting of
a \emph{legal aid complaint} relating to the conduct of
a \emph{respondent} who is a \emph{barrister} may if it thinks fit (and
whether or not it sentences the \emph{respondent} in accordance with
rE207.1 in respect of any conduct arising out of the same \emph{legal
aid complaint}) order that any such fees as are referred to in Section
41(2) of the Act of 1985 shall be reduced or cancelled;\item where a \emph{Disciplinary Tribunal} hears a charge
of \emph{professional misconduct} against a \emph{respondent} who is
a \emph{barrister} it may (in addition to, or instead of, sentencing
that \emph{respondent} in accordance with rE206.1) order that they be
excluded from providing representation funded by the \emph{Legal Aid
Agency} as part of the Community Legal Service, or Criminal Defence
Service, either temporarily, or for a specified period, if it determines
that there is a good reason to exclude them arising from:\al
\item their conduct in connection with any such services as are mentioned
in Section 40(1) of the Act of 1985; or\\
\item their professional conduct generally.\la\ln
\rulesubsection{rE218}\par
Whether or not a \emph{Disciplinary Tribunal} finds any charge or
application proved against a \emph{barrister} who is a \emph{pupil
supervisor}, if the \emph{Disciplinary Tribunal} considers that the
circumstances of the \emph{allegation} are relevant to
the\emph{ respondent} in their capacity as a \emph{pupil supervisor}, it
may notify the \emph{respondent's} \emph{AETO} and/or the \emph{BSB }of
those concerns in such manner as it sees fit.\\
\rulesubsection{rE219}\par
If a \emph{barrister} is a member of more than one \emph{Inn},
each \emph{Inn }of which they are a member  \textcolor{myred}{\textbf{must}} be mentioned in the
sanction imposed on them.\\
\rulesubsection{regulations E220-E224 - Sanction of suspension from practice or
from authorisation or licensing or imposition of
conditions }\par
\rulesubsubsection{rE220}
For the purposes of rE222 to rE224:\\\nl \item The effect of a sanction of suspension for a \emph{BSB authorised
individual} is that:\al
\item the \emph{respondent's} \emph{practising
certificate} is \emph{suspended} by the Bar Standards Board for the
period of the suspension;\\
\item  the \emph{respondent} is prohibited from practising as
a \emph{barrister}, or holding themselves out as being
a \emph{barrister }when providing \emph{legal services} or as otherwise
being authorised by the \emph{Bar Standards Board} to
provide \emph{reserved} \emph{legal activities }or when describing
themselves as a \emph{barrister} in providing services other
than \emph{legal services} (whether or not for reward) unless they
disclose the suspension;\la
\item The effect of a sanction of suspension for a \emph{registered
European lawyer} shall mean that
the \emph{respondent} is \emph{suspended} from the \emph{register of
European lawyers} maintained by the \emph{Bar Standards Board} and is,
for so long as they remain \emph{suspended}:\al
\item prohibited from holding themselves out as registered with
the \emph{Bar Standards Board}; and;\\
\item not authorised to \emph{practise}.\la \item The effect of a sanction of suspension for a \emph{BSB entity} shall mean that the body's authorisation or licence is \emph{suspended} for the period of the suspension such that the \emph{respondent} is not an \emph{authorised person }for that period;
\item The effect of a sanction on a \emph{BSB authorised individual} or
a \emph{registered European lawyer} requiring completion of continuing
professional development shall be in addition to the mandatory
requirements set out in the continuing professional development rules at
Part 4 of this \emph{Handbook}.\ln
\rulesubsection{rE221}\par
In exceptional circumstances, where the total suspension is three months
or less, the Tribunal may postpone the commencement of the suspension
for a period as it deems fit.\\
\rulesubsection{rE222}\par
The period for which a sanction of suspension from \emph{practice} is
expressed to run may be:\\\nl \item a fixed period; or\item until the \emph{respondent} has complied with any conditions
specified in the order imposing the sanction of suspension.\ln
\rulesubsection{rE223}\par
Conditions may be imposed on a\emph{ barrister's} \emph{practising
certificate} or on the authorisation or licence of a \emph{BSB
entity}:\\\nl \item without its being \emph{suspended}; or\item to take effect on a \emph{barrister's} \emph{practising
certificate} or on the authorisation or licence of a \emph{BSB
entity} when a period of suspension ends.\ln
\rulesubsection{rE224}\par
Conditions may (depending on the circumstances) include:\\\nl \item conditions limiting the scope of the \emph{respondent's
practice} (after the end of any suspension, if relevant) to such part as
the \emph{Disciplinary Tribunal} may determine, either indefinitely or
for a defined period; and/or\item imposing requirements that the \emph{respondent}, or in the case of
a \emph{BSB entity}, its \emph{managers} or employees, undergo such
further \emph{training} as the \emph{Disciplinary Tribunal} may
determine; and/or\item prohibiting the \emph{respondent} from accepting or carrying out
any \emph{public access} \emph{instructions}; and/or\item such other matters as the \emph{Disciplinary Tribunal} may consider
appropriate for the purpose of protecting the public and/or preventing a
repetition of the conduct in question.\ln
\rulesubsection{regulations E225-E233 - Suspension/withdrawal of practising
rights pending the hearing of any appeal }\par
\rulesubsubsection{rE225}
rE226 to rE233\textbf{ }below apply to any \emph{respondent }who:\\\nl \item is a \emph{barrister, }who has been sanctioned to be disbarred or to
be \emph{suspended }or to be prohibited from accepting or carrying out
any public access work or \emph{instructions }for more than twelve
months;\item is a \emph{BSB }authorised\emph{ individual, }who has been sanctioned
to be disqualified\emph{ }or to be \emph{suspended }for more than twelve
months;\item is a \emph{BSB entity, }which has been sanctioned to have its
authorisation or licence revoked or \emph{suspended }for more than
twelve months; or\item is a \emph{BSB authorised person, }who has been sanctioned to have
conditions placed on
their \emph{practising }certificate\emph{, }authorisation or licence (as
appropriate) prohibiting them from accepting any \emph{public access
instructions }or \emph{conducting any litigation }or for more than
twelve months.\ln
\rulesubsection{rE226}\par
Where rE225\textbf{ }applies, the \emph{Disciplinary Tribunal }must seek
representations from the \emph{respondent }and from the \emph{Bar
Standards Board }on the appropriateness or otherwise of taking action
under rE227 below.\\
\rulesubsection{rE227}\par
Having heard any representations under
rE225\textbf{ }the \emph{Disciplinary Tribunal }must (unless in the
circumstances of the case it appears to the \emph{Disciplinary
Tribunal }to be inappropriate to do so), either:\\\nl \item in relation to rE225.1 to rE225.3, require the \emph{respondent} to
suspend their \emph{practice} immediately, in which case the \emph{Bar
Standards Board}  \textcolor{myred}{\textbf{must}} suspend that \emph{respondent's} \emph{practising
certificate} with immediate effect; or\item in relation to rE225.4 decide that the condition prohibiting
the \emph{respondent} from accepting \emph{public access
instructions} or conducting any litigation, shall take effect
immediately; or\item where the \emph{respondent} has been sanctioned to be disbarred or to
be suspended, and where that \emph{respondent} does not currently hold
a \emph{practising certificate}, require the \emph{Bar Standards
Board} not to issue any \emph{practising certificate} to them.\ln
\rulesubsection{rE228}\par
If the \emph{Disciplinary Tribunal }decides that it would be
inappropriate to require immediate \emph{suspension }or immediate
imposition of conditions (as the case may be) it may nonetheless require
the \emph{respondent }to suspend their \emph{practice }or to impose
conditions, from such date as the \emph{Disciplinary Tribunal }may
specify.\\
\rulesubsection{rE229}\par
Where the \emph{respondent }is permitted to continue to practise for any
period before being \emph{suspended }under
rE228\textbf{ }the \emph{Disciplinary Tribunal }may require
the \emph{Bar Standards Board }to impose such terms on
the \emph{respondent's practice }as the \emph{Disciplinary
Tribunal }deems necessary to protect \emph{the public }until the
suspension comes into effect.\\
\rulesubsection{rE230}\par
Where an order is made in respect of a \emph{respondent }under
rE225\textbf{ }and that \emph{respondent }considers that, due to a
change in the circumstances, it would be appropriate for that order to
be varied, they may apply to \emph{the President }in writing for it to
be varied.\\
\rulesubsection{rE231}\par
When \emph{the President }receives an application made under rE230, they
must refer it to the Chair and to one of the \emph{lay members }of
the \emph{Disciplinary Tribunal }which originally made the order to make
a decision on the application.\\
\rulesubsection{rE232}\par
Any application made under rE230  \textcolor{myred}{\textbf{must}} be sent by the applicant, on the
day that it is made, to the \emph{Bar Standards Board. }The \emph{Bar
Standards Board may }make such representations as they think fit on that
application to those to whom the application has been referred by
the \emph{President.}\par
\rulesubsection{rE233}\par
The persons to whom an application made under rE230 above is referred
may vary or confirm the order in relation to which the application has
been made.\\
\rulesubsection{regulation E234 - Wording of the sanction when respondent not
present }\par
\rulesubsubsection{rE234}
If a \emph{respondent} has not been present throughout the proceedings,
the sanction in respect of that \emph{respondent}  \textcolor{myred}{\textbf{must}} include one or
more of the following statements:\\\nl \item if the relevant procedure under rE183 has been complied with, that
the finding and sanction were made in the absence of
the \emph{respondent} in accordance with rE183;\item if the procedure under rE184 has been complied with, that the finding
and the sanction were made in the absence of the \emph{respondent} and
that they have the right to apply to the \emph{Directions Judge} for an
order that there should be a new hearing before a
fresh \emph{Disciplinary Tribunal};\item if the relevant procedure under rE213 has been complied with, that
the sanction was made in the absence of the \emph{respondent} in
accordance with rE214;\item if the procedure under rE215 has been complied with, that the
sanction was made in the absence of the \emph{respondent} and that they
may apply to the \emph{Directions Judge} for an order that there should
be a new hearing before a fresh \emph{Disciplinary Tribunal}.\ln
\rulesubsection{regulation E235 - Report of Finding and Sanction }\par
\rulesubsubsection{rE235}
As soon as is practicable after the end of the proceedings of
a \emph{Disciplinary Tribunal}, the Chair  \textcolor{myred}{\textbf{must}} prepare a report in
writing of the finding(s) on the charge(s) of \emph{professional
misconduct} and/or on any applications, and the reasons for those
findings and the sanction. At the discretion of the Chair, the report
may also refer to matters which, in the light of the evidence given to
the \emph{Disciplinary Tribunal}, appear to require investigation or
comment. The Chair  \textcolor{myred}{\textbf{must}} send copies of the report to:\\\nl \item the \emph{respondent};\item the Director General of the \emph{Bar Standards Board};\item the Chair of the \emph{Bar Standards Board}; and\item where a \emph{barrister} has been disbarred,
the \emph{respondent's} \emph{Inn} of Call and to any other Inns of
which they are a member; and\item where a \emph{HOLP} or \emph{HOFA} or \emph{manager} or employee of
a \emph{licensed} \emph{body} has been
disqualified, \emph{the} \emph{LSB}; and\item in cases where one or more charges
of \emph{professional} \emph{misconduct} have been found proved:\al
\item the \emph{respondent's} head of chambers, \emph{HOLP}, or employer
(as appropriate); and\\
\item  in the case of a \emph{registered} \emph{European} \emph{lawyer},
their home professional body; and\la \item in cases where one or more charges
of \emph{professional} \emph{misconduct} have been found proved and any
such charge constitutes, or arises out of, \emph{a legal aid complaint},
and/or the sanction includes an order under rE217, \emph{the Legal Aid
Agency}; and\item any other person or bodies that the \emph{President} deems, in their
absolute discretion, to be appropriate, taking into account the
circumstances.\ln
\rulesubsection{regulations E236-E238 - Appeals }\par
\rulesubsubsection{rE236}
In cases where one or more charges of \emph{professional
misconduct} have been proved, and/or a \emph{disqualification order} has
been made, an appeal may be lodged with the High Court in accordance
with the Civil Procedure Rules:\\\nl \item by the \emph{respondent} against finding and/or sanction;\item with the consent of the\emph{ Commissioner}, by the \emph{Bar
Standards Board} against sanction.\ln
\rulesubsection{rE237}\par
In any case where any charge of \emph{professional misconduct} or
application to \emph{disqualify} has been dismissed, the \emph{Bar
Standards Board} may (with the consent of the\emph{ Commissioner}) lodge
an appeal with the High Court in accordance with the Civil Procedure
Rules.\\
\rulesubsection{rE238}\par
Where a \emph{respondent} lodges an appeal against a disbarment
or \emph{Disqualification Order} or the revocation of a licence or
authorisation, they may at the same time lodge with the High Court an
appeal against any requirement imposed under rE227 to rE229 as
appropriate.\\
\rulesubsection{regulations E239-E240 - Action to be taken by the Inn (in
circumstances where a barrister has been sanctioned to be
disbarred) }\par
\rulesubsubsection{rE239}
The Treasurer of the \emph{respondent's Inn }of Call  \textcolor{myred}{\textbf{must not}} fewer than
21 days, or more than 35 days, after the end of the \emph{Disciplinary
Tribunal's} proceedings (or, where the \emph{respondent} has given
notice of appeal to High Court against the finding and/or sanction, once
the time for appeal to the High Court has expired and any appeal to the
High Court has been disposed of) pronounce the sanction of disbarment
decided on by the \emph{Disciplinary Tribunal}, and take such further
action as may be required to carry the sanction into effect. The
Treasurer  \textcolor{myred}{\textbf{must}} inform the \emph{persons} specified in rE235 of the date
on which the sanction is to take effect, (which  \textcolor{myred}{\textbf{must}} be no later than
two working days after the date when that sanction is pronounced).\\
\rulesubsection{rE240}\par
In any case in which the \emph{respondent} has given notice of appeal to
the High Court against the finding and/or sanction of
the \emph{Disciplinary Tribunal} on the charges of \emph{professional
misconduct}, no action referred to in rE239 may be taken until the
appeal has been heard by the High Court, or otherwise disposed of
without a hearing.\\
\rulesubsection{regulations E241-E242 - Action to be taken by the Bar Standards
Board }\par
\rulesubsubsection{rE241}
Subject to rE242, the \emph{Bar Standards Board }must take the
appropriate steps to put the finding and/or sanction of
the \emph{Disciplinary Tribunal }into effect, except that in any case in
which an \emph{applicable person }has given notice of appeal to the High
Court against the finding and/or sanction of the \emph{Disciplinary
Tribunal }on the charges of \emph{professional
misconduct }or \emph{disqualification order, }no action may be taken
until the appeal has been heard by the High Court or otherwise disposed
of without a hearing.\\
\rulesubsection{rE242}\par
Where the finding and/or sanction of the \emph{Disciplinary Tribunal }is
that the \emph{BSB regulated person }should be subject to an immediate
suspension and/or immediate imposition of conditions in accordance with
rE226\textbf{ }the actions of the \emph{Bar Standards Board }must not be
deferred even if the \emph{BSB regulated person }has given notice of
appeal to the High Court against the finding and/or sanction of
the \emph{Disciplinary Tribunal }on the charges of \emph{professional
misconduct.}\par
\rulesubsection{regulations E243-E243A - Publication of finding, sanction and
report of the Disciplinary Tribunal }\par
\rulesubsubsection{rE243}
The following procedures apply to the publication of the finding and
sanction of a \emph{Disciplinary Tribunal:}\par

\nl \item \emph{BTAS:}\al
\item  \textcolor{myred}{\textbf{must}}, where charges are proved, publish the finding and sanction of
the \emph{Disciplinary Tribunal }on its website within 14 days of the
date when the \emph{Disciplinary Tribunal's }proceedings end, unless, on
application by the \emph{respondent} at the hearing,
the \emph{Disciplinary Tribunal} directs that it is not in the public
interest to publish the finding and/or sanction; and\\
\item  \textcolor{myred}{\textbf{must}}, where charges have been dismissed, including following an
application under rE127.2, not publish the finding on its website,
unless the \emph{respondent} so requests; and\la
\item The \emph{Bar Standards Board} is free to publish the findings and
sanction of a \emph{Disciplinary Tribunal} on its website in accordance
with rE243.1.\ln
\rulesubsection{rE243A}\par
The following procedures apply to the publication of the report of
the \emph{Disciplinary Tribunal} Decision:\\\nl \item \emph{BTAS}:\\\al
\item  \textcolor{myred}{\textbf{must}}, where charges are proved, publish the report of
the \emph{Disciplinary} \emph{Tribunal} decision on its website within a
reasonable time after the date when
the \emph{Disciplinary} \emph{Tribunal's} proceedings end, unless, on
application by the \emph{respondent} at the hearing,
the \emph{Disciplinary} \emph{Tribunal} directs that it is not in the
public interest to publish the report; and\\
\item  \textcolor{myred}{\textbf{must}}, where charges have been dismissed, including following an
application under rE127.2, not publish the report on its website, unless
the \emph{respondent} so requests; and\la
\item  \textcolor{myred}{\textbf{must}}, where charges have been dismissed, including following an
application under rE127.2, publish an anonymised summary of the report
on its website, unless on application by the \emph{respondent} at the
hearing, the \emph{Disciplinary} \emph{Tribunal} directs that it is not
in the public interest to publish the anonymised summary; and\\
\item may, where charges have been dismissed, publish the report of
the \emph{Disciplinary} \emph{Tribunal} on their websites at any time,
provided that in this case all details of the relevant parties involved
in the hearing are anonymised.\ln
\rulesubsection{regulations E244-E248 - Costs }\par
\rulesubsubsection{rE244}
A \emph{Disciplinary Tribunal} or Directions Judge may make such Orders
for costs, whether against or in favour of a \emph{respondent}, as
the \emph{Disciplinary Tribunal} or Directions Judge shall think fit.\\
\rulesubsection{rE245}\par
A party who wishes to make an application for costs  \textcolor{myred}{\textbf{must}}, no later than
24 hours before the commencement of the hearing, serve upon any other
party and file with \emph{BTAS }a schedule setting out the costs they
seek.\\
\rulesubsection{rE246}\par
Where it exercises its discretion to make an Order for costs,
a \emph{Disciplinary Tribunal }must either itself decide the amount of
such costs or direct \emph{BTAS }to appoint a suitably
qualified \emph{person }to do so on its behalf.\\
\rulesubsection{rE247}\par
Any costs ordered to be paid by or to a \emph{respondent}  \textcolor{myred}{\textbf{must}} be paid
to or by the \emph{Bar Standards Board}.\\
\rulesubsection{rE248}\par
All costs incurred by the \emph{Bar Standards Board} preparatory to the
hearing before the \emph{Disciplinary Tribunal}  \textcolor{myred}{\textbf{must}} be borne by
the \emph{Bar Standards Board}.\\
\rulesubsection{regulations E249-E250 - Service of documents }\par
\rulesubsubsection{rE249}
Any documents required to be served on a \emph{respondent} in connection
with proceedings under these Regulations shall be deemed to have been
validly served:\\\nl \item If sent by guaranteed delivery post, or other guaranteed or
acknowledged delivery, or receipted hand delivery to:\\\al
\item in the case of a \emph{BSB authorised individual}, the address
notified by them pursuant to the requirements of Part 2 of
this \emph{Handbook} (or any provisions amending or replacing it) as
their \emph{practising address}; or\\
\item in the case of a \emph{BSB entity}, its registered office address or
its principal office; or\\
\item in the case of a \emph{BSB regulated person} or non-\emph{authorised
individual} acting as a \emph{manager} or employee of a \emph{BSB
entity}, the address provided by the \emph{BSB entity} as their home
address or, in the absence of such information, the address of the
relevant \emph{BSB entity} notified pursuant to the requirements of Part
2 of this \emph{Handbook}; or\\
\item in either case, an address to which the \emph{respondent} has asked
in writing that such documents be sent; or\\
\item in the absence of any of the above, to their last known address;
or;\\
\item in the case of a \emph{BSB regulated person} or non-\emph{authorised
individual} acting as a \emph{manager }or employee of a \emph{BSB
entity}, the last known address of the relevant \emph{BSB entity},\la
and such service shall be deemed to have been made on the second working
day after the date of posting or on the next working day after receipted
hand delivery;\item If served by e-mail, where:\al
\item  the \emph{respondent's} e-mail address is known to the \emph{Bar
Standards Board}; and\\
\item the \emph{respondent} has asked for or agreed to service by e-mail,
or it is not possible to serve by other means;\la
and such service shall be deemed to have been made on the second working
day after the date the e-mail is sent;\item If actually served;\item If served in any way which may be directed by the \emph{Directions
judge} or the Chair of the\emph{ Disciplinary Tribunal}.\ln
\rulesubsection{rE250}\par
For the purpose of rE249.1, "receipted hand delivery" means  a delivery
by hand which is acknowledged by a receipt signed by
the \emph{respondent} or by a relevant representative of
the \emph{respondent} (including, for example,
the \emph{respondent's} clerk, or a \emph{manager} or employee of
the \emph{BSB entity} at which the \emph{respondent} work).\\
\rulesubsection{regulations E251-E253 - Delegation }\par
\rulesubsubsection{rE251}
The powers and functions conferred by these Regulations on
a \emph{Directions judge} may be exercised by any other \emph{Judge} or
Queen's Counsel nominated by \emph{the President}, including
the \emph{Judge} or Queen's Counsel designated in the \emph{Convening
Order} as Chair of\emph{ the Disciplinary Tribunal} appointed to hear
and determine the charge or charges against the \emph{respondent}, if
the \emph{Directions Judge} is unable to act due to absence, or for any
other reason.\\
\rulesubsection{rE252}\par
Any duty or function or step which, under these regulations, is to be
discharged or carried out by \emph{the President} may, if they are
unable to act due to absence or to any other reason, be discharged or
carried out by the Registrar of \emph{BTAS, }the Chair of
the \emph{Tribunal, }or by any other \emph{person }nominated in writing
by \emph{the President }for any specific purpose.\\
\rulesubsection{rE253}\par
Anything required by these Regulations to be done or any discretion
required to be exercised by, and any notice required to be given
to, \emph{the President} may be done or exercised by, or given to,
any \emph{person} authorised by \emph{the President}, either
prospectively or retrospectively and either generally or for a
particular purpose. Any authorisations given by the President under this
regulation  \textcolor{myred}{\textbf{must}} be in writing.\\
\rulesubsection{regulations E254-E258 - Exclusion from providing representation
funded by the Legal Aid Agency (Application for
termination) }\par
\rulesubsubsection{rE254}
A \emph{respondent} who has been excluded from legal aid work under
Section 42 of the Administration of Justice Act 1985 may apply for an
order ending their exclusion from providing representation funded by
the \emph{Legal Aid Agency} as part of the Community Legal Service or
Criminal Defence Service in accordance with rE256 below.\\
\rulesubsection{rE255}\par
Any such application  \textcolor{myred}{\textbf{must}} be in writing and addressed to the Chair of
the\emph{ Disciplinary Tribunal }that made the original order.\\
\rulesubsection{rE256}\par
\emph{The President} may dismiss the application, or may decide that
the \emph{respondent's} exclusion from providing representation funded
by the \emph{Legal Aid Agency} as part of the Criminal Legal Aid or
Civil Legal Aid be ended forthwith, or on a specified future date .\\
\rulesubsection{rE257}\par
The Chair of the\emph{ Disciplinary Tribunal}  \textcolor{myred}{\textbf{must notify}} their decision
in writing to all those \emph{persons} who received copies of the report
of the \emph{Disciplinary Tribunal} under rE235.\\
\rulesubsection{rE258}\par
\emph{The Disciplinary Tribunal }may make such order for costs in
relation to an application under rE244 as it thinks fit and rE244 to
rE248 apply with all necessary modifications.\\
\rulesubsection{regulation E259 - Interpretation }\par
\rulesubsubsection{rE259}
In Section 5.B all italicised terms shall be interpreted in accordance
with the definitions in Part 6. \\
\rulesection{Part 5 - B2. Citation and commencement }\par
\rulesubsubsection{rE260}
These Regulations may be cited as ``The \emph{Disciplinary
Tribunal} Regulations 2017''.\\
\rulesubsection{rE261}\par
These Regulations will come into effect on 1 November 2017 and shall
apply to all cases referred to a \emph{Disciplinary Tribunal }prior to
that date under the Regulations then applying, and any step taken in
relation to any \emph{Disciplinary Tribunal }pursuant to those
Regulations shall be regarded as having been taken pursuant to the
equivalent provisions of these Regulations.\\
\rulesubsection{rE261A}\par
Notwithstanding the provisions in rE164 and rE261,
the \emph{Disciplinary Tribunal}  \textcolor{myred}{\textbf{must}} apply the criminal standard of
proof when deciding:\\\nl \item charges of \emph{professional misconduct} where the conduct alleged
within that charge occurred prior to 1 April 2019, including where the
same alleged conduct continued beyond 31 March 2019 and forms the basis
of a single charge of \emph{professional misconduct}; and\item whether the \emph{disqualification condition} has been established,
in relation to an applicable person's alleged breach of duty or other
conduct which occurred prior to 1 April 2019, including where the same
alleged conduct continued beyond 31 March 2019.\ln
\rulesection{Part 5 - B3. Annexes to the Disciplinary Tribunals
Regulations }\mysec{Annex 1 - Sentencing Powers Against Barristers}

When a charge of \emph{professional misconduct} has been found proved
against a \emph{barrister }\footnote{If an application to disqualify the \emph{Barrister }from acting
as \emph{HOLP, }manager or employee of an \emph{authorised person }is
made in the same proceedings, the \emph{Disciplinary Tribunal }may also
disqualify the \emph{Barrister }in accordance with the provisions of
Annex 5.} by a \emph{Disciplinary Tribunal},
the \emph{Disciplinary Tribunal} may decide to:\\\nl \item order that they be disbarred;\item order that their \emph{practising certificate }be suspended for a
prescribed period;\item order that their \emph{practising certificate} should not be
renewed;\item order that conditions be imposed on their \emph{practising
certificate};\item order that they be prohibited, either indefinitely or for a
prescribed period and either unconditionally or subject to conditions,
from accepting or carrying out
any \emph{public} \emph{access} \emph{instructions};\item order that their authorisation to \emph{conduct} \emph{litigation} be
removed or suspended, or be subject to conditions imposed;\item order them to pay a fine of up to £50,000 to
the \emph{Bar} \emph{Standards} \emph{Board} (or up to £50,000,000 if
the charges relate to his or her time as an employee or manager of a
licensed body);\item order them to complete continuing professional development of such
nature and duration as the \emph{Disciplinary} \emph{Tribunal} may
direct, whether outstanding or additional requirements, and to provide
satisfactory proof of compliance with this order to
the \emph{supervision} \emph{team};\item reprimand them;\item give them advice about their future conduct;\item order them to attend on a nominated \emph{person} to be reprimanded;
or\item order them to attend on a nominated \emph{person} to be given advice
about their future conduct.\ln
\mysec{Annex 2 - Sentencing Powers Against BSB Authorised
Bodies }\par
If a \emph{Disciplinary Tribunal} finds a charge of \emph{professional
misconduct} proved against a \emph{BSB} \emph{authorised body},
the \emph{Disciplinary Tribunal} may decide to :\\\nl \item order that its authorisation to practise as \emph{a BSB authorised
body  }be removed;\item order that conditions be imposed on its authorisation to practise as
a \emph{BSB authorised body;}\item order that its authorisation to \emph{practise }for a prescribed
period be suspended (either unconditionally or subject to conditions);\item order that it\emph{ }be re-classified as a BSB licenced body (either
unconditionally or with conditions imposed on its licence to practise as
a \emph{BSB licensed body);}\item  order that its authorisation to \emph{conduct litigation }be
withdrawn or suspended, or be subject to conditions on it;\item order it to pay a fine of up to £250,000 to the \emph{Bar Standards
Board;}\par
.7 order that its \emph{managers }or employees\emph{ }complete
continuing professional development of such nature and duration as
the \emph{Disciplinary Tribunal may }direct and to provide satisfactory
proof of compliance with this order to the \emph{supervision team;}\par
.8 reprimand it;\item give it advice about its future conduct; or\\\item order it to attend (by its \emph{HOLP }or
other \emph{person }identified in the order) on a
nominated \emph{person }to be given advice about its future conduct.\ln
\mysec{Annex 3 - Sentencing Powers Against BSB Licensed
Bodies }\par
If a \emph{Disciplinary Tribunal} finds a charge of \emph{professional
misconduct} proved, against a \emph{BSB licensed
body }the \emph{Disciplinary Tribunal} may decide to:\\\nl \item revoke its licence to practise;\item suspend its licence to practise for a prescribed period (either
unconditionally or subject to conditions);\item impose conditions on its licence to practise;\item withdraw or suspend its \emph{right to conduct litigation} or to
impose conditions on it;\item order it to pay a fine of up to £250,000,000 to the \emph{Bar
Standards Board};\item order it to ensure that its \emph{managers} or employees complete
continuing professional development of such nature and duration as the
Tribunal shall direct and to provide satisfactory proof of compliance
with this order to the \emph{supervision team};\item reprimand it;\item give advice to it about its future conduct; or\item order it to attend (by its \emph{HOLP} or
other \emph{person} identified in the order) on a
nominated \emph{person} to be given advice about its future conduct.\\
\mysec{Annex 4 - Sentencing Powers Against Registered European
Lawyers }\par
If a \emph{Disciplinary Tribunal} finds a charge of \emph{professional
misconduct} proved against a \emph{registered European lawyer},
the \emph{Disciplinary Tribunal} may decide to:\\\nl \item order that they be removed from the \emph{register of European
lawyers;}\item order that they be suspended from the \emph{register of European
lawyers }for a prescribed period (either unconditionally or subject to
conditions);\item order a condition to be imposed on them prohibiting them, either
indefinitely or for a prescribed period and either unconditionally or
subject to conditions, from accepting or carrying out any \emph{public
access instructions;}\par
\item  order them to pay a fine of up to £50,000 to the \emph{Bar Standards
Board }(or of up to £50,000,000 if, the charges relate to their time as
an employee\emph{ }or \emph{manager }of \emph{a licensed body);}\par
\item  order them to complete continuing professional development of such
nature and duration as the \emph{Disciplinary Tribunal }shall direct,
whether outstanding or additional requirements, and to provide
satisfactory proof of compliance with this order to
the \emph{supervision team;}\par
\item  reprimand them;\item give them advice about their future conduct;\item order them to attend on a nominated \emph{person }to be reprimanded;
or\item order them to attend on a nominated \emph{person }to be given advice
about their future conduct.\ln
\mysec{Annex 5 - Sentencing Powers Against All Other BSB Regulated
Persons }\par
If a \emph{Disciplinary Tribunal} finds a charge of \emph{professional
misconduct} proved against any other \emph{BSB regulated
person }\footnote{If an application to disqualify is made in the same proceedings,
the \emph{Disciplinary Tribunal} may also disqualify a \emph{BSB
regulated person} in accordance with these Regulations.} the \emph{Disciplinary Tribunal} may decide\emph{ }to:\\\nl \item order them to pay a fine of up to £50,000 to the \emph{Bar Standards
Board }(or up to £50,000,000 if the charges relate to their time as an
employee\emph{ }or \emph{manager }of a \emph{licensed body);}\item reprimand them;\item give them advice about their future conduct;\item order them to attend on a nominated \emph{person }to be
reprimanded;\item order them to attend on a nominated \emph{person }to be given advice
about their future conduct.\ln

\mysec{Annex 6 - Standard Directions }\par
The \emph{standard directions }as referred to in rE103.3 are as follows
:\\
1. The hearing will be in public;\\
2. This timetable will commence on the second working day after filing
of these directions with the \emph{BTAS} and all time limits will run
from that date, unless stated otherwise ;\\
3. Within 28 days, ie by {[}date{]}:\\
3.1 all parties will provide \emph{BTAS }with dates when they are
available for the substantive hearing in the period between
{[}month/year{]} and {[}month/year{]}, failing which \emph{BTAS }may fix
the hearing without reference to the availability of any party;\\
3.2 the \emph{respondent }will specify:\\
(a) whether they admit the charges;\\
(b) if not, which areas of fact and/or law are in dispute;\\
4. Within 42 days, ie by {[}date{]}, the \emph{respondent  \textcolor{myred}{\textbf{must}} }provide
a copy of the documents and a list of witnesses, on which and on whom
they intend to rely, and copies of any witness statements on which they
intend to rely. The BSB is to provide copies of any witness statements
on which it intends to rely within 42 days, i.e. by {[}date{]}, if
required;\\
5. Within 56 days, ie by {[}date{]}, both the \emph{Bar Standards
Board} and the \emph{respondent}  \textcolor{myred}{\textbf{must}}:\\
5.1 serve written notice of the witnesses (if any) whom they require the
other party to tender for cross-examination;\\
5.2 provide a schedule setting out details of the witnesses they intend
to call and a time estimate for the evidence of each of their
witnesses;\\
6. At least 14 days before the date fixed for the substantive hearing:\\
6.1 the \emph{respondent} will provide to \emph{BTAS} {[}four/six{]}
copies of any defence bundle already provided under direction (5) for
circulation to the \emph{Disciplinary Tribunal} members, and at the same
time send a copy to the \emph{Bar Standards Board};\\
6.2 where the \emph{respondent} has indicated an intention to admit the
charge(s), the \emph{respondent} will provide
to \emph{BTAS} {[}four/six{]} copies of any financial documents or other
documentation the \emph{respondent }wishes to rely on in mitigation, in
the event that the charge(s) is found proved;\\
6.3 the \emph{Bar Standards Board }will provide
to \emph{BTAS }{[}four/six{]} copies of any bundle of evidence as
originally served under rE103 for circulation to the \emph{Disciplinary
Tribunal }members;\\
7. If either party seeks reasonable adjustments, to enable a person with
a disability to participate in the hearing, or measures under rE179 to
rE181, they  \textcolor{myred}{\textbf{must notify}} \emph{BTAS }as soon as possible and no later
than 21 days before the date fixed for the substantive hearing;\\
8. The estimated duration of the hearing is {[}number{]} days/hours;\\
9. Any skeleton argument to be relied on at the hearing be filed
with \emph{BTAS} and served on the other parties at least 48 hours
before the time fixed for the hearing;\\
10. There is liberty to apply to the \emph{Directions Judge }for further
directions.\\
\chap{Part 5 - C. The Interim Suspension and Disqualification
Regulations }\rulesection{Part 5 - C1. Application }\par
\rulesubsubsection{rE262}
This Section 5.C prescribes the manner in which the BSB may seek to take
interim action to:\\\nl \item suspend a \emph{BSB authorised person }(excluding, for the avoidance
of doubt, any \emph{unregistered barrister}); or\item \emph{disqualify} any \emph{applicable person} from acting as an
a \emph{HOLP} or a \emph{HOFA} or from working as a \emph{manager} or
employee of a \emph{BSB authorised person};\ln
subject to the criteria outlined at rE268 and rE269 below, and pending
consideration by a \emph{Disciplinary Tribunal} under Section 5.B.
\rulesubsection{rE263}\par
In addition to the above, this Section 5.C sets out the basis upon which
the Chair of the \emph{Independent Decision-Making Body} may impose an
immediate interim \emph{suspension} or \emph{disqualification} on
any \emph{applicable person} subject to the criteria outlined at rE270
to rE272 below, and pending consideration by an \emph{interim panel} in
accordance with this Section 5.C.\\
\rulesubsection{rE264}\par
Anything required by this Section 5.C to be done or any discretion
required to be exercised by, and any notice required to be given to,
the \emph{President} or the Chair of the \emph{Independent
Decision-Making Body }and\emph{ Commissioner}, may be done or exercised
by, or given to, any \emph{person} or body authorised by
the \emph{President} or by the Chair of the \emph{Independent
Decision-Making Body }and\emph{ Commissioner}as the case may be (either
prospectively or retrospectively and either generally or for a
particular purpose).\\
\rulesection{Part 5 - C2. The Regulations }\rulesubsection{regulations E265-E267 - Composition of panels }\par
\rulesubsubsection{rE265}
An \emph{interim panel} shall consist of three members nominated by
the \emph{President} being a Chair (who shall be a Queen's Counsel) and
two others, of whom at least one  \textcolor{myred}{\textbf{must}} be a \emph{lay member}. Provided
that:\\\nl \item the proceedings of an \emph{interim panel} shall be valid
notwithstanding that one of the members becomes unable to act or
is \emph{disqualified} from acting, so long as the number of members
present throughout the substantive hearing is not reduced below two and
continues to include the Chair and one \emph{lay member};\item no \emph{person} shall be appointed to serve on a panel if they:\al
\item are a member of the \emph{Bar }Council or of any of its committees;
or\\
\item are a member of the \emph{Bar Standards Board} or of any of its
committees; or\\
\item are a member of the \emph{Bar Standards Board} or any of its
committees at any time when the matter was being considered by
the \emph{Bar Standards Board}.\la\ln
\rulesubsection{rE266}\par
A \emph{review panel} shall consist of three members nominated
by \emph{the President} being a Chair (who shall be a Queen's Counsel)
and two others, of whom at least one  \textcolor{myred}{\textbf{must}} be a \emph{lay member}.
Provided that:\\\nl \item the proceedings of a \emph{review panel} shall be valid
notwithstanding that one of the members becomes unable to act or
is \emph{disqualified} from acting, so long as the number of members
present throughout the substantive hearing is not reduced below two and
continues to include the Chair and one \emph{lay member};\item no \emph{person} shall be appointed to serve on a panel if they:\al
\item are a member of the \emph{Bar Council} or of any of its committees;
or\\
\item are a member of the \emph{Bar Standards Board} or of any of its
committees; or\\
\item were a member of the \emph{Bar Standards Board} or any of its
committees at any time when the matter was being considered by
the \emph{Bar Standards Board};\la\item no individual who is intended to sit on the \emph{review panel} shall
have sat on either the \emph{interim panel} or the \emph{appeal
panel} considering the same matter.\ln
\rulesubsection{rE267}\par
An \emph{appeal panel} shall consist of three members nominated
by \emph{the President} being:\\\nl \item two Queen's Counsel, each of whom is entitled to sit as a Recorder or
a Deputy High Court Judge or who has been Queen's Counsel for at least
ten years. Unless the \emph{appeal panel} otherwise decides, the
senior \emph{barrister} member will be the Chair of the \emph{appeal
panel}; and\item a \emph{lay member}.\\
Provided that:\al
\item the proceedings of an \emph{appeal panel} shall be valid
notwithstanding that one of the members, becomes unable to act or
is \emph{disqualified} from acting, so long as the number of members
present throughout the substantive hearing is not reduced below two and
continues to include the Chair and the \emph{lay member};\\
\item no \emph{person} shall be appointed to serve on an \emph{appeal
panel} if they:\rl
\item are a member of the \emph{Bar }Council or of any of its committees;
or\\
\item are a member of the \emph{Bar Standards Board} or of any of its
committees; or\\
\item were a member of the \emph{Bar Standards Board} or any of its
committees at any time when the matter was being considered by
the \emph{Bar Standards Board};\lr\la\item no individual who is intended to sit on the \emph{appeal panel} shall
have sat on either the \emph{interim panel} or the \emph{review
panel} considering the same matter.\ln
\rulesubsection{regulations E268-E272 - Referral to an interim
panel }\par
\rulesubsubsection{rE268}
On receipt of a referral\emph{ }or any other information,
the \emph{Commissioner} may refer a \emph{respondent} to
an \emph{interim panel} if:\\\nl \item subject to rE269:\al
\item the \emph{respondent }has been convicted of, or charged with,
a \emph{criminal offence} in any jurisdiction other than a \emph{minor
criminal offence}; or\\
\item the \emph{respondent} has been convicted by another \emph{Approved
Regulator}, for which they have been sentenced to a period of suspension
or termination of the right to practise; or\\
\item the \emph{respondent} has been intervened into by the \emph{Bar
Standards Board}; or\\
\item  removed;\\
\item the referral is necessary to protect the interests
of \emph{clients} (or former or potential \emph{clients}); and\la\item the \emph{Commissioner} decides having regard to the \emph{regulatory
objectives} that pursuing an interim \emph{suspension} or an
interim \emph{disqualification order} is appropriate in all the
circumstances.\ln
\rulesubsection{rE269}\par
No matter shall be referred to an \emph{interim panel}  on any of the
grounds of referral set out in rE268.1.a to rE268.1.b unless
the \emph{Commissioner} considers that, whether singly or collectively,
 the relevant grounds of referral would warrant, in the case of
a \emph{BSB authorised person}, a charge of \emph{professional
misconduct} and referral to a \emph{Disciplinary Tribunal}, or, in the
case of a \emph{applicable person}, an application to
a \emph{Disciplinary Tribunal} for \emph{disqualification} (in each case
such referral or application to be made in accordance with Section
5.B).\\
\rulesubsection{rE270}\par
If the \emph{Commissioner} refers a \emph{respondent} to
an \emph{interim panel} under\textbf{ }rE268, the Chair of
the\emph{ Independent Decision-Making Body} shall consider whether or
not the \emph{respondent} should be subject to an immediate
interim \emph{suspension} or \emph{disqualification} under rE272 pending
disposal by the \emph{interim panel}.\\
\rulesubsection{rE271}\par
An immediate interim \emph{suspension} or \emph{disqualification} may
only be imposed if the Chair of the \emph{Independent Decision-Making
Body} is satisfied that such a course of action is justified having
considered the risk posed to the \emph{public} if such
interim \emph{suspension} or \emph{disqualification} were not
implemented and having regard to the \emph{regulatory objectives}.\\
\rulesubsection{rE272}\par
Any immediate
interim \emph{suspension} or \emph{disqualification} imposed by the
Chair of the \emph{Independent Decision-Making Body} shall:\\\nl \item take immediate effect;\item be notified in writing by the \emph{Commissioner} to
the \emph{respondent};\item remain in force until the earlier of:\al
\item  such time as an \emph{interim panel} has considered the matter; or\\
\item  the date falling four weeks after the date on which the immediate
interim \emph{suspension} or \emph{disqualification} is originally
imposed;\la\item where relevant, result in the removal of the relevant \emph{BSB
authorised individual's} \emph{practising certificate}, \emph{litigation
extension} and/or right to undertake public access work (as
appropriate);\item where relevant, result in the imposition of conditions on the
relevant \emph{BSB authorised person's} authorisation\textbf{ }and/or
licence (as appropriate)\item be published on the \emph{Bar Standards Board's} website; and\item be annotated on the \emph{Bar Standards Board's} register
of \emph{BSB authorised persons} which is to be maintained by
the \emph{Bar Standards Board} in accordance with rS60.2 and rS129 or be
included on the \emph{Bar }Standards Board's register of individuals
that are the subject of a \emph{disqualification order} (as
appropriate).\ln
\guidancesection{Guidance to Regulations E268-E272 }
\guidancesection{gE1}
If an immediate interim \emph{suspension} or \emph{disqualification} has
been imposed by the Chair of the \emph{Independent Decision-Making
Body} it  \textcolor{myred}{\textbf{must}} be considered by an \emph{interim panel} within four weeks
of the date that that the immediate
interim \emph{suspension} or \emph{disqualification} is originally
imposed. If it is not considered by an \emph{interim panel} within that
period, it shall automatically fall away and no further period of
interim \emph{suspension} or \emph{disqualification} may be imposed on
the \emph{respondent} until the matter is considered by an \emph{interim
panel}.\\
\guidancesection{gE2}\par
If, subsequent to the imposition of an
immediate \emph{suspension} or \emph{disqualification} under \textbf{ } rE271,
the \emph{applicable person} agrees to provide to
the \emph{Commissioner} an undertaking in written terms in accordance
with the provisions of rE274.4 below which is satisfactory to
the \emph{Commissioner} and which is subject to such conditions and for
such period as the \emph{Commissioner} may agree,
the \emph{Commissioner} may elect to remove or qualify the immediate
interim \emph{suspension} or \emph{disqualification} pending the
disposal of any charges or application by a \emph{Disciplinary
Tribunal}. For the avoidance of doubt, in these circumstances the
referral to the \emph{interim panel} shall also be withdrawn in
accordance with the provisions of rE275 below.\\
\rulesubsection{regulations E273-E275 - Procedure after referral to an Interim
Panel and, where relevant, the decision to impose an immediate interim
suspension or disqualification }\par
\rulesubsubsection{rE273}
As soon as practicable after the \emph{Commissioner} has made a decision
to refer a \emph{respondent} to an \emph{interim
panel},\emph{ }the \emph{Bar Standards Board} shall write to
the \emph{President} notifying them of the decision and informing them
about whether or not an immediate
interim \emph{suspension} or \emph{disqualification} has also been
imposed on such \emph{respondent}.\\
\rulesubsection{rE274}\par
As soon as practicable after receipt of the notice referred to in rE273,
the \emph{President} shall write to the \emph{respondent} notifying them
of the decision, together with a copy of these \emph{Enforcement
Regulations}, and briefly setting out the details that have caused the
referral to the \emph{interim panel}. The letter of notification
shall:\\\nl \item where relevant, inform the \emph{respondent} that they are the
subject of an immediate
interim \emph{suspension} or \emph{disqualification} (as appropriate)
together with a summary of the consequences of that decision;\item lay down a fixed time and date (normally not less than 14 and not
more than twenty-one days from the date of the letter) for the hearing
to take place. One alternative shall be given;\item invite the \emph{respondent} to accept one or other of the dates
proposed or to provide a written\emph{ }representation to
the \emph{President}, which should be copied to the \emph{Commissioner},
objecting to both dates with reasons and providing two further
alternative dates which shall be not more than:\al
\item  four weeks after the date of the imposition of the immediate
interim \emph{suspension} or \emph{disqualification}, where relevant;
or\\
\item in all other cases, twenty-one days from the date of the letter of
notification;\la
Any such representation  \textcolor{myred}{\textbf{must}} be received by the \emph{President} not
more than ten days from the date of the letter of notification.
The \emph{President} shall consider any such representation together
with any representations from the Commissioner, and either confirm one
of the original dates or re-fix the hearing. If no such representation
is received within ten days of the date of the letter of notification
the hearing shall take place at the time and date first fixed pursuant
to rE274.2 above. The \emph{President's} decision, which shall be
notified in writing to the \emph{respondent} by the \emph{President},
shall be final. Once fixed, a hearing date shall be vacated only in
exceptional circumstances and with the agreement of
the \emph{President}:
\item inform the \emph{respondent} that they may by letter to
the \emph{Commissioner} undertake, pending the disposal of any charge(s)
or application(s) by a \emph{Disciplinary Tribunal}:
\al \item to be immediately \emph{suspended} or \emph{disqualified} (in which
case the consequences set out at rE272.4 to rE272.7 would apply);\\
\item not to accept or carry out any \emph{public
access} \emph{instructions}; and/or\\
\item to inform their professional and/or lay \emph{clients} about
any \emph{convictions}, charges or other matters leading to a referral,
in written terms satisfactory to the \emph{Commissioner;}\la
and summarising the consequences of the \emph{respondent} electing to
make such an undertaking (which for the avoidance of doubt, may include
those set out at rE272.4 to rE272.7 above);\item shall inform the \emph{respondent} that they are entitled to make
representations in writing or orally, by themselves or by others on
their behalf; and\item removed.\ln
\rulesubsection{rE275}\par
If a \emph{respondent} sends a letter in accordance with rE274.4 above
which is satisfactory to the \emph{Commissioner}, the Chair shall accept
the undertaking contained in the letter in lieu of the \emph{interim
panel }imposing any period of interim \emph{suspension} or
interim \emph{disqualification} pending the disposal by
a \emph{Disciplinary Tribunal} of any charges of \emph{professional
misconduct} or applications for a \emph{disqualification order} (as the
case may be).\\
\rulesubsection{regulations E276-E278 - Procedure and powers of interim
panels }\par
\rulesubsubsection{rE276}
At any hearing of an \emph{interim panel} the proceedings shall be
governed by the rules of natural justice, subject to which:\\\nl \item the procedure shall be informal, the details being at the discretion
of the Chair of the \emph{interim panel};\item the \emph{respondent} shall be entitled to make representations in
writing or orally, by themselves or by another on their behalf, as to;\al
\item why a period of interim \emph{suspension} or
interim \emph{disqualification} should not be imposed; or\\
\item  why the \emph{interim panel} should not direct
the \emph{respondent} to notify their professional \emph{clients} and/or
lay \emph{clients} about any \emph{convictions}, charges or other
matters leading to a referral; or\\
\item  any further or alternative direction which the \emph{interim
panel }is empowered to make in relation to the \emph{respondent }under
rE278.3 below;\la
pending the disposal of any charges or applications by
a \emph{Disciplinary Tribunal};\item no witnesses may be called without the prior consent of the Chair of
the Panel and without the submission of a proof of evidence;\item the attendance of the \emph{respondent} shall be required. Should
they nevertheless fail to attend, the hearing may proceed in their
absence subject to the \emph{interim panel} being satisfied that this
course is appropriate. Should the \emph{interim panel} not be so
satisfied, it shall have the power to adjourn the hearing;\item the hearing shall not be in public unless so requested by
the \emph{respondent} and a record shall be taken electronically; and\item if the \emph{interim panel} decides an adjournment is necessary for
any reason, it may adjourn the hearing for such period and to such time
and place, and upon such terms, as it may think fit.\ln
\rulesubsection{rE277}\par
If the members of the \emph{interim panel} are not unanimous as to any
decision, the decision made shall be that of the majority of them. If
the members of the \emph{interim panel} are equally divided the decision
shall be that which is most favourable to the \emph{respondent}.\\
\rulesubsection{rE278}\par
At the conclusion of the hearing the \emph{interim panel}:\\\nl \item may decide not to impose any period of interim \emph{suspension},
interim \emph{disqualification} or other order;\item may impose a period of interim \emph{suspension} or
interim \emph{disqualification} (in each case, either unconditionally or
subject to conditions) pending the hearing before a \emph{Disciplinary
Tribunal}, provided that no interim \emph{suspension} or
interim \emph{disqualification} may be imposed unless the \emph{interim
panel }considers that:\al \item were a \emph{Disciplinary Tribunal} to find a related charge
of \emph{professional misconduct} proven, it would be likely to impose a
sentence of disbarment (with respect to \emph{barrister respondents}), a
sentence of \emph{suspension} (with respect to \emph{barrister
respondents} or\emph{ registered European lawyer
respondents} or \emph{BSB entity respondents}), revocation of the
licence or authorisation (with respect to \emph{BSB entity respondents})
or a \emph{disqualification order} (with respect to \emph{applicable
person respondents}); and\item such interim \emph{suspension} or interim \emph{disqualification} is
in the public interest;\la \item in lieu of imposing a period of interim \emph{suspension} or
interim \emph{disqualification}, the \emph{interim panel} may either:\al
\item where the \emph{respondent} is a \emph{BSB authorised person}, direct
the \emph{respondent} to carry out their or its future activities in
accordance with such interim conditions on
the \emph{respondent's }authorisation or licence as the \emph{interim
panel} may think fit pending final disposal of the charges or
application against them or it; or\\
\item where the \emph{respondent }is a \emph{manager} or employee of
a \emph{BSB authorised person}, direct such \emph{person} (after
affording the \emph{BSB authorised person} an opportunity to be heard)
to take such steps in relation to the \emph{respondent }as
the \emph{interim panel} may think fit, which may include limits on the
type of work the \emph{respondent }is to be permitted to do, or
requirements as to their supervision or training, pending final disposal
of the charges or application against them;\\
\item accept from the \emph{respondent }an undertaking in written terms
satisfactory to the \emph{interim panel} (and subject to such conditions
and for such period as the \emph{interim panel} may agree):\rl
\item to be immediately \emph{suspended} or \emph{disqualified}; or\\
\item  not to accept or carry out any \emph{public
access} \emph{instructions} or to \emph{conduct litigation}; or\\
\item  to inform their professional and lay \emph{clients} about
any \emph{convictions}, charges or other matters leading to a
referral;\lr\la\ln
pending the disposal of any charges or application by
a \emph{Disciplinary Tribunal} provided always that
the \emph{respondent }accepts that the following consequences may arise
as a result of such undertaking being provided depending on the nature
of the undertaking being provided:\item the removal of the relevant \emph{BSB authorised
individual's} \emph{practising certificate}, \emph{litigation
extension} and/or right to undertake \emph{public access} work (as
appropriate);\item the imposition of conditions on the relevant \emph{BSB authorised
person's} authorisation\textbf{ }and/or licence (as appropriate);\item publication of the details of such
interim \emph{suspension} or \emph{disqualification} on the \emph{Bar
Standards Board's} website; and\item either the inclusion of a note on the \emph{Bar Standards
Board's} register of \emph{BSB authorised persons} to the effect that
such \emph{BSB authorised person} is
temporarily \emph{suspended} from \emph{practice} or the inclusion of
the details of such interim \emph{disqualification} on
the \emph{Bar }Standards Board's register of individuals that are the
subject of a \emph{disqualification order};\item shall set down in writing signed by the Chair of the \emph{interim
panel} the decision of the \emph{interim panel} and the terms of any
period of interim \emph{suspension}, interim \emph{disqualification} or
interim condition imposed under these \emph{Interim Suspension and
Disqualification Regulations} or accepted (in the form of an
undertaking) under rE278.3.c above.\al
\item  Where the \emph{respondent} is a \emph{BSB authorised individual},
the imposition of any period of \emph{suspension} shall be recorded as
follows:\\
``That\ldots\ldots\ldots..be \emph{suspended} from \emph{practice} as a
\ldots\ldots\ldots\ldots\ldots\ldots\ldots{} and be prohibited from
holding themselves out as being a
\ldots\ldots\ldots\ldots\ldots\ldots\ldots{} for a period expiring on
{[}the \ldots\ldots.. day of\ldots\ldots\ldots\ldots\ldots/{[}insert
applicable condition/event on which expiry is contingent{]} or such
earlier date as a \emph{Disciplinary Tribunal} shall have disposed of
any charges that have caused the interim \emph{suspension} or
such \emph{Disciplinary Tribunal} may otherwise direct." (Note: If the
Panel decides that the \emph{suspension} should apply to only part of
the \emph{respondent}'\emph{s} \emph{practice} or shall be subject to
conditions, such part or such conditions (as the case may be) shall be
recorded);\\
\item  Where the \emph{respondent }is a \emph{BSB entity}, the imposition of
any period of \emph{suspension} shall be recorded as follows:\\
``That \ldots\ldots\ldots\ldots.. have its BSB
licence/authorisation \emph{suspended} for a period expiring on {[}the
\ldots\ldots. day of\ldots\ldots\ldots\ldots\ldots\ldots/{[}insert
applicable condition/event on which expiry is contingent{]} or such
earlier date as a \emph{Disciplinary Tribunal} shall have disposed of
any charges that have caused the interim \emph{suspension} or
such \emph{Disciplinary Tribunal} may otherwise direct." (Note: If the
Panel decides that the \emph{suspension} should apply to only part of
the \emph{respondent's practice} or shall be subject to conditions, such
part or such conditions (as the case may be) shall be recorded);\\
\item  Where the \emph{respondent }is an \emph{applicable person}, the
imposition of any period of \emph{disqualification} shall be recorded as
follows:\\
``That \ldots\ldots\ldots. be \emph{disqualified} from {[}\emph{specify
here the relevant capacities in respect of which the order applies,
which may be some or all of: }acting as
a \emph{HOLP}, \emph{HOFA} or \emph{manager} of any \emph{BSB entity} or
being an employee of any \emph{BSB authorised person}{]} and that
any \emph{BSB regulated person} is prohibited from permitting
the \emph{respondent }to work in any such capacity for a period expiring
on {[}the \ldots\ldots\ldots{} day
of\ldots\ldots\ldots\ldots\ldots\ldots/{[}insert applicable
condition/event on which expiry is contingent{]} or such earlier date as
a \emph{Disciplinary Tribunal} shall have disposed of any charges that
have caused the interim \emph{disqualification} or
such \emph{Disciplinary Tribunal} may otherwise direct";\la \item shall, if a period of interim \emph{suspension} or
interim \emph{disqualification} or an interim condition is imposed or a
written undertaking is accepted under these \emph{Interim Suspension and
Disqualification Rules}:\al \item a inform the \emph{respondent} of their right to request a \emph{review
panel} to review the matter as provided in rE279 below;\\
\item  inform the \emph{respondent} of their right of appeal as provided in
rE284 below;\\
\item  removed;\item may, if it has not already been referred to a \emph{Disciplinary
Tribunal}, refer the matter to a \emph{Disciplinary Tribunal}.\la\ln
\rulesubsection{regulations E279-E283 - Review }\par
\rulesubsubsection{rE279}
In the event of a significant change in circumstances or other good
reason the \emph{respondent} may at any time while on
interim \emph{suspension}, interim \emph{disqualification} or subject to
interim conditions make a request in writing to \emph{the President} for
a \emph{review panel} to be convened to review the matter.\\
\rulesubsection{rE280}\par
The letter  \textcolor{myred}{\textbf{must}} set out the details of any alleged change in
circumstances or good reason.  On receipt of such a letter
the \emph{President} may seek representations from
the \emph{Commissioner }and may in their discretion convene
a \emph{review panel} or refuse the request. In either case \emph{the
President} shall notify the \emph{respondent} in writing of the
decision. If the \emph{President} decides to convene a \emph{review
panel} the procedure to be followed for fixing the time and date of the
hearing shall be as set out in rE274.2 and rE274.3.\\
\rulesubsection{rE281}\par
The proceedings before a \emph{review panel} shall be by way of a
rehearing and the provisions of \textbf{ } rE276 above shall apply as if
for references therein to the \emph{interim panel} and the Chair of
the \emph{interim panel} there were substituted references respectively
to the \emph{review panel} and the Chair of the \emph{review panel}.\\
\rulesubsection{rE282}\par
Unless in the meantime the hearing before a \emph{Disciplinary
Tribunal} of any charges or applications arising from and/or related to
the referral to an \emph{interim panel} has commenced, a hearing by
a \emph{review panel} convened pursuant to rE279 above shall take place
at the time and date fixed. Such hearing shall be a rehearing of the
matter by the \emph{review panel} which may reconsider the matter as if
there had been no previous hearing.\\
\rulesubsection{rE283}\par
If the hearing before a \emph{Disciplinary Tribunal} of any charges or
applications arising from and/or related to the referral to
an \emph{interim panel} has commenced before the date fixed for a
rehearing by a \emph{review panel}, the date fixed for the rehearing
shall be vacated and any interim \emph{suspension},
interim \emph{disqualification} or interim conditions made or
undertaking accepted by the \emph{interim panel} shall continue until
such charges or applications have been disposed of by
the \emph{Disciplinary Tribunal}.\\
\rulesubsection{regulations E284-E289 - Appeals }\par
\rulesubsubsection{rE284}
A \emph{respondent} may by letter served on the \emph{President} and on
the \emph{Commissioner }not more than 14 days after the date of
the \emph{relevant decision} of an \emph{interim panel} give notice of
their wish to appeal against the decision.\\
\rulesubsection{rE285}\par
As soon as practicable after receipt of a letter in accordance with
rE284 above the \emph{President} shall convene an \emph{appeal
panel} and write to the \emph{respondent} notifying them of a fixed time
and date (normally not less than 14 and not more than twenty-one days
from the date of receipt of the letter) for the hearing to take place.
The \emph{respondent} may make a written representation, addressed to
the Chair of the proposed \emph{appeal panel}, objecting to the date
with reasons and providing two further alternative dates. Any such
representation  \textcolor{myred}{\textbf{must}} be received by the Chair of the \emph{appeal
panel} not more than 14 days from the date of the letter of
notification. The Chair shall consider any such representation and
either confirm the original date or re-fix the hearing. If no such
representation is received within 14 days of the date of the letter of
notification the hearing shall take place at the time and date
originally notified to the \emph{respondent}. The Chair's decision,
which shall be notified in writing to the \emph{respondent} shall be
final. Once fixed, a hearing date shall be vacated only in exceptional
circumstances and with the agreement of the Chair of the \emph{appeal
panel}.\\
\rulesubsection{rE286}\par
The proceedings before an \emph{appeal panel} shall be by way of a
rehearing and the provisions of rE276 above shall apply as if for
references therein to the \emph{interim panel} and the Chair of
the \emph{interim panel} there were substituted references respectively
to the \emph{appeal panel} and the Chair of the \emph{appeal panel}.\\
\rulesubsection{rE287}\par
At the conclusion of the hearing the \emph{appeal panel}:\\\nl \item may remove the period of interim \emph{suspension} or
interim \emph{disqualification} and/or any interim conditions imposed
under this\textbf{ }Section\textbf{ }5.C;\item may confirm the period of interim \emph{suspension} or
interim \emph{disqualification} or impose further or alternative interim
conditions, or substitute such shorter period (either unconditionally or
subject to conditions) as may be thought fit;\item in lieu of confirming or imposing a period of
interim \emph{suspension} or interim \emph{disqualification} or imposing
interim conditions, may accept from the \emph{respondent} in terms
satisfactory to the Chair of the Panel an undertaking in writing to
continue to be \emph{suspended}, \emph{disqualified} and/or to submit to
such conditions and for such period as the \emph{appeal panel} may
agree, pending the disposal of any charges by a \emph{Disciplinary
Tribunal};\item shall set down in writing signed by the Chair of the \emph{appeal
panel} the decision of the \emph{appeal panel} and the terms of any
interim \emph{suspension}, interim \emph{disqualification} or interim
conditions confirmed or imposed under rE287.2 above or undertaking
accepted under rE287.3 above;\item may, if it has not already been referred to a \emph{Disciplinary
Tribunal}, refer the matter to a \emph{Disciplinary Tribunal};\ln

If the members of the \emph{appeal panel} are not unanimous as to the
decision, the decision made shall be that of the majority of them. If
the members of the \emph{appeal panel} are equally divided, the decision
shall be that which is most favourable to the \emph{respondent}. Any
period of interim \emph{suspension} or
interim \emph{disqualification} or interim conditions having been set,
which is confirmed or imposed, shall be recorded as set out in rE278.4
above.
\rulesubsection{rE288}\par
A pending appeal to an \emph{appeal panel} shall not operate as a stay
of any period of interim \emph{suspension} or
interim \emph{disqualification} or interim conditions having been set or
the terms of any direction or undertaking which is/are the subject of
the appeal.\\
\rulesubsection{rE289}\par
There shall be no right of appeal from the decision of an \emph{appeal
panel}.\\
\rulesubsection{regulation E290 - Suspension or disqualification ceases to have
effect }\par
\rulesubsubsection{rE290}
Unless a \emph{Disciplinary Tribunal} shall otherwise direct, any period
of interim \emph{suspension} or \emph{disqualification} and any interim
conditions imposed by the \emph{interim panel} or \emph{appeal
panel} under this Section 5.C shall cease and
the \emph{respondent} shall cease to be bound by the terms of any
direction made or undertaking accepted by a \emph{interim panel} or
an \emph{appeal panel} immediately upon:\\\nl \item a \emph{Disciplinary Tribunal} dismissing or making an order
disposing of all charges of \emph{professional misconduct} or
applications for \emph{disqualification} based on the referral from
the \emph{interim panel};\item any appeal by the \emph{respondent} against the \emph{conviction} or
all the \emph{conviction(s)} which had caused the referral to
a \emph{interim panel} being successful;\item the acquittal of the \emph{respondent} of the criminal charge or (as
the case may be) all of the criminal charges which had caused the
referral to a \emph{interim panel};\item the criminal charge or (as the case may be) all of the criminal
charges which had caused the referral to an \emph{interim panel} being
withdrawn.\ln
\rulesubsection{regulation E291 - Costs }\par
\rulesubsubsection{rE291}
An \emph{interim panel}, \emph{review panel} and an \emph{appeal
panel} shall have no power to award costs.\\
\rulesubsection{regulations E292-E294 - Report and Publication of
Decisions }\par
\rulesubsubsection{rE292}
As soon as practicable after the conclusion of an\emph{ interim
panel} hearing or an \emph{appeal panel} hearing,
the \emph{President} shall confirm the decision to
the \emph{respondent} in writing.\\
\rulesubsection{rE293}\par
In any case where a period of interim \emph{suspension} or
interim \emph{disqualification} is imposed or an interim condition is
imposed under this Section 5.C or a direction is made requiring
notification to lay and/or professional \emph{clients} or an undertaking
from a \emph{respondent} is accepted, the \emph{President} shall
communicate brief details in writing of the fact that
the \emph{respondent} is on an interim
basis \emph{suspended}, \emph{disqualified} and/or subject to conditions
(as the case may be) to:\\\nl \item the \emph{respondent};\item the \emph{Chair of the Bar Standards Board and Commissioner};\item the \emph{respondent}'s head of \emph{chambers,
HOLP, }or \emph{employer} (as appropriate);\item in the case of a \emph{registered European lawyer}, their \emph{home
professional body};\item the Treasurers of the \emph{respondent's Inn }of Call\emph{ }and of
any other Inns of which they are a member;\item other \emph{Approved Regulators} and the \emph{LSB}; and\item those of the following whom the President deems, in their absolute
discretion, to be appropriate taking into account the particular
circumstances:\al
\item the Lord Chancellor;\\
\item the Lord Chief Justice;\\
\item the Attorney General;\\
\item the Director of Public Prosecutions;\\
\item the Chair of the Bar Council;\\
\item the Leaders of the six circuits;\\
\item such one or more press agencies or other publications, as
the \emph{Commissioner} may direct.\la\ln
\rulesubsection{rE294}\par
The \emph{Bar Standards Board} shall keep a record of those who are
subject to \emph{suspension} or \emph{disqualification order}s or
conditions imposed on their authorisation made under the procedures in
this \emph{Handbook} and shall publish details of any
interim \emph{suspension}, interim \emph{disqualification} or interim
conditions on its website and in such of its registers as it considers
appropriate, for as long as they remain in effect.\\
\rulesubsection{regulation E295-E296 - Service of documents }\par
\rulesubsubsection{rE295}
Any documents required to be served on a \emph{respondent} arising out
of or in connection with proceedings under these Regulations shall be
deemed to have been validly served:\\\nl \item If sent by registered post, or recorded delivery post, or receipted
hand delivery to:\al
\item  in the case of a \emph{BSB authorised individual}, the address
notified by such \emph{respondent} pursuant to the requirements of Part
2 of this \emph{Handbook} (or any provisions amending or replacing the
same) as being their \emph{practising address}; or\\
\item  in the case of a \emph{BSB regulated person} or \emph{non-authorised
individual} acting as a \emph{manager} or\textbf{ }employee of
a \emph{BSB entity}, the address provided by the \emph{BSB entity} as
being their home address or, in the absence of such information, the
address of the relevant \emph{BSB entity} notified pursuant to the
requirements of Part 2 of this \emph{Handbook}; or\\
\item  in either case, an address to which the \emph{respondent} may request
in writing that such documents be sent; or\\
\item in the absence of any of the above, to their last known address or;
in the case of a \emph{BSB regulated person} or \emph{non-authorised
individual} acting as a \emph{manager} or\textbf{ }employee of
a \emph{BSB entity}, the last known address of the relevant \emph{BSB
entity};\la
and such service shall be deemed to have been made on the second day
after it was posted, left with, delivered to or collected by the
relevant service provider, (provided that that day is a business day,
or, if not, the next business day after that day) or on the next working
day after receipted hand delivery;\item If served by e-mail, where:\al
\item  the \emph{respondent}'s e-mail address is known to the \emph{Bar
Standards Board}; and\\
\item the \emph{respondent} has requested or agreed to service by e-mail,
or it is not possible to serve by other means;\la
the \emph{respondent} has requested or agreed to service by e-mail, or
it is not possible to serve by other means;\item if actually served;\item if served in any way which may be directed by the \emph{President} of
the \emph{Council of the Inns of Court}.\ln
\rulesubsection{rE296}\par
For the purpose of this regulation "receipted hand delivery" means by a
delivery by hand which is acknowledged by a receipt signed by
the \emph{respondent} or a relevant representative of
such \emph{respondent} (including, for example, their clerk and
a \emph{manager} or employee of the \emph{BSB entity} at which they
work).\\
\rulesection{Part 5 - C3. Interpretation }\par
\rulesubsubsection{rE297}
In this Section 5.C unless the context otherwise requires all italicized
terms shall be defined and all terms shall be interpreted in accordance
with the definitions in Part 6.\\
\rulesection{Part 5 - C4. Commencement }\par
\rulesubsubsection{rE298}
These rules shall come into force in accordance with the provisions of
Part 1 of this \emph{Handbook}.\\
\rulesection{Part 5 - D. The Fitness to Practise Regulations }\par
These Regulations, commencing 6 January 2014, are made by the \emph{Bar
Standards Board}, in liaison with the \emph{Council of the Inns of
Court}, under section 21 (regulatory arrangements) Legal Services Act
2007, under authority delegated by the General Council of the Bar as the
Approved Regulator of the Bar under Part 1 of Schedule 4 to the Legal
Services Act 2007, and with the approval of the \emph{Legal Services
Board} under Paragraph 19 of Schedule 4 to the Legal Services Act
2007.\\
\rulesection{Part 5 - D1. Preliminaries }\par
Commencement and application\\
\rulesubsection{rE299}\par
These Regulations will come into effect on 6 January 2014 and shall
apply to all cases referred to a \emph{Fitness to Practise
Panel} or \emph{an Appeal Panel} prior to that date under the
Regulations then applying, and any step taken in relation to
any \emph{Fitness to Practise Pane}l or \emph{Appeal Panel} pursuant to
those Regulations shall be regarded as having been taken pursuant to the
equivalent provisions of these Regulations.\\
\rulesubsection{rE300}\par
Anything required by these Regulations to be done or any discretion
required to be exercised by, and any notice required to be given to,
the \emph{President} of the Council of the Inns of Court or
the \emph{Commissioner}, may be done or exercised by, or given to,
any \emph{person} or body authorised by the \emph{President} or by
the \emph{Commissioner} as the case may be (either prospectively or
retrospectively and either generally or for a particular purpose).\\
\mysec{Definitions}
\rulesubsection{rE301}\par
Any term defined in \emph{Definitions Section }of
the \emph{Handbook} shall carry the same meaning in these Regulations.
For the purpose of Section D4 of these Regulations alone, ``Individual''
includes anyone who was \emph{`BSB authorised individual'- }at the time
of any decisions taken by a \emph{Fitness to Practise Panel}.\\
\rulesection{Part 5 - D2. Constitution of Panels }\par
\rulesubsubsection{rE302}
The \emph{President} shall constitute \emph{Fitness to Practise
Panels} and \emph{Appeal Panels} (Panels) to exercise the functions
afforded to those Panels under these Regulations, in accordance with the
provisions set out Schedule 1.\\
\hspace*{0.333em}\par
\rulesection{Part 5 - D3. The Fitness to Practise Procedure }\rulesubsection{regulations E303-E308 - Referral to a Fitness to Practise
Panel }\par
\rulesubsubsection{rE303}
Where the \emph{Commissioner} receives information suggesting that
an \emph{Individual} is \emph{unfit to practise}, the matter shall be
considered under Regulation E305.\\
\rulesubsection{rE304}\par
The \emph{Commissioner} may carry out any investigation, appropriate to
the consideration of whether the \emph{Individual} may be \emph{unfit to
practise}, prior to consideration of any referral under Regulation
E306.\\
\rulesubsection{rE305}\par
Where  the \emph{Commissioner} receives information under Regulation
E303, the \emph{Commissioner} shall, subject to Regulation E307, as soon
as reasonably practicable, write to the \emph{Individual} concerned:\\\nl \item notifying them that information has been received which appears to
raise a question of whether they are \emph{unfit to practise}; and,\item providing them with copies of any information received under
Regulation E303 or obtained under Regulation E304.\ln
\rulesubsection{rE306}\par
Where the \emph{Commissioner}, following receipt of information under
Regulation E303 or during the \emph{Commissioner's} consideration of a
referral under the Enforcement Decision Regulations, considers that
an \emph{Individual} may be \emph{unfit to practise}, they shall refer
the matter to a \emph{Fitness to Practise Panel} for determination.\\
\rulesubsection{rE307}\par
No decision to refer shall be taken under Regulation E306 without
the \emph{Individual} having been provided with a reasonable opportunity
(as to the circumstance) to make representations on the matter.\\
\rulesubsection{rE308}\par
In reaching a decision under Regulation E306,
the \emph{Commissioner} shall take into account any information received
under Regulation E303 or obtained under Regulation E304, and any
representations submitted by the \emph{Individual}.\\
\rulesubsection{regulation E309 - Preliminary Hearings }\par
\rulesubsubsection{rE309}
As soon as reasonably practicable after referral of a matter by
the \emph{Commissioner} to a \emph{Fitness to Practise Panel}, the Chair
of the Panel shall send a notice in writing of the referral to
the \emph{Individual} which shall:\\\nl \item contain a summary of the case and the reasons why it has been
referred to a \emph{Fitness to Practise Panel};\item inform the \emph{Individual} of the time and date for a preliminary
hearing before the Panel;\item inform the \emph{Individual} of their right to attend and be
represented at the preliminary hearing, and to produce evidence at the
preliminary hearing, in accordance with Regulations E335.2 and E335.3
below;\item inform the \emph{Individual} of the Panel's powers at a preliminary
hearing under Regulations E310 and rE313 to rE316 below; and,\item inform the \emph{Individual} of their right to appeal under
Regulation E328 below.\ln
\rulesubsection{regulation E310 - Directions }\par
\rulesubsubsection{rE310}
At a preliminary hearing, the \emph{Fitness to Practise Panel} may give
directions for the full hearing before the Panel, which may include
that:\\\nl \item the \emph{Individual}, within a specified period of time, submit to a
relevant medical examination to be carried out by a \emph{Medical
Examiner} nominated by the Panel;\item the \emph{Bar Standards Board} instruct a \emph{Medical Examiner} to
conduct such examination and to provide a report setting out an opinion
as to whether the \emph{Individual} is \emph{unfit to practise} and as
to any other matters as may be specified by the Panel;\item the \emph{Individual} authorise disclosure to the \emph{Bar Standards
Board }and the \emph{Medical Examiner}, of such of their relevant
medical records as may be reasonably required for the purposes of the
medical examination and subsequent report; and,\item the \emph{Bar Standards Board }carry out such other investigations or
seek such advice or assistance as the Panel considers appropriate to the
matters for consideration at the full hearing, and where it gives a
direction under Paragraph .1 or .3 above, it shall inform
the \emph{Individual} that failure to comply with the direction may be
taken into account by the Panel in accordance with Regulation E319.2\ln

\rulesubsection{regulations E311-E312 - Medical Examinations}\par
\rulesubsubsection{rE311}
Where a \emph{Medical Examiner} is nominated by a Panel under Regulation
E310.1 or E320.2.a, the \emph{Medical Examiner} shall:\\\nl \item within the period specified by the Panel, undertake a relevant
medical examination of the \emph{Individual} in accordance with any
directions from the Panel;\item prepare a report which shall express an opinion as to:\al \item whether the \emph{Individual} has a physical or mental condition;\\
\item  whether the \emph{Individual} is fit to practise either generally or
on a restricted basis; and\\
\item any other matters which they have been instructed to address, in
accordance with any directions of the Panel; and\la
\item where requested by the \emph{Commissioner} to do so, attend a hearing
to present their findings.\ln
\rulesubsection{rE312}\par
An \emph{Individual's} medical records and any report prepared by
a \emph{Medical Examiner} under these Regulations shall not be used for
any other purpose than is provided for in these Regulations and shall
not be disclosed to any other \emph{person} or body without the consent
in writing of the \emph{Individual.}\par
\rulesubsection{regulations E313-E317 - Interim Restrictions }\par
\rulesubsubsection{rE313}
At a preliminary hearing, a \emph{Fitness to Practise Panel} may, where
it is satisfied that it is necessary to protect the public, is otherwise
in the public interest or is in the \emph{Individual's} own interests to
do so, direct that the \emph{Individual }is subject to an
interim \emph{restriction}.\\
\rulesubsection{rE314}\par
An interim \emph{restriction} may be imposed subject to such conditions
as the Panel may consider appropriate, and shall have effect pending the
determination of the matter at a full hearing before the \emph{Fitness
to Practise Panel} for a specified period, which shall not, save in
exceptional circumstances, exceed 3 months.\\
\rulesubsection{rE315}\par
In lieu of imposing an interim \emph{restriction} under Regulation E313
above, the Panel may accept from the \emph{Individual} an undertaking in
writing on terms satisfactory to the Panel:\\\nl \item  agreeing to an immediate interim \emph{restriction} for such period
as may be agreed; or,\item as to the \emph{Individual's} conduct or behaviour pending the
conclusion of the full hearing.\ln
\rulesubsection{rE316}\par
Where it has directed an interim \emph{restriction} under Regulation
E313 or accepted undertakings under Regulation E315, a Panel may, at any
point during the period of an interim restriction:\\\nl \item at the request of the \emph{Commissioner} or of the Individual,
direct that the interim \emph{restriction} or undertaking be reviewed at
a further hearing of the Panel, on such date as the Panel shall specify,
or on an unspecified date provided that the \emph{Individual} is served
with no less than 14 days' notice in writing of the hearing;\item at the request of the \emph{Individual}, direct an expedited full
hearing of the \emph{Fitness to Practise Panel};\\
and, shall;\al
\item inform the \emph{Individual} of their right to request
a \emph{Fitness to Practise Panel }to review the
interim \emph{restriction} or undertaking under Regulation E324 below;\\
\item inform the \emph{Individual} of their right of appeal under
Regulation E328 below.\la\ln
\rulesubsection{rE317}\par
The Chair of the Panel shall record, in writing, the decision of the
Panel, together with its reasons and the terms of any direction made,
interim \emph{restriction} imposed or undertakings accepted.
\rulesubsection{regulation E318 - Full Hearings before a Fitness to Practise
Panel }\par
\rulesubsubsection{rE318}
As soon as reasonably practicable after receipt of any report prepared
by a \emph{Medical Examiner} or, where no report has been prepared,
the \emph{Commissioner} considers that the case is ready for hearing,
the Chair of the Panel shall send a notice in writing of hearing to
the \emph{Individual} which shall:\\\nl \item contain a summary of the case and a copy of the report, where
applicable;\item inform the \emph{Individual} of the time and date of the full
hearing;\item inform the \emph{Individual} of their right to attend and be
represented at the hearing, and to produce evidence at the hearing, in
accordance with Regulations E335.2 and .3 below;\item inform the \emph{Individual} of the Panel's powers at a full hearing
under Regulations E319 to E321 below; and,\item inform the \emph{Individual} of their right to appeal under
Regulation E328 below.\ln
\rulesubsection{regulations E319-E323 - Decisions of a Fitness to Practise
Panel }\par
\rulesubsubsection{rE319}
At a full hearing, the \emph{Fitness to Practise Panel} shall decide
whether the \emph{Individual} is \emph{unfit to practise} and, in
reaching its decision, shall be entitled to take into account:\\\nl \item the \emph{Individual's} current physical or mental condition, any
continuing or episodic condition experienced by the \emph{Individual},
or any condition experienced by the \emph{Individual} which, although
currently in remission, may be expected to cause impairment if it
recurs; and\item any failure by the \emph{Individual} to comply with a direction to
undergo a relevant medical examination made under Regulation E310.1.\ln
\rulesubsection{rE320}\par
Where a \emph{Fitness to Practise Panel} has decided that
an \emph{Individual} is \emph{unfit to practise}, the Panel may
direct:\\\nl \item that the \emph{Individual} be subject to a \emph{restriction} which
may be subject to such conditions as the Panel may consider appropriate,
and which may be imposed indefinitely or for such period, not exceeding
six months, as shall be specified in the direction;\item that the \emph{Individual's} right to continue to practise, or to
resume practice after any period of \emph{restriction} shall be subject
to such conditions as the Panel may think fit, including that
the \emph{Individual}:\al
\item  submit for regular examination before one or more \emph{Medical
Examiners} nominated by the Panel,\\
\item  authorise disclosure to the \emph{Commissioner} and the \emph{Medical
Examiner }such of their medical records as may be reasonably required
for the purposes of the medical examination and subsequent report,\\
\item  is reviewed by a registered medical practitioner and shall follow the
treatment they recommend in respect of any physical or mental condition,
which the Panel consider may be a cause of the Individual being unfit to
practice.\la\ln
\rulesubsection{rE321}\par
In lieu of imposing any direction under Regulation E320 above, the Panel
may accept from the \emph{Individual} one or more undertakings in
writing in which the \emph{Individual} agrees to such period
of \emph{restriction}, or such conditions, as the Panel would otherwise
have imposed.\\
\rulesubsection{rE322}\par
Where it has made a direction under Regulation E320 or agreed
undertakings under Regulation E321, the Panel shall inform
the \emph{Individual}:\\\nl \item of their right to request a \emph{Fitness to Practise Panel} to
review any direction made, or undertakings agreed, under Regulation E324
below;\item of their right of appeal under Regulation E328 below; and\item that a failure to comply with the direction or undertakings would be
likely to result in a charge of professional misconduct being brought
against the \emph{Individual} before a Disciplinary Tribunal.\ln
\rulesubsection{rE323}\par
The Chair of the Panel shall record, in writing, the decision of the
Panel, together with its reasons and the terms of any direction made or
undertakings accepted.\\
\rulesection{Part 5 - D4. Reviews and Appeals }\rulesubsection{regulations E324-E327 - Review of decisions made by a Fitness to
Practise Panel }\par
\rulesubsubsection{rE324}
At any time during which an \emph{Individual} is subject to a period
of \emph{restriction} or conditions, directed or undertaken pursuant to
these Regulations, the \emph{Commisioner} may, of their own motion, or
at the request of the \emph{Individual}, refer the matter to be reviewed
before a \emph{Fitness to Practise Panel}, where they consider there has
been a significant change in the \emph{Individual 's} circumstances or
that there is some other good reason for a review to be undertaken.\\
\rulesubsection{rE325}\par
Where a case has been referred to a \emph{Fitness to Practise Panel} for
a review hearing under Regulation E324, Regulations E309 to E323 and
E335 shall apply, save that the Chair of the Panel and
the \emph{Individual} may agree in writing that no preliminary hearing
shall be held.\\
\hspace*{0.333em}\par
\rulesubsection{rE326}\par
At the conclusion of a review hearing, the \emph{Fitness to Practise
Panel} may:\\\nl \item confirm or revoke the direction made or undertakings agreed;\item extend or vary (or further extend or vary) the period for which the
direction has effect, or agree with the \emph{Individual} concerned an
extension or variation of the period for which an undertaking has been
agreed;\item replace the direction or undertakings, exercising any of the powers
of a \emph{Fitness to Practise Panel} under Regulations E313, E315, E320
or E321 above.\ln
\rulesubsection{rE327}\par
Where a case has been referred to a \emph{Fitness to Practise Panel} for
a review hearing under Regulation E324 above and the review hearing
cannot be concluded before the expiry of any period
of \emph{restriction} imposed under Regulation E314 or E320.1, or agreed
under Regulation E315.1 or E321, the Panel may extend
the \emph{restriction} for such period as it considers necessary to
allow for the conclusion of the review hearing.\\
\rulesubsection{regulations E328-E330 - Appeals before an Appeal
Panel }\par
\rulesubsubsection{rE328}
An \emph{Individual} may appeal a decision of a \emph{Fitness to
Practise Panel} to impose, extend, vary or replace a period
of \emph{restriction} by notifying the \emph{President} in writing that
they wish to do so, no more than 14 days after the date of the decision
subject to appeal.\\
\rulesubsection{rE329}\par
As soon as reasonably practicable after receipt of an appeal under
Regulation E328, the Chair of the \emph{Appeal Panel} shall send a
notice in writing of the appeal hearing to the \emph{Individual}, which
shall:\\\nl \item inform the \emph{Individual } of the time and date of the appeal
hearing;\item inform the \emph{Individual } of their right to attend and be
represented at the hearing, and to produce evidence at the hearing, in
accordance with Regulations E335.2 and .3 below; and\item inform the \emph{Individual} of the Panel's powers under Regulation
E331 below.\ln
\rulesubsection{rE330}\par
A pending appeal to an \emph{Appeal} \emph{Panel} shall not operate as a
stay of the decision subject to appeal.\\
\rulesubsection{regulations E331-E334 - Decisions of an Appeal
Panel }\par
\rulesubsubsection{rE331}
At the conclusion of an appeal hearing, the \emph{Appeal Panel} may:\\\nl \item allow the appeal;\item confirm the decision that is subject to appeal;\item exercise any of the powers of a \emph{Fitness to Practise
Panel} under Regulations E320 or E321 above;\ln
\rulesubsection{rE332}\par
The \emph{Appeal Panel} shall inform the \emph{Individual}:\\\nl \item of their right to request a \emph{Fitness to Practise Panel} to
review any direction made, or undertakings agreed, under Regulation E324
above; and\item that failure to comply with a \emph{restriction} or condition imposed
under Regulation E331.3 above would be likely to result in a charge of
professional misconduct being brought before a \emph{Disciplinary
Tribunal}.\ln
\rulesubsection{rE333}\par
The Chair of the Panel shall record, in writing, the decision of the
Panel, together with its reasons, and the terms of
any \emph{restriction} imposed or undertakings accepted.\\
\rulesubsection{rE334}\par
There shall be no right of appeal from a decision of an \emph{Appeal
Panel}.\\
\rulesection{Part 5 - D5. Conduct of Fitness to Practise and Review Panel
Hearings }\rulesubsection{regulations E335-E341 - Procedure before a Panel }\par
\rulesubsubsection{rE335}
At any hearing before a \emph{Fitness to
Practise} or \emph{Appeal Panel}, the proceedings shall be governed by
the rules of natural justice, subject to which:\\\nl \item the procedure shall be informal, the details being at the discretion
of the Chair of the Panel;\item the \emph{Individual} shall attend the hearing and may be represented
by another member of the bar or a solicitor, save that where
the \emph{Individual} does not attend and is not represented, the
hearing may nevertheless proceed if the Panel is satisfied that it is
appropriate to do so and that all reasonable efforts have been made to
serve the \emph{Individual} with notice in writing of the hearing in
accordance with these Regulations;\item the \emph{Individual }may, on their own behalf or through their
representative:\al
\item  make representations in writing or orally,\\
\item  produce evidence, provided (but subject to the discretion of the
Chair) that a proof of such evidence has been submitted no less than 24
hours prior to the hearing, and\\
\item put questions to any \emph{Medical Examiner} whose report is in
evidence before the Panel;\la
\item the hearing shall be in private, unless
the \emph{Individual }requests a public hearing, and shall be recorded
electronically;\item decisions shall be taken by simple majority;\item where the votes are equal the issue shall be decided, at a hearing
before a \emph{Fitness to Practise Panel}, in
the \emph{Individual 's} favour and, in an appeal case, against
the \emph{Individual}.\ln
\rulesubsection{rE336}\par
If at any time it appears to a Panel that it would be appropriate to do
so, the Panel may refer the case to the \emph{Commissioner} for
consideration of whether to refer any matter for a hearing before
a \emph{Disciplinary Tribunal}. 
\rulesubsection{rE337}\par
Where it considers it necessary, a Panel may appoint a practising
barrister or solicitor to assist it on any question of law or
interpretation of these Regulations, by providing an independent advice
either orally or in writing, such advice to be tendered in the presence
of the parties, or, where the parties are not present at the hearing,
copied to the parties as soon as reasonably practicable.\\
\rulesubsection{rE338}\par
A Panel shall have no power to award costs.\\
\rulesubsection{rE339}\par
The proceedings before an \emph{Appeal} \emph{Panel} shall be by way of
a rehearing.\\
\rulesubsection{rE340}\par
At any review hearing before a \emph{Fitness to Practise Panel} or
appeal hearing before an \emph{Appeal Panel,} copies of the report of
any expert or any proof of evidence referred to at any previous hearing
of the Panel in respect of the same case may be referred to by the
Panel.\\
\rulesubsection{rE341}\par
In the arrangements that it makes to perform its functions, and in
undertaking its functions, in particular, in reaching any decision
concerning an \emph{ Individual's} fitness to practise, a Panel shall:\\\nl \item take into account its duties to make reasonable adjustments which
arise under the Equality Act 2010; and\item have due regard to the need to:\al
\item  eliminate unlawful discrimination and other conduct prohibited by the
Equality Act 2010, and\\
\item  advance equality of opportunity and foster good relations
between \emph{persons} who share a relevant protected characteristic as
set out in Section 149 of the Equality Act 2010 and those who do not.\la\ln
\rulesubsection{regulations E342-E346 - Postponement, adjournment and
cancellation }\par
\rulesubsubsection{rE342}
Before the opening of any hearing in which notice has been served in
writing in accordance with these Regulations, the Chair of the Panel
may, of their motion or on the application of the \emph{Bar Standards
Board} or the \emph{Individual}, postpone the hearing until such time
and date as they think fit.\\
\rulesubsection{rE343}\par
Where any hearing under these Regulations has commenced, the Panel
considering the matter may, at any stage in the proceedings, whether of
its own motion or on the application of the \emph{Bar Standards
Board} or the \emph{ Individual}, adjourn the hearing until such time
and date as it thinks fit.\\
\rulesubsection{rE344}\par
No hearing shall be postponed or adjourned under Regulations E342 or
E343 unless the \emph{ Individual} has been given reasonable opportunity
to make representations on the matter. \\
\rulesubsection{rE345}\par
Where a hearing has been postponed or adjourned, the parties shall be
notified as soon as reasonably practicable of the time, of the date and
place at which the hearing is to take place or to resume.\\
\rulesubsection{rE346}\par
Where notice of hearing has been served in writing under these
Regulations, the Chair of the Panel may, on application of the \emph{Bar
Standards Board} or the \emph{Individual}, cancel the hearing where the
Chair considers that there are no reasonable grounds for questioning
whether the \emph{Individual} is \emph{unfit to practise}.\\
\rulesubsection{regulations E347-E348 - Notice and publication of
Decisions }\par
\rulesubsubsection{rE347}
Where a decision has been taken by \emph{Fitness to Practise Panel} or
an \emph{Appeal Panel} under these Regulations, the Chair of the Panel
shall, as soon as reasonably practicable, serve notice in writing of the
decision on the \emph{Individual} concerned.\\
\rulesubsection{rE348}\par
Where a decision is taken at a full hearing of a \emph{Fitness to
Practise Panel} or at an \emph{Appeal Panel} hearing, unless the
decision is to take no action and the \emph{Individual} is permitted to
continue to practise without \emph{restriction}, the Chair shall provide
notice in writing of the decision to any \emph{person} to whom they
consider it to be in the public interest to do so.\\
\rulesubsection{regulations E349-E351 - Service of documents }\par
\rulesubsubsection{rE349}
Regulation rE249 of the Disciplinary Tribunals Regulations 2017 (section
5.B) shall apply for the purposes of the service of any notices or
documents under these Regulations save that, for the reference in
Regulation rE249.4 to the ``\emph{Directions Judge} or the Chair of
the \emph{Disciplinary Tribunal}'', there shall be substituted the
``Chair of the Panel''.\\
\rulesubsection{rE350}\par
Where a Panel directs that an \emph{Individual's} ability to practise be
subject to \emph{restrictions,} conditions or agreed undertakings,
the \emph{President} shall always communicate brief details of the
decision, in writing to:\\\nl \item the \emph{Individual};\item the \emph{Commissioner} of the \emph{Bar Standards Board};\item  the Barrister's Head of Chambers, where relevant;\item  the Treasurers of the Barrister's \emph{Inn} of Call and of any
other Inns of which they are a member, where relevant; and\item  other regulators, where relevant.\ln
\rulesubsection{rE351}

The following shall have details of the decision of the Panel
communicated to them in writing, at the discretion of
the\emph{ President}:\\\nl \item  the Chair of the \emph{Bar Council};\item the Lord Chancellor;\item the Lord Chief Justice;\item the Attorney General;\item the Director of Public Prosecutions; and,\item the Leaders of the six circuits.\ln
\hspace*{0.333em}\par
\mysec{Schedule 1 - Constitution of Fitness to Practise and Appeal
Panels}\par
1. The \emph{President} shall appoint and maintain:\\
(a) a list of barristers and \emph{lay persons} eligible to be members
of \emph{Fitness to Practise Panel};\\
(b) a list of barristers and \emph{lay persons} eligible to be members
of an \emph{Appeal Panel}; and,\\
(c) from the lists at (a) and (b), lists of Queen's Counsel eligible to
act as Chairs of a \emph{Fitness to Practise Panel} and an \emph{Appeal
Panel} respectively. \\
2. The \emph{President} shall remove from the lists at Paragraph
1 \emph{persons}:\\
(a) whose term of appointment has come to an end, unless that term is
renewed;\\
(b) who resign from the relevant list by giving notice in writing to
that effect to the \emph{President}; or\\
(c) who in the opinion of the \emph{President} have ceased to be
eligible for appointment.\\
3. The \emph{President} shall appoint, and ensure that arrangements are
in place to be able to access suitably qualified \emph{medical
members} to sit on Fitness to Practise and Appeal Panels.\\
4. A \emph{Fitness to Practise Panel} shall consist of five members
selected by the \emph{President} from the list of \emph{persons} under
Paragraph 1(a) and in line with the arrangements arising from paragraph
3, being:\\
(a) a Chair whose name appears on the relevant list at Paragraph 1(c);\\
(b) two practising barristers;\\
(c) a \emph{medical member}; and\\
(d) a \emph{lay member}.\\
5. An \emph{Appeal Panel} shall consist of four members selected by
the \emph{President} from the list of \emph{persons} under Paragraph
1(b) and in line with paragraph 3, being:\\
(a) two practising barristers, including a Chair whose name appears on
the relevant list at Paragraph 1(c), and who shall, unless
the \emph{Appeal Panel} decide otherwise, be the most senior of the
barrister members;\\
(b) a \emph{medical member}; and\\
(c) a \emph{lay member}.\\
6. No \emph{person} shall be selected to sit on a \emph{Fitness to
Practise Panel} or an \emph{Appeal Panel} if:\\
(a) they are a member of the \emph{BSB} or any of its other Committees
or the Independent Decision-Making Body; or\\
(b) removed.\\
7. No \emph{person} shall sit on a \emph{Fitness to Practise Panel }or
an \emph{Appeal Panel} for the hearing of a case that they have
previously considered or adjudicated upon in any other capacity.\\
8. The proceedings of a \emph{Fitness to Practise Panel }or
an \emph{Appeal Panel} shall be valid notwithstanding that one or more
members of the Panel become unable to sit or disqualified from sitting
on the Panel, or are replaced by another member from the appropriate
list or by the arrangement at paragraph 3, subject to there being a
minimum of three Members which shall include a Chair from the relevant
list held under Paragraph 1(c), a \emph{medical member} and a \emph{lay
member}.\\
9. The validity of the proceedings of a Panel shall not be affected by
any defect in the appointment of a member.\\
\rulesection{Part 5 - E. Interventions and Divestiture }\rulesection{Part 5 - E1. Interventions }\par
\rulesubsubsection{rE352}
The \emph{Bar Standards Board} has the statutory power under Schedule 14
of the Legal Services Act 2007 (as amended by the Legal Services Act
2007 (General Council of the Bar) (Modification of Functions) Order
2018) to intervene into a \emph{BSB authorised person}.\\
\rulesubsection{rE353}\par
The \emph{Bar Standards Board} may authorise an intervention into
a \emph{BSB authorised person} where:\\\nl \item in relation to the \emph{BSB authorised person}, one or more of the
intervention conditions (as such term is defined in the Legal Services
Act 2007) is satisfied; or\item the licence, authorisation or \emph{practising certificate} granted
to the \emph{BSB authorised person} has expired and has not been renewed
or replaced by the \emph{Bar Standards Board}.\ln
\rulesubsection{rE354}\par
In circumstances where the \emph{Bar Standards Board} authorises an
intervention under rE353 above, such intervention shall be carried out
in accordance with the provisions of the Legal Services Act 2007 and
the \emph{Bar Standards Board's} Statutory Interventions Strategy.\\
\rulesection{Part 5 - E2. Divestiture }\par
\rulesubsubsection{rE355}
The \emph{Bar Standards Board} has the statutory power under Schedule 13
of the Legal Services Act 2007 to make an application for divestiture in
relation to a \emph{non-authorised person} and a \emph{BSB licensed
body}.\\
\rulesubsection{rE356}\par
The \emph{Bar Standards Board} may make an application for divestiture
if the divestiture condition (as such term is defined in the Legal
Services Act 2007) is satisfied in relation to such \emph{non-authorised
person} and a \emph{BSB licensed body} (as the case may be).\\
\rulesubsection{rE357}\par
In circumstances where the \emph{Bar Standards Board} elects to make an
application for divestiture under rE356 above, such application shall be
carried out in accordance with the provisions of the Legal Services Act
2007.\\
