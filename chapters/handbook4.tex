\part{Qualification Rules}
\rulesection{Part 4
- A. Application }


\rulesubsubsection{rQ1}

Section 4.B applies to all individuals who wish to be called to the
\emph{Bar} and to become qualified to practise as a \emph{barrister} and
to \emph{Authorised Education and Training Organisations (AETOs)}.~Until
1 January 2020, for the purposes of any proceedings of the Inns Conduct
Committee, Part 4 applies as if version 3.5 of the BSB Handbook were in
force.

\rulesubsubsection{rQ2}

Section 4.C applies to all \emph{practising barristers}.

\rulesection{Part 4
- B. Bar Qualification Rules }



\ocsection{Part 4
- B1. Purpose of the Bar Qualification Rules} 


\ocsubsection{oC1}

To provide routes for the qualification of \emph{barristers} that enable
them to meet the Professional Statement and to provide for the
regulation of AETOs.

\rulesection{Part 4
- B2. Routes to Qualification as a barrister and authorised person
}


\rulesubsubsection{rQ3}

To be called to the \emph{Bar} by an \emph{Inn} an individual must have
successfully completed the following:

\nl\item academic legal training;

\item vocational training;

\item the number of qualifying sessions as a student member of an
\emph{Inn} as prescribed from time to time by the \emph{BSB}; and

\item pay such fee or fees as may be prescribed.\ln

\rulesubsubsection{rQ4}

To obtain a \emph{provisional practising certificate} a \emph{barrister}
must:
\nl
\item have successfully completed a period of \emph{pupillage} satisfactory
to the \emph{BSB};

\item pay such fee or fees as may be prescribed.\ln

\rulesubsubsection{rQ5}

To obtain a \emph{full practising certificate} a \emph{barrister} must:

\nl \item have successfully completed a further period of \emph{pupillage}
satisfactory to the \emph{BSB};

 \item pay such fee or fees as may be prescribed.\ln

\rulesubsubsection{rQ6}

The BSB shall set out in writing:

\nl  \item the requirements to be met by an \emph{Inn} in admitting student
members and calling individuals to the \emph{Bar};

 \item the manner in which an \emph{Inn} shall assess whether such
individuals are fit and proper; and

 \item the minimum requirements for the delivery of qualifying sessions by
an \emph{Inn}.\ln

\rulesubsubsection{rQ6A}

Where it is alleged that the \emph{call declaration} made by a
\emph{barrister} on \emph{call} was false in any material respect or
that the \emph{barrister} has engaged before \emph{call} in conduct
which is dishonest or otherwise discreditable to a \emph{barrister} and
which was not, before \emph{call}, fairly disclosed in writing to the
\emph{Inn} calling them or where any undertaking given by a
\emph{barrister} on \emph{call} to the \emph{Bar} is breached in any
material respect that shall be treated as an allegation of a breach of
this \emph{Handbook} and will be subject to the provisions of Part 5.





\rulesection{Exemptions
}


\rulesubsubsection{rQ7}

The \emph{BSB} may grant exemptions from all or part of the requirements
set out in rQ3 to rQ5 above.

\rulesubsubsection{rQ8}

In deciding whether to grant an exemption from part or all of any
component of training, the \emph{BSB} will determine whether the
relevant knowledge and experience of the applicant make it unnecessary
for further training to be required.

\rulesubsubsection{rQ9}

An exemption from part or all components of training may be granted
unconditionally or subject to conditions, which may include in an
appropriate case:
\nl
 \item a requirement to do training instead of the training prescribed by
this Section; and/or

 \item a condition that the applicant must pass a \emph{Bar Transfer Test}.
\ln
\rulesubsubsection{rQ10}

Where the BSB exempts an individual pursuant to rQ7 above, it may also:
\nl
 \item grant exemption in whole or in part from the requirement to attend
\emph{qualifying sessions}; and

 \item specify the period within which any requirement to attend
\emph{qualifying sessions} must be fulfilled, which may be a period
ending after the individual concerned has been called to the \emph{Bar}.
\ln
\rulesubsubsection{rQ11}

An application for exemption under this Section must be in such form as
may be prescribed by the \emph{BSB} and contain or be accompanied by the
following:
\nl
 \item details of the applicant's educational and professional
qualifications and experience that meets the standards required of
candidates;

 \item evidence (where applicable) that the applicant is or has been
entitled to exercise rights of audience before any court, specifying the
rights concerned and the basis of the applicant's entitlement to
exercise such rights;

 \item any other representations or evidence on which the applicant wishes
to rely in support of the application;

 \item verified English translations of every document relied on which is
not in the English language; and

 \item payment of such fee or fees as may be prescribed.\ln

\rulesubsubsection{rQ12}

Before deciding whether to grant any exemption under this Section, the
\emph{BSB} may make any further enquiries or require the applicant to
provide any further information that it considers relevant.

\rulesection{Full
exemption }


\rulesubsubsection{rQ13}

If the \emph{BSB} is satisfied that an applicant falls within Rule Q14,
the \emph{BSB} will:

\nl  \item exempt the applicant from any component of training prescribed by
this Section which the applicant has not fulfilled; and

 \item authorise the applicant to practise as a \emph{barrister} on their
being admitted to an Inn and called to the \emph{Bar} subject to
complying with the Handbook.\ln

\rulesubsubsection{rQ14}

The following categories of individual fall within this Rule:

\nl  \item an individual who has been granted rights of audience by an
\emph{approved regulator} and who is entitled to exercise those rights
in relation to all proceedings in all courts of England and Wales;

 \item subject to Rule Q15, an individual who has been granted rights of
audience by an \emph{approved regulator} and who is entitled to exercise
those rights in relation to either all proceedings in the High Court or
all proceedings in the Crown Court of England and Wales (but not both);

 \item a \emph{barrister} of Northern Ireland who has successfully completed
pupillage in accordance with the rules of the Bar of Northern Ireland;

 \item subject to Rule Q16, a \emph{Qualified Swiss~Lawyer}.
\ln
\rulesubsubsection{rQ15}

The \emph{BSB} may exceptionally require an applicant who falls within
Rule Q14.2 to do part of \emph{pupillage} if it considers this necessary
having regard particularly to the knowledge, professional experience and
intended future \emph{practice} of the applicant.

\rulesubsubsection{rQ16}

Subject to Rules Q18 to Q20, the \emph{BSB} may require a
\emph{Qualified Swiss~Lawyer} to pass a \emph{Bar Transfer Test} if the
\emph{BSB} determines that:

\nl  \item the matters covered by the education and training of the applicant
differ substantially from those covered by the \emph{academic legal
training} and the \emph{vocational training}; and

\item the knowledge acquired by the applicant throughout their professional
experience does not fully cover this substantial difference.\ln

\rulesection{Registered
European Lawyers }


\rulesubsubsection{rQ17}

The Rules governing registration as a \emph{Registered European Lawyer}
are in Section 3.D of this \emph{Handbook}.

\rulesubsubsection{rQ18}

To the extent provided in the \emph{Swiss Citizens' Rights Agreement},
the BSB may not require an applicant who is a \emph{Registered European
Lawyer} and who falls within Rule Q20 or Q21 to pass a \emph{Bar
Transfer Test} unless it considers that the applicant is unfit to
practise as a \emph{barrister}.

\rulesubsubsection{rQ19}

In considering whether to require an applicant who falls within Rule Q21
to pass a \emph{Bar Transfer Test}, the \emph{BSB} must:

\nl \item take into account the professional activities the applicant has
pursued while a \emph{Registered European Lawyer} and any knowledge and
professional experience gained of, and any training received in, the law
of any part of the United Kingdom and of the rules of professional
conduct of the \emph{Bar}; and

\item assess and verify at an interview the applicant's effective and
regular pursuit of professional activities and capacity to continue the
activities pursued.\ln

\rulesubsubsection{rQ20}

To fall within this Rule an applicant must have:

\nl \item for a period of at least three years been a \emph{Registered European
Lawyer}; and

 \item for a period of at least three years effectively and regularly
pursued in England and Wales under a \emph{Home Professional Title}
professional activities in the law of England and Wales; and

\item applied for admission to the \emph{Bar} before 1 January 2025.\ln

\rulesubsubsection{rQ21}

To fall within this Rule an applicant must have:

\nl  \item for a period of at least three years been a \emph{Registered European
Lawyer}; and

 \item for a period of at least three years effectively and regularly
pursued in England and Wales professional activities under a \emph{Home
Professional Title}; and

 \item  for a period of less than three years effectively and regularly
pursued in England and Wales under a \emph{Home Professional Title}
professional activities in the law of England and Wales;~and

 \item applied for admission to the \emph{Bar} before 1 January 2025.\ln

\rulesubsubsection{rQ22}

For the purpose of this Section, activities are to be regarded as
effectively and regularly pursued if they are actually exercised without
any interruptions other than those resulting from the events of everyday
life such as absence through illness or bereavement, customary annual
leave or parental leave.

\rulesection{Partial
exemption }


\rulesubsubsection{rQ23}

If the \emph{BSB} is satisfied that an applicant falls within Rule Q24,
the \emph{BSB} will exempt the applicant from the \emph{academic legal
training} and the \emph{vocational training} and, if the \emph{BSB}
thinks fit, from part or all of \emph{pupillage}.

\rulesubsubsection{rQ24}

If the \emph{BSB} is satisfied that an applicant falls within Rule Q24,
the \emph{BSB} will exempt the applicant from the \emph{academic legal
training} and the \emph{vocational training} and, if the \emph{BSB}
thinks fit, from part or all of \emph{pupillage}. The following
categories of individual fall within this Rule:
\nl
 \item an individual who has been granted rights of audience by another
\emph{Approved Regulator} and is entitled to exercise those rights in
relation to any class of proceedings in any of the Senior Courts or all
proceedings in county courts or magistrates' courts in England and
Wales;

 \item a \emph{Qualified Foreign Lawyer} who has for a period of at least
three years regularly exercised full rights of audience in courts which
administer law substantially similar to the common law of England and
Wales;

 \item a teacher of the law of England and Wales of experience and academic
distinction.\ln

\rulesection{Temporary
call to the Bar of Qualified Foreign Lawyers }


\rulesubsubsection{rQ25}

A \emph{Qualified Foreign Lawyer} (``the applicant'') who falls within
Rule Q24.2 may apply to be called to the \emph{Bar} by an \emph{Inn} on
a temporary basis for the purpose of appearing as counsel in a
particular case before a \emph{court} of England and Wales without being
required to satisfy any other requirements of this Section if the
applicant has:

 \nl  \item obtained from the \emph{BSB} and submitted to an \emph{Inn} a
\emph{Temporary Qualification Certificate~}specifying the case for the
purposes of which the applicant is authorised to be~called to the
\emph{Bar};

 \item duly completed and signed a \emph{call declaration} in the form
prescribed by the \emph{BSB} from time to time; and

 \item paid such fee or fees as may be prescribed.\ln

\rulesubsubsection{rQ26}

The \emph{BSB} will issue a \emph{Temporary Qualification Certificate}
if the applicant submits to the \emph{BSB}:

\nl
 \item evidence which establishes that the applicant is a \emph{Qualified
Swiss~Lawyer} or falls within Rule Q24.2;

 \item a \emph{certificate of good standing}; and

 \item evidence which establishes that a \emph{professional client} wishes
to instruct the applicant to appear as counsel in the case or cases for
the purposes of which the applicant seeks temporary \emph{call} to the
\emph{Bar}.\ln

\rulesubsubsection{rQ27}

Admission to an \emph{Inn} and \emph{call} to the \emph{Bar} under Rule
Q25 take effect when the applicant is given notice in writing by the
\emph{Inn} that the applicant has been admitted to the \emph{Inn} and
called to the \emph{Bar}~under Rule Q26 and automatically cease to have
effect on conclusion of the case or cases specified in the applicant's
\emph{Temporary Qualification Certificate}.

\rulesubsubsection{rQ28}

Where an individual is dissatisfied with a decision by either the
\emph{BSB} or an \emph{Inn} in relation to rQ3 to rQ5 and rQ7 to rQ26
above they may apply to the \emph{BSB} for a review.

\rulesection{Part 4
- B3. Authorised Education and Training Organisations }


\rulesubsubsection{rQ29}

Providers of \emph{vocational training} and \emph{pupillage} must be
authorised by the \emph{BSB} as an \emph{AETO}.

\rulesubsubsection{rQ30}

An application to become an \emph{AETO} must be made in such form and be
accompanied by payment of such fee or fees as may be prescribed by the
\emph{BSB}.

\rulesubsubsection{rQ31}

In determining an application from an applicant to become an
\emph{AETO}, the \emph{BSB} will have regard to the \emph{Authorisation
Framework} and in particular the mandatory criteria. The BSB will not
approve an application to become an \emph{AETO} unless it is satisfied
that it is:

\nl  \item able to meet the mandatory criteria set out in the
\emph{Authorisation Framework} relevant to the application; and

 \item a suitable provider for the purposes of the \emph{Authorisation
Framework}.\ln

\rulesubsubsection{rQ32}

The \emph{BSB} may grant authorisation to an \emph{AETO} on such terms
and conditions as it considers appropriate including the period of
authorisation.

\rulesubsubsection{rQ33}

The BSB may vary, amend, suspend or withdraw authorisation of an
\emph{AETO} in the following circumstances:
\nl
 \item the \emph{AETO} has applied for such variation, amendment, suspension
or withdrawal;

 \item the \emph{AETO} ceases to exist, becomes insolvent or merges;

 \item the \emph{AETO} fails to comply with conditions imposed upon its
authorisation;

 \item the \emph{BSB} is of the view that the \emph{AETO} has failed or will
fail to fulfil the mandatory requirements set out in the
\emph{Authorisation Framework};

 \item the \emph{BSB} is of the view that the \emph{AETO} is not providing
the training for which it was authorised to an adequate standard or
there has been a material change in the training provided; or

 \item the \emph{BSB} is of the view that the continued authorisation of the
\emph{AETO} would inhibit the \emph{Regulatory Objectives}.\ln

\rulesubsubsection{rQ34}

An \emph{AETO} which is dissatisfied by a decision in relation to rQ31
-- rQ33 above may apply to the \emph{BSB} for a review.

\rulesection{Part 4
- B4. Review and Appeals }


\rulesubsubsection{rQ35}

Where provision is made under this Section for a review by the
\emph{BSB} of a decision, any request for such a review must be
accompanied by:
\nl
 \item a copy of any notice of the decision and the reasons for it received
by the person requesting the review (``the applicant'');

 \item where the decision is a decision of an \emph{Inn} or the \emph{ICC},
copies of all documents submitted or received by the applicant which
were before the \emph{Inn} or the \emph{ICC};

 \item any further representations and evidence which the applicant wishes
the \emph{BSB} to take into account; and

 \item payment of such fee or fees as may be prescribed.\ln

\rulesubsubsection{rQ36}

Where the decision under review is a decision of an \emph{Inn}, the
\emph{BSB} will invite the \emph{Inn} to comment on any further
representations and evidence which the applicant submits under Rule
Q35.3.

\rulesubsubsection{rQ37}

On a review under this Section the \emph{BSB}:

\nl  \item may affirm the decision under review or substitute any other decision
which could have been made on the original application;

 \item may in an appropriate case reimburse the fee paid under Rule Q35.4;
and

 \item will inform the applicant and any other interested \emph{person} of
its decision and the reasons for it.\ln

\rulesubsubsection{rQ38}

Where provision is made under this Section for a review of a decision by
the \emph{BSB}, this review may be delegated to an \emph{Independent
Decision-Making Panel}, where specified by the \emph{BSB}.

\rulesubsubsection{rQ39}

Where under this Section provision is made for a review by the
\emph{BSB} of a decision, no appeal may be made to the High Court unless
such a review has taken place.

\rulesubsubsection{rQ40}

An individual who is adversely affected by a decision of the \emph{BSB}
under Section B.2 may appeal to the High Court against the decision.

\rulesubsubsection{rQ41-rQ129}

Removed.

\rulesection{Part 4
- C. The CPD Rules }



\rulesection{The
mandatory continuing professional development requirements (Rules
Q130-Q131) }


\rulesubsubsection{rQ130}

For the purpose of this Section 4.C:
\nl
\item ``calendar year'' means a period of one year starting on 1 January in
the year in question;

\item ``continuing professional development'' (``CPD'') means work
undertaken over and above the normal commitments of a \emph{barrister}
and is work undertaken with a view to developing the \emph{barrister's}
skills, knowledge and professional standards in areas relevant to their
present or proposed area of practice in order to keep the
\emph{barrister} up to date and maintain the highest standards of
professional practice.

\item ``CPD Guidance'' means guidance issued by the Bar Standards Board
from time to time which sets out the CPD structure with which an EPP
\emph{barrister} should have regard to.

\item ``EPP'' means the Established Practitioners Programme which requires
\emph{barristers}, once they have completed the NPP, to undertake CPD
during each calendar year in accordance with these Rules.

\item the ``mandatory requirements'' are those in Rules Q131 to Q138 below.

\item ``NPP'' means the New Practitioner Programme which requires
\emph{barristers} to complete CPD in their first three calendar years of
practice in accordance with these rules.

\item a ``\emph{pupillage} year'' is any calendar year in which a
\emph{barrister} is at any time a \emph{pupil}.

\item a `` learning objective'' is a statement of what a \emph{barrister}
intends to achieve through their CPD activities for that calendar year
with reference to a specific aim and one or more outcomes.
\ln
\rulesubsubsection{rQ131}

Any practising \emph{barrister} who, as at 1 October 2001, had started
but not completed the period of three years referred to in the
Continuing Education Scheme Rules at Annex Q to the Sixth Edition of the
Code of Conduct must complete a minimum of 42 hours of CPD during their
first three years of \emph{practice}.


\guidancesection{Guidance
to Rule Q131 }


\guidancesubsubsection{gQ1}

Rule rQ131 is intended to apply only in those limited circumstances
where a \emph{barrister} started \emph{practice} before 1 October 2001
but after the NPP first came into force, left \emph{practice} before
completing the NPP, but has since returned. Rule rQ131 requires them to
finish their NPP during whatever is left of their first three years of
\emph{practice}.

\rulesection{The
mandatory continuing professional development requirements (Rule Q132)
}


\rulesubsubsection{rQ132}

Any practising NPP \emph{barrister} who starts \emph{practice} on or
after 1 October 2001 must during the first three calendar years in which
the \emph{barrister} holds a \emph{practising certificate} after any
\emph{pupillage} year complete a minimum of 45 hours of CPD.


\guidancesection{Guidance
to Rule Q132 }


\guidancesubsubsection{gQ2}

NPP \emph{barristers} should have regard to rQ137 and the NPP guidance
which will note the details of any compulsory courses the NPP
\emph{barristers} must complete. It also provides guidance as to the
types of activities that count towards CPD.

\rulesection{The
mandatory continuing professional development requirements (Rules
Q133-Q134) }


\rulesubsubsection{rQ133}

Subject to Rule Q136, any EPP \emph{barrister} who holds a
\emph{practising certificate} or certificates during a calendar year
must undertake CPD.

\rulesubsubsection{rQ134}

An EPP \emph{barrister} who is required to undertake CPD must:
\nl
 \item prepare a written CPD Plan setting out the \emph{barrister's}
learning objectives and the types of CPD activities they propose to
undertake during the calendar year

 \item keep a written record of the CPD activities the \emph{barrister} has
undertaken in the calendar year

 \item keep a written record in the CPD Plan for each calendar year of:
\al
 \item the \emph{barrister's} reflection on the CPD they have undertaken;

 \item any variation in the \emph{barrister's} planned CPD activities; and

 \item the \emph{barrister's} assessment of their future learning
objectives.\la

 \item  Retain a record of the CPD Plan and completed CPD activities for
three years.

 \item submit to the Bar Standards Board an annual declaration of completion
of CPD in the form specified by the BSB.\ln


\guidancesection{Guidance
to Rules Q133-Q134 }


\guidancesubsubsection{gQ3}

EPP \emph{barristers} who are required by these Rules to undertake CPD
should refer to the CPD Guidance. The CPD Guidance provides further
detailed information which EPP \emph{barristers} should have regard to
when planning, undertaking and recording their CPD. The CPD Guidance is
not prescriptive. Its purpose is to provide a structure that would
represent good practice for most \emph{barristers} when considering
their CPD requirements.

\guidancesubsubsection{gQ4}

The CPD Guidance explains that these Rules do not specify a minimum
number of CPD hours which an EPP \emph{barrister} must undertake in a
calendar year: it is the responsibility of the individual
\emph{barrister} to determine the CPD activities they will undertake in
order meet the requirements of CPD. The Bar Standards Board will assess
and monitor \emph{barristers'} compliance with CPD.

\guidancesubsubsection{gQ5}

The underlying principle behind the requirement to plan CPD and set
learning objectives is that \emph{barristers} consider their own
circumstances and development needs when they complete CPD activities.
This best ensures that activities completed contribute to the
development of the barrister's practice.

\rulesection{The
mandatory continuing professional development requirements (Rules
Q135-Q138) }


\rulesubsubsection{rQ135}

Upon the request of the Bar Standards Board, a \emph{barrister} must
produce their CPD Plan and record of CPD activities for assessment.

\rulesubsubsection{rQ136}

Rule Q133 does not apply:
\nl
 \item in the case of a \emph{barrister} to whom Rule Q131 applies, to any
calendar year forming or containing part of the period of 3 years
referred to in Rule Q131; or

 \item in the case of a \emph{barrister} to whom Rule Q132 applies, during
any \emph{pupillage} year or during the first three calendar years in
which the \emph{barrister} holds a \emph{practising certificate}.\ln

\rulesubsubsection{rQ137}

The \emph{Bar Standards Board} may, by resolution, specify the nature,
content and format of courses and other activities which may be
undertaken by \emph{barristers} (or by any category of
\emph{barristers}) in order to satisfy the mandatory requirements.

\rulesubsubsection{rQ138}

The \emph{Bar Standards Board} may, by resolution and after consultation
with the Inns, Circuits and other providers as appropriate, vary the
minimum number of hours of CPD which must be completed by an NPP
\emph{barrister} in order to satisfy any of the mandatory requirements.


